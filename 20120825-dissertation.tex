\documentclass[12pt,letterpaper,oneside,final]{memoir}
%M-x shell
%latexmk -pdf -e '$pdflatex=q/xelatex %O %S/' 20120825-dissertation.tex
\usepackage[no-math]{fontspec}
\usepackage{xltxtra}
\defaultfontfeatures{Scale=MatchLowercase,Mapping=tex-text}
\setmainfont[Mapping=tex-text]{Junicode} %CMU Serif
\setsansfont[Mapping=tex-text]{Junicode} %CMU Sans Serif
\newfontfamily\greekfont{Linux Libertine O} %CMU Serif,Linux Libertine O,Junicode
%\setmainfont[Mapping=tex-text]{CMU Serif}
\usepackage{xkeyval}
\usepackage{polyglossia}
\setdefaultlanguage[variant=american]{english}
\setotherlanguage[variant=ancient,numerals=arabic]{greek}
\setotherlanguage[spelling=new]{german}
\setotherlanguages{latin,french,italian,spanish}
\usepackage[babel=once,english=american,autostyle=tryonce,strict=true]{csquotes}
\usepackage[backend=biber,style=authoryear,sorting=debug,bibstyle=authoryear,citestyle=authoryear,useprefix=false,firstinits=false,url=false,usetranslator=true]{biblatex}%was: firstinits=true
%%%%%%%%%%%%%%%%%%%%%%%%%%%%%
%sorting=debug: by sort key
%sorting=nyt: what I need
%%%%%%%%%%%%%%%%%%%
%\DeclareAutoCiteCommand{plain}{\cite}{\cites}
\DeclareAutoCiteCommand{plain}{\textcite}{\textcites}
\DeclareAutoCiteCommand{inline}{\textcite}{\textcites}
%\DeclareAutoCiteCommand{footnote}[l]{\footcite}{\footcites}
%\DeclareAutoCiteCommand{footnote}[f]{\smartcite}{\smartcites}
\bibliography{masterbib}
\usepackage[final]{hyperref}%?%hyperfootnotes=false
\hypersetup{bookmarks=false,        % show bookmarks bar?
    unicode=true,           % non-Latin characters in Acrobat’s bookmarks
    pdftoolbar=true,        % show Acrobat’s toolbar?
    pdfmenubar=true,        % show Acrobat’s menu?
    pdffitwindow=false,     % window fit to page when opened
    pdfstartview={FitH},    % fits the width of the page to the window
    pdftitle={Ethics of Leadership: Organization and Decision--Making in Caesar's \emph{Bellum~Gallicum}},
    pdfauthor={Kyle P. Johnson},     % author
    pdfsubject={Dissertation on organization and decision--making in Julius Caesar's Bellum Gallicum},   % subject of the document
    pdfcreator={Kyle P. Johnson},   % creator of the document
    pdfproducer={Kyle P. Johnson}, % producer of the document
    pdfkeywords={Julius Caesar, Bellum Gallicum, Gallic War, communication, deliberation, decision--making, leadership, organization, Xenophon, Anabasis}, % list of keywords
    pdfnewwindow=true,      % links in new window
    colorlinks=true,       % false: boxed links; true: colored links
    linkcolor=black,          % color of internal links
    citecolor=black,        % color of links to bibliography
    filecolor=black,      % color of file links
    urlcolor=black           % color of external links
}
\usepackage{microtype}
\usepackage{xecolor}
\definergbcolor{blue}{0000FF}
\definergbcolor{red}{FF0000} %\textxecolor{colorname}{text}
\XeTeXdashbreakstate=1
\usepackage{indentfirst}
\usepackage{outline}
\usepackage{verbatim}
\usepackage{enumerate}
%\usepackage{tikz}
%\usetikzlibrary{shapes,backgrounds}
\usepackage{longtable}
%\usepackage{lscape}
%\usepackage{verse}
%\usepackage{rotating}
\usepackage{ccicons}
\usepackage{bookmark}
%\usepackage{ledmac,ledpar}
\usepackage{etoolbox}
%!%%%%%%%%%%%%%%%%%%%%%for non-italicized headings%%%%%
% http://tex.stackexchange.com/questions/32655/remove-italic-from-memoir-headings-pagestyle
\makeevenhead{headings}{\leftmark}{}{}
\makeoddhead{headings}{\rightmark}{}{}
\makeevenfoot{headings}{}{\thepage}{}
\makeoddfoot{headings}{}{\thepage}{}

%!%%%%%%%%%%%%%%%%%%%%%%%%%%%%%%%%%%%%%%%%%%%%%%%%%%%%%%
%!%%%%%%%%%%%%%%%%%%%%%for margins, l=1.5; rest=1.0, maybe add 0.1in%%%%%
\setstocksize{11in}{8.5in}
\settrimmedsize{11.0in}{8.5in}{*} %\settrimmedsize{ height }{ width }{ ratio }
\settypeblocksize{7.75in}{5.8in}{*} %\settypeblocksize{ height }{ width }{ ratio } %note: 7.25h gives margins of 1.5 on top/bottom w/ ubratio of 1
\setlrmargins{1.5in}{*}{*} %\setlrmargins{ spine }{ edge }{ ratio } %spine = left, edge = right; only answer one or two of these values
%\setlrmarginsandblock{1.5in}{1.0in}{*} %\setlrmarginsandblock{ spine }{ edge }{ ratio } %
%\setulmargins{*}{*}{*} %\setulmargins{ upper }{ lower }{ ratio }
%\setulmarginsandblock{0.5in}{*}{*} %\setulmarginsandblock{ upper }{ lower }{ ratio }
%\setheadfoot{*}{*} %\setheadfoot{ headheight }{ footskip }
%\setheaderspaces{1.0in}{*}{*} %\setheaderspaces{ headdrop }{ headsep }{ ratio }
%\checkandfixthelayout[lines]
\flushbottom
%!%%%%%%%%%%%%%%%%%%%%%%%%%%%%%%%%%%%%%%%%%%%%%%%%%%%%%%
\apptocmd{\sloppy}{\hbadness 10000\relax}{}{}
%\renewcommand{\@pnumwidth}{3em} %taken from memman p. 153
%\renewcommand{\@tocrmarg}{4em} %taken from memman p. 153
\setsecnumdepth{subparagraph}
\makeatletter
\renewcommand\@makefntext{\hspace*{2em}\@thefnmark. }
\newenvironment*{singlespcquote}
        {\quote\SingleSpacing}
        {\endquote}
\SetBlockThreshold{0}
\SetBlockEnvironment{singlespcquote}
\SetCiteCommand{\parencite} %default is \cite
  %csquote + biblatex; see csquotes.pdf section 5 + 8.6
  %\textcquote[prenote][postnote]{key}[punct]{text}[tpunct]
  %\blockcquote[prenote][postnote]{key}[punct]{text}[tpunct]
  %usually:
  %\textcquote[page#]{key}{quote}
  %\blockcquote[page#]{key}[.]{quote}
  %\textcites(pre)(post)[pre][post]{key}...[pre][post]{key}
  %example: \textcites(and chapter 3)[35]{riggsby2006}[78]{hammond1996}[23]{levene2010}
\title{Ethics of Leadership: Organization and Decision--Making in Caesar's \emph{Bellum~Gallicum} \\ 
  \ifdraftdoc
  (version: \today)
  \fi
}
\author{Kyle~P.~Johnson}
%\date{January 31, 2008}
\begin{document}
%\fussy
%\hyphenpenalty=5000   %1000 default=?
%\tolerance=1000        %1000 %200= default
%\setlength{\emergencystretch}{3em}
%\midsloppy
\fussy
\vbadness=10000 % badness above which bad vboxes are shown. (Default = 10000?)
\frontmatter
\hyphenation{im-pe-ri-um le-ga-tus le-ga-ti prae-to-re La-bi-en-us ha-bi-tus auc-to-ri-tas po-tes-tas am-i-ci-ti-a  Der-ri-da Pom-pey Com-men-ta-ri-i mi-me-sis Saus-sure Aug-us-tus im-per-at-or Bel-lum Gal-li-cum Cor-pus Cae-sa-ri-an-um vir-tus ci-vi-le im-per-at-or Ner-vi-i Ver-cin-get-or-ix Gal-ba Ar-i-o-vis-tus Trans-al-pine Gal-ba Xe-no-phon An-ab-as-is Var-ro Al-ex-an-dri-num Bel-lum Af-ric-um His-pan-i-en-se eth-no-gra-phies con-si-li-um con-ci-li-um con-lo-qui-um fa-ma Vel-lau-no-dun-um Ce-na-bum No-vi-od-un-um Ju-li-us Cae-sar Quin-tus Mar-cus Ci-ce-ro pri-mo-rum or-di-num cen-tu-ri-on-es en-thu-mē-ma and pa-ra-deig-ma ep-i-chei-re-ma ep-i-chei-re-ma-ta en-thu-meme pro-po-si-ti-o ra-ti-o ra-ti-on-is con-fir-ma-ti-o com-fir-ma-ti-o ex-or-na-ti-o con-ple-xi-o ra-ti-o-ci-na-ti-o ra-ti-on-es con-fir-ma-ti-on-es dol-us ep-i-chei-re-ma Ca-ti-li-na Iu-gur-tha res ges-tae Her-en-ni-um Prin-ce-ton Rhe-tor-ic Twel-ve Mne-mo-sy-ne Ha-bi-nek Od-ys-sey Twel-ve thu-mos phro-ne-sis Od-ys-se-us Pol-yb-i-us Mal-den Leip-zig Teub-ner Plu-tarch}

\DoubleSpacing

%\raggedright turn on with %\usepackage[none]{hyphenat}
%\parindent=0.5in turn on with %\usepackage[none]{hyphenat}

\begin{center}

\thispagestyle{empty}
Ethics of Leadership: Organization and Decision--Making in Caesar's \emph{Bellum~Gallicum}
  \ifdraftdoc \\ (version: \today)
  \fi
\\
\vspace{20mm}
by\\
\vspace{10mm}
Kyle~P.~Johnson\\
\vspace{10mm}
A dissertation submitted in partial fulfillment\\
of the requirements for the degree of\\
Doctor of Philosophy\\
Department of Classics\\
New York University\\
September, 2012
\end{center}
\vspace{35mm}
\begin{flushright}
{\rule[0pt]{45mm}{0.1mm}}\\ %rule[raise-height]{width}{height} * raise-height specifies how high to raise the rule (optional) * width specifies the length of the rule (mandatory) * height specifies the height of the rule (mandatory)

Joy~Connolly
\end{flushright}

\newpage

\thispagestyle{empty}

\begin{center} This dissertation is licensed under a Creative Commons Attribution--NonCommercial--ShareAlike 3.0 Unported License \ccbyncsa{}\,(GPL--compatible). For the full text of this license, see Appendix~\ref{license}. \\ The dissertation is typeset with \XeLaTeX\ (MIT/X11 License; GPL--compatible) with the Junicode and Linux Libertine fonts (both covered under the SIL Open Font License; GPL--compatible). \\ Copyright © 2012 Kyle P. Johnson. \end{center}

\newpage

%\end{comment}
\thispagestyle{empty}

\begin{comment}
\chapter{Dedication}
\SingleSpacing
\end{comment}
\DoubleSpacing

\newpage

\chapter{Acknowledgments}%aaaa

%NYU
I thank the advisors to this dissertation, Professors Joy Connolly and David Levene, plus committee members Michael Peachin, Peter Meineck, and Christina Kraus. In the years to come, I will consider their commitment to high standards a model to live up to. I also thank the good people of New York University's Classics Department, and in particular Larissa Bonfante, whose kindness and encouragement have been crucial to my academic development. Early in the research process for this dissertation, generous grants, from the New York Classical Club and N.Y.U.'s Global Fellowship, allowed me to spend two summers in Rome, where I was fortunate enough to study with Myles McDonnell, and in Florence.

%family
My family's love has seen me through not only graduate school but nurtured me from birth. As best as I am able, I express gratitude to my late father Jim and mother Cindy, grandmothers Marilyn and Marylass, brother Adam, and auntie Julie. The newest addition to the Johnson family, my beautiful fiancée Jamie, is a true blessing. Jahan, Jeremy, Will, and Amit -- who are like family by this point -- have offered steadfast friendship.

%dharma
Last but not least, I thank Khenpo Pema Wangdak for so generously sharing the Buddha's teachings of wisdom and compassion. Khenpo Pema's embodiment of these teachings is a thing of wonder to me.


\newpage

\chapter{Abstract}
\DoubleSpacing

%!check word requirements for submission
This dissertation studies how Julius Caesar represents himself as an ideal leader in the \emph{Bellum~Gallicum}. It argues that Caesar achieves this idealized self--portrait through interactional relationships between the general's mind and his subordinate officers. Chapter~\ref{ch-pres} analyzes how Caesar communicates with his army and foreigners. In reference to the Greco--Roman literary tradition, Chapter~\ref{ch-delib} is about the general's mind and the importance that the text gives to his decision--making process. Chapter~\ref{ch-delib-subs} looks at episodes in which subordinate officers must deliberate for themselves, and how these independent deliberations ultimately demonstrate the brilliance of Caesar's organization of the army and his centrality to it. Considered together, the three chapters demonstrate that Caesar's mind and its manifestation portray Caesar as a uniquely qualified bearer of Roman \emph{imperium}.

\newpage
\renewcommand*{\cftappendixname}{Appendix\space}
\renewcommand*{\contentsname}{Table of contents}
\setcounter{tocdepth}{2}% chapters and above
%this removes TOC listing from TOC
\begin{KeepFromToc}
\tableofcontents
\end{KeepFromToc}
\clearpage
\newpage
\clearpage

\chapter{Abbreviations}
\SingleSpacing
\begin{longtable}{ll}
\emph{Ad~Att.} & Cicero, \emph{Epistulae ad Atticum}\\
\emph{Ad~fam.} & Cicero, \emph{Epistulae ad familiares}\\
\emph{Ad~Her.} & \emph{Rhetorica ad Herennium}\\
\emph{Anab.} & Xenophon, \emph{Anabasis}\\
\emph{BC} & Caesar, \emph{Bellum~civile} \\
\emph{BG} & Caesar, \emph{Bellum~Gallicum} \\
\emph{Brut.} & Cicero, \emph{Brutus} \\
\emph{Commentarii} & The \emph{Bellum~Gallicum} and \\
 & \emph{Bellum~civile} of Caesar\\
Caesar & C. Julius Caesar \\
Cicero & M. Tullius Cicero\\
\emph{De~inv.} & Cicero, \emph{De inventione}\\
\emph{De~or.} & Cicero, \emph{De oratore} \\
\emph{De~sen.} & Cicero, \emph{Cato maior de senectute}\\
\emph{Hist.} & Polybius, \emph{Historiae} \\
\emph{Il.} & Homer, \emph{Iliad} \\
\emph{LSJ} & \emph{A Greek--English Lexicon} \\
 & \parencite{lsj1996} \\
\emph{OCD}\textsuperscript{3} & \emph{Oxford Classical Dictionary} \\
 & \parencite{ocd3}\\
\emph{Od.} & Homer, \emph{Odyssey} \\
\emph{OLD} & \emph{Oxford Latin Dictionary} \\
 & \parencite{old} \\
\emph{Phil.} & Cicero, \emph{Orationes Philippicae}\\
Pompey & Cn. Pompeius Magnus \\
\emph{Prov.~cons.} & Cicero, \emph{Provinciis consularibus}\\
Q.~Cicero & Q. Tullius Cicero\\
\emph{Rhet.} & Aristotle, \emph{Rhetorica}\\
Scipio & P. Cornelius Scipio Africanus\\
\end{longtable}

\DoubleSpacing
\mainmatter
\clearpage
\cleardoublepage  
\phantomsection  
\addcontentsline{toc}{chapter}{Introduction}

\chapter*[Introduction]{Introduction}
\label{ch-intro}
In Cicero's \emph{De senectute}, Cato the Elder argues against those who insist upon the vicissitudes of old age. He discourses, instead, on the benefits belonging to the elderly.\footnote{The \emph{De senectute} dates to 44 B.C. \parencite[438]{lintott2008}. For summary, see \textcite[438--439]{lintott2008} and, for commentary, \textcite{powell1988}.} In making his point, Cato scrutinizes concepts of agency and how agency is valued: In their aging, the old do not lose the ability to act, but they perform different kinds of labor, which in fact are more important than the physical activity of the young. Through a ship of state metaphor, an old man is likened to a helmsman and young men to deckhands. \blockquote[\emph{De~sen.}~17]{\textlatin{Nihil igitur adferunt qui in re gerunda versari senectutem negant, similesque sunt ut si qui gubernatorem in navigando nihil agere dicant, cum alii malos scandant, alii per foros cursent, alii sentinam exhauriant, ille autem clavum tenens quietus sedeat in puppi. Non facit ea quae iuvenes; at vero multo maiora et meliora facit. Non viribus aut velocitate aut celeritate corporum res magnae geruntur, sed consilio auctoritate sententia.}\footnote{Text from \textcite{powell2006}.}} \blockquote[\emph{De~sen.}~17]{They offer nothing, those who deny that old age engages in the carrying on of affairs, and are like those who would say that the pilot of a ship does nothing in steering a ship -- since some scale masts, others rush about the deck, and others pump bilge water -- but he holding the tiller sits quietly on the stern. He does not do what the young men do, but surely what he does is far more important and better. Not with strength, speed, or adroitness of body are great things done, but with advice, influence, and opinion.\footnote{All translations are my own, unless otherwise noted.}} The metaphor explains the agencies of the old and young through a series of contrasts between the respective activities of the captain and deckhands.\footnote{For precedent ship of state metaphors, see \textcite[139--140]{powell1988}.} Among these contrasts are those between a singular leader (\textquote{\textlatin{ille}}) and many others (\textquote{\textlatin{alii \ldots{} alii \ldots{} alii}}); motionless (\textquote{\textlatin{clavum tenens quietus sedeat}}) and physical action (\textquote{\textlatin{scandant \ldots{} cursent \ldots{} exhauriant}}); and action of broad versus narrow effect (\textquote{\textlatin{in navigando}} versus scaling masts, etc.). While the deckhands' agency is located in their physical bodies (\textquote{\textlatin{viribus aut velocitate aut celeritate corporum}}), the helmsman's lies in his mind and the communication of it (\textquote{\textlatin{consilio auctoritate sententia}}).\footnote{In the \emph{De~senectute}, expressions similar to \textquote{\textlatin{consilio auctoritate sententia}} also include \textquote[e.g., 19 and 67; {\cite[140--141]{powell1988}}]{\textlatin{ratio}}.} Upon this division of action, Cato projects values, considering the efforts of the helmsman as more important and better (\textquote{\textlatin{vero multo maiora et meliora}}). Important action begins in the mind of the quiet helmsman and is then expressed in language to his subordinate workers. Ultimately, Cato's ship of state metaphor is an argument for how \emph{\textlatin{res gestae}} (\textquote[][,]{\textlatin{re gerunda}} \textquote{\textlatin{res magnae geruntur}}; i.e., actions of politics and war) happen. Since this ship is a metaphor for how political communities work, the \emph{gubernator} is an even stronger analog to the Roman senator, who functioned in an advice--offering capacity.\footnote{\textquote{\textlatin{Consilio auctoritate sententia}} \textquote[{\cite[141]{powell1988}}]{strongly suggest the Roman senate}.} From these abstractions about actors, action, agency, division of labor, and valuation of labor, Cicero's Cato offers a rich set of ideas and associations concerning the Roman state.

This dissertation argues that Julius Caesar's \emph{Bellum~Gallicum} (or \emph{Gallic~War}) may be read, like Cato's ship, for what its abstractions and particular embodiments of human action express about the Roman republic, its leaders, and the subordinates to these leaders.\footnote{\textcite{wiseman1998} is the most recent and thorough account of the publication date of the \emph{Bellum~Gallicum}. He convincingly argues that the work was gradually written and incrementally published during Caesar's years in Gaul.} As an example of what I am not trying to do, consider \textcite{goldsworthy1998}, an excellent essay on Caesar's portrayal of himself as a general (his \textquote{instinctive genius}), which grounds its arguments in Caesar alone.\footnote{Goldsworthy argues that the presentation of Caesar in the \emph{Bellum~Gallicum} accords to a Roman audience's expectations and his own actual manner of waging war. While this article does mention Roman soldiers, it touches upon Caesar's non--use of them (e.g., interviewing captives himself, p.~203) or his stalwart use of them in battle (e.g., p.~206--207), in which \textquote[{\cite[206]{goldsworthy1998}}]{Caesar was even more active}.} Instead, this dissertation emphasizes Caesar in relationship to others, with the notion that different qualities of his idealized generalship will thus come to light. The armies, officers, and general of the \emph{Bellum~Gallicum}, I argue, are crucial to the text's portrait of the author's \emph{res gestae}, intended to persuade a readership of more than just facts about a series of events that took place in northwestern Europe in the 50's B.C., but elaborate arguments on behalf of its author Julius Caesar, his conceptualization of how \emph{\textlatin{res gestae}} happen, and what he has to offer the senate and people of Rome.

Of two conclusions that each chapter of this dissertation makes, one is obvious, the other perhaps not. Not surprisingly, like Cato's helmsman/politician, Caesar represents himself (incidentally, also a \emph{gubernator}) as the most important laborer in the Roman army. The \emph{Bellum~Gallicum}'s division of labor even approximates Cato's above: Chapters~\ref{ch-pres} and~\ref{ch-delib} show, respectively, the importance that Caesar's communicative and mental activity plays in the text's construction of events. But the \emph{Bellum~Gallicum} is not all about Caesar but also the soldiers who do his bidding. A second persistent thesis of this dissertation, emerging especially in chapters~\ref{ch-pres} and~\ref{ch-delib-subs}, is that Roman subordinate officers and soldiers play a critical role in this text's narratives, even if only to cast Caesar in a good light. In better recognizing the (occasionally significant) roles that these subordinate figures play in the \emph{Bellum~Gallicum}, this dissertation offers some improved explanations of how and why how Caesar takes center stage. For one actor (like the \textquote[][,]{\textlatin{ille}} above) can only be considered great in reference to other actors (the \textquote{\textlatin{alii}}); it is therefore necessary to understand who these others are, what they do, and how the protagonist interacts with them. In a relative subordination of Caesar, I hope to reassess the significance of overlooked actors, especially subordinate officers, in the narratives of the \emph{Bellum~Gallicum}.

%Dissertation summary
Cato's ship runs as a harmoniously functioning system in which each individual has a role with attendant responsibilities. Within the major process of the sailing ship, there are many sub--processes performed by others. In understanding the larger process of Caesar's army, as depicted in the \emph{Bellum~Gallicum}, this dissertation focuses on the various processes of others within his army. Two particular types of processes -- deliberative and communicative -- are of prime importance to the text, as I show in the following three chapters. Chapter~\ref{ch-pres} begins the dissertation as a study of the communicative relationships between Caesar and his army (section~\ref{bg-legati}), and Caesar and non--Romans (sections~\ref{bg-concilium}, \ref{bg-consilium}, \ref{bg-conloquium}, and \ref{comm-patt}). The next two chapters turn to deliberative activity by Caesar and his subordinates. Caesar's decision--making labor, Chapter~\ref{ch-delib} shows, is unique in the army, and emerges from a literary tradition of deliberating heroes. Chapter~\ref{ch-delib-subs} turns to the soldiers of the army and looks at a particular set of circumstances, of communicative failure within the Roman army, under which subordinate officers take on Caesar's role as deliberator.

%subordinate officers in \emph{BG}
Before summarizing the dissertation and drawing some preliminary conclusions, a look at relevant scholarship on the \emph{Commentarii} will help to locate my concerns among those of other researchers. Scholarship divides \textquote{subordination} into two parts, which I argue are in fact part of one phenomenon: Caesar's relationship to his subordinate officers and Caesar's communicative methods. Scholarship has touched upon Caesar's relationship, in the \emph{Commentarii}, to subordinate officers. By and large, this scholarship has understood a fairly stable and natural relationship between \emph{imperator} and subordinate (\emph{legatus}, \emph{tribunus}, etc.). \textcite{adcock1956} sets the tone for much scholarship that has followed, that: \blockquote[{\cite[56]{adcock1956}}]{Caesar is the \emph{imperator}, his generals are his \emph{legati}, his lieutenants only. \ldots{} To Caesar the \emph{res gestae} of his \emph{legati} can be subsumed under his own \emph{res gestae}, of which they are reckoned as a part.} This statement may be accurate, yet overlooks what I will insist upon in the course of this dissertation, that the text's construction of Caesar's accomplishments is done through its portrayal of interactional relationships between him and others. \textcite{adcock1956} elsewhere characterizes the goings on of \emph{legati} as of secondary importance, essentially dismissing them.\footnote{See \textcite[73--76]{adcock1956} on Caesar's manner of description of lieutenants: \textquote[{\cite[73--74]{adcock1956}}]{When he is describing the doings of his lieutenants the style is, in general, less emphatic, less vigorous, though even in these, as in the account of Curio's campaign, or again, in that of the disaster to the army of Sabinus and Cotta and the events that led to it, there is a more dramatic treatment of the situation. It becomes more personal as Caesar's imagination of what must have happened is engaged. On the whole, though, the operations of \emph{legati} are described so that the military quality of their actions, their \emph{consilia}, so far as these are their own and not Caesar's at one remove, can be appraised, but that is all}.} The most popular observation about Caesar and his subordinates has been of Caesar's instillation of \emph{virtus} into his officers and soldiers, such as \textcite[131-133]{goldsworthy1996} on officers' ideal dissemination of a commander's order; \textcite{brown1999} on Caesar inspiring \emph{virtus} into subordinates; \textcite[304--322]{lendon1999} on soldiers' \emph{virtus} and \emph{animus} under Caesar's leadership during battle narratives;\footnote{\textquote[{\cite[325]{lendon1999}}]{To Caesar the movement of troops over terrain was not necessarily the most important aspect of battle, the description of these movements and that terrain not necessarily the most useful way of getting at the heart of what happened in battle. Caesar's understanding of warfare suggested to him alternative topographies -- those that psychology and bravery inscribe upon the land}.} \textcite{brown2004} on how Caesar leads centurions and \emph{legati} through \emph{virtus} and \emph{consilium}; \textcite[98--99]{welch1998} on how the author praises subordinates for their \emph{virtus};\footnote{\textcite{welch1998} focuses on the relationship of Caesar to his subordinate officers, and the import of this representation on the reader's appreciation of Caesar's importance: \textquote[{\cite[85]{welch1998}}]{The argument will be based on Caesar's treatment of his officers in the \emph{Bellum~Gallicum}, on the contrasting ways Caesar writes of himself, his centurions and tribunes and his Gallic enemies, and on omissions which can be demonstrated}.} and \textcite[293--319]{mcdonnell2006} and \textcite[83--96]{riggsby2006} on competing definitions of \emph{virtus} as courage and self--restraint.\footnote{There has also been interest in the \emph{virtus} of Gauls in the \emph{Bellum~Gallicum}: for example, \textcite{barlow1998}, \textcite{rawlings1998}, \textcite{jervis2001}, \textcite[96--100]{riggsby2006}, and \textcite[247--251]{vasaly2009}.} Not explicitly about \emph{virtus}, \textcite[116--123]{powell1998} is about Caesar blaming defeats upon individual subordinates. One exception is \textcite{welch1998}, about the \emph{Bellum~Gallicum}, which looks at \textquote[{\cite[85]{welch1998}}]{the contrasting ways Caesar writes of himself, his centurions and tribunes and his Gallic enemies}. The argument amounts to a claim that political issues at Rome influenced the presence of officers, especially in the later books, which reflect \textquote[{\cite[102]{welch1998}}]{an author in political as well as military turmoil, far more consciously naming and acknowledging the members of his own class}. \textcite[esp.~pp.~166--168]{kraus2009} is another exception, in its recognizing that Caesar is not the only important character in the \emph{Bellum~Gallicum}, but shares the stories with many other Gauls and Romans. This scholarship all shares that it is interested in Caesar and his subordinate officers, often in contrast to one another, yet does not explicitly concern itself with Caesar's relationship to these actors.

%chapter 1
Chapter~\ref{ch-pres} looks at precisely the relationships that bind Caesar to the actors who populate his stories. Focusing on an extended episode from \emph{Bellum~Gallicum}~1, this chapter explains Caesar's communicative interactions with the German Ariovistus and also his own army. It is useful to look at this episode because it becomes apparent that two attitudes toward communication, corresponding to two voices of the narrator and actor Caesar, reside in the text. The \textquote{Caesar} of the stories expresses an open and eager attitude to communication, especially dialog with his foreign allies and even enemies. The narrator, however, expresses to the reader Caesar's interior doubts about the efficacy of conciliation through spoken words (sections~\ref{bg-concilium}, \ref{bg-consilium}, \ref{bg-conloquium}, and \ref{comm-patt}). The text, then, offers a portrait of Caesar as both shrewd yet dedicated to nonviolent solutions. To those within his army and through the pretense to dialogism, Caesar practices a similar dedication to communication. As I show at the insurrection at Vesontio, the order of the Roman army is constituted by communicative practices; if soldiers practice speech acts with those outside of a power hierarchy, all organization in this army will necessarily unravel (section~\ref{bg-legati}). Caesar, in his representation of himself and his army, is to show communications moving  (almost) flawlessly through his army. Caesar's orders are always present with his men even when the general is not physically present. In all, the chapter shows a dynamic cynicism and optimism which operate to portray Caesar (author and character) as one able to secure obedience from others through an implicit threat of violence, yet develop as much as possible their willing subordination.

%Caesar's communication
Regarding the communicative relationship of Caesar to other actors, there is some scholarship, which this chapter builds upon. Several scholars have anticipated my understanding of communication as one of many forms of action in the \emph{Commentarii} \parencite{oldsjo2001,batstonedamon2006}.\footnote{\textcite[401--454]{oldsjo2001} surveys the verbs that the \emph{Bellum~Gallicum} uses to describe action, communicative or not. \textcite[162--164]{batstonedamon2006} briefly treat the way that the text articulates Caesar's actions in the \emph{Bellum~civile}, including verbs of communication and speeches. On the role of Caesar's letter to the senate at the opening of the \emph{Bellum~civile}, and attendant thematic issues of free speech in this text, see \textcite[43--49]{batstonedamon2006}. \textcite{nordling1991} covers indirect discourse in the \emph{Commentarii}.} This scholarship has emphasized Caesar as a communicator, yet has not prioritized those with whom he communicates. The most noteworthy work of this type, sometimes mentioning the matter only in passing, includes: \textcite{bertrand1997}, \textcite{ezov1996}, and \textcite[125--131]{goldsworthy1996} on communications to Caesar, such as that through \emph{exploratores}; \textcite[262--272]{henderson1996} on Caesar's attempted negotiations with Pompeians in the \emph{Bellum~civile};\footnote{Henderson holds that orders are an attempt to portray Caesar, both the represented and the historical author, as holding an ethical high ground in his purported attempts at reconciliation through dialog. See especially \textcite[262--276]{henderson1996} on writing. \textquote[{\cite[269--270]{henderson1996}}]{So Caesar writes his instructions, demands, letters. They invite readers to come to talk, if only about talks. His \emph{Commentarii} do not pretend to be other than documentary drafts, a condensed saturation of documentation}.} \textcite[25--26]{hall1998} on Caesar's historical use of a \textquote{literary bureau}; \textcite[330--338]{osgood2009} on the historical military advantage of Caesar's literacy in conquering Gaul;\footnote{The stories in this text preserve for us, argues Osgood, how Caesar documented the expedition's events, sent messages to subordinates, and reported to the Roman senate. These practices of the written word all contributed to Caesar's successful (historical) subjugation of Gaul: \textquote[{\cite[329]{osgood2009}}]{Simply put, writing helped to facilitate the Roman conquest of the Gallic peoples. It allowed Caesar to send messages within his own theater of operations, sometimes with distinctive advantages not offered by purely oral communication; it helped him to stay in touch and maintain intimate relations with Senators and other personal agents back in Rome, so that he could retain his command and obtain fresh resources; and it helped him, in the commentaries above all, to turn the story of his scattered campaigns into a grand and enduring narrative of how a vast territory now called \textquote{Gaul} was subjected to Roman rule}. Though intended as an interpretation of the \emph{Bellum~Gallicum}, this article, by necessity of limited evidence, is an exposition of letter writing in its narratives.} \textcite{james2000} on Caesar's speech in control of troop morale; and \textcite{nordling2005} on Caesar's speech before battle at Pharsalus. Of communication--based research, there is some that focuses on specifically speeches and other oral communication in the texts.\footnote{\textcite[56--57, 108--109, 139--142]{batstonedamon2006} analyzes speeches in the course of its study of the style and themes of the \emph{Bellum~civile}. \textcite[107--132, 184--188]{riggsby2006} includes several extended readings of speeches in the \emph{Bellum~Gallicum}.} There is also historical scholarship on military espionage, some of which uses the \emph{Commentarii} as evidence.\footnote{Most recently \textcite{ezov1996}. For the limited bibliography, see \textcite[p.~64, n.~1]{ezov1996}. \textcite{brizzi2010} looks at Caesar's biographies, by Suetonius and Plutarch, for clues to the evolution of his style as a military strategist.} Finally, \textcite{zadorojnyi2005} and \textcite{pelling2009} look at some communicative aspects, literacy and vision respectively, in representations of Caesar in later Roman biographies. In this dissertation's first chapter, I build from and contribute to this scholarship by joining observations on Caesar's communicative skills by understanding the communication in reference to others with whom he has or attempts to have a power relationship. It is not for its own sake that I am subordinating Caesar, but rather to demonstrate his interactional, specifically communicative, relationship to other figures in the text.

%chapter 2
The first chapter explains how Caesar communicates in the \emph{Bellum~Gallicum}. The next two look at deliberation which, as will be shown, is a closely related topic. Chapter~\ref{ch-delib} focuses upon the mind of Caesar, as represented by the narrator, and shows it to be the originating point for the ideas Caesar communicates. The text spends a great deal of space showcasing Caesar thinking to himself, deliberating the best choice of multiple options. The significance of the text's modeling of decision--making is that it offers a portrait of the army as a realization, accomplished through the perfect communication and obedience seen in Chapter~1, of Caesar's mind. Outside of its significance to the text itself, I show that Caesar crafts himself according to a literary tradition of the deliberating hero (section~\ref{comm}, on Homer and Polybius). As implicit narrator and actor, Caesar portrays himself as an excellent solver of problems, contributing to implicit notions about the excellence of the directives he gives to his soldiers, allies, and enemies. The \emph{Bellum~Gallicum} goes to great lengths to convince readers of the superior rationality of its general.

%jccorr:paragraph needs help ... see comments
%kjcorr: I said that I think this scholarship is right and useful
%rationalization
These conclusions about an openly deliberating Caesar help to situate the text of \emph{Bellum~Gallicum} in the larger intellectual movement of rationalization in the late Republic. The argument about rationalization runs that, by subjecting areas of traditional knowledge to abstract and universal principles, new independent bodies of knowledge could be dislodged from traditional to a rational-legal authority.\footnote{I here use Weber's terms for convenience's sake. The two most important overviews of this codification of knowledge are \textcite{moatti1997} and \textcite{wallacehadrill2008}. See also \textcite{rawson1985} on history and portraits of intellectuals at Rome, \textcite{nicolet1988} on geography, \textcite{wallacehadrill1997} on emerging culture during the 1st century revolutions, \textcite{kastely2002} on Cicero's \emph{De inventione}, and \textcite[65--76]{connolly2007} on Roman rhetorical treatises.} The movement sought to categorize and codify traditional knowledge such as history, linguistics, and law; and though not limited to Julius Caesar, scholarship agrees that Caesar was one of those central to this phenomenon. Scholars, I think convincingly, have made some arguments placing Caesar's difficulty reconstructed rationalizing texts and political actions within this movement.,\footnote{Concerning Caesar in particular, see \textcite[109--114]{rawson1985} and \textcite{fantham2009} on his intellectual life at Rome, \textcite{sinclair1993,sinclair1994} on the \emph{\textlatin{De analogia}}, and \textcite{feeney2007} on standardization of the calendar.} There are also some intriguing, yet limited, arguments aligning the style of the \emph{Bellum~Gallicum} to the \emph{\textlatin{De analogia}}, whose system intended to reduce the irregularities of human language to an all--encompassing \emph{ratio}.\footnote{Comparative studies speculating on the influence of Caesar's system on the \emph{Bellum~Gallicum} have focused almost entirely on vocabulary and grammar \parencite[e.g.,][17--19]{hall1998}. For bibliography on vocabulary and grammar in Caesar's thinking, see \textcite[p.~229, n.~1]{willi2010}. See also \textcite[94--95]{sinclair1994} on \emph{De~analogia} fragment 2 (Klotz), about avoidance of unusual words. Summarizing the regularity of language that Caesar advocated, \textcite[95]{sinclair1994} writes: \textquote{\emph{De analogia} came complete with an entire and evidently rather elaborate and detailed system by which one could correct many of the obtrusive and offensive irregularities of Latin by subordinating them to a logical system. It was only in those cases where the system (\emph{ratio}) could not provide a satisfactory resolution of a problem that Caesar allowed \emph{consuetudo} to serve as arbiter.}} As shown in Chapter~\ref{ch-delib}, in displaying for his readership his reasons and the good logic that unites them, Caesar crafts a portrait of himself as an exceptionally brilliant commander. Looking at heroes and generals from previous literature (such as Odysseus and Hannibal), Caesar clearly appears to be fashioning himself as someone whose shrewd intellect, not inherited strength or wealth, allows him to achieve his goals. The \emph{Bellum~Gallicum} doesn't just claim but proves that its \emph{gubernator} is a skilled thinker.

%chapter 3
In Chapter~\ref{ch-delib-subs}, I turn to the crew of Caesar's ship and look at how it functions with and without Caesar's guidance. Of particular interest here is what happens when, in rare instances, Caesar's commands to his men do not succeed as intended, nor can he communicate with his subordinates. Turning to such an episode in \emph{Bellum~Gallicum}~5, in which subordinate officers must deliberate for themselves what to do, I show how Caesar's officers ought to act and think in their master's absence. Caesar's skilled decision--making becomes all the more apparent in juxtaposition to this attempt by subordinates to make their own decisions. As in Chapter~\ref{ch-delib}, this chapter turns to an important literary precedent for deliberating subordinate officers. I argue that Xenophon's \emph{Anabasis} is a probable template for the \emph{Bellum~Gallicum}, and despite their similar construction of subordinates' deliberations, the two texts represent the decision--making process very differently (section~\ref{delib-xen}). The major difference is that, in the \emph{Anabasis}, group decision–making is the norm. However, in parallelism with the \emph{Bellum~Gallicum}, these group meetings are non–confrontational and top–down in nature (as shown in this dissertation's first chapter). Both texts show a strong interest in foregrounding decision--making -- the \emph{Bellum~Gallicum} through Caesar's vigorous debate within himself, the \emph{Anabasis} through non--divisive collaboration.

Caesar's Roman army is never intended to think for itself, though it must in situations when Caesar cannot communicate with his men, so as to give them orders. Scholarship has picked up on several variations of Caesar's incommuncativity. \textcite[93--95]{riggsby2006} looks at Caesar's inability to send and receive messages from Q. Cicero at \emph{Bellum~Gallicum} 5.40--48. \textcite[99--200]{kagan2006} reads the \emph{Bellum~Gallicum} as an historical record of Caesar's inability to perceive events or send and receive messages during battle.\footnote{\textquote[{\cite[117]{kagan2006}}]{Each of his narratives, therefore, must be examined separately to see how its character, his experience, his physical location, and his role shaped his perception and description of the events he described. This chapter examines four battles that illustrate the relationship between perception and narration. The goal remains to identify whether and how Caesar perceived and described the essential concatenations of a battle's events}. For summary, see \textcite[108--115]{kagan2006}. \textcite{glucklich2004} is another take on the indeterminacy of Caesar's vision, initially obscured to actor and reader.} \textcite[pp.~45--96, esp.~51--54]{rambaud1966} takes another approach to events at which Caesar was present, trying to determine which parts of the \emph{Commentarii} were composed from the reports of \emph{legati}.\footnote{For the overview of reporting in the \emph{Commentarii}, see \textcite[45--96]{rambaud1966}, and for a concise overview of reported events in the \emph{Bellum~Gallicum}, see \textcite[51--54]{rambaud1966}: \textquote[{\cite[51]{rambaud1966}}]{\textfrench{Dans le récit de César comme dans les lettres de Cicéron, les informations sont logiquement subordonnées à l'activité du général; elles en sont présentées comme la cause. Dans un livre, cette subordination subsiste quoique sous d'autres formes. L'activité du général, activité principale, fournit le canevas du récit, et les rapports de légats y sont insérés et rattachés comme des prolongements ou des compléments}}.} This chapter looks at what appear to be uncomfortable situations, when a leader cannot control the minds and actions of subordinates. Here, Caesar argues, officers must follow a set of protocols in order to gain control of their own minds, in what amounts to a second--rate \emph{mimesis} of Caesar's deliberative process.

This dissertation is thoroughly devoted to the subject of Caesar's representation of himself as an idealized figure, but argues that Caesar explains who he is through his subordinates and his army's procedures, as much as through his own activities in the text. In the following three chapters, I intend to prove that understandings of Caesar's construction of himself in the \emph{Bellum~Gallicum} must account for his interactions with other characters of the text, especially his subordinate officers. I also aim to show some of the careful ways that the laborers of Caesar's ship (so to speak) are deployed, and what this means about the text's construction of its main figure. In the dissertation's Conclusion, I turn to the traditional and unavoidable dilemma of the \emph{Bellum~Gallicum} as propaganda. With several elements of the text better accounted for, and with an intentionally anti--teleological attitude toward Caesar's eventual autocracy, I argue that my conclusions help to situate this \emph{commentarius} among Romans' historical concerns about the abuse of power by would--be autocrats. If there is a propagandistic intent to the \emph{Bellum~Gallicum}, it is to convince its readership that Caesar is an eminently conventional, responsible, skillful, and successful delegate upon whom the \emph{imperium Romanum} had been bestowed.

\chapter{Communicative presence}%1111
\label{ch-pres}
This chapter approaches communication as a mechanism by which Caesar, in the narratives of the \emph{Bellum~Gallicum}, leads his army. While some scholarly approaches to the topic of communication and leadership have prioritized battle narratives \parencite{lendon1999,kagan2006}, I emphasize inter--battle operations, specifically those in the Ariovistus narrative. Indicative of the importance of diplomatic and operational activities, in the case of the Ariovistus narrative, the lead--up to the battle is seven times longer than the battle itself.\footnote{As counted in the \textcite{hering2008} edition. The lead--up is between \emph{Bellum~Gallicum} 1.31--48 and the battle itself \emph{Bellum~Gallicum} 1.49--53.} This chapter introduces a notion of communicative presence, whereby Caesar is at once a physically located (and thus physically limited) actor, though one which maintains a omnipresent mastery over others -- his army, allies, and enemies -- by means of communicative practices. Thus, though not always in direct perception or participation of events, he controls the activities of his men, especially by means of influencing their psychological states, with speeches of encouragement. I observe the communicative practices of Caesar in the Ariovistus episode of the first book of the \emph{Bellum~Gallicum}. Communication, by which his thinking, commands, and words are available to all, overcomes the limitations of Caesar's physical existence. Physical absence is mediated through written letters and human messengers. This is not to say, however, that physical presence is not important in the text. Whenever possible, Caesar attempts to exchange linguistic information face to face, for it is in these situations that he may best adopt a set of communicative practices which intervene in the psychological states of his interlocutors. In order to gain an appreciation of Caesar's communicative presence as it is sustained over time, this chapter studies the entirety of the narrative of the conflict between Caesar and Ariovistus (\emph{BG}~1.31--53). Whether operating in harmony or conflict with others, communicative presence allows for Caesar to be many places at the same time.

By documenting these phenomena -- of physical presence, communication, and  psychological control -- several insights emerge. This chapter argues that Caesar's \emph{Bellum~Gallicum} presents a particular ethics of leadership previously unrecognized. When dealing with his own men, Caesar's communicative strategies, and namely his control over their communications with one another, are central in creating an (almost) perfectly harmonious army which (almost always) unflinchingly follows his orders. In dealing with conflict, certain aspects of his leadership qualities become apparent. It is recognized that Caesar, in the \emph{Bellum~Gallicum}, represents himself wanting just war; and, throughout the \emph{Bellum~civile}, negotiated peace.\footnote{For Caesar and just war theory, see \textcite{siebenborn1990} and \textcite[157--189]{riggsby2006}. On peaceful reconciliation in the \emph{Bellum~civile}, see, for example, \textcite[57--60]{batstonedamon2006} on \emph{Bellum~civile} 1.8--11, where Caesar makes his initial attempts at reconciliation with Pompey.} This chapter shows how, in dealing with outsiders or those he may be in conflict with, Caesar (as he represents it) resolves disputes by means of his personal communication, physical presence, and psychological interventions. 

In demonstrating this chapter's thesis, a related sub--theme also emerges. There are two opposing trends in the text's explanation of others' obedience to Caesar, whether it is willing obedience or their fear of him. While one may imagine an explanation that accounts for both of these forces, there appear instead two distinct voices which articulate these explanations as distinct from one another. Caesar, as a speaking agent within narrative, expresses others' obedience as elective; but the narrator's scrutiny of Caesar's leadership, and characters' obedience, is a more coercive one. There is, in other words, a tension between whether Caesar rules with the coercive power of physical force or by an elective loyalty in others. These two, apparently competing, strands of thought are recognized and reflected upon as they emerge in the latter half of \emph{Bellum~Gallicum}~1. A partial explanation of the origin of this tension may lie in the text's distinction between the narrative voice and actor Caesar. As has been observed, most recently in \textcite[150--155]{riggsby2006} and \textcite[148--156]{batstonedamon2006}, the voice of the narrator does not engage in ethical evaluations of characters and events, thus producing an \textquote[{\cite[151]{riggsby2006}}]{objectivity effect}. In order to avoid confusion of these two voices, I use \textquote[conventionally enough]{narrator} for the voice that records the activity of the narrative, and for the represented Caesar within the narrative, I adopt the term \textquote[][.]{actor Caesar}\footnote{The narratological vocabulary of \textcite{bal1997} informs my choice of these two terms. On the several first--person intrusions by the narrator in the \emph{Bellum~Gallicum}, see \textcite[pp.~150--151 and 242--243, nn.~90--94]{riggsby2006}.} The narrator displays a more cynical attitude about the role that power, fear, and self--interest play in human motivations. Caesar the speaking actor, however, consistently reflects a less cynical attitude, and holds that human relationships bond people together and motivate them into mutually beneficial partnerships.

In support of my thesis, I make another observation: the text arranges the Ariovistus narrative according to Caesar's communicative medium. That the text structures historical events as a series of the actor Caesar's types of information--exchange is an indication of the importance that communication plays in the \emph{Bellum~Gallicum}. In an elucidation of this, the organization of the narrative is reflected in this chapter's organization. It progresses from a \emph{concilium} (\textquote{public gathering,} \emph{BG}~1.31--33; section~\ref{bg-concilium}) to \emph{legati} (\textquote{messengers,} \emph{BG}~1.34--38; section~\ref{bg-legati}), a \emph{consilium} (\textquote{advisory body,} \emph{BG}~1.39--41; section~\ref{bg-consilium}), and a \emph{conloquium} (\textquote{meeting,} \emph{BG}~1.42--47, section~\ref{bg-conloquium}). Finally, section~\ref{comm-patt} gives an overview of the commonality, across the entirety of the \emph{Bellum~Gallicum}, of the narrative pattern and communicative presence seen in the Ariovistus episode.

Communicative presence is a mechanism that the \emph{Bellum~Gallicum} uses to explain Caesar's success in leadership, to highlight a willingness to settle disputes without violence, and to justify the wars he fights. It also gives rise to the derivative topic of obedience, whether Caesar is obeyed due to others' loyalty or their fear. The communicative acts identified in this chapter find some commonality with a particular adoption of Jürgen Habermas's theories of communication, by \textcite[73--80]{ando2000}. Ando's explanation of the stability of the Roman empire -- a result of interactive communications between Roman emperors and provincials -- finds some thematic resonances with Julius Caesar's portrait of himself as ruling, at least in part, by means of free communication between rational agents. The double explanation of obedience allows the text to offer a compelling portrait of Caesar as an actor capable of dynamic interpersonal relationships, while also preserving a notion of the narrator as a shrewd interpreter of human motivations.

Considering Julius Caesar's audience at Rome, the significance of his interactions with Ariovistus is all the more significant, since Ariovistus was in communication with Caesar's political enemies. The German king (or so the text has him claim) tells Caesar that he is in contact with his rivals at Rome (\textquote{\textlatin{nobilibus principibusque}}) and would secure their friendship (\textquote{\textlatin{gratiam atque amicitiam}}) by killing him (\emph{BG}~1.44.12). Ariovistus has learned this, he says, through these aristocrats' messages to him (\textquote[][,]{\textlatin{id se ab ipsis per eorum nuntios compertum habere}} \textquote{he had discovered this from those very men through messengers}). The message that the text sends about the nature of Caesar's communication during this episode is even more significant knowing that his reading public would be aware of the efforts of some to kill him. Even if one is skeptical whether Ariovistus in fact made such a claim or that certain Romans sent such letters, the fact that as an author Caesar stages the German and his Roman enemies as such grave threats only enhances the contrast that I draw between the two leaders.\footnote{\textcite[228--229]{goldsworthy2006} reserves some doubts, yet \textcite[pp.~110--111, n.~1]{gelzer1968} does not.} In its construction of speech participants and their communicative strategies, the \emph{Bellum~Gallicum} is an argument of how Caesar reasons and compromises with even the bitterest of his foes.

\section{\emph{Concilium}}
\label{bg-concilium}

This and the following three sections explore one episode that is exemplary of Caesar's communicative presence during the Ariovistus episode. Section~\ref{comm-patt} explains the frequency of this basic narrative pattern throughout the \emph{Bellum~Gallicum} and shows that it is not unique in the text.

The first segment of the Ariovistus episode (\emph{BG}~1.31--33) displays Caesar's communicative presence. In this example, he is in physical proximity to his interlocutors. Here, one of Lendon's observations, that Caesar intervenes in the psychological states of his soldiers, is shown to be relevant when Caesar deals with outsiders to his army, here Gallic allies.\footnote{For the heart of this argument, see \textcite[pp.~295--304 on \emph{animus} and pp.~306--316 on \emph{virtus}]{lendon1999}.} The explanation for the Gauls' following of Caesar is explained in two manners. There is a tension between the text's explanation of Caesar's success as a result of others' fear of his army's physical power, and its explanation that it is their elective decision to defer to him.

At the inception of the Ariovistus episode, the text constructs historical events according to Caesar's communicative mode, here a \emph{concilium}.\footnote{There are actually two \emph{concilia}, the first the petitioned meeting at 1.30.4 (also called a \emph{concilium} at 1.30.5 and 1.31.1), of which nothing is reported. The \emph{concilium} explained at length by the text (and of concern here) is called a \emph{concilium} at 1.33.2.} After the initial \emph{concilium}, Gallic leaders approach Caesar for a secret meeting. The assembled Gauls ask for Caesar's intervention in pushing Ariovistus and the Germans out of Gallic territories. After listening to them, Caesar agrees to see to their problem, and dismisses the \emph{concilium} (\emph{BG}~1.33.1). Several sentences explain Caesar's thinking about why he should intervene in this affair (\emph{BG}~1.33.2--5). In light of his considerations, he concedes to the Gauls' request, and immediately begins his next communicative activity, corresponding with Ariovistus via messengers: \textquote[\textquote{wherefore it was pleasing to him that he send \emph{legati} to Ariovistus} \emph{BG}~1.34.1]{\textlatin{quamobrem placuit ei, ut ad Ariovistum legatos mitteret}}.\footnote{All \emph{Bellum~Gallicum} text from \textcite{hering2008}.} Caesar's communicative presence operates within both the \emph{concilium} itself, as well as this narratival framework that prioritizes his communicative contexts.

In the assembly, in the physical presence of his interlocutors, there are several important aspects of Caesar's communicative practices. The actor Caesar shows himself willing to intervene in the petitioners' mental states for the purpose of soothing their fear and anxiety. In describing the meeting, the narrator also highlights some non--linguistic interactions which hint at the value, to Caesar, of face to face meetings. As a speaking subject, Caesar listens to the story behind the assembled Gauls' problems. The tale of Ariovistus's cruelty is told at length between \emph{Bellum~Gallicum} 1.31.3--16. Caesar's reply is only one sentence (1.33.1). The \emph{concilium} begins with physical interaction between Caesar and the others. \textquote[][,]{All crying} the Gauls \textquote[\textquote{\textlatin{sese omnes flentes Caesari ad pedes proiecerunt}} \emph{BG}~1.31.2]{threw themselves at Caesar's feet}. This type of extra--linguistic communication is not possible outside of communication in physical presence. The narrator's inclusion of this action offers insight into the importance of the psychological state of the speakers.\footnote{Such bodily supplication occurs earlier, too, at \emph{Bellum~Gallicum} 1.27.2. See also 2.13.2--3, where the people of besieged Bratuspantium hold out their hands begging for peace. At 1.20.5, in response to Caesar reaches out his right hand to Dumnorix.} Along with the narrator, the actor Caesar is also sensitive to the psychology of his interlocutors. He notices that the Sequani have not said anything and appear disturbed. \blockquote[\emph{BG}~1.32.2--3]{\textlatin{Animadvertit Caesar unos ex omnibus Sequanos nihil earum rerum facere quas ceteri facerent, sed tristes capite demisso terram intueri. eius rei quae causa esset miratus ex ipsis quaesiit. nihil Sequani respondere, sed in eadem tristitia taciti permanere.}} \blockquote[\emph{BG}~1.32.2--3]{\textlatin{Caesar noticed that the Sequani alone of them all did nothing of these things which the others did, but sad, looked at the ground with their heads hung low. Wondering, he asked them about the cause of this. The Sequani did not respond at all, but remained silent in the same sadness.}} From their external appearance, Caesar intuits their inner turmoil. The narrator makes it clear that Caesar is an active subject, who \textquote{noticed} and was \textquote[][.]{wondering} He observes their physical actions, such as standing apart, not doing like the others, holding their heads down, and looking at the ground. From this, their psychological characterization as sad is physically grounded. In response, he seeks dialog with the Sequani, who do not reciprocate. Caesar's suspicions are confirmed when Diviciacus speaks for the Sequani and says that they are in fact have a \textquote{more miserable and graver fortune} than the rest, and \textquote{were afraid} of the \textquote[][.]{cruelty of Ariovistus} With it established that the Sequani are miserable, and that Caesar perceives it, the narrator next explains how Caesar allays their fears.

Caesar convenes such assemblies of Gauls somewhat regularly.\footnote{Other assemblies that Caesar calls include \emph{Bellum~Gallicum} 4.6.5, 5.2.4 (Treveri don't come to Caesar's \emph{concilia}), 5.24.1 (a \emph{concilium}), 5.54.1, and 6.3.4--6 (a \emph{concilium}).} In several of these, he intervenes in the psychological state of his Gallic audience. At \emph{Bellum~Gallicum} 4.6.5, Caesar calls together Gallic leaders and soothes and encourages their minds (\textquote{\textlatin{eorumque animis permulsis et confirmatis}}). In another meeting, Caesar acts similarly to the convened Gauls: \textquote[\textquote{by terrifying some \ldots{} and exhorting others he held a large part of Gaul in loyalty} \emph{Bellum~Gallicum} 5.54.1]{\textlatin{alias territando \ldots{} alias cohortando magnam partem Galliae in officio tenuit}}. In two of these assemblies, in addition to the one at \emph{Bellum~Gallicum} 1.31--33, the narrator explains a different intention within Caesar's mind than what he speaks to them. At \emph{Bellum~Gallicum} 4.6.5, Caesar deceives his listeners by concealing his thoughts (\textquote{\textlatin{Caesar ea quae cognoverat dissimulanda sibi existimavit,}} \textquote{Caesar thought that what he had learned ought to be concealed}); and at 6.3.4, where he, thinking that several tribes had defected (\textquote{\textlatin{initium belli ac defectionis hoc esse arbitratus}}), moved the location of the \emph{concilium}. Assemblies like this \emph{concilium} are present elsewhere in the narratives of the \emph{Bellum~Gallicum}, and in those Caesar intervenes in the psychological states of his listeners and as well as dissimulates.

In explaining Caesar's ability to soothe these Gauls, a similarity and contrast is drawn between Caesar and Ariovistus: their power over others corresponds to their presence, yet Caesar's is to reassure the Gauls and Ariovistus's to terrify. The German is feared as if present, even when absent: \textquote[\textquote{they feared the cruelty of an absent Ariovistus as if he were present face to face} \emph{Bellum~Gallicum} 1.32.5]{\textlatin{absentisque Ariovisti crudelitatem velut si coram adesset horrerent}}.\footnote{This thematic of a population terrified by the absent leader appears in the \emph{Bellum~civile}, too, as when the Spanish troops remember and fear \textquote[\textquote{the name and rule of <Pompey> absent} \emph{BG}~1.61.3][]{\textlatin{nomen atque imperium absentis <Pompei>}}. For a similar instance of fear, this time the senate's for Pompey, see also \emph{Bellum~civile} 1.33.} This is in contrast to Caesar's communicative presence, by which his words are made available to actors of the \emph{Bellum~Gallicum}. Face to face, Caesar soothes the Gauls' concerns. \blockquote[\emph{BG}~1.33.1]{\textlatin{His rebus cognitis Caesar Gallorum animos verbis confirmavit pollicitusque est sibi eam rem curae futuram.}} \blockquote[\emph{BG}~1.33.1]{When these matters were known, Caesar reassured the Gauls' minds with his words and promised that the matter would be a subject of his care.} Whereas with Ariovistus fear increases with his physical presence, with Caesar there appears an inverse relationship of the Gauls' fear to his proximity. Importantly for my thesis about communicative presence, it is not Caesar's physical being that brings reassurance to them but his words (\textquote{\textlatin{verbis}}). Several key aspects of Caesar's communicative presence, namely that of a face to face communicative presence, are apparent here. Caesar, here as a leader of foreign allies, patiently listens, physically interacts with them, and is perceptive of and intervenes in their psychological states.

There is remarkable parallelism between these observations of Caesar's intervention into the psychological states of these Gauls and that achieved with his own soldiers. \textcite{lendon1999}\label{lendon-sum} identifies a soldier--author tradition, preceding the \emph{Commentarii}, of Thucydides, Xenophon, and Polybius. What distinguishes these four is that they \textquote[{\cite[276]{lendon1999}}]{had all witnessed battles, and all had a professional interest in how they worked}. Lendon identifies three aspects of battle as found in the \emph{Commentarii} -- tactics, \emph{animus} (soldiers' morale), and \emph{virtus} (soldiers' courage) -- that are portrayed as influential to the outcome of conflicts. Whereas Thucydides and Polybius placed primary importance in abstract laws of battle tactics, Caesar, who shares with Xenophon an appreciation for an army's psychology, makes room for the morale and bravery of soldiers. The purpose of the general's many exhortations is to strengthen the hearts and minds of Roman soldiers.\footnote{Lendon summarizes, drawing from the victory over the Nervii in \emph{Bellum~Gallicum}~2: \textquote[{\cite[320]{lendon1999}}]{The story that Caesar is telling is not just that of military movements, blunders, flank attacks, tactics. The story Caesar \emph{is} telling is signaled early in the battle by his exhortation to his troops \textquote{that they should maintain a memory of their ancient \emph{virtus}, that they should not be perturbed in \emph{animus}, and that they should bravely sustain the \emph{impetus} of the enemies (\emph{\textlatin{impetum fortiter sustinerent}})} (\emph{G} 2.21; cf. 3.19). To Caesar tactics (\emph{impetus}), \emph{animus}, and \emph{virtus} share the battle equally, and by dividing the battle description into tactical, \emph{animus}, and \emph{virtus} segments, the very structure of the Roman general's account elegantly reflects that fact. To Caesar the best victory is victory in tactics, \emph{animus}, and \emph{virtus} all at once}.} This chapter's second thesis extends Lendon's observation from his relationship with the Roman army to the psychological states of his allies. Whereas Lendon sees this intervention in formal speeches, such intervention does not need long speeches, and Caesar's audience need not be soldiers. The subject matter at 1.33.1, however, is identical to one of the three main concerns -- \emph{animus} -- of Caesar with his own men in battle. In fact, the \emph{Bellum~Gallicum} often uses pairs of \emph{animus} and \emph{confirmare} for Caesar encouraging his men, as \textquote[\textquote{thinking enough had been done for strengthening the minds of the soldiers} \emph{BG}~7.53.3]{\textlatin{satis \ldots{} militumque animos confirmandos factum existimans}}.\footnote{See \textcite[299--301]{lendon1999} on how panicked soldiers are \textquote{firmed up} with \textquote{firmare} or \textquote[][.]{confirmare}} Lendon writes that, for Caesar, military victories are explainable partly by strengthening the minds of his men, and the same is true of Caesar's diplomatic victory in the \emph{consilium} under consideration here. The mental state of Gallic allies in the \emph{Bellum~Gallicum} is an object of concern for Caesar, similar to his concern for that of Roman soldiers. 

Caesar is successful in his psychological intervention into the minds of the Gauls. Two competing strands of explanation for Caesar's success -- authority and coercive power -- appear, albeit in a limited form, at this initial episode of the Ariovistus narrative. The Gauls, in their speech, evidently believe that Caesar can rectify their situation by one of several means. They say: \blockquote[\emph{BG}~1.31.16]{\textlatin{Caesarem vel auctoritate sua atque exercitus vel recenti victoria vel nomine p. R. deterrere posse.}} \blockquote[\emph{BG}~1.31.16]{Caesar would be able to deter <Ariovistus> either by his authority and that of his army, or by his recent victory, or by the name of the Roman people.} According to Diviciacus, the possible persuasive forces will be either \emph{auctoritas}, news of military success over the Helvetii, or the name of the Roman people. These options boil down to two very different sorts of influence. Persuasion by \emph{auctoritas} may invoke an idea of prestige or reputation, while the \textquote[\textquote{\textlatin{recenti victoria}}]{recent victory} leaves little room in the imagination except for physical, brute might. The third option, \textquote[\textquote{\textlatin{nomine p.~R.}}]{the name of the Roman people}, may perhaps fall either way. If the Gauls mean that Ariovistus will have respect for the Roman people, then the sense is closer to \emph{auctoritas}; if they mean that he will fear physical harm by the hand of the Roman people (and their army), then it may be construed as closer to the \textquote{recent victory} line.  Either way, the first two options are not diametrically opposed, as they contribute to the force of the name of Rome. Whatever the exact qualities of these three, all I hope to show here is that the reasons for others' obedience to Caesar are multiple. As the episode develops, a dual explanation emerges why actors in the narrative obey Caesar.

Caesar's conflicting thoughts and words, in response to the request, underscore the contradictory persuasive forces suggested by Diviciacus. As a speaker, the actor Caesar says that he will take the \emph{auctoritas} route, but when the narrator explains the actor's inner thought process, the calculations are more cynical, or power--based. In his verbal response, Caesar says that Ariovistus will cooperate due to Caesar's \emph{auctoritas} and kindness. % \begin{midsloppypar}
 \blockquote[\emph{BG} 1.33.1--2]{\textlatin{(1) Magnam se habere spem et beneficio suo et auctoritate adductum Ariovistum finem iniuriis facturum. (2) hac oratione habita concilium dimisit.}} 
% \end{midsloppypar}
 \blockquote[\emph{BG}~1.33.1--2]{He had great hope that Ariovistus, led both by Caesar's kindness and \emph{auctoritas}, would make an end to his injuries. Having given this speech, he dismissed the \emph{concilium}.}  The \textquote{leading} of \textquote{\textlatin{adductum}} underscores the cooperative relationship signified by \textquote{\textlatin{beneficio}} and \textquote[][.]{\textlatin{auctoritate}} Caesar's reported words to his interlocutors rest upon the Romans' and Gauls' mutual goodwill.

The elaboration of Caesar's calculations (\emph{BG}~1.33.2--5), voiced by the narrator, is much shrewder, signifying a different attitude to elective or compelled obedience. Special emphasis is given by the text that the calculations are those of Caesar. The narrator marks repeatedly that these are Caesar's thoughts, occurring at the time of the \emph{concilium}. In the four sentences, there are ten words that denote thinking or consideration: \textquote[][,]{\textlatin{eam rem cogitandam et suscipiendam putaret}} \textquote[][,]{\textlatin{videbat}} \textquote[][,]{\textlatin{intellegebat}} \textquote[][,]{\textlatin{arbitrabatur}} \textquote[][,]{\textlatin{videbat}} \textquote[][,]{\textlatin{existimabat}} \textquote[][,]{\textlatin{putabat}} and \textquote[][.]{\textlatin{videretur}}\footnote{\textquote[\emph{BG}~1.33.2--5]{\textlatin{(2)\label{bg-1-33} et secundum ea multae res eum hortabantur, quare sibi eam \textbf{rem cogitandam et suscipiendam putaret}, inprimis, quod Haeduos fratres consanguineosque saepe numero a senatu appellatos in servitute atque in dicione \textbf{videbat} Germanorum teneri eorumque obsides esse apud Ariovistum ac Sequanos \textbf{intellegebat}; quod in tanto imperio p.~R.~turpissimum sibi et rei p. esse \textbf{arbitrabatur}. (3) paulatim autem Germanos consuescere Rhenum transire et in Galliam magnam eorum multitudinem venire p.~R.~periculosum \textbf{videbat}, (4) neque sibi homines feros ac barbaros [ob]temperaturos \textbf{existimabat}, quin cum omnem Galliam occupavissent, ut ante Cimbri Teutonique fecissent, in provinciam exirent atque inde in Italiam contenderent, praesertim cum Sequanos a provincia nostra Rhodanus divideret; quibus rebus quam maturrime occurrendum \textbf{putabat}. (5) ipse autem Ariovistus tantos sibi spiritus, tantam arrogantiam sumpserat, ut ferendus non \textbf{videretur}.}}} Of the \textquote[\textquote{\textlatin{multae res}}]{many reasons} that induce him to intervene, all pertain to the calculations of relative power between Germans and Gauls. First, because of the great power of the Roman people (\textquote{\textlatin{in tanto imperio populi Romani}}), Caesar takes the Germans' incursion as a slight (\emph{BG}~1.33.2). The rest of his reasoning lies with the gradual threat to Romans that the Germans would pose if left unchecked. He considers the Germans \textquote[\textquote{\textlatin{homines feros ac barbaros}}]{savage and barbarian men} who must be stopped, or else they would not stop migrating until they had reached Transalpine Gaul and Italy itself (\emph{BG}~1.33.4). In terms that have nothing to do with \emph{auctoritas}, and everything to do with military power dynamics, Caesar sees the Germans as a \textquote[\textquote{\textlatin{periculosum}} \emph{BG}~1.33.3]{danger}. Considering that Ariovistus is described as having \textquote[\textquote{\textlatin{tantos sibi spiritus, tantam arrogantiam}}]{such great pride, such arrogance} he cannot be allowed to continue. In his spoken words in the \emph{concilium}, Caesar's foreign policy de--emphasizes military force and promotes mutual relationships of respect. But secretly, we discover the real, inner reasons for Caesar doing what he does. In his mind, he fears a growing and dangerous German incursion that could harm the Roman people. This double explanation, reflected also in the Gauls' request, appears in various guises throughout the Ariovistus episode, and explored in each section below.

The text offers competing models that explain how Caesar leads, a double articulation of his benevolence (revealed in the actor's words) and shrewdness (revealed by the narrator). If there are two perspectives on Caesar's leadership, then one may choose to evaluate which of the two is, say, the author's own. This chapter sidesteps any inquiry into any ultimate truth of the text, be it Caesar's historical actions or Julius Caesar's intentions in writing, and prefers to reflect on the choice that the text presents its readers. Concerning a different idea (subjectivity versus objectivity), \textcite[150--155]{riggsby2006} also disentangles the narrator from the actor Caesar in the \emph{Bellum~Gallicum}: he concludes that the \textquote[][,]{Caesar (the author) wants to have it both ways} in order to be at the same time a \textquote{moralizing figure} and an \textquote[][.]{\textquote{objective} technocrat}\footnote{\textcite[152]{riggsby2006}.} Beyond forging an epistemic objectivity, the \emph{Bellum~Gallicum} displays two different idealized (in transparency or cynicism of rule) voices, neither of which may completely subordinated to the other. As propaganda, the \emph{Bellum~Gallicum} displays two attractive and, to some degree, irreconcilable messages for readers to choose from.

This dynamic is developed in the following sections. In this \emph{consilium}, the \emph{Bellum~Gallicum} expresses Caesar's fundamental two--part relationship to almost all others, of how and why they follow his leadership. In the next section, we see a rare breakdown of another's obedience to his \emph{auctoritas}.

\section{\emph{Legati}}
\label{bg-legati}
The next segment of text (\emph{BG}~1.34--38) consists of a series of exchanges, between Caesar and Ariovistus, by means of messengers. In these exchanges through \emph{legati}, which by necessity mean that they are not proximate, Caesar seeks a face to face meeting with the German leader. In Caesar's documented language to Ariovistus, he pursues a mutually beneficial, reciprocal relationship. The narrator, however, as in the \emph{concilium} above, speaks to a more cynical, power--minded reality to Caesar's motivations.

As seen with the \emph{concilium}, the narrative organizes all action around Caesar's communicative context. There are two pairs of messages sent by \emph{legati}. These messengers, whom Caesar and Ariovistus send much like letters, constitute the communicative medium of this next portion of the narrative: \textquote[\textquote{he sends \emph{legati} to Ariovistus} \emph{BG}~1.34.1]{\textlatin{ad Ariovistum legatos mitteret}}, \textquote[\textquote{to this legation Ariovistus responds} \emph{BG}~1.34.2]{\textlatin{ei legationi Ariovistus respondit}}, and \textquote[\textquote{to him Caesar sends \emph{legati}} \emph{BG}~1.35.2]{\textlatin{ad eum Caesar legatos \ldots{} mittit}}. During this space of text, Caesar communicates with other Gauls, though also by messenger: \textquote[\textquote{\emph{legati} came from the Haedui and Treveri} \emph{BG}~1.37.2]{\textlatin{legati ab Haeduis et a Treveris veniebant}}. While sending these messages, no other action occurs, except some movements by Caesar toward the Germans. This section ends with the beginning of the insurrection at Vesontio and ends with a shift of speaking agents (Roman soldiers, Gauls, and traders \emph{BG}~1.39.2). Here, not only does Caesar communicate solely through \emph{legati}, but the represented events are structured by the narrative around his communicative context.

In this series of exchanges, several new aspects of Caesar's communicative priorities are expressed. Caesar seeks a meeting in person, suggests a mutually convenient location for it, and shows signs that the meeting will be a two--way exchange of information. Unable to attain such a meeting, Caesar's wish to converse transforms into a set of demands. In the first exchange, Caesar attempts dialog in the form of a \emph{conloquium}. He sends messengers and asks \textquote[\textquote{that he choose some place equidistant to them both for a \emph{conloquium}} \emph{BG}~1.34.1]{\textlatin{uti aliquem locum medium utriusque conloquio deligeret}}. That the location is Ariovistus's choice, and in the \textquote[\textquote{middle} or \textquote{equidistant place}]{\textlatin{locum medium}} denotes some willingness to facilitate the meeting. When Ariovistus refuses the \emph{conloquium}, Caesar replies that he must then make demands. \blockquote[\emph{BG}~1.35.2]{\textlatin{In conloquium venire invitatus gravaretur neque de communi re dicendum sibi et cognoscendum putaret, haec esse, quae ab eo postularet.}} \blockquote[\emph{BG}~1.35.2]{Though invited he was reluctant to come to a \emph{\textlatin{conloquium}}, nor thought their common matter ought to be discussed and considered by him [Ariovistus], he [Caesar] demanded these things from him.} Caesar presumes that each would want to speak with (\textquote{\textlatin{dicendum}}) and listen to (\textquote{\textlatin{cognoscendum}}) one another about something shared by them both (\textquote{\textlatin{de communi re}}).\footnote{I follow (along with \cite{klotz1926} and \cite{seel1961}) MSS in reading \textquote[][,]{\textlatin{dicendum}} not the emendation of \textquote{\textlatin{di<s>cendum}} in \textcite{hering2008}.} Caesar displays here both an eagerness to meet in talk in person with Ariovistus, and certainly some interest in exchanging information freely with him.

Ariovistus's refusal to meet is instructive in that it offers a contrast to Caesar's relative openness. In the first response, he says that only the petitioner, who needs something from the other, should approach: \textquote[\textquote{if he needed anything from Caesar himself, then he would have come to him} \emph{BG}~1.34.2]{\textlatin{si quid ipsi a Caesare opus esset, sese ad eum venturum fuisse}}. Motion language is invoked by Ariovistus (\textquote{\textlatin{venturum}}), for him travel ought to occur from one extreme end to another (\textquote{\textlatin{sese ad eum}}). Concerning need (\textquote{\textlatin{opus esset}}), the implicit meaning is that Caesar is in a weaker position than Ariovistus, in that Caesar is apparently dependent upon him. He continues this characterization in terms of want or desire for something: \textquote[\textquote{if he wants anything, it is right that he come to him [Ariovistus]} \emph{BG}~1.34.2]{\textlatin{si quid ille se velit, illum ad se venire oportere}}. The seeker of communication is construed as the weaker. This construing of communication is quite similar to that by the antagonist of the \emph{Bellum~civile}, Pompey, who is at one point said to attribute the request for communication as a sign of fear: \textquote[\textquote{and fear is indicated by those who send <\emph{legati}>} \emph{BC}~1.32.8]{\textlatin{timoremque eorum qui mitterent significari}}. Ariovistus's second refusal to meet reiterates what he said earlier (\emph{BG}~1.34.4) that he held the disputed territory by \textquote[\textquote{law of war}]{\textlatin{ius \ldots{} belli}}, that the conqueror may rule the defeated as they wish (1.36.1). Physical proximity, Ariovistus concludes, is for fighting: \textquote[\textquote{when he wants, he may approach/engage; he would learn why the Germans were undefeated} \emph{BG}~1.36.7]{\textlatin{cum vellet, congrederetur: intellecturum quid invicti Germani}}. For Ariovistus, \textquote{walking together} (\textquote{con--gredi}) means fighting.

Caesar's reported messages to Ariovistus outwardly express a desire to have a fair and equal exchange of information and, perhaps, a mutually agreeable resolution. The narrator, however, expresses a very different Caesar. Caesar's interior thoughts at \emph{Bellum~Gallicum} 1.33.2--5, looked at above, are in this section used as the catalyst for action: \textquote[\textquote{on account of which it was pleasing to him that \ldots{}} \emph{BG}~1.34.1]{\textlatin{quamobrem placuit ei ut \ldots{}}}. For the remainder of the \emph{legati} exchanges, Caesar's thinking does not intrude, but instead the narrator instead articulates Caesar's words in indirect discourse. Caesar's tone in his speech is very different. \blockquote[\emph{BG}~1.34.1]{\textlatin{Quamobrem placuit ei ut ad Ariovistum legatos mitteret, qui ab eo postularent, uti aliquem locum medium utriusque conloquio deligeret: velle sese de re publica et summis utriusque rebus cum eo agere.}} \blockquote[\emph{BG}~1.34.1]{On account of which it was pleasing to him that he send to Ariovistus \emph{legati}, who would demand that he choose some place equidistant to them both for a \emph{conloquium}: He wished to do some business with him concerning the Republic and matter of the highest importance to them both.} Caesar's request for the meeting is couched in language which can be read two ways, as either asking or demanding. The full \emph{OLD} definition gets at the nuance of the word as \textquote[\emph{OLD}, s.~v.~\textquote{\textlatin{postulo}}]{to ask for (generally as something to which one is entitled)}.\footnote{\textcite{meusel1887} divides \textquote{\textlatin{postulare}} between uses when grammatical subjects are human or objects. When it is humans doing the asking, he glosses \textquote{\textlatin{flagitare}} and \textquote[def.~A.]{\textlatin{poscere}}.} The term, then, though making a strong claim for a response, indicates some degree of reciprocity.\footnote{This reciprocity is reflected in Caesar's earlier request also (\emph{BG}~1.35.2).} Caesar asks (without a power relation necessarily implied), but strongly expects a reply (indicating some dialogic relationship). This dialogic valence of \textquote{\textlatin{postulare}} (though perhaps slight) mirrors the double explanation of the obedience of Caesar's soldiers. While the meeting is somehow demanded, the content of the meeting is expressed as a wish (\textquote{\textlatin{velle}}) to deal with common matters (\textquote{\textlatin{de \ldots{} summis utriusque rebus}}). As noted above, the \textquote{middle ground} pursued by Caesar shows some interest in reciprocity too. Caesar's words to Ariovistus demonstrate some degree of openness and interest in an exchange of information, which differs from his mind's calculations of power and threat at 1.33.

Ariovistus is not receptive to Caesar's suggestions, making for a strong contrast between their professed intentions. In his first response, Ariovistus denies any matter for discussion common to them both, since he himself laid claim to northeastern Gaul (\textquote{\textlatin{in sua Gallia}}).\footnote{\textquote[\textquote{It seemed to him, however, amazing that there would be any business in his own Gaul, which he had won in war, for either Caesar or the Roman people} \emph{BG}~1.34.4]{\textlatin{Sibi autem mirum videri, quid in sua Gallia, quam bello vicisset, aut Caesari aut omnino populo Romano negotii esset}}.} If Ariovistus posses this part of Gaul, then there no middle on which to meet. To the primary point of settling beyond the Rhine, Ariovistus claims he may do so because he defeated certain Gauls in war: \textquote[\textquote{in his [Ariovistus's] Gaul, which he had won in war} \emph{BG}~1.34.4][]{\textlatin{in sua Gallia, quam bello vicisset}}. What Ariovistus means by this, and as he repeats twice later, is that he has the right to settle in the disputed territory \textquote{by right of war}: \textquote[\emph{BG}~1.36.1][]{\textlatin{ius \ldots{} belli}} and \textquote[\emph{BG}~1.44.2][]{\textlatin{iure belli}}. In the first exchange of \emph{legati}, there is a clear distinction in the words offered by each leader. Ariovistus shows no interest in or valuation of the exchange of information in pursuit of a resolution of dispute.

In the next round of messengers (\emph{BG}~1.35--36), Caesar's words to Ariovistus fail again to bring about a meeting. This second exchange of \emph{legati} elaborates Caesar's communicative and leadership style. Caesar promotes a two--way relationship based on \emph{amicitia} (\textquote{friendship}), while Ariovistus again disregards any need for the \emph{conloquium}, compromise, or dialog. Ariovistus repeats what he had said about the absolute right conferred by military victory, and accordingly does not engage further in any communicative action. Caesar's second message highlights the Roman concept of \emph{amicitia} and the responsibilities that Ariovistus has to the Romans because of their bestowal of the title \emph{amicus} (\textquote{friend}). \blockquote[\emph{BG}~1.35.2]{\textlatin{Quoniam tanto suo populique Romani beneficio adfectus, cum in consulatu suo rex atque amicus ab senatu appellatus esset, hanc sibi populoque Romano gratiam referret, ut in conloquium venire invitatus gravaretur neque de communi re dicendum sibi et cognoscendum putaret, haec esse, quae ab eo postularet.}} \blockquote[\emph{BG}~1.35.2]{Since, having been affected by <Caesar's> own great kindness and that of the Roman people, though in <Caesar's> consulship he had been called king and friend by the senate, he returned this kindness to <Caesar> and the Roman people; with the result that, though invited, he was reluctant to come into a \emph{conloquium}, nor thought their common matter ought to be discussed and considered by him, <Caesar> demanded these things from him.} Because Caesar and the Roman state have shown kindness (\textquote{\textlatin{beneficio}}), and since Ariovistus has been called a friend (\textquote{\textlatin{amicus ab senatu appellatus}}), Caesar claims that he is obliged to reciprocate, in this case to engage at least in a meeting (\textquote{\textlatin{in conloquium venire}}). Ariovistus has instead, Caesar says, returned a refusal to compromise and communicate as \textquote{gratitude} (\textquote{\textlatin{hanc \ldots{} gratiam referret}}). Caesar's letter refers to Ariovistus's \emph{appellatio} (official designation) by the senate as an \emph{amicus}, a so--called \textquote{client king,} commonly signified by expressions such as \textquote[][.]{\textlatin{rex sociusque et amicus}}\footnote{\emph{OCD}\textsuperscript{3} \label{ft-amicita} \parencite[s.~v.~\textquote{client kings}]{ocd3}. \textquote[{\cite[55]{gruen1984}}][]{Military cooperation, with or without \emph{foedus}, made the partners \emph{amici}. A state that sought Roman assistance and was accepted into \emph{fides} would henceforth be adjudged an \emph{amicus}}. On \emph{amicitia} in Roman foreign policy, see \textcite{badian1958} and \textcite[55--69]{gruen1984}. On \emph{amicitia} within Roman politics, see comment by \textcite[157]{syme1939} and \textcite[351--381]{brunt1988}, which argues that \emph{amicitia} does not alone explain political relationships, though it does often oblige the exchange of mutual services (\emph{officia}).} Though when used of foreigners, \emph{amicus} could by Caesar's time \textquote[{\cite[12]{badian1958}}]{be a polite term for an inferior}, the term implies a relationship which is \textquote[{\cite[13]{badian1958}}]{ready enough to fit into the requirements of \emph{beneficium} and \emph{officium} that form the moral basis of that category}.\footnote{\textquote[{\cite[13]{badian1958}}]{Thus, to the mixed and disparate category that may be called `\emph{clientela} proper,' there are added, as the rise of Roman power and prestige makes equality a pretence, other relationships formally of a different nature, but ready enough to fit into the requirements of \emph{beneficium} and \emph{officium} that form the moral basis of that category}. \textcite[158--200]{gruen1984}, however, argues against over--extrapolating Roman social relations onto international relations in the Republic.} The text's union of Roman foreign policy with the inter--personal concept of \emph{amicitia} (\textquote{friendship}) is significant, since reciprocity, or an \textquote[{\cite[127]{konstan1997}}]{ethic of obligation}, is central to \emph{amicitia}.\footnote{\textquote[{\cite[123]{konstan1997}}]{This ethic of obligation was assumed to be associated particularly with relations between friends, which accorded with the picture of pragmatic give and take as the foundation of personal alliances among political leaders}. On friendship in Roman culture, see \textcite[122--148]{konstan1997}. For the related notion of \textquote[\textquote{loyalty}]{\textlatin{fides}} in the \emph{Bellum~civile}, see \textcite[136--138]{batstonedamon2006}.} Caesar holds, then, that Ariovistus's relationship with him and the Roman state demands two--way relations, not unidirectional and unilateral activity. Again, the meeting is called a \emph{conloquium}, and here the activities of it are elaborated. To enter into a \emph{conloquium} is to \textquote[][.]{to speak about and listen to a common matter} \textquote{\textlatin{De communi re}} (\textquote{about a common matter}) and the gerundives \textquote{\textlatin{dicendum}} and \textquote{\textlatin{cognoscendum}} signify two--way communication. As gerundives these words reveal a presumption by the actor Caesar (who composes this message) that there is in fact an obligation to communicate in this manner. Finally, Caesar strongly requests (\textquote{\textlatin{postularet}}) that Ariovistus engage in dialog, a move which does not invoke any absolute claims to authoritative action by virtue of military dominance advanced by the German at \emph{Bellum~Gallicum}~1.34.4. The verb is appropriate here thematically, and also because \textquote{\textlatin{postulare}} is not uncommon in conjunction with \emph{amicitia} in first century discourse.\footnote{Observe, for example, the congruence of the two terms in a letter from Cicero to Caelius: \textquote[\textquote{Nevertheless I beg you, in whatever land I go, that you watch over me and my children as our friendship and your trust requires (\textquote{\textlatin{postulabit}})} \emph{Ad~fam.}~2.16.8]{\textlatin{te tamen oramus, quibuscumque erimus in terris, ut nos liberosque nostros ita tueare ut amicitia nostra et tua fides postulabit}}. For similar constructions, see Cicero \emph{Ad~familares} 5.7.3 and \emph{Ad~Atticum} 11.12.8.} In other words, \textquote{\textlatin{postulare}} appears with \emph{amicitia} because of the obligations (sometimes mutual, sometimes skewed toward one party) it confers. This second message from Caesar to Ariovistus further establishes an actor Caesar whose will is to negotiate in lieu of fighting.

Ariovistus's second response to Caesar's appeal for dialog displays a sort of might--makes--right argument, according to which communication has no place in a dominate actor's command of the weaker. The catch is that Ariovistus does not acknowledge that his right to conquest interferes with the Romans' obligations to their allies. Not responding directly to Caesar's call for a reciprocal relationship founded on \emph{amicitia}, Ariovistus claims that since he has defeated the Haedui, he may do whatever he wants to them, for law of war means that the winners \textquote[\textquote{\textlatin{imperarent}}]{may command} the defeated as they see fit: \textquote[\textquote{The law of war <means> that they may command those whom they have defeated, in whatever way the want} \emph{BG}~1.36.1]{\textlatin{ius esse belli, ut qui vicissent, iis quos vicissent, quemadmodum vellent, imperarent}}. \textquote[][,]{\textlatin{Imperare}} throughout the \emph{Bellum~Gallicum}, is used for Caesar's orders to defeated or submissive, though allies.\footnote{E.g., \emph{BG}~4.22.2, 5.20.3, 5.22.4, and 6.4.6.} The word is also commonly used for generals to their officers, showing that Ariovistus makes complete and total claim to the Haedui. Since Caesar frequently \textquote{orders} his officers with \textquote[][,]{\textlatin{imperat}} it seems that Ariovistus makes a similar all--encompassing claim to his Gallic subjects.\footnote{For \textquote{\textlatin{Caesar \ldots{} imperat}} or similar at e.g. \emph{BG}~5.1.1, 5.1.8, 5.7.7, 7.45.1, and 7.86.2.} Romans, Ariovistus points out, operate by this very principle, by which the more powerful has the privilege of ordering (\textquote{\textlatin{imperare}} the weaker according to its own command (\textquote{\textlatin{ad \ldots{} praescriptum}} and \textquote{\textlatin{ad suum arbitrium}}). \blockquote[\emph{BG}~1.36.1]{\textlatin{Item p.~R.~victis non ad alterius praescriptum, sed ad suum arbitrium imperare consuesse.}} \blockquote[\emph{BG}~1.36.1]{Likewise the Roman people were accustomed to order the defeated not according to the command of another, but according to their own judgment.} There is some sense to Ariovistus's point here, that Romans rule unilaterally over the defeated. \textquote{\textlatin{Praescriptum}} and \textquote[][,]{\textlatin{arbitrium}} though uncommon in the \emph{Commentarii}, also have the ring of absolute imperative.\footnote{\textquote{\textlatin{Praescriptum}} appears once at \emph{Bellum~civile} 3.51.4, and \textquote{\textlatin{arbitrium}} two other times in the \emph{Bellum~Gallicum}, at 6.11.3 and 7.75.5.} Unlike Caesar, who calls for an exchange of information, Ariovistus prefers one--way demands.

When Ariovistus does acknowledge Caesar's conception of relationship, he comes to an opposite conclusion that mutual understandings do not inspire any real ethic of obligation (such as the free exchange of information), but allows for them to not communicate. Were he to return the Haedui's hostages without tribute, then the \textquote[\textquote{the title of Brother of the Roman people would mean nothing} \emph{BG}~1.36.5]{\textlatin{fraternum nomen p.~R.~afuturum}}. Whereas Rome's treaties with Gauls appear to facilitate communication with Caesar (as at \emph{BG}~1.31), for Ariovistus a formal relationship displaces any need for communication. \blockquote[\emph{BG}~1.36.2]{\textlatin{Si ipse populo Romano non praescriberet quemadmodum suo iure uteretur, non oportere se a p.~R.~in suo iure impediri.}} \blockquote[\emph{BG}~1.36.2]{If he himself [Ariovistus] was not ordering the Roman people how it should enjoy its own right, then it is not right for him to be hampered by the Roman people in his own right.} His version of reciprocity is fundamentally one of non--interference, with little or no emphasis on mutual obligations or benefits, as Caesar has expressed. Should conflict arise, it may be resolved in war, not dialog. The appearance of \textquote[\textquote{ordering}]{\textlatin{praescriberet}} echoes \textquote{\textlatin{praescriptum}} above. This and the previous sentence from Ariovistus are interesting also because they cite the Roman people (here, \textquote{\textlatin{populo Romano}} and \textquote{\textlatin{a p.~R.}}), whereas Caesar twice appeals to Ariovistus's obligations to both the Roman people and himself (\textquote{\textlatin{tanto suo populique Romani beneficio}} and \textquote{\textlatin{sibi populoque Romano}} \emph{BG}~1.35.2). For Ariovistus, there is no room for interpersonal relationships. Ariovistus's version of \emph{amicitia} is that it prevents not only war, but also the need for communication. To meet upon middle ground, he holds, is a sacrifice not to be demanded of a friend.

Ariovistus's two responses to Caesar's two messages make clear the contrast between the actors. Caesar seeks a middle ground in disputes, face to face, in a \emph{conloquium}, while Ariovistus holds that communication is only necessary for the making of demands, by the stronger, of the weaker. Within the discussions about holding a \emph{conloquium}, the text offers elaboration upon the respective characters' attitudes toward communication and power. The first two episodes of the latter half of \emph{Bellum~Gallicum}~1 -- the \emph{concilium} of Gauls and the exchanged messages with Ariovistus -- are structured according to the communicative style adopted by the actor Caesar's verbal communications. The distinction between what the narrator records as going on in Caesar's mind is very different from the words the actor uses with others.

In this episode, Caesar does not have the opportunity to engage with the psychological state of another in a face to face situation. Without an in--person meeting, Caesar's leadership and persuasive abilities appear compromised. In the next section of the text, Caesar engages in another communicative medium and context, in which he successfully brings others into obedience.

\section{\emph{Consilium}}
\label{bg-consilium}
The following episode highlights another variation on Caesar's communicative presence, the principle identified in this chapter by which Caesar interacts with other actors in the stories of the \emph{Bellum~Gallicum}. Face to face, Caesar demonstrates the techniques of psychological intervention pointed out by Lendon as central to his leadership strategy in the \emph{Bellum~Gallicum}. Caesar uses techniques similar to those found when engaging with the Gauls during the earlier \emph{concilium}. Here, the actor Caesar accomplishes what he does through the control of the bodies, speech, and minds of his subordinates. The need for this \emph{consilium} is explained by a corruption of regular hierarchical communicative practices in Caesar's army. If information is communicated outside of the control of the general, the entire army ceases to function. This hierarchical system of control comes undone through a corruption or misuse of communicative and affective processes, resulting in an intervention by Caesar, who regains control over this network. And as seen in the \emph{concilium} and \emph{legati} episodes, there is a discrepancy between explanations for others' obedience of the general. In this case, the narrator ascribes the soldiers' disobedience to fear, and their loyalty to shame; whereas Caesar and the soldiers' words speak to an obedience arising from their \emph{virtus}. The relevance of this section to my larger argument is that the text shows again a two--part explanation of how Caesar leads others.

Caesar's communicative mode is so important that the entire Ariovistus episode falls into discrete components according to units corresponding to them. Here, the context is a \emph{consilium} (\textquote{advisory body} \emph{BG}~1.37--41). In this case, there is some narrative setup (\emph{BG}~1.37--39) to the communicative context proper, though the problem culminates and is resolved through this assembly of Caesar and certain of his men (\emph{BG}~1.40--41). After the previous communications through \emph{legati}, messengers from the Haedui and Treveri tell him that more Germans were coming into Gallic territory (\emph{BG}~1.37). Caesar then moves quickly for Vesontio, where he has heard Ariovistus was heading, and occupies it (\emph{BG}~1.38). Here, the soldiers hear from Gallic traders that the Germans are strong and skillful in battle  (\emph{BG}~1.39). The soldiers then begin talking among themselves, exaggerating rumors of the Germans, and thus become reluctant to fight (\emph{BG}~1.39). This prompts Caesar to call a \emph{consilium} (\textquote{\textlatin{convocato consilio}} and \textquote{\textlatin{ad id consilium}}) in order to bring the soldiers back into line (\emph{BG}~1.40.1). Here, Caesar gives a long speech (over a page in the Hering text). After this speech, the soldiers pledge allegiance to Caesar and the army moves on in pursuit of Ariovistus (\emph{BG}~1.41.4). Immediately after this, Caesar contacts Ariovistus by \emph{legati} and a face to face meeting termed a \emph{conloquium} is arranged (\emph{BG}~1.42). As with the \emph{concilium} and \emph{legati} episodes, this portion of the narrative is structured according to Caesar's communicative context.

The most complex example of psychological intervention via communicative presence is found in the \emph{consilium} at Vesontio. There is an interrelated, two--part problem. As demonstrated in this section, the initial problem is that hierarchically--controlled communication patterns disintegrate within the Roman army, which immediately results in an infirmity of mind among the soldiers, whose fear makes them unable to march upon the Germans. This falling apart of communication patterns is documented at \emph{Bellum~Gallicum} 1.39 and acts as a set--up to the \emph{consilium}. In a prime example of Caesar's communicative presence, along the lines of that seen in the \emph{concilium} above, the commander linguistically intervenes in the psychological habits of his soldiers. To finalize the reestablishment of the power structure, the army verbally expresses their loyalty to Caesar. This part of the episode demonstrates another facet of communication in the \emph{Bellum~Gallicum} -- its role in maintaining power structures.

The soldiers' insurrection arises in the deviation of their communicative habits. They begin conversing with outsiders to the army and then with one another, which leads to fearful mental states.\footnote{See also the narrator on rumor creating fear about soldiers at \emph{Bellum~Gallicum} 6.37.7--9.} This discourse is described as uncontrolled gossip in impromptu meetings: \blockquote[\emph{BG}~1.39.1]{\textlatin{Ex percontatione nostrorum vocibusque Gallorum ac mercatorum \ldots{} tantus subito timor omnem exercitum occupavit}}. \blockquote[\emph{BG}~1.39.1]{From inquiry of our men and gossip of Gauls and traders \ldots{} such a great fear suddenly occupied all the army.} The fear of the Roman soldiers is a result of three communicative problems. First, the soldiers are receiving information from outsiders of the army, and not from Caesar or their direct superiors. \textquote{\textlatin{Percontatio}} can have the special sense of being an \textquote{interrogation} of an enemy (\emph{OLD} def.~a). Caesar uses \textquote{\textlatin{percontatio}} once elsewhere in the \emph{Commentarii}, of questioning Britons (perhaps captives) about the geography of their island (\emph{BG}~5.13.4). Similarly, at \emph{Ad~Herennium} 2.13 a \textquote{\textlatin{percontatio}} is questioning against legal \textquote[\textquote{adversaries}]{\textlatin{adversarii}}, and in Livy: a \textquote[\textquote{an interrogation <is> made of captives} 27.43.5]{\textlatin{ex captiuis percontatio facta}}. This kind of exchange of information, in general in Latin literature and in the \emph{Bellum~Gallicum} in particular, does not appropriately fall to individual soldiers. Second, the result of this type of exchange of information is rumor, a decentralized communicative mode. When rumor prevails, Caesar and subordinate officers are cut out of this movement of information within the army. With fear and information moving out of the centralized control of the general, the organization and chain of command of the army is ruined. Eventually, from the tears and complaints of these inexperienced officers infect the courage of experienced soldiers (\emph{BG}~1.39.3--4). \blockquote[\emph{BG} 1.39.5]{\textlatin{Horum vocibus ac timore paulatim etiam ii qui magnum in castris usum habebant, milites centurionesque quique equitatui praeerant, perturbabantur.}} \blockquote[\emph{BG}~1.39.5]{From the gossip (\emph{voces}) and fear (\emph{timor}) of these men, little by little even they, who had great experience in military camps -- soldiers and centurions and those in charge of the cavalry -- became disturbed.} Here, \textquote{\textlatin{voces}} is best translated as \textquote[][.]{gossip}\footnote{See \textcite{kranerdittenbergermeusel1967} for \textquote{\textlatin{vocibus}} as \textquote[][.]{\textgerman{Gerede}}} Like an infection, \textquote[\textquote{\textlatin{vocibus ac timore}}]{gossip and fear} reproduce themselves and spread through the army. They are so closely associated here that the narrator expresses them in conjunction. Who exactly conversed with the Gauls is not made clear, but it is these actors who invoke the emotions and set off fear throughout the entirety of the army. That this imputation of gossip is leveled at Romans is serious, since elsewhere in the \emph{Bellum~Gallicum}, rumor is a communicative method attributed solely to Gauls. For example, all five instances of \emph{rumor} in the \emph{Bellum~Gallicum} are used in reference to Gallic communication.\footnote{\emph{BG}~2.1.1, 4.5.3, 6.20.1, 7.1.2, and 7.59.1.} Misinformation is such a problem among the Gauls, in fact, that they have a system by which to minimize its damaging effects (\emph{BG}~6.20.1–3). \textquote[\textquote{\textlatin{timor}}]{Fear} which has seized the army, and misinformation are endemic to rumor--like communication (\emph{BG}~1.39.1). Unidentified members of the army tell Caesar \textquote[\textquote{that the soldiers would not obey his order}]{\textlatin{non fore dicto audientes milites}}. Again, the narrator makes it clear that this is on \textquote{account of fear} (\textquote{\textlatin{propter timorem}}).\footnote{\textquote[\textquote{Furthermore some announced to Caesar that the soldiers, although he had ordered the camps to be moved and the standards to be carried, would not listen to the order nor carry the standards on account of fear} \emph{BG}~1.39.7][]{\textlatin{Nonnulli etiam Caesari nuntiabant, cum castra moveri ac signa ferri iussisset, non fore dicto audientes milites neque propter timorem signa laturos}}.} Third, the gossip and fear spread among not only the soldiers but even into the officer ranks.\footnote{\textquote[\textquote{This <fear> first arose from the \emph{tribuni militum}, \emph{praefecti}, and others who, having followed Caesar from the city [Rome] for the sake of friendship, did not have much experience in military affairs} \emph{BG}~1.39.2]{\textlatin{Hic primum ortus est a tribunis militum, praefectis reliquisque, qui ex urbe amicitiae causa Caesarem secuti non magnum in re militari usum habebant}}.} These three parts of this setup show the danger of communication outside of a hierarchical chain of commands. Inappropriate information comes from the outside, spreads among the soldiers and creates fear, and then this gossip--fear travels up the chain of command. Of the many ways of possibly explaining the insurrection at Vesontio, it is important that the narrator has expressed this as a corruption of communicative practices, for it is only then that one may understand the correctives, communicative in nature, that Caesar applies to the situation in the following \emph{consilium}.

When he hears of this disobedience stemming from irregular communication between soldier and Gauls, and between soldiers themselves, Caesar immediately reestablishes a hierarchical communication pattern. Caesar is the ultimate source of all information, and it is according to his orders that this information is sent and received among the members of the army. This meeting is neither a \emph{concilium} nor \emph{conloquium} as above, but labeled a \emph{consilium}. According to \textcite{meusel1887}, \emph{consilium} has three meanings, all of which appear in the \emph{Commentarii}, of either the act of \textquote[][,]{deliberation}  or the \textquote{deliberative assembly} in which deliberation occurs, or the \textquote{counsel} that emerges from such assemblies. These correspond, respectively, to definitions A, B, and C \parencite[s.~v.~\textquote{\textlatin{consilium}}]{meusel1887}. For the second definition, other translations by the historian Adrian Goldsworthy are \textquote[{\cite[226]{goldsworthy2006}}]{a council or briefing}. All three meanings are common in the \emph{Commentarii}. Here, the sense of \emph{consilium} is used in two ways, twice of the second sense (\textquote{deliberative assembly}) and once of the third (\textquote{counsel}). \blockquote[\emph{BG}~1.40.1]{Haec cum animadvertisset, convocato consilio omniumque ordinum ad id consilium adhibitis centurionibus vehementer eos incusavit: primum quod aut quam in partem aut quo consilio ducerentur, sibi quaerendum aut cogitandum putarent.} \blockquote[\emph{BG}~1.40.1]{When he had noticed these, with a \emph{consilium} (\textquote{council}) called and centurions of all ranks called to this \emph{consilium} (\textquote{council}), he vehemently criticized them, first, for thinking that they may presume to inquire into or think about either into which area or by which \emph{consilium} (\textquote{counsel}) they should be led.} In an army, a \emph{consilium} is often translated as a \textquote[][,]{council of war} was led by an \emph{imperator}, and was constituted of officers and senior centurions.\footnote{On \emph{consilia} within armies in the late republic, see \textcite[131--133]{goldsworthy1996}: \textquote[p.~132]{The officers present often seem to have expressed opinions on the tactical situation and recommend certain courses of action. \ldots{} What is important to note is that almost invariably the commander is represented as having the final decision after the other officers had spoken, and he might reject all their suggestions}.} The distinction between the two \emph{consilia} is parallel to that in other authors.\footnote{For example, the \emph{OLD} divides the first definition between \textquote[def.~a]{debate, discussion, deliberation} and \textquote[def.~b]{a meeting for deliberation}. For martial examples of the former, see Livy 21.41.2, Tacitus \emph{Historia} 2.40, and of the latter Livy 9.15.1. Livy 22.53.6 is interesting in this context, as it uses both definitions both definitions, like \emph{BG}~1.40.1: \textquote[\textquote{but they who thought a council ought to be called about it, the young [Scipio], destined leader of this war, refused any counsel: he said it ought to be dared and done, not deliberated (\textquote{\textlatin{consultandum}}) in response to such a great evil}]{\textlatin{qui aderant et consilium aduocandum de eo censerent, negat consilii rem esse [Scipio] iuuenis, fatalis dux huiusce belli: audendum atque agendum, non consultandum ait in tanto malo esse}}.} The power of this term \emph{consilium} is that it implies a venue for some exchange of information, while in fact, Caesar ends up berating them, thus not satisfying any definition of such an assembly.

The \emph{consilium} should be distinguished from other types of assemblies which the actor Caesar could have called. For example, it is not a \emph{contio}, in which Caesar similarly upbraids his soldiers after a defeat (\emph{BG}~7.52.1); a \emph{contio} implies a meeting in which one person gives a speech to a body of people, and not any dialogic deliberation or advice--giving.\footnote{A \emph{contio} is a meeting called by and organized around one leading, speaking individual: \textquote[\emph{OLD}, s.~v.~\textquote{\textlatin{contio}}]{any public meeting, assembly; (esp.~mil.) a parade addressed by a general}.} The audience of this assembly is important. Instead of addressing the entire army, Caesar addresses the centurions, which would amount to roughly 420 centurions.\footnote{So \textcite{kranerdittenbergermeusel1967}: \textquote[][.]{\textgerman{Caesar läßt an der Versammlung, die nicht eine Beratung, sondern nur Ermahnung und Ermutigung bei der allgemeinen Furcht und Verwirrung zum Zweck hat, alle Centurionen (60 in jeder Legion) teilnehmen, während zum eigentlichen Kriegsrat nur die primorum ordinum centuriones außer den Legaten und Tribunen zugezogen wurden}} On  the historical composition of wartime \emph{consilia}, see \textcite[131--133]{goldsworthy1996}. Therefore, 420 centurions: 60 centurions per legion and a total of seven legions, the number at \emph{Bellum~Gallicum} 1.10.3.} These addressees will in turn disseminate the general's reprimand according to the established chain of command, thus reinstating the proper hierarchical flow of information. And though there will be a response from the soldiers, centurions, and subordinate officers, there is practically no dialog or middle ground, but only Caesar's single position. This \emph{consilium} corresponds loosely to a \textquote{deliberative assembly} in outer form only, for the acts of deliberation and counsel do not accord with expectations of such a meeting. This ambiguously labeled meeting is Caesar's first step in the reestablishment of communication patterns in the army.

The second step in the restoration of order is a reestablishment of communication, which is at the same time an intervention in the officers' and soldiers' minds. Following Caesar's upbraiding speech (\emph{BG}~1.40), the minds of all were astonishingly turned around: \textquote[\textquote{the minds of all were astonishingly turned around} \emph{BG}~1.41.1]{\textlatin{mirum in modum conversae sunt omnium mentes}}. Caesar's control of the army is complete with all (\textquote{{omnium}}) the minds of his soldiers returned to obedience. With this control, they will not consider gossiping with outsiders or with themselves, nor thinking about this information, nor acting upon the conclusions of their deliberation.

Finally, following this reversal, their reaffirmation of Caesar's authority is explained. First, the tenth legion gives thanks to Caesar for having praised their loyalty to him: \textquote[\textquote{he was confident on account of their \emph{virtus}} \emph{BG}~1.40.15]{\textlatin{propter virtutem confidebat maxime}}. \emph{Virtus} in Caesar is more than just bravery, but obedience, or something a quality held when obedience is questioned. The Roman soldier, argues \textcite[89]{riggsby2006}, is praised not only by individual bravery, but also through also through collective \textquote{submission to authority.}\footnote{Riggsby continues on \emph{virtus}: \textquote[89]{Roman soldiers do what they must because they are told to do so, whether that is by fighting or not fighting. \ldots{} The act of fighting, though, is not the touchstone, as in a more traditional view. Rather, it is obedience under difficult circumstances}.} In an important corrective to this stance, \textcite[303--304]{mcdonnell2006} points out that \emph{virtus} need not be the same property as obedience, but something they can have when their obedience is challenged.\footnote{Citing \emph{Bellum~Gallicum} 7.52.4, where Caesar rebukes soldiers for presuming to do battle without his instructions. Caesar says that \textquote[\textquote{he desires in a soldier modesty and self--control not less than \emph{virtus} and greatness of courage}]{\textlatin{non minus se in milite modestiam et continentiam quam virtutem atque animi magnitudinem desiderare}}. For more on \emph{virtus} in Caesar's writings, see \textcite[300--319]{mcdonnell2006}.} The tenth legion, which Caesar singled out for their steadfast loyalty in this insurrection (\emph{BG}~1.40.15), first professes obedience to Caesar. \blockquote[\emph{BG}~1.41.2]{\textlatin{Princepsque decima legio per tribunos militum ei gratias egit, quod de se optimum iudicium fecisset, seque esse ad bellum gerendum paratissimam confirmavit.}} \blockquote[\emph{BG}~1.41.2]{And the tenth legion first gave thanks to him through the tribunes of the soldiers (\emph{tribuni militum}), that he had formed the best judgment about them, and confirmed that they were most prepared for waging war.} That the tenth legion gives thanks is clear, though how they do so is less. The centurions, the bulk of the audience in the \emph{consilium}, must make some sort of collective decision to give thanks, as this thanksgiving is done \textquote[\textquote{to him through the \emph{tribuni militum}}]{\textlatin{per tribunos militum ei}}.\footnote{It is unclear how many military tribunes would have been present in the tenth legion (perhaps six total, with only two actively serving at a given time), though it was surely far fewer than all sixty centurions. \textcite[57]{holmes1914a} thinks all six were present here because \textquote[][.]{this council \ldots{} was not a council of war} \label{tribuni-militum} For an overview of military tribunes in the late Republic, see \textcite[41--42]{suolahti1955}. See also \textcite[136--141]{rosenstein2007}.} Here, we see a reverse flow of information to Caesar's speech, with his subordinates communicating with him, this time through hierarchical channels. For the loyal tenth legion, \emph{tribuni militum} give thanks directly to the \emph{imperator}, in apparently universal obedience, also implied by the unqualified \textquote{\textlatin{decima legio}} and \textquote[\textquote{most prepared}]{\textlatin{paratissimam}}, is expressed. In the way that Riggsby observes centurions' \emph{virtus} acting with obedience to military hierarchy, this \emph{virtus} may be extended beyond action in battle (Riggsby's concern) and to their response to orders (or here, \textquote{advice}). Further, this obedience to hierarchy expresses itself in the upward flow of information. Order is restored, since the centurions speak \textquote[\textquote{\textlatin{per}} \emph{BG}~1.41.2]{through} their superiors, not, say, \textquote[\textquote{\textlatin{ex percontatione \ldots{} vocibusque}} \emph{BG}~1.39.1]{from inquiry \ldots{} and gossip}. So, communication must trace patterns of power, the Roman army's chain of command, or this network fails to function as intended, \textquote[\textquote{for waging war}]{\textlatin{ad bellum gerendum}}. The restoration of power, in this episode of the \emph{Bellum~Gallicum}, is the restoration of authorized communicative patterns, for which Caesar is the center and source of control.

The reestablishment of power is more difficult with the remaining legions, which the narrator portrays as guilty (\emph{BG}~1.41.3--4). These soldiers talk with the \emph{tribuni militum} and \emph{\textlatin{primorum ordinum centuriones}} and then pledge allegiance to Caesar. \blockquote[\emph{BG}~1.41.3]{\textlatin{Deinde reliquae legiones cum tribunis militum et primorum ordinum centurionibus egerunt, uti per eos Caesari satis facerent; se neque umquam dubitasse neque timuisse neque de summa belli suum iudicium, sed imperatoris esse existimavisse.}} \blockquote[\emph{BG}~1.41.3]{Then the remaining legions discussed with military tribunes (\emph{tribuni militum}) and centurions of the first orders (\emph{\textlatin{centuriones primorum ordinum}}), that they pledged allegiance to Caesar through them. They said that they had never hesitated nor feared nor thought that the highest judgment about the management of the war was their own, but belonged to the general.} The expression of allegiance by the rebellious legions shows some differences to the faithful tenth legion. First, there are added two classes of actors, the \emph{\textlatin{centuriones primorum ordinum}}, who consisted of the five senior centurions of the first cohort, and Roman soldiers.\footnote{See \textcite[191--192]{gilliver2007} on centurions in the Augustan system, which followed the Republican with respect to centurions. Some think that this number is six, including the \emph{primus pilum} \parencite[191]{gilliver2007}. \textcite[58-59]{holmes1914a} considers the evidence and speculates that the \emph{\textlatin{centuriones primorum ordinum}} were six.} With six \emph{tribuni militum} and five \emph{\textlatin{centuriones primorum ordinum}} per legion, and with a total of these six legions (excluding the tenth), the \textquote{remaining legions} put forward sixty--six officers and centurions to reconcile with Caesar.\footnote{Assuming two \emph{tribuni militum} per legion.} The phraseology of \textquote[\textquote{the remaining legions}]{\textlatin{reliquae legiones}} is not direct, though refers to Roman soldiers of the rebellious legions. This is clear from the expression \textquote[\textquote{negotiated} or \textquote{discussed with}]{\textlatin{cum \ldots{} egerunt}}.\footnote{\textquote[{\cite{kranerdittenbergermeusel1967}}]{\textgerman{\textquote{Verhandelten, besprachen sich} mit ihnen}}.} That these are soldiers is also denoted by the generalizing \textquote[][,]{\textlatin{legiones}} signifying that the soldiers as some whole are having a communicative relationship with their respective, immediate leaders, \emph{tribuni militum} and \emph{primorum ordinum centuriones}. That these are negotiations and not direct orders from the officers implies that order has not yet been fully reinstated. Why are these thirty additional officers pledging loyalty? Considering the dissension among the ranks, perhaps there is need for a greater conduit through which the rebellious legions may reestablish communicative bonds. The implication of this is that it is not just the seditious content of their talk, but the channels through which their communication moves that is problematic. In this narrative, the flow of information is further elaborated. The soldiers' negotiations with their officers is a separate event in a separate clause (connected by an \textquote{\textlatin{ut}}). Next, the text clearly explains that the reconciliation is done \textquote[\textquote{to Caesar through them}]{\textlatin{per eos Caesari}}. In this \emph{consilium} is a complete communicative cycle from the top to the bottom of the army. Information moves from Caesar to his subordinate officers to their subordinates and then back up from soldiers to the officers and to Caesar. Due to a lack of control over soldiers' communicative habits, their minds become unfit for duty due to fear. As shown next, the soldiers' fear goes so far as to make them forget their proper role in the army as actors, not decision makers. Caesar's communicative presence is a remedy to the soldiers' delusion.

It is noteworthy that there is an additional concept added to the soldiers' rebellion: their fear is so great that it has led them to forget their proper role in the army. Not surprising in a military context, the text portrays this failure to perform according to one's rank as negatively. The \emph{Bellum~Gallicum} stresses in multiple places the danger that fear may play in the soldiers of the Roman army, especially in the absence of their general.\footnote{See, for example, the gnomic statements about fear at \emph{Bellum~Gallicum} 5.33.1 (impact on decision--making; see section~\ref{cotta-sabinus} on Sabinus's fear) and 7.84.5: \textquote[\textquote{For everything, more or less, when is absent, violently disturbs the minds of men}]{\textlatin{omnia enim plerumque, quae absunt, vehementius hominum mentes perturbant}}. See also \emph{Bellum~Gallicum} 6.41.2--4 on soldiers' fears put aside by Caesar's presence. See also related comments on fear and absent leaders at \emph{Bellum~civile} 1.32.8--9, 1.33.1, and 1.61.3.} Caesar's first accusation is that the centurions (and soldiers by association) had mistaken their roles, which entail the following of orders, with that of Caesar's, which entails decisions about where the army goes and by whose advice they act (\textquote{\textlatin{primum quod aut quam in partem aut quo consilio ducerentur}}). Improper action is highlighted at \emph{Bellum~Gallicum} 1.40.10, where their wrongful action is described either as a loss of faith in their general's sense of duty (\textquote{\textlatin{de officio imperatoris desperarent}}) or the simple wrongful action of prescribing (\textquote{\textlatin{praescribere}}) Caesar, the decision--maker, what to do.\footnote{\label{aliae-partes}For a narratorial statement on roles and obligation, see \emph{Bellum~civile} 3.51.4: \textquote[\textquote{The responsibilities of the legate and the general are different: one ought to do everything according to the order, the other to freely consider the most important matters}]{Aliae enim sunt legati partes atque imperatoris: alter omnia agere ad praescriptum, alter libere ad summam rerum consulere debet}.} In the same vein, Caesar's challenge of his centurions' sense of shame (\emph{pudor}) and obligation to duty (\emph{officium}) at \emph{Bellum~Gallicum} 1.40.14 calls to his audience's minds their proper role in the army, the duty assigned to it, and their sense of obligation to this role and its obligations. The soldiers' disobedience highlights the crucial relationship between authorized speech acts and their dependence upon networks of actors fulfilling their obligations within an authorized prescribed of action. In his response to a breaking down of the system that is this Roman army in Gaul, Caesar explains in explicit terms how the system works and why it ought to be sustained.

Though the passage emphasizes the rectification of communication patterns, the content of what the subordinate officers and centurions say is meaningful, too. Their disobedience, according to the text, is turned to absolute obedience. Notably, the expression \textquote[\textquote{they met his desires}][]{\textlatin{satis facerent}} means that the legions are stating and pledging absolute obedience. This expression is especially strong form of commitment: according to \textcite{kranerdittenbergermeusel1967}, it denotes a pledge for the past, present, and future.\footnote{Translation from \emph{OLD} def.~6.b, though see also \textcite{kranerdittenbergermeusel1967}: \textquote{\textgerman{Dies Verbum wird hier und V 54, 3 gewöhnlich erklärt als gleichbedeutend mit \emph{\textlatin{se excusare}}; es ist aber mehr als dies: es bedeutet, einem andern gegenüber sich so verhalten, daß dieser für die Vergangenheit, Gegenwart und Zukunft befriedigt ist, also hier nicht bloß sich wegen des Vorgefallenen entschuldigen, sondern auch verbindliche Erklärungen wegen der Zukunft abgeben}.}} The strength of this pleading with Caesar signifies both the extent to which the legions have done wrong and how greatly with wish to restore relations. This confirms what has been observed in the narrator's initial insights into the soldiers fearfulness (at \emph{BG}~1.39), and shows something of the extent to which the speaking actor Caesar softens the blow of his knowledge in his address of the \emph{consilium}. In response, the soldiers' say that they have not attempted to usurp the general's role. They deny that they had any ambition to make decisions (\textquote{\textlatin{neque de summa belli suum iudicium}}) and affirm that decision--making belongs to the general (\textquote{\textlatin{imperatoris esse}}). The expression \textquote{\textlatin{de summa belli}} pertains to the highest levels of military strategy and decision--making.\footnote{\textquote[{\cite{kranerdittenbergermeusel1967}}]{\textgerman{Die oberste Leitung des Krieges}}. For a similar use of \textquote[][,]{\textlatin{summa}} in the context of decision making, see \emph{Bellum~civile} 3.51.4 at footnote~\ref{aliae-partes}.}

Fittingly, the tenth legion thanks Caesar for his \textquote{\textlatin{iudicium}} about them. In this sentence documenting the legions' reconciliation with power, there are several important characterizations of this process by the text. First, there is a multi--part reestablishment of authorized communicative networks, from Caesar through the officers to soldiers and back up. This rectification is done in response to an initial corruption of communication, which led to weakened mental states of the soldiers. Caesar's communicative presence may perhaps seem to fail at this moment of sedition, though it is reestablished by means of his lecturing of them in the \emph{concilium}. In Caesar's speech and the officers' response, the role of the general is to think about what to do, the officers' duty is to tell soldiers this will, and the soldiers' duty is to follow orders. 

As earlier in the Ariovistus episode, there is a double explanation offered for others' obedience. The narrator portrays Caesar thinking one thing and saying another. Before calling the \emph{consilium}, the text records Caesar as thinking that \emph{timor} prevents them from marching (\emph{BG}~1.39.1, 1.39.5, and 1.39.7). In the course of his speech, however, he does not directly address their fear of the enemy, but mentions it only briefly at the end, putting it in juxtaposition to their \textquote[\textquote{senses of shame and duty} \emph{BG}~1.40.14]{\textlatin{pudor atque officium}}. He thus does not directly accuse them of being afraid (which he believes) but instead emhasizes other terms. The soldiers and centurions are at fault in not acting in accordance with the specific actions entailed by their roles. As the label \emph{consilium} misleadingly implies a forum for the exchange of information, the actor Caesar is not directly honest about why he thinks they refuse to fight. His restraint in disclosure offers insight into how the text as a whole is interested in representing its author, Julius Caesar, as a communicative, yet shrewd, leader of men.

The \emph{concilium} and \emph{consilium} are two successful communicative acts in the Ariovistus episode, and thinking about communication and loyalty is perhaps easier here than in the failed communications with Ariovistus. \textcite{ando2000} explains the longstanding stability of the Roman empire as an effect of the meeting, in many linguistic acts by rational subjects in the Roman empire (provincial citizens, in particular) with another rational agent, the emperor.\footnote{See especially pp.~73--80. \textquote[{\cite[75]{ando2000}}]{What we require is a model of social action in general, and of communicative action in particular, that reveals what promises the Romans made when they published their laws, letters, and regulations. After all, relations between emperor and governor, and between governor and procurator, and between procurator and villagers, were all conducted through language, whether written or spoken: something about what the Romans said made the villagers think it worth their while to say something back}.} In a Habermasian mode, he describes how a normative consensus developed through a process of intersubjective communication, which he explains in contrast to subject--centered reason. In Habermas's own words, intersubjective reason is explained as the meeting of validity claims. \blockquote[{\cite[314]{habermas1987a}}]{\textquote{Rationality} refers in the first instance to the disposition of speaking and acting subjects to acquire and use fallible knowledge. \ldots{} Subject--centered reason finds its criteria in standards of truth and success that govern the relationships of knowing and purposively acting subjects to the world of possible objects or states of affairs. By contrast, as soon as we conceive of knowledge as communicatively mediated, rationality is assessed in terms of the capacity of responsible participants in interaction to orient themselves in relation to validity claims geared to intersubjective recognition. Communicative reason finds its criteria in the argumentative procedures for directly or indirectly redeeming claims to propositional truth, normative rightness, subjective truthfulness, and aesthetic theory.} The actor of the \emph{Bellum~Gallicum} -- in all four of the communicative environments in the Ariovistus episode -- forges normative consensus of rational agents, who meet together in attempts to understand the subjective reality of one another. Though the actor Caesar is no ideal Habermasian interlocutor, he does embody some of these characteristics, especially in contrast to the narrator's projections upon events. Ando explains provincial loyalty as a result of a normativity that arose out of people, who had different goals, coming to understand others and themselves through dialogic communication; for Caesar, the shared lifeworld is experienced by Caesar and others (officers, soldiers, allies and would--be allies). The actor Caesar's open efforts at free and open communication fail.

\section{\emph{Conloquium}}
\label{bg-conloquium}
So far, the Ariovistus narrative has built up to a face to face meeting identified as a \emph{conloquium}, which Caesar had sought at \emph{Bellum~Gallicum} 1.34.1 and 1.35.2. The importance of Caesar's presence is again apparent, as is his seemingly open communicativity. The actor Caesar shows a willingness to exchange information, while Ariovistus refuses to respond directly to him. There is good reason, however, to doubt the sincerity of Caesar's sincerity in this dialog, though. As in the previous exchange of \emph{legati}, he emphasizes reciprocity and relationships, while Ariovistus argues that friendship with the Romans is a contract of non--harming between two parties. 

This communicative context begins with the conclusion of the \emph{consilium} and ends with the battle (\emph{BG}~1.42--47). Within this section of text, there are preparations for the meeting (\emph{BG}~1.42), the \emph{conloquium} itself (\emph{BG}~1.43--45), a martial skirmish which ends the \emph{conloquium} (\emph{BG}~1.46), and a failed attempt at another \emph{conloquium} (\emph{BG}~1.47). The identification of this meeting as a \emph{conloquium} is very strong. For in addition to the labeling of the meeting as such at \emph{Bellum~Gallicum} 1.34.1 and 1.35.2, the word appears eleven times in the course of the meeting.\footnote{\emph{BG}~1.42.1, 1.42.3, 1.42.4, 1.42.5, 1.43.2, 1.43.3, 1.46.1, 1.46.3, 1.46.4 (twice), and 1.47.1.} Reflecting the text's clarity of the meeting's label, the term \emph{conloquium} itself is fairly narrowly defined as, generally, \textquote{talk, conversation (between private persons)} and, in military context, \textquote[\emph{OLD}, s.~v.~\textquote{\textlatin{conloquium}} defs.~1.a.~and 2.c., respectively]{a meeting for discussion of terms, etc., esp.~with an enemy; a parley}. In the fourth consecutive example of the sort, the text recounts Caesar's interaction with the Germans in book one by means of his communicative mode.

In preparation to the meeting, Caesar seems to take pains to equalize power relations between the two speakers. He thinks that Ariovistus has returned to a reciprocal relationship, returning cooperation for the good deeds previously shown by the Roman people, \textquote[\textquote{on behalf of such great known deeds, of the Roman people toward him} \emph{BG}~1.42.3]{\textlatin{pro suis tantis populique Romani in eum beneficiis cognitis}}. Again, Caesar finds connectivity in the obligations of \emph{amicitia}. The text construes the agreement to meet as Caesar agreeing to several of Ariovistus's conditions, showing again his willingness to negotiate a middle ground between them both. \textquote[\textquote{\textlatin{non respuit condicionem Caesar}} \emph{BG}~1.42.2]{Caesar did not refuse the proposal} and he agrees to Ariovistus's condition that there be no soldiers at the meeting, but only cavalry. As in the earlier messages, some sort of reconciliation is sought, as he repeats here (\textquote{\textlatin{condicionem}}) and in his recent speech.\footnote{See also \textquote[\textquote{with the equality of conditions seen}]{\textlatin{aequitate condicionum perspecta}} at \emph{Bellum~Gallicum} 1.40.3.} The narrator identifies the unacceptability of Ariovistus's previous un--cooperative behavior, saying that \textquote[\textquote{he [Caesar] thought he had returned to sanity} \emph{BG}~1.42.2]{\textlatin{eum ad sanitatem reverti arbitrabatur}}. In Caesar's preparation for the meeting, he has established a \emph{conloquium} of equal relations. First, the meeting happens on even ground, \textquote[\textquote{nearly an equal space from both camps} \emph{BG}~1.43.1]{\textlatin{aequum fere spatium a castris utriusque}}. At the face to face meeting, both have an equal number of horses (ten). The literal equality of this meeting is another aspect of the text's projection of fairness onto Caesar and is similar to his earlier request for a meeting in \textquote{\textlatin{aliquem locum medium utriusque conloquio}} (\emph{BG}~1.34.1). From this set--up, which has made possible the exchange of information, the conversation is not in fact dialogic.

In the lead up to the \emph{conloquium}, especially in Caesar's shrewd thinking at \emph{Bellum~Gallicum} 1.33.2--5 and Ariovistus's resistant communicativity at \emph{Bellum~Gallicum} 1.34-38, readers understand a mutual insincerity in both actors' intentions for engaging in dialog. Ariovistus makes a condition that soldiers not be present at the meeting, but only cavalry. The narrator details Caesar's reservations about being accompanied only by Gallic cavalry, so devises a plan to put members of his trusted tenth legion on the Gauls' horses. In this meeting, Caesar fears for his safety (\textquote{\textlatin{salutem suam}}) and makes plans that he has a \textquote[\textquote{guard}]{\textlatin{praesidium}} \textquote[\textquote{if in fact he needed anything} \emph{BG}~1.42.5]{\textlatin{siquid opus facto esset}}. Caesar's inner doubts about Ariovistus's intentions to violence--free communication are not reflected in his words to the German, though they are later validated in the German cavalry's attack of the Romans and the seizing of Caesar's Gallic representatives.

Accordant to the apparently equal conditions for the \emph{conloquium}, Caesar attempts dialog, while Ariovistus refuses to address the particular issues and instead attempts to contextualize their disagreement.\footnote{The \emph{conloquium} itself is between \emph{Bellum~Gallicum} 1.43--46.} Caesar again emphasizes the reciprocal relationship that \emph{amicitia} demands (\emph{BG}~1.43.4--9). He \textquote{tells} and \textquote{reminds} Ariovistus of the benefits given by the Roman state: \textquote[\emph{BG}~1.43.4]{\textlatin{commemoravit}}, \textquote[\emph{BG}~1.43.4]{\textlatin{docebat}}, and \textquote[\emph{BG}~1.43.6]{\textlatin{docebat}}. Cataloging the good deeds done towards Ariovistus by him, Caesar mentions \textquote[\textquote{his own good deeds and those of the senate toward him} \emph{BG}~1.43.4]{\textlatin{sua senatusque in eum beneficia}}, \textquote[\textquote{greatest number of gifts sent} \emph{BG}~1.43.4]{\textlatin{munera amplissime missa}}, \textquote[\textquote{although he had neither approach [to the senate] nor a just cause for demanding, he [Ariovistus] acquired gifts from his own generosity and that of the senate} \emph{BG}~1.43.5]{\textlatin{illum, cum neque aditum neque causam postulandi iustam haberet, beneficio ac liberalitate sua ac senatus ea praemia consecutum}}. Fitting with the theme of reciprocity, Caesar speaks as before about \emph{amicitia}, both that between Rome and Ariovistus, and between Rome and the Haedui. Of Ariovistus's personal relationship with the city, he speaks of \textquote[\textquote{how he had been called king and \emph{amicus} by the senate} \emph{BG}~1.43.4]{\textlatin{quod rex appellatus esset a senatu, quod \emph{amicus}}}. Of his tribe's collective relationship to Rome, there are also established bonds of \emph{amicitia}: \textquote[\textquote{they had sought earlier our \emph{amicitia}} \emph{BG}~1.43.7]{\textlatin{prius etiam quam nostram amicitiam adpetissent}}. For Caesar, friendship requires equal reciprocity, seen elsewhere in his attempts at geographical equality of the meeting.\footnote{On \emph{amicitia} and the reciprocity it obliges: \textquote[\textquote{he wished for this custom of the Roman people, that allies and \emph{amici} not only do nothing to harm one another, but to aggrandize with thanks, status and honor. Who would be able to allow what they had brought to the \emph{amicitia} of the Roman people to be taken from them?} \emph{BG}~1.43.8]{\textlatin{populi Romani hanc esse consuetudinem, ut socios atque amicos non modo sui nihil deperdere, sed gratia, dignitate, honore auctiores velit esse; quod vero ad amicitiam populi Romani attulissent, id iis eripi quis pati posset?}}.} Third, Caesar does not order Ariovistus to do anything, but \textquote[though forcefully; \textquote{\textlatin{postulavit}} \emph{BG}~1.43.9]{asks}. Fourth, Caesar situates their conflict in a larger context, a complex network of responsibilities to both the Gauls and to the Germans (\emph{BG}~1.43.6--9). Some middle ground, argues Caesar, must be found because he cannot allow anything to happen, because of the \textquote[\textquote{custom of the Roman people}]{\textlatin{p.~R.~\ldots{} consuetudinem}} to not allow \emph{amici} to be harmed (\emph{BG}~1.43.8). Caesar's speech highlights an expressed dialogic communicativity, reluctance to use force, reciprocity, and acknowledgment of complex diplomatic obligations.

Ariovistus does not respond to Caesar in the terms addressed nor offers any middle ground or concessions to Caesar's conditions. Instead, Ariovistus \textquote[\textquote{responds with very few words to Caesar's request}]{\textlatin{ad postulata Caesaris pauca respondit}}, but, according to the narrator, says \textquote[\textquote{many things about his own courageous deeds} \emph{BG}~1.44.1]{\textlatin{de suis virtutibus multa}}. The narrator thus implies, by drawing this contrast between their words, that Ariovistus is more concerned with his own high esteem. He repeats what he said earlier about the absolute power of the \textquote[\emph{BG}~1.44.2; earlier at \emph{BG}~1.36.1]{\textlatin{ius belli}}. Ariovistus also reworks Caesar's concept of friendship. He emphasizes benefits accrued through \emph{amicitia}, not on the reciprocity that Caesar extols: \textquote[\textquote{the \emph{amicitia} of the Roman people ought to be an ornament and help, not a detriment; he had sought it for this hope} \emph{BG}~1.44.5]{\textlatin{amicitiam populi Romani sibi ornamento et praesidio, non detrimento esse oportere, idque se hac spe petisse}}. The result of friendship, he says, is prestige and assistance, and should not be a hindrance. Thus, \emph{amicitia} is a relationship to be enjoyed as long as it confers benefits upon oneself.\footnote{A reasonable opinion with support by historians, for example Peter Brunt: \textquote[{\cite[355]{brunt1988}}]{It is beyond question that \emph{amicitia}, for whatever reasons the relationship was formed, was not a relationship either of mere affection or of mere reciprocal interest; if it was more than an empty name, it bound the friends together in bonds of obligation and honour. It was supposed to be founded on fdes, or, as Cicero says in the \emph{Laelius} (92, 97) and quite casually in a forensic speech (\emph{Quinct.}~6), on \emph{ueritas}. The author of the rhetorical work \emph{ad Herennium} (iv. 19) ranks betrayal of friends with breach of an oath or violence to parents among acts of moral obliquity}. For more bibliography, see footnote~\ref{ft-amicita} (p.~\pageref{ft-amicita}).} Ariovistus continues: \textquote[\textquote{\ldots{} he would reject the \emph{amicitia} of the Roman people as readily as he had sought it} \emph{BG}~1.44.5]{\textlatin{\ldots{} non minus se libenter recusaturum populi Romani amicitiam quam adpetierit}}. For not according to this vision of \emph{amicitia}, he calls Caesar's friendship false (\textquote{\textlatin{simulata \ldots{} amicitia}}) and even says that he could gain that of Caesar's enemies at Rome by killing him (\emph{BG}~1.44.10 and 1.44.12). For Ariovistus \emph{amicitia} does not suppose any compromise (only benefit).

Caesar's reply underscores a contrast with Ariovistus, in direct response to what Ariovistus says, \textquote[\textquote{to this opinion} \emph{BG}~1.45.1]{\textlatin{in eam sententiam}}. Though Ariovistus changes the terms of debate as first offered, Caesar shows his commitment to peaceful reconciliation by nevertheless addressing those concerns. Caesar explains again that, according to \textquote[\textquote{\textlatin{consuetudinem}} similarly at \emph{BG}~1.43.8]{custom}, the Romans cannot abandon their Gallic allies \emph{amici} (\emph{BG}~1.45.1). Unlike his communications with the Gauls in the \emph{concilium} and soldiers at the \emph{consilium}, he cannot find any mutually acceptable terms of agreement, any \textquote[][.]{middle ground} Without common terms, such as differing conceptions of \emph{amicitia}, and willing interlocutors, it seems that Caesar's communicative intervention cannot occur.

This is not to say that Ariovistus's complaints are not valid. According to \textcite[246--247]{meier1995} and \textcite[184--187]{riggsby2006}, Ariovistus successfully counters his arguments. He does so, however, by expanding topics of conversation, while Caesar's arguments are \textquote[{\cite[184]{riggsby2006}}]{fairly simple} and \textquote[p.~187]{strikingly unresponsive} to the broad claims. Riggsby holds that Ariovistus's arguments, while persuasive to some readers today, were not to Caesar's readers.\footnote{\textquote[{\cite[184]{riggsby2006}}]{In this light, it is worth noting that all of Caesar’s (and Cicero’s) \textquote{arguments} consist of lists of Roman motivations for war. They do not allow for the possibility of contradictory accounts or countervailing factors}.} But instead of Riggsby's suggestion that this encounter highlights a blind spot in Julius Caesar's ideology, I think that the encounter needs to be investigated a bit more, since here as elsewhere, the \emph{Bellum~Gallicum} sends mixed messages about Caesar's use of language in the settling of disagreements. In the staging of this conflict, both actors perhaps make persuasive arguments, though their sincerity is thrown into suspicion: Caesar for switching out Gallic cavalry for Roman troops, and Ariovistus for how the meeting dissolves. The actor Caesar has perhaps found an equal in an ability to say one thing and plan another.

Before another reply by Ariovistus, the meeting is interrupted when Ariovistus's cavalry begins assaulting the Roman guard. The conclusion of the \emph{conloquium} is revealing of another aspect of the leaders' respective communicative habits. The German cavalry present at the meeting begin harassing the Romans. This occurs without Ariovistus's direct knowledge, though readers may suspect otherwise. \blockquote[\emph{BG}~1.46.1]{\textlatin{Dum haec in conloquio geruntur, Caesari nuntiatum est equites Ariovisti propius tumulum accedere et ad nostros adequitare, lapides telaque in nostros conicere}}. \blockquote[\emph{BG}~1.46.1]{While this was happening in the \emph{conloquium}, it was announced to Caesar that the cavalry of Ariovistus was coming near the mound and approaching our men, throwing stones and javelins at them.} A reader may suspect that this assault is planned by Ariovistus, and this would confirm Caesar's earlier suspicions about his good faith at \emph{Bellum~Gallicum} 1.42. Caesar immediately ends the meeting \textquote[\textquote{and brought himself to his men and ordered them to throw not even one spear against the enemy} \emph{BG}~1.46.2]{\textlatin{seque ad suos recepit suisque imperavit ne quod omnino telum in hostes reicerent}}. The simplicity of the order to his men (\textquote{\textlatin{suis imperavit}}) results in perfect obedience. The narrator offers Caesar's motivation for refusing this skirmish, that the enemy was not \textquote[\textquote{surrounded by him in the \emph{conloquium} when under terms} \emph{BG}~1.46.3]{\textlatin{eos ab se per fidem in conloquio circumventos}}. That is, the narrator says that Caesar did not want there to be any reason for one to say that he did not engage in the \emph{conloquium} in good faith (another statement of the narrator's expression of the actor Caesar's shrewdness).

After this event, Ariovistus attempts another \emph{conloquium}.\footnote{\textquote[\textquote{Ariovistus sends legates to Caesar} \emph{BG}~1.47.1]{\textlatin{Ariovistus ad Caesarem legatos mittit}}.} He proposes to Caesar that this be done face to face or through legates: \textquote[\textquote{that he [Caesar] decide another day for a \emph{conloquium} or, if he wished, he send to him someone from among his legates} \emph{BG}~1.47.1]{\textlatin{uti aut iterum conloquio diem constitueret aut, si id minus vellet, ex suis legatis aliquem ad se mitteret}}. For reasons not articulated (but presumably due to the multiple failures of the \emph{conloquium}), the actor Caesar sees further conversation (\textquote{\textlatin{conloquendi}}) as pointless.\footnote{\textquote[\textquote{there did not seem to Caesar reason for talking together} \emph{BG}~1.47.2]{\textlatin{conloquendi Caesari causa visa non est}}.} Thinking it too dangerous to send one of his own legates, he does send two in his party \textquote[\textquote{on account of their loyalty and knowledge of the Gallic tongue} \emph{BG}~1.47.4]{\textlatin{et propter fidem et propter linguae Gallicae scientiam}}. This request for communication by Ariovistus, appears to again have been a pretense, since when the Gauls arrive to the German camp, he claims that they are spies and imprisons them (\emph{BG}~1.47.6). Caesar's shrewdness is revealed in his decision not to send one of his own officers (\emph{BG}~1.47.3).

With the foil of Ariovistus, the \emph{Bellum~Gallicum} again illustrates a double explanation of Caesar's methods as a leader, one being characterized by an open exchange of information, the other (reflected in his actions) being far more shrewd in his motives and expectations of others'. The communicative presence which the actor Caesar extends here fails to achieve what it does earlier in the Ariovistus episode.

\section{A communicative pattern}
\label{comm-patt}
The Ariovistus episode explained in detail in this chapter follows a narrative pattern that is dominant in the first half of the \emph{Bellum~Gallicum} and, in the latter half, prominent.\footnote{For overviews of the scope and episodes of the \emph{Bellum~Gallicum}, see \textcite[49--56]{rambaud1966} for a particular take on Caesar's composition of material in the \emph{Commentarii}. He identifies three sources for Caesar's actual composition and identifies their place in the narratives; the three are reports from legates, his own letters to the Senate and others, and literary developments. See also \textcite[25--49]{adcock1956} for thematically minded summaries of the books of the \emph{Bellum~Gallicum}.} The pattern goes something like this: all is peaceful in Gaul, until Caesar receives reports of trouble somewhere in or around Gaul, decides that his intervention is necessary, approaches barbarian leadership to come to a language--based settlement to the perceived problem, and the barbarian refuses to cooperate with or cede to Caesar, who finally settles dispute with the physical force of the Roman army. Gaul is once again peaceful. Caesar is victorious and achieves the acquiescence that he initially sought. In book 1, this pattern appears with Caesar engaging with the Helvetii, whom he refuses to allow to migrate through Gallia Transalpina (\emph{BG}~1.2--30); against the Germans for the purposes of defending Gallic allies and alleged protection of Italy (\emph{BG}~1.31--54). In book 2, Caesar engages in a series of actions against various tribes of the Belgae, who, Caesar hears \textquote[\textquote{\textlatin{contra p.~R.~coniurare}} \emph{BG}~2.1.1]{are plotting against the Roman people}. Caesar makes peace with some Belgae (the Remi) early on (\emph{BG}~2.3--5) and then slowly defeats the tribes and comes to a peace settlement with them (e.g., the Nervii, who surrender at \emph{BG}~2.28). As in these alliances and peace treaties, a certain drive to dialogism characterizes most of Caesar's interactions with barbarians in this book. For example, even with the defeated Aduatuci, in conversation with Caesar, make a request (\textquote{\textlatin{unum petere ac deprecari}} \emph{BG}~2.31.4) that Caesar grants in a spoken response (\emph{BG}~2.32). The book ends with Caesar's accomplishment of a pacified Gaul: \textquote[\textquote{once all Gaul had been pacified} \emph{BG}~2.35.1]{\textlatin{omni Gallia pacata}}.

The narrative pattern changes slightly in parts of books 3 and 4, for in these barbarian enemies often do some deed that reveals them to be so hostile that Caesar cannot communicate with them. Book 3, which documents the Romans' destruction of the Veneti, begins like the others, with a state of seeming peace: \textquote[\textquote{Caesar thought Gaul was pacified} \emph{BG}~3.7.1]{\textlatin{Caesar pacatam Galliam existimaret}}. The narrative pattern takes a different course when the Veneti kidnap Roman \emph{legati} (\emph{BG}~3.8.2), such a serious offense -- in the ancient world and in the \emph{Bellum~Gallicum} -- that Caesar marches upon them without any attempt at dialog. After their defeat, many other tribes surrender according to Caesar's demands (\emph{BG}~3.27).\footnote{On the Veneti's imprisonment of \emph{legati} at 3.8--9 in the \emph{Bellum~Gallicum}, see \textcite[120]{riggsby2006}. See narrator's comments at 3.9.3: \textquote[\textquote{legates, a name which had always been sacred and inviolable among all nations}]{\textlatin{legatos, quod nomen apud omnes nationes sanctum inviolatumque semper fuisset}}.} Book 4 contains two episodes about Caesar's interaction with two small German tribes (the Usipetes and Tenctheri \emph{BG}~4.1--19) and Britanni (\emph{BG}~4.20--37). The Germans, much like the Helvetii in book 1, threaten Gallic stability and thus he must intervene (\emph{BG}~4.5.1). While in difficult negotiations that bring about a temporary truce (\emph{BG}~4.7--11), they nevertheless attack the Romans (\emph{BG}~4.12), after which Caesar decides that further communication is futile: \textquote[\textquote{Caesar now thinks that neither \emph{legati} ought to be heard nor terms accepted from them} \emph{BG}~4.13.1]{\textlatin{Caesar neque iam legatos audiendos neque condiciones accipiendas arbitratur ab iis}}. Caesar makes a similar abrupt refusal to negotiate with Ariovistus at \emph{Bellum~Gallicum} 1.47 (see section~\ref{bg-conloquium}). Prior to sailing to Britannia, Caesar receives word from several tribes who say that they will obey Roman rule (\emph{BG}~4.21.5), though they back down once Caesar arrives.\footnote{See his reprimand, in negotiations, of them at \emph{BG}~4.27.5.} Terms of peace with these Britanni are nevertheless agreed upon (\emph{BG}~4.28.1), though they again turn on the Romans (\emph{BG}~4.30--35) and are compelled to treat for peace (\emph{BG}~4.36.1). On the whole, the first four books of the \emph{Bellum~Gallicum} are comprised of smaller episodes, each of which corresponds in part or almost entirely to the pattern of conflict--negotiation--battle. As seen in books 3 and 4, Caesar's desire to communicate with outsiders decreases when dealing with unfaithful or conspiring barbarians.

Books 5 through 7 continue the trend of Caesar's decreased communication correlated to increased barbarian hostility. The privileging of communication by Caesar does not disappear, but, beginning in book 5, various Gallic states are actively conspiring against Caesar, leaving apparently no room for pre--battle negotiations. Unique among these last three books, the British opponents are not explained as having actively harmed the Romans. The first half of book 5 covers several small conflicts and issues with which Caesar deals (\emph{BG}~5.1--7), then his second campaign in Britain (\emph{BG}~5.8--23). In this expedition, the text does not exhibit Caesar engaging in dialog with his opponents (though there is some, e.g., with the Trinovantes at \emph{BG}~5.20). The second half of book 5 (\emph{BG}~5.24--58) mostly explains events for which Caesar was not present. While in previous episodes, a tremendous act of deceit precipitates Caesar's ending of negotiations (e.g., Ariovistus at \emph{BG}~1.47 and the Veneti in book~3), Cassivellaunus and his people are not offered a chance to negotiate before battle. The \emph{Bellum~Gallicum} portrays events for which Caesar is absent very differently than those resembling the narrative pattern identified here.\footnote{Events for which subordinate officers are responsible explored in Chapter~\ref{ch-delib}.}

In books 6 and 7, the revolt of Gaul continues, and with it some alteration of the narrative pattern. Early on in book 6, Caesar communicates with Gallic allies (e.g., \emph{consilia} of all Gallic allies at \emph{BG}~6.3.4--4.6 and 6.44.1--2), and resembles his communicative activities early on (e.g., the \emph{concilium} of \emph{BG}~1.33--33). Other elements of the pattern hold, as well. After the \emph{consilium} and once pledges of allegiance secured, the narrator explains that peace (though now only partial) was achieved: \textquote[\textquote{after this part of Gaul was pacified} \emph{BG}~6.5.1]{\textlatin{hac parte Galliae pacata}}. He then pursues Ambiorix, who has proved himself an enemy of Rome in book 5 (\emph{BG}~5--6, 9--10, 29--43). Book 7, which chronicles the great uprising of Gauls against Rome, and the Rome's siege of Avaricum, battle of Gergovia, and battle of Alesia. In this last book, there is little communication between because Caesar and his enemies because they are so clearly their enemies. The narrative pattern of the Ariovistus episode is the standard pattern for the \emph{Bellum~Gallicum} dominant in books 1 and 2, prominent in 3 and 4, and partially applicable for books 5--7. The change in the relative prominence of the pattern can be explained as a difference in the characterization of certain antagonists, those who in their outright insurrections prove themselves to be uncompromising enemies to the Roman people. When able, Caesar still engages in acts which portray the actor Caesar as forging communicative bonds with foreigners.\footnote{Reasons for this change in representation may also well have some origin in the oft--noted development of style in the \emph{Commentarii} from something more resembling the \emph{commentarius} genre to historiography. See, for example, \textcite[142]{riggsby2006} on the increase of direct speeches, which begin in book 4.}

\section{Conclusion}

Caesar's contradictory representation of communicative action allows for a two--part argument for himself as a shrewd general and a credible man of peace. As an actor, Caesar's communicative acts are on the conciliatory end of the spectrum. He may be understood as a type of ideal communicator, even with sly barbarians like Ariovistus, to find \textquote{middle ground} and achieve, in Habermas's language \textquote[{\cite[314]{habermas1987a}}]{intersubjective recognition}. This communicative presence functions within the Roman army too, where Caesar's psychological interventions, such as those seen in the \emph{consilium} and in \textcite[patrolling soldiers' \emph{animus} and \emph{virtus}]{lendon1999}, play an important role in keeping other agents in a state in which they may obey.\footnote{Another take on this is \textcite[330--338]{osgood2009}, on the tactical advantage writing gave Caesar over the Gauls.} This extends observations on the limit of Caesar's knowledge and activity by \textcite{kagan2006}, who notices that in battle narratives, \textquote[p.~151]{Caesar does not report himself in the right place at the right time, nor does he claim to control events that he does not}.\footnote{Kagan continues:  Caesar (the author) writes of himself (the character) as imperfect. \blockquote[{\cite[151]{kagan2006}}]{Caesar reports his limitations -- in visibility as in action -- as many times he does his influence. Caesar's appearance in battle narrative at an appropriate place and time, illustrating \textquote{Roman} qualities of providence, valor, or bravery, does not constitute his \textquote{instinctive genius}}.} Kagan's arguments (pp.~99--180, esp.~99--115) about Caesar's imperfect knowledge during battles are convincing, though they do not account for how the text represents Caesar's communicative action as fundamental to his \textquote{eye of command.} This narrator--actor explanation distinguishes the actor Caesar's speech acts (geared to mutual understanding and nonviolent reconciliation) and his interior thoughts (expressed by the narrator) offer a doubled portrait of Julius Caesar as a humane leader, yet one persistently in control of events under his supervision.

% The narrative displays the general's inadequacy of sight and action. Despite a naivete concerning the relationship of \textquote[{\cite[119]{kagan2006}}]{real circumstances} to narrative prose, Kagan's observation -- that the actor Caesar is not omniscient or omnipotent within the \emph{Bellum~Gallicum} -- is useful. The psychological interventions, achieved by means of direct communications with individuals, allow Caesar a mechanism to sustain a paradoxical representation of himself as a lover of peace and of war.

% Information comes to Caesar through a steady stream of information, some spoken\footnote{Oral communication mentioned in this chapter include: \textquote{\textlatin{nuntiatum est ei}} \emph{BG}~1.38.1; \textquote{\textlatin{Caesari nuntiabant}} \emph{BG}~1.39.7; and \textquote{\textlatin{Caesari nuntiatum est}} \emph{BG}~1.46.1.} some written.\footnote{On the alteration of the written word to the limitations of physical location, see \textcite{osgood2009}.} Typical notions of presence and absence fall away when communicative networks bind together disparate actors.


\chapter{Internal deliberation}%2222
\label{ch-delib}
Chapter~\ref{ch-pres} introduced the notion of intersubjective communication, suggesting that Caesar's communications with other rational agents express his leadership style, whereby individuals may come to mutual understanding and agreement. While some of those contexts (e.g., a \emph{consilium} or \emph{conloquium}) appear to facilitate communication between speech actors, I argued that this representation of communication demonstrates Caesar's disingenuous approach to intersubjective communication. The resultant view is a portrait of Caesar's mastery of others through language.

This chapter looks at Caesar's thought process, another facet of the \emph{Bellum~Gallicum}'s expression of his mastery of his context, army, enemies, and even his own mind. Caesar's consideration of two sides of an argument is unique in the \emph{Bellum~Gallicum}. For unlike the Gallic leaders who deliberate together,\footnote{E.g., \textquote[\emph{BG}~7.15.3]{\textlatin{in communi concilio}} and the Germans, too, at 4.9.1.} Caesar performs this deliberative action with only himself. Furthermore, as I explain in the next chapter, deliberations by subordinate Roman officers are noteworthy for their collaborative nature.\footnote{For these group deliberations, see page~\pageref{del-list}.} And while Greek and Roman rhetoric does not adress internal deliberation,\footnote{On the social and civic nature of ancient rhetoric, see \textcite[pp.~20--21, 45--46, and 81--85]{habinek2005} and \textcite[pp.~70--74 and 145--146]{connolly2007}. \textcite[81--85]{habinek2005} highlights one partial exception in Isocrates's \emph{Nicocles} 5--9, which offers some possibility of non--collaborative deliberation for certain individuals. See also the essential role of \textquote[\textquote{\textgreek{οἱ ἀκρoaταί}} \emph{Rhet.}~1.3.1, 1358a37]{listeners} in the determination of speech type.} the character Caesar nevertheless frequently takes up the role of what might be called a \textquote{deliberating hero,} a protagonist whose thought process is explained to a text's audience. By considering several previous deliberating heroes, I discuss Caesar's internal deliberations in the \emph{Bellum~Gallicum} and consider how they work hand in hand with his authorial goal to justify his controversial proconsulship in Gaul.

I argue that Caesar may be understood as unifying two roles, of deliberating commander and deliberating narrator, a strategy that functions to express Caesar's power in the narratives of the \emph{Bellum~Gallicum} as a function of his brilliant rationality. By way of introducing the ideas behind this dual--role, I first consider two descriptions of Odysseus by Polybius, who sees the deliberating hero as an ideal commander and ideal historian (\ref{comm}). Next, I look at several examples of deliberating leaders from Greco--Roman literature (Homer and Polybius, sections~\ref{hom-comm} and~\ref{polyb-comm}). I then turn to explore in detail the \textquote{Odyssean} aspects of Caesar and how these have been informed by the tradition of the deliberating hero (\ref{ch2-delib-caesar}). As implicit narrator Caesar acts like a forensic investigator of history, and as military commander he is a highly rational interpreter and planner of events. Both roles, I will show, contribute directly to the apologetic goals of the \emph{Bellum~Gallicum}. In essence, in thinking about what to do, much like heroes of earlier literature, the commander's deliberations significantly contribute to how the \emph{Bellum~Gallicum} makes meaning. Finally, in this chapter's conclusion (\ref{dec-conc}), I tie these ideas together and consider several avenues of thought -- parallel concerns, in rhetorical texts, about political reconciliation through debate and about mental self--discipline, and how Caesar's deliberations lend his narratives authority.

As with the previous chapter, and in fact the third too, this chapter explains an under--recognized way that Caesar crafts himself as all--powerful figure. These moments of deliberation are the first and central step in his leadership of the Roman army. His mastery of the army is beyond just an exposition of his mind but his mind's process, his intelligence, which arises in the moments he vacillates between two courses of action. In the words of \textcite{rambaud1966}, the \emph{Commentarii} are a \textquote{tableau of Caesar's mind.}\footnote{\textquote[{\cite[250]{rambaud1966}}]{\textfrench{Pour la postérité, le grand homme a posé comme une intelligence qui dominait l'événement, et les \emph{Commentaires} sont le tableau de son esprit, inscrit dans les faits}}.}

\section{Odyssean deliberators}
\label{comm}
I will eventually show that Caesar's representations of himself as an author and military general are informed by a tradition of deliberating heroes in earlier literature. Here, I take a close look at the status and qualities of deliberating by individuals in two genres, epic and history. For representatives of the tradition of the deliberating hero, I consider several leaders from Homeric epic (Achilles, Zeus, and Odysseus, section~\ref{hom-comm}) and Polybius's \emph{Histories} (Scipio Africanus and Hannibal, section~\ref{polyb-comm}). In addition to being Odyssean commanders according to Polybius's conception, Odysseus and Scipio share some specific traits to the deliberating Caesar of the \emph{Bellum~Gallicum}, who will be looked at below. For Homer, and Odysseus in the \emph{Odyssey} in particular, the deliberations show a balanced and proof--like consideration of multiple options; Odysseus himself, in moments of deliberation, shows foresight (physically into the distance, and metaphorically into the future) and rationally makes plans about what to do. Polybius's Scipio and Hannibal also look to the future in these ways (rational weighing of options, foresight, planning), with the one distinguishing factor being that his ideal deliberating figures lack emotion (namely fear) in this process. For both Homer's \emph{Odyssey} and Polybius's \emph{Histories}, I consider the importance of these deliberating figures to the larger thematic interests of each work. In short, Odysseus in his deliberations takes \textquote[{\cite[79]{stanford1963}}]{the long view}, whereby he uses rational self--reflection to negotiate interpersonal relationships in the \emph{Iliad} and journey home to suppress the suitors in the \emph{Odyssey}. For Polybius, the deliberating commander plays into Polybius's larger project of elevating the author to the distinguished position of military commander. The similarities and divergences of Julius Caesar's representation of himself as a deliberator will be considered below, in section~\ref{ch2-delib-caesar}. 

At several points in the \emph{Histories}, Polybius explains the ideal commander in reference to Homer's Odysseus. The Odyssean commander is a thinking hero and the Odyssean author is, like this hero, ever--suffering and much experienced. When discussing the ideal commander, Polybius invokes Odysseus during his lengthy discussion of generalship in book~9. In book~9's digression on ideal commanders, Polybius describes how Homer portrays Odysseus, who is here construed as an ideal commander. \blockquote[\emph{Hist.}~9.16.1--3]{\textgreek{(1) ᾗ καὶ τὸν ποιητὴν ἄν τις ἐπαινέσειε, διότι παρεισάγει τὸν Ὀδυσσέα, τὸν ἡγεμονικώτατον ἄνδρα, τεκμαιρόμενον ἐκ τῶν ἄστρων οὐ μόνον τὰ κατὰ τοὺς πλοῦς, ἀλλὰ καὶ τὰ περὶ τὰς ἐν τῇ γῇ πράξεις. (2) ἱκανὰ γὰρ καὶ τὰ παρὰ δόξαν γινόμενα (μὴ δυνάμενα) τυγχάνειν προνοίας ἀκριβοῦς εἰς τὸ πολλὴν ἀπορίαν παρασκευάζειν καὶ πολλάκις, (3) οἷον ὄμβρων καὶ ποταμῶν ἐπιφοραὶ καὶ πάγων ὑπερβολαὶ καὶ χιόνες, ἔτι δ' ὁ καπνώδης καὶ συννεφὴς ἀὴρ καὶ τἄλλα τὰ παραπλήσια τούτοις.}} \blockquote[\emph{Hist.}~9.16.1--3]{(1) Wherefore one would praise also the poet, since he portrays Odysseus as a man most capable of command, judging from the stars not only actions at sea but on land too. (2) For it is impossible for sharp foresight (preparing against great difficulty) to light upon unexpected events, sufficiently great and frequent, (3) such as sudden bursts of rain and rivers, excesses of frost, snows, even dark and cloudy air, and things nearly equal to those.\footnote{The passage continues: \textquote[\textquote{(4) If it is possible to foresee even these, and we neglect them, how is it not fitting that we fail in most ventures events by ourselves? (5) Wherefore nobody should disregard the above mentioned, in order that we not fall foul of these errors, of what sort they say many others have fallen upon, and those who now are about to be spoken by me as examples} 9.16.4--5]{\textgreek{(4) εἰ δὲ καὶ περὶ ὧν δυνατόν ἐστι προϊδέσθαι, καὶ τούτων ὀλιγωρήσομεν, πῶς οὐκ εἰκότως ἐν τοῖς πλείστοις ἀποτευξόμεθα δι' αὑτούς; (5) διόπερ οὐκ ἀφροντιστητέον οὐδενὸς τῶν προειρημένων, ἵνα μὴ τοιούτοις ἀλογήμασι περιπίπτωμεν οἵοις φασὶ περιπεσεῖν ἑτέρους τε πλείους καὶ τοὺς νῦν ὑφ' ἡμῶν λέγεσθαι μέλλοντας ὑποδείγματος χάριν}}. Polybius text from \textcite{buttner1889}.}} Though not always sufficient, Polybius here says that a leader must use powers of foresight (\textquote{\textgreek{προνοίας ἀκριβοῦς}}) in order to be successful. While navigating by sea, Odysseus literally and metaphorically \textquote[\textquote{\textgreek{προϊδέσθαι}}]{looks forward} for physical weather storms while sailing and into the future in devising clever plans. The commander, says Polybius, must prepare (\textquote{\textgreek{εἰς τὸ πολλὴν ἀπορίαν παρασκευάζειν}}) against contingency (\textquote{\textgreek{τὰ παρὰ δόξαν γινόμενα}}). As an ideal commander (\textquote{\textgreek{ἡγεμονικώτατον}}), Odysseus showed shrewdness in preparation for and reaction to the unexpected. On this passage, Walbank cites Odysseus as \textquote[{\cite[142]{walbank1967}}]{the embodiment of energy and strategical skill}.\footnote{As examples of Odysseus's later reputation for craftiness, \textcite[142]{walbank1967} cites \emph{Histories} 12.27.10 and Horace 1.2.17--18. On Homer in particular, he cites \emph{Iliad} 10.251--3 and \emph{Odyssey} 5.270--275 as examples of navigation by stars, at land and under sail, respectively; and \textcite[118--127]{stanford1968} generally.} Because Polybius writes of Odysseus as a real historical (and not a fictional or mythological) figure, his relevance in the discussion of the ideal general is like that of more contemporary generals, three of whose failures of foresight Polybius turns to next (9.17.1--9.19).\footnote{The most important surveys of Polybean historiography are \textcite{pedech1964}, \textcite{walbank1972}, and \textcite{sacks1981}, plus the commentaries of \textcite{walbank1957,walbank1967,walbank1979}. For more bibliography, see footnote~\ref{ft-polyb-biblio}. \textcite[179--180]{walbank2002} writes of how mythological figures are written of by Polybius, especially Odysseus, as when crossing the Pillars of Hercules: \textquote[180]{Following Eratosthenes, Polybius interprets Odysseus’ voyages along these lines. Aeolus, who gave Odysseus the bag of the winds, was in reality a man who instructed him about sailing around the Straits of Messina; only legend had turned him into a king and \textquote{guardian of the winds}}.}

Odysseus is not only an exemplary commander, according to Polybius, but he also possesses the characteristics of an ideal author of history. In writing about the prime importance of an historian to have been present at events and well--versed in the pursuit he is writing about, Polybius contrasts investigative historians with armchair historians, who \textquote[\textquote{\textgreek{κατακείμενον}}]{lying down} research from books and \textquote[\textquote{\textgreek{πολυπραγμονεῖσθαι χωρὶς κινδύνου καὶ κακοπαθείας,}} \emph{Hist.}~12.27.4]{busy about without danger and suffering}. These men get out into the field and suffer, since \textquote[\textquote{\textgreek{ἡ δὲ πολυπραγμοσύνη πολλῆς μὲν προσδεῖται ταλαιπωρίας καὶ δαπάνης}} 12.27.6]{research requires great hardship and expenditure}. Polybius joins this toiling historian with the original suffering Greek hero. In discussing how to write history Odysseus appears again. \blockquote[\emph{Hist.}~12.27.10--11]{\textgreek{(10) ἔτι δὲ τούτων ἐμφαντικώτερον ὁ ποιητὴς εἴρηκε περὶ τούτου τοῦ μέρους. ἐκεῖνος γὰρ βουλόμενος ὑποδεικνύειν ἡμῖν οἷον δεῖ τὸν ἄνδρα τὸν πραγματικὸν εἶναι, προθέμενος τὸ τοῦ Ὀδυσσέως πρόσωπον λέγει πως οὕτως· 
\begin{verse} 
ἄνδρα μοι ἔννεπε, Μοῦσα, πολύτροπον, ὃς μάλα \\ 
πολλὰ πλάγχθη, 
\end{verse} 
(11) καὶ προβάς, 
\begin{verse} 
πολλῶν δ' ἀνθρώπων ἴδεν ἄστεα καὶ νόον ἔγνω, \\ 
πολλὰ δ' ὅγ' ἐν πόντῳ πάθεν ἄλγεα ὃν κατὰ θυμόν, 
\end{verse} 
καὶ ἔτι 
\begin{verse} 
ἀνδρῶν τε πτολέμους ἀλεγεινά τε κύματα πείρων.
\end{verse}}} \blockquote[\emph{Hist.}~12.27.10--11]{(10) The poet has spoken even more emphatically about this matter. For wishing to reveal to us of what sort it is necessary for the man of action to be, setting forth the character of Odysseus, he speaks thus:
\begin{verse} 
Sing to me the many--turned man, Muse, who \\
suffered many ills
\end{verse} 
(11) and further on, 
\begin{verse} 
He saw cities and knew the mind of many men,\\
and suffered many pains in his heart at sea,
\end{verse} 
and also 
\begin{verse} 
suffering wars and painful waves.
\end{verse}} 
Polybius's  \textquote{poly--} vocabulary for historical research (\textquote{\textgreek{πολυπραγμονεῖσθαι,}} \textquote{\textgreek{πολυπραγμοσύνη}}) seems pulled straight from the invocation of the \emph{Odyssey} (\textquote{\textgreek{πολύτροπον,}} \textquote{\textgreek{πολλὰ,}} \textquote{\textgreek{πολλῶν,}} \textquote{\textgreek{πολλὰ}}). He picks up on the travels of Odysseus as analogous to the first-person investigations of the true historian (\textquote{\textgreek{ἴδεν ἄστεα καὶ νόον ἔγνω}}). Perhaps the most striking parallel is the association of suffering to, in the \emph{Histories}, inquiry (i.e., not \textquote{\textgreek{χωρὶς κινδύνου καὶ κακοπαθείας,}} \textquote{\textgreek{ταλαιπωρίας καὶ δαπάνης}}); and, in the \emph{Odyssey}, adventure (\textquote{\textgreek{πολλὰ πλάγχθη,}} \textquote{\textgreek{πολλὰ \ldots{} πάθεν ἄλγεα,}} \textquote{\textgreek{ἀλεγεινά τε κύματα πείρων}}). Some contrast may be drawn between Polybius's rationalizing tendencies and what Walbank calls \textquote[{\cite[44]{walbank2002}}]{a glimpse of a romantic, who imagines himself in the rôle of a second Odysseus}. \textcite{luce1997} construes Polybius as less of a romantic by tying his biography to these views.\footnote{See \textcite[6--13]{walbank2002} for an overview of Polybius's methodology. \textquote[{\cite[92]{luce1997}}]{Polybius himself had not only been involved in politics and military life, but was famous for his extensive travels. An honorary inscription set up in his native Megalopolis described him as one who had \textquote{wandered over every land and sea} (Pausanias 8.30), the fruits of which can be seen in Book 34, the whole of which was an excursus on the geography of Europe and Africa}.}

Before turning to the \emph{Bellum~Gallicum} and my argument that in Caesar there is a pairing of both ideals (commander and author) in one person, I want to investigate what Polybius has in mind when talking about the two ideal types in Homer's \emph{Iliad} and \emph{Odyssey} (section~\ref{hom-comm}) and within his own \emph{Histories} (\ref{polyb-comm}). With a conception of something of this tradition of deliberating heroes/commanders and investigative historians, I may then turn to how exactly the two roles unite in the \emph{Bellum~Gallicum} and what significance this unification has to the text as a whole.

\subsection{Homer}
\label{hom-comm}
Be they in assemblies (i.e, the Greeks' \textgreek{βουλαί} in the \emph{Iliad}), small groups (i.e., the embassy to Achilles), or in private (as Odysseus to himself above at \emph{Odyssey}~9.299--318 and 420--424), deliberation is generally important to the Homeric epics. \textcite{edwards1992}, for instance, divides Homeric narrative into five categories, one of which is deliberation.\footnote{The other four types are battle, social intercourse, travel, and ritual. See \textcite[316--319]{edwards1992} on deliberation and for relevant bibliography.} Privately and in the form of self--reflection, a character thinks to himself, often speaking aloud his thought process. While most scholarship on the matter of deliberation concerns the Homeric construction of self (especially Bruno Snell's arguments about the non--unified conception of personhood), I here pay special attention to deliberation in its incarnation as self--reflection.\footnote{\textcite{snell1953} is most associated with this line of thought. Within a larger conception of a disjointed self that he advances, Snell points out that Homeric heroes address address not themselves precisely, but parts of their body, such as the \emph{thumos}. \textcite[5]{dodds1951} echoes similar thoughts. On Homeric divisions of the body and how this influences deliberation, see most recently \textcite{gill1996} and \textcite[pp.~149--176, with bibliography pp.~161--162 and 175--176]{barnouw2004}. See \textcite[105]{janko1992} for German bibliography on self--reflection. For surveys of deliberation and self--reflection in scholarship, see \textcite[317--318]{edwards1992}. See \textcite{fenik1978} on deliberation in the \emph{Iliad}, and on the \emph{Odyssey}, \textcite[pp.~30, 35, 45, 53, 67, and 92]{heubeck1989} and \textcite[pp.~30, 108, 110, and 387--88]{russo1992}. Also commenting on and in response to Snell on internal deliberation are \textcite[487]{russosimon1968}, \textcite{fenik1978}, \textcite[5]{russo1982}, \textcite{jarratt1991}, \textcite[10--11]{nienkamp2001}, and \textcite[21--56]{stefanson2004}.} I will in this section look at some of the fundamental features of individuals' deliberation in the Homeric texts generally, and then turn to how Odysseus, in particular, deliberates. When we turn next to Polybius and later Caesar, I will have shown the ways that this manner of self--reflection is characteristic of heroes (and other important actors), dating back to the beginnings of Greek literature.

In the \emph{Iliad} and \emph{Odyssey}, when characters come to a dilemma, they often reflect in speech what is happening and what course of action they should take. The formal features of Homeric deliberations have been well mapped out by \textcite[11]{scully1984}, who offers the following as essential to a Homeric self--reflection: being in first-- or third--person narration, \textquote{an either/or (\textgreek{ἤ/ἦε}) outline of the perceived alternative,} and preferable result given by a comparative. Often, these deliberations are preceded by the participle \textquote[\textquote{perplexed}]{\textgreek{ὀχθήσας}}.\footnote{The examples of actors deliberating in the \emph{Iliad} are, according to \textcite[p.~12, nn. 1]{scully1984}, are 1.188-93 (Achilles); 5.671-76 (Odysseus); 8.167-70 (Diomedes); 10.503-7 (Diomedes); 13.455-59 (Deiphobus); 14.20-24 (Nestor); 16.644-55 (Zeus); 16.712-15 (Hector).} To these remarks of Scully, I add that most of the examples here communicate some sort of emotion for the deliberator, usually fear or grief. In the \emph{Iliad}, these deliberations typically arise: \blockquote[{\cite[11]{scully1984}}]{when a hero is unexpectedly caught in the forefront of battle and the narrator describes the indecision whether to stand and fight or to retreat back into the melee and safety.} While this marital context is  not directly applicable to all of the examples from the \emph{Iliad} that I look at below, we do see this basic premise, with the author choosing to underline deliberation at these moments of duress.

Deliberations in the \emph{Iliad} follow this pattern identified by Scully. Consider one of the more memorable passages from the opening of the \emph{Iliad}, when Achilles is on the verge of killing Agamemnon for taking Briseis. The narrator says that Achilles is enraged by the seizure, falls into a fit of grief, and begins to reflect upon what to do. \begin{verse}\SingleSpacing \textgreek{ Ὣς φάτο· Πηλεΐωνι δ' ἄχος γένετ', ἐν δέ οἱ ἦτορ \\ στήθεσσιν λασίοισι διάνδιχα μερμήριξεν, \\ ἢ ὅ γε φάσγανον ὀξὺ ἐρυσσάμενος παρὰ μηροῦ \\ τοὺς μὲν ἀναστήσειεν, ὃ δ' Ἀτρεΐδην ἐναρίζοι, \\ ἦε χόλον παύσειεν ἐρητύσειέ τε θυμόν. \\ ἧος ὃ ταῦθ' ὥρμαινε κατὰ φρένα καὶ κατὰ θυμόν, \\ ἕλκετο δ' ἐκ κολεοῖο μέγα ξίφος, ἦλθε δ' Ἀθήνη.} \\ (\emph{Il.}~1.188--94)\end{verse} \begin{verse} \SingleSpacing Thus he spoke, and grief came to the son of Peleus, and in his shaggy breast \\ his heart was debating in two ways, \\ whether taking his sharp sword from his side \\ he should make them rise, and slay the son of Atreus, \\ or stop his anger and restrain his heart. \\ For a while he debated these in his mind and heart, and grabbed his great sword, and Athena came. \\ (\emph{Il.}~1.188--94)\end{verse} \DoubleSpacing Evident here are typical Homeric terminology and characterization of the mind. Ach\-il\-les's mind is located in the location of the \emph{thumos} (\textquote{\textgreek{θυμόν}}; \textquote{\textgreek{στήθεσσιν λασίοισι}}), here also called \textquote{\textgreek{ἦτορ}} and \textquote{\textgreek{φρένα.}}\footnote{Though separated here, the \emph{thumos} is closely associated with the \textquote{\textgreek{στῆθος}}, as in expressions like \textquote{\textgreek{θυμὸς ἐνὶ στήθεσσι}} (\emph{Il.}~4.152, 4.289, 4.313, 4.360, 7.68, 7.216, 7.349, 7.369, 8.6, 9.8, 9.703, 13.73, 13.494, 14.39, 15.629, 15.701, 17.22, 17.68, 19.102, 19.328; \emph{Od.}~5.191, 7.187, 8.27, 11.566, 14.169, 14.391, 15.20, 16.141, 17.469, 18.352, 20.9, 20.217, 20.328, 21.96, 21.276, 23.105, 23.215). The other \textquote{\textgreek{θυμόν}} here, on line 192 is used in the other sense of \textquote{emotion.}} The metaphorical division of the mind, as with \textquote{\textgreek{διάνδιχα}} here, signifies an actor torn between one of two possible courses of action, as does the either/or \textquote{\textgreek{ἤ/ἦε}.}\footnote{\textquote{\textgreek{διάνδιχα}} occurs only in the \emph{Iliad}, with \textquote{\textgreek{μερμήριξεν}} in three of the four other times (\emph{Il.}~8.167 and 13.455). Synonyms \textquote{\textgreek{δίχα}} and \textquote{\textgreek{διχθά}} appear in deliberative constructions at \emph{Il.}~16.435, 18.510, 20.32, 21.386; \emph{Od.}~3.150, 17.73, 19.524, 22.33. The division \textquote{two ways} (\emph{LSJ}) of this deliberation has an almost physical sense apparent at \emph{Iliad} 9.37, in Diomedes's address to Agamemnon, when he says that the leader has honor (\textquote{\textgreek{τετιμῆσθαι}}) but not physical might (\textquote{\textgreek{ἀλκὴν}}): \textquote[\textquote{the crooked minded son of Zeus gave you <power/qualities> in two parts}]{\textgreek{σοὶ δὲ διάνδιχα δῶκε Κρόνου πάϊς ἀγκυλομήτεω}}. Of this use, \textcite{hainsworth1993} explains the term as \textquote{one thing not another}.} Even without \textquote{\textgreek{διάνδιχα}} term for thinking, \textquote{\textgreek{μερμηρίζω}} denotes an actor in the process of choosing between one of two or more possible courses of action. Take, for example, Zeus pondering the deaths of Patroclus and Sarpedon: \textquote[\textquote{he was pondering in his heart, | debating many things about the death of Patroclus, | whether <Sarpedon should die> \ldots{} or \ldots{} <others die>} \emph{Il.}~16.646--651]{\textgreek{φράζετο θυμῷ, | πολλὰ μάλ' ἀμφὶ φόνῳ Πατρόκλου μερμηρίζων, | ἢ \ldots{} ἦ}}.\footnote{Other examples of \textquote{\textgreek{μερμηρίζω}} with an either/or construction (\textquote{\textgreek{ἢ \ldots{} ἦ}} or similar) are \emph{Il.}~ 5.671--673, 8.167--168 (with infinitive construction), 10.503--506 (Diomedes debates between three choices), 16.647--651; \emph{Od.}~6.141--144, 16.73--76, 18.90--92, 22.333--337, 24.235--238 (only one \textquote{\textgreek{ἦ}}). In several places, there is not a clear either/or explained by the text, but the thinking appears more open--ended (e.g., \emph{Il.}~ 2.3, 20.17; \emph{Od.}~10.438).} Finally, Achilles's deliberation here begins with the arrival of a state of emotion, here grief (\textquote{\textgreek{Πηλεΐωνι δ' ἄχος γένετ'}}). Since Achilles is interrupted mid--thought, the final comparative is missing here. Athena's appearance and subsequent conversation with Achilles (\emph{Il.}~1.194--222) is a perfect example of what scholars in the wake of \textcite{dodds1951} and \textcite{snell1953} explain as the externalization of thought in Homer, how an individual's ideas are represented as arising from a divine agent planting such thoughts in their head.\footnote{\textquote[{\cite[19]{snell1953}}]{Whenever a man accomplishes, or pronounces, more than his previous attitude had led others to expect, Homer connects this, in so far as he tries to supply an explanation, with the interference of a god. It should be noted especially that Homer does not know genuine personal decisions; even where a hero is shown pondering two alternatives the intervention of the gods plays the key role. This divine meddling is, of course, a necessary complement of Homer's notions regarding the human mind and the soul. The \emph{thymos} and the \emph{noos} are so very little different from other physical organs that they cannot very well be looked upon as a genuine source of impulses}. \textcite[e.g., pp.~15--16]{dodds1951} follows Snell's thinking about the disunity of the Homeric person. For what claims to be a clarification of Snell, see \textcite[163--176]{barnouw2004}.} Though many have disagreed with Snell on the Homeric mind,\footnote{For bibliography against Snell, see \textcite[p.~5, n.~1]{lawrence2002} and \textcite[p.~175, n.~2]{barnouw2004}, and more generally \textcite[pp.~21--50, esp.~pp.~21--26]{williams1993}, along with essay and critique by \textcite[165--166]{long2007}. For arguments concerning reflective deliberation in particular, see, \textcite[37--39]{jarratt1991}, \textcite[10--16]{nienkamp2001}, and \textcite{lawrence2002}.} and here how this thought comes to Achilles, this example shows that deliberation does occur in Homer and that it takes place in the Homeric mind, or \emph{thumos}. The key elements of heroic self--reflection are all here in Achilles's deliberation. In searching out the best course of action, we see his choosing between one of multiple options, being motivated by emotion, and considering contingencies.

The syntactical structure of Achilles's deliberation also illustrates the balance and relative sophistication of the decision--making process. Between two statements of Achilles's mental state (1.188 and 193), the narrator describes the two choices occurring being evaluated in his mind. \begin{greek} \begin{enumerate} \SingleSpacing
\item Πηλεΐωνι δ' ἄχος γένετ', 
\item ἐν δέ οἱ ἦτορ | στήθεσσιν λασίοισι διάνδιχα μερμήριξεν, |
  \begin{enumerate}
  \item ἢ ὅ γε φάσγανον ὀξὺ ἐρυσσάμενος παρὰ μηροῦ |
    \begin{enumerate}
    \item τοὺς μὲν ἀναστήσειεν,
    \item ὃ δ' Ἀτρεΐδην ἐναρίζοι, |
    \end{enumerate}
  \item ἦε χόλον παύσειεν ἐρητύσειέ τε θυμόν. |
  \end{enumerate}
\item ἧος ὃ ταῦθ' ὥρμαινε κατὰ φρένα καὶ κατὰ θυμόν
\end{enumerate} \DoubleSpacing \end{greek}
The major choice lies between 2.a) (drawing his sword) and 2.b) (putting a stop to his anger). Aorist optative verbs structure the sentence, which provides for parallelism between choices. They convey the unreality and equal potentiality of each course of each course of action. The optative combined with \textquote{\textgreek{ἢ \ldots{} ἦ}} specifically \textquote[{\cite[220]{monro1891}}]{expresses a doubt or deliberation thrown back into the past}.\footnote{As \label{fthomdel} examples of this, Monro cites \emph{Il.}~1.189, 5.671, 16.713; \emph{Od.}~4.117, 6.141, 10.50.  Other examples of deliberation offered by \textcite[220]{monro1891} are similar in that these deliberations are described by the narrator, are between two choices, and begin (unsurprisingly) with a verb or expression denoting some sort of mental division: \textquote[\emph{Il.}~5.671--673]{\textgreek{μερμήριξε δ' ἔπειτα κατὰ φρένα καὶ κατὰ θυμὸν | ἢ \ldots{} | ἦ}}, \textquote[\emph{Il.}~16.713--714]{\textgreek{δίζε γὰρ ἠὲ μάχοιτο \ldots{} | ἦ ὁμοκλήσειεν ἀλῆναι}}, \textquote[\emph{Od.}~4.117--119]{\textgreek{μερμήριξε δ' ἔπειτα κατὰ φρένα καὶ κατὰ θυμόν, | ἠέ \ldots{} | ἦ}}, and \textquote[\emph{Od.}~6.141--143]{\textgreek{ὁ δὲ μερμήριξεν Ὀδυσσεύς, | ἢ \ldots{} | ἦ}}.} Within the first option, two clausulae, governed with a \textquote{\textgreek{μὲν \ldots{} δέ}} construction, explain a twofold result or intention of this course of action -- scaring the others and killing Agamemnon.\footnote{\textcite[73]{hainsworth1993} explains that Aristarchus had a different interpretation, that Achilles was deciding between stirring others to kill Agamemnon or to do it himself. The contrast instead, I think, in this \textquote{\textgreek{μὲν \ldots{} δέ}} is between the movement of others versus the stillness of a dead Agamemnon.} Though Achilles's deliberations are not representative of all of the \emph{Iliad} \parencite[23--26]{scully1984}, the fundamental vacillation between a problem at a moment of crisis is clear.

A god, Zeus, also deliberates in Homer. In the following, Zeus reflects to himself, considering whether to allow Hector to kill Patroclus (\emph{Il.}~16.647--651).\footnote{\textquote[\emph{Il.}~16.647--651]{But he always looked down upon them and pondering in his heart, | debating many things about the death of Patroclus, | whether now in strong battle famous Hector should slay him, | <standing> there over godlike Sarpedon, | with bronze and seize the arms from his shoulders, | or <Zeus> should increase sharp toil for even more men}. See also Aristotle on similar deliberation about whether to allow Patroclus to slay Sarpedon (16.433--438)} \begin{greek} \begin{enumerate} \SingleSpacing
\item ἀλλὰ κατ' αὐτοὺς αἰὲν ὅρα καὶ φράζετο θυμῷ, |
\item πολλὰ μάλ' ἀμφὶ φόνῳ Πατρόκλου μερμηρίζων, |
  \begin{enumerate}
  \item ἢ ἤδη καὶ κεῖνον ἐνὶ κρατερῇ ὑσμίνῃ |
αὐτοῦ ἐπ' ἀντιθέῳ Σαρπηδόνι φαίδιμος Ἕκτωρ |
χαλκῷ δῃώσῃ, 
    \begin{enumerate}
    \item ἀπό τ' ὤμων τεύχε' ἕληται, |
    \end{enumerate}
  \item ἦ ἔτι καὶ πλεόνεσσιν ὀφέλλειεν πόνον αἰπύν. 
  \end{enumerate} 
\end{enumerate} \DoubleSpacing \end{greek} As in the Achilles example above, the decision is structured as an either/or (\textquote{\textgreek{ἢ \ldots{} ἦ}}) construction, between bringing about the death of his son or delaying it. While Zeus is not a hero, as is Achilles above or the Odysseus below, this deliberation is quite typical, what \textcite[393]{janko1992} calls a \textquote{normal example} of \textquote{pondering--scenes}. Like Achilles's deliberation in book~1, the first option has two parts, stabbing with bronze and taking of his armor (\textquote{\textgreek{δῃώσῃ \ldots{} ἕληται}}). \textcite[393]{janko1992} explains the briefer second option as a feature common to deliberations when the second of two choices are chosen.\footnote{\textquote[{\cite[393]{janko1992}}]{The second option is given briefly (651) because it is stated in full when Zeus selects it, whereas Homer readies us for Patroklos' death and the loss of his armour by describing that eventuality in detail}.} In addition to his deliberations taking a similar form, he also feels anguish as mortal heroes do, as evidenced by his grief expressed moments earlier during another self--reflection.\footnote{E.g., \textquote{\textgreek{ὤ μοι ἐγών}} (\emph{Il}.~16.433) during his deliberation 16.433--438; Hera's acknowledgment of this emotion at 16.450--452); and Zeus's tears of blood (16.459--460).} Some may object that Zeus is not a commander or leader in the way that Homeric heroes are, though scholarship has come to understand Zeus, in the \emph{Iliad}, as a bringer--about of fate and plot (i.e., \textquote[\textquote{the plan of Zeus}]{\textgreek{Διὸς \ldots{} βουλή}} \emph{Il.}~1.5).\footnote{For bibliography and concept of this \emph{Dios boulē} argument, see two lengthy notes on the significance of the \emph{Dios boulē} to the \emph{Iliad} and the significance of its absence in the \emph{Odyssey}, \textcite[pp.~19--20, nn. 19--20]{pucci1998}, as well as \textcite[pp.~271--272, n.~9]{redfield1994}. The idea goes back to the commentators, some of whom followed in the tradition of what Aristarchus called \textquote{\textgreek{οἱ νεώτεροι},} who held that the \textquote{\textgreek{Διὸς \ldots{} βουλή}} was part of Zeus's larger project of reducing the Earth's population through the Trojan war \parencite[this expression being analogous to \emph{Kypria} fr. 1,][p.~53]{kirk1985}.} \textcite[23]{scully1984} agrees, writing that what separates Zeus from others is not deliberation, but omniscience, which he argues Achilles shares to some extent. The two deliberators, hero or god, share that they are both important agents of action within the story of the \emph{Iliad}.

I next turn to deliberation by Odysseus in the \emph{Iliad} and \emph{Odyssey}. Deliberation by Odysseus follows the style of those of Achilles and Zeus above. I will first look at an example of Odysseus's deliberation in the \emph{Iliad}, then turn to look at a series of deliberations in the \emph{Odyssey}, which exhibit again his deliberative style as well as the importance of foresight and planning to this character. Cast about at sea and facing the unknown (\textquote{\textgreek{τὰ παρὰ δόξαν γινόμενα}}), Odysseus's preparation is mental, as Polybius says (\textquote{\textgreek{προνοίας ἀκριβοῦς εἰς τὸ πολλὴν ἀπορίαν παρασκευάζειν καὶ πολλάκις}} \emph{Hist}.~9.2), and, as I say, this preparation is deliberative.

%Odysseus, Iliad 11
Odysseus deliberates in battle when confronted with a difficult situation of choosing whether to fight or run (\emph{Iliad}~11). By the ships, he is left alone to face the Trojans. In speaking to himself, he reminds himself of his duty to fight. \begin{verse}\SingleSpacing \textgreek{Οἰώθη δ' Ὀδυσεὺς δουρὶ κλυτός, οὐδέ τις αὐτῷ \\ Ἀργείων παρέμεινεν, ἐπεὶ φόβος ἔλλαβε πάντας· \\ ὀχθήσας δ' ἄρα εἶπε πρὸς ὃν μεγαλήτορα θυμόν· \\ ὤ μοι ἐγὼ τί πάθω; μέγα μὲν κακὸν αἴ κε φέβωμαι \\ πληθὺν ταρβήσας· τὸ δὲ ῥίγιον αἴ κεν ἁλώω \\ μοῦνος· τοὺς δ' ἄλλους Δαναοὺς ἐφόβησε Κρονίων. \\ ἀλλὰ τί ἤ μοι ταῦτα φίλος διελέξατο θυμός; \\ οἶδα γὰρ ὅττι κακοὶ μὲν ἀποίχονται πολέμοιο, \\ ὃς δέ κ' ἀριστεύῃσι μάχῃ ἔνι τὸν δὲ μάλα χρεὼ \\ ἑστάμεναι κρατερῶς, ἤ τ' ἔβλητ' ἤ τ' ἔβαλ' ἄλλον.} \\ (\emph{Il.}~11.401--410)\end{verse} \begin{verse}\SingleSpacing Odysseus famous for the spear was left alone, nor did anyone \\ of the Argives remain, since fear seized them all; \\ and then perplexed he spoke to his great heart: \\ \textquote{Oh me, what will I suffer? <I will suffer> a great evil if \\ I, frightened by the crowd, am afraid; but <I will suffer> something more terrible if I am captured \\ alone; and the Kronion has frightened all of the Danaans. \\ But why does my heart reason about these things? \\ For I know that bad men stay away from war, \\ and for he who is best in battle, it is necessary for him \\ to stand bravely, if he is either harmed or harms another.} \\ (\emph{Il.}~11.401--410)\end{verse} \DoubleSpacing The characteristics of internal deliberation are apparent: he is uncertain about what to do (\textquote{\textgreek{ὀχθήσας}} and \textquote{\textgreek{διελέξατο θυμός}}), addresses his \emph{thumos} (\textquote{\textgreek{εἶπε πρὸς ὃν μεγαλήτορα θυμόν}}), and weighs options between two choices (\textquote{\textgreek{αἴ κε φέβωμαι}} and \textquote{\textgreek{αἴ κεν ἁλώω | μοῦνος}}). \textcite{hainsworth1993} calls this passage a \textquote{monologue} and an example of the externalization of \textquote[p.~270]{what goes on in the mind \ldots{} as a dialogue between the person and a personified entity}.\footnote{Parallel examples of type of what Hainsworth calls the \textquote{\textquote{Shall I stand and fight or withdraw?}} type appear at \emph{Iliad} 17.90--105, 21.553--570, and 22.98--130 \parencite[270]{hainsworth1993}. For more on the types of monologues, see \textcite{fenik1978}.} The comparative (\textquote{\textgreek{ῥίγιον}}) is clear and indicates the better of his two decisions. There is not a strong emotional aspect to Odysseus's deliberation here (though some emotion indicated, perhaps, with \textquote{\textgreek{ὤ μοι ἐγὼ τί πάθω; μέγα μὲν κακὸν}} and the \textquote{\textgreek{φόβος}}). Broken down into outline form to highlight its syntax, there is a balance of two possible options, between which Odysseus vacillates. \begin{greek} \begin{enumerate} \SingleSpacing
\item ὤ μοι ἐγὼ τί πάθω; 
  \begin{enumerate}
  \item μέγα μὲν κακὸν αἴ κε φέβωμαι | πληθὺν ταρβήσας· 
  \item τὸ δὲ ῥίγιον αἴ κεν ἁλώω | μοῦνος· 
    \begin{enumerate}
    \item τοὺς δ' ἄλλους Δαναοὺς ἐφόβησε Κρονίων. |
    \end{enumerate}
  \end{enumerate}
\item ἀλλὰ τί ἤ μοι ταῦτα φίλος διελέξατο θυμός; |
\item οἶδα γὰρ 
  \begin{enumerate} 
  \item ὅττι κακοὶ μὲν ἀποίχονται πολέμοιο, | 
  \item ὃς δέ κ' ἀριστεύῃσι μάχῃ 
    \begin{enumerate} 
    \item ἔνι τὸν δὲ μάλα χρεὼ | ἑστάμεναι κρατερῶς, 
      \begin{enumerate} 
      \item ἤ τ' ἔβλητ' 
      \item ἤ τ' ἔβαλ' ἄλλον.
      \end{enumerate}
    \end{enumerate}
  \end{enumerate}
\end{enumerate} \DoubleSpacing \end{greek} There is a structured logic to this thinking, in that it resembles something of an argument or proof. The deliberation begins with open credence to both options. The decision between fleeing and staying are contrasted in two different clauses. The first are two conditional statements (\textquote{\textgreek{αἴ κε φέβωμαι}} and \textquote{\textgreek{αἴ κεν ἁλώω | μοῦνος}}), with each of their apodoseis given first (\textquote{\textgreek{κακὸν}} and \textquote{\textgreek{ῥίγιον}}). To add to this parallelism, the two conditionals are joined with a \textquote{\textgreek{μὲν \ldots{} δὲ}} clause. In the process of coming to this decision, he considers one fact, being the circumstance at hand that the other Greeks have fled (\textquote{\textgreek{τοὺς δ' ἄλλους Δαναοὺς ἐφόβησε Κρονίων}}). At the end of the deliberation, Odysseus seems to have made up his mind, considering the two options (\textquote{\textgreek{κακοὶ μὲν ἀποίχονται πολέμοιο}} and \textquote{\textgreek{ὃς δέ κ' ἀριστεύῃσι μάχῃ }}), again structured with \textquote{\textgreek{μὲν \ldots{} δὲ.}}. Within the choice of staying in battle, some balance is considered with \textquote{\textgreek{ἤ τ' ἔβλητ'}} and \textquote{\textgreek{ἤ τ' ἔβαλ' ἄλλον,}} as either harming or being harm are the two options for one staying in battle. Despite the relatively balanced and logical consideration of these conditional possibilities and facts on the ground, this thinking is also strongly associated with Odysseus's own person through the four first--person verbs. These features of Odysseus's in--battle deliberation formally correspond to what \textcite{scully1984} calls typical self--reflexive thought. I add that this and the others take the form of logically structured thought.

%odyssey 9
%foresight + planning: Od. 9.270--5 (Cyclops?)
%stanford example:
Odysseus also shows good foresight and planning during the following two instances of deliberation by Odysseus, first in deciding how to injure the Cyclops (9.299--318), then how to escape the cave (\emph{Od.}~9.420--424). The text structures Odysseus's deliberation in a different manner than above by being in a first--person memory narrated by Odysseus himself and by not expressing as clearly the structure of the thought process. Still, structured and rational thought are present. According to \textcite[35]{heubeck1989}, these two deliberations have a similar structure: 299 \& 316 $\approx$ 420; 317 $\approx$ 421--423; and 318 = 424.\footnote{\textcite[35]{heubeck1989} incorrectly cite line 424 (\textquote{\textgreek{ἥδε δέ μοι κατὰ θυμὸν ἀρίστη φαίνετο βουλή}}) as 427.} In each instance, Odysseus (here the narrator of the tale) clearly states that he was deliberating at the opening of the decision--making process: \textquote[\textquote{I planned} 9.299]{\textgreek{ἐγὼ βούλευσα}} and \textquote[\textquote{I was planning} 9.420]{\textgreek{ἐγὼ βούλευον}}. In both deliberations, he considers conditional situations. \begin{verse}\SingleSpacing \textgreek{\hspace{2.0cm} \textquote{ἐγὼ λιπόμην κακὰ βυσσοδομεύων \\ εἴ πως τεισαίμην, δοίη δέ μοι εὖχος Ἀθήνη.}} \\ (\emph{Od.}~9.316--317) \end{verse} \begin{verse}\SingleSpacing \hspace{2.0cm} \textquote{I remained, deeply pondering evils, \\ if I might somehow exact revenge, and Athena grant this achievement to me.} \\ (\emph{Od.}~9.316--317) \end{verse} The second conditional is longer, but similar in form and syntax (aorist optative protasis). \begin{verse}\SingleSpacing \textquote{\textgreek{αὐτὰρ ἐγὼ βούλευον, ὅπως ὄχ' ἄριστα γένοιτο, \\ εἴ τιν' ἑταίροισιν θανάτου λύσιν ἠδ' ἐμοὶ αὐτῷ \\ εὑροίμην· πάντας δὲ δόλους καὶ μῆτιν ὕφαινον, \\ ὥς τε περὶ ψυχῆς·}} \\ (\emph{Od.}~9.420--423)\end{verse} \begin{verse} \SingleSpacing \textquote{I, however, was planning how all would be best, \\ if I might find some escape from death for my comrades \\ and myself; I was contriving all tricks and plan, \\ as a man is wont concerning his life.} \\ (\emph{Od.}~9.420--423)\end{verse} Each deliberation concludes with the following line, which also functions as an introduction to the following execution of what to do: \textquote[\textquote{This plan seemed best to my heart} \emph{Od.}~9.318 = 9.424]{\textgreek{ἥδε δέ μοι κατὰ θυμὸν ἀρίστη φαίνετο βουλή}}. The terminology here is that of planning (\textquote{\textgreek{βούλευσα,}} \textquote{\textgreek{βούλευον,}} \textquote{\textgreek{βουλή}} $\times$ 2) and contrivance (\textquote{\textgreek{βυσσοδομεύων,}} \textquote{\textgreek{εὑροίμην,}} \textquote{\textgreek{πάντας δὲ δόλους καὶ μῆτιν,}} \textquote{\textgreek{ὕφαινον}}). These words denote active agency in the subject's deliberation.\footnote{\textcite[30]{heubeck1989} translate 316 as \textquote{planning evil in the depths of my heart,} noting that the root of \textquote{\textgreek{βρυσσοδομεύω}} is \textquote[\textquote{building}]{\textgreek{δέμω}} and in parallel passages does not pertain to doing \textquote{evils.} The sense, then, is primarily one of thinking. See also \textcite[35]{heubeck1989} on \textquote{\textgreek{ὕφαινον.}} } Of course, the \textquote{\textgreek{μῆτιν}} of his planing looks ahead to the famous \textquote[\emph{Od.}~9.366]{\textgreek{Οὖτις}} trick.\footnote{For a famous take on Odysseus's cleverness in this and other passages, see \textcite[\textquote{Excursus I: Odysseus or Myth of Enlightenment}][35--62]{horkheimeradorno2002}: \textquote[p.~39]{The faculty by which the self survives adventures, throwing itself away in order to preserve itself, is cunning. The seafarer Odysseus outwits the natural deities as the civilized traveler was later to swindle savages, offering them colored beads for ivory}. A more recent take on the \textquote{Nobody} trick is \textcite[322--323]{slatkin2005}.} Once successful, it seems as if the entire episode has been an expression of his deliberation, which he later calls \textquote[\textquote{blameless plan} \emph{Od.}~9.414]{\textgreek{μῆτις ἀμύμων}}. According to \textcite[79--79]{schein1970}, Odysseus's deliberation is thematically significant, as it shows Odysseus resisting an initial, traditionally heroic, impulse to kill the Cyclops without thought for the blocked door. \begin{verse}\SingleSpacing \textgreek{\hspace{3.5cm} \textquote{ἕτερος δέ με θυμὸς ἔρυκεν. \\ αὐτοῦ γάρ κε καὶ ἄμμες ἀπωλόμεθ' αἰπὺν ὄλεθρον· \\ οὐ γάρ κεν δυνάμεσθα θυράων ὑψηλάων \\ χερσὶν ἀπώσασθαι λίθον ὄβριμον, ὃν προσέθηκεν.}} \\ (\emph{Od.}~9.302--305)\end{verse} \begin{verse} \SingleSpacing \hspace{3.5cm} \textquote{But another impulse checked me. \\ For we would have perished in utter ruin there; \\ for we would not have been able to push back \\ the mighty stone, which he put in front, from the high door.} \\ (\emph{Od.}~9.302--305)\end{verse} Before taking action, he considers once possible (\textquote{\textgreek{κε}} $\times$ 2) yet presently unrealized situations in which he and his men would be stuck. To add a sense of rationality to his thinking, Odysseus offers reasons why he fears these situations (\textquote{\textgreek{γάρ}} $\times$ 2). As is widely recognized about this passage, Odysseus's ingenuity is what helps him escape, though his plans do not emerge out of nowhere. Rather, his thought process, if not spelled out in detail, is at least signaled by the text. In all these examples, the text presents an Odysseus who is aware of his surroundings and whose consciousness, processing the facts of his circumstances, explains his survival and success.

%passages walbank says polyb has in mind: Iliad 10 and Od 5
Finally, I will consider two examples from Homer that Polybius seems to have in mind in the passage above  (\emph{Hist.}~9.16.1--3). These show in Odysseus the qualities of foresight and planning. Commenting upon Polybius's assertion that Odysseus was an ideal commander (\textquote{\textgreek{τὸν ἡγεμονικώτατον ἄνδρα}}) (\emph{Hist.}~9.16.1), \textcite[142]{walbank1967} cites two passages -- \emph{Iliad} 10.251--253 and \emph{Odyssey} 5.270--275 -- of Odysseus navigating by stars, the former while at sea and latter on land. In neither of these examples does the text offer a complete representation of the character's self--reflection, though it does make clear that the character is thinking, and in particular using foresight, to solve a problem. 
%iliad 10 
 On land, Odysseus makes plans by using stars in looking into the future and making plans at \emph{Iliad} 10.248--253. Here, he speaks to Diomedes, who has encouraged Odysseus to make a nighttime raid behind enemy lines (the so--called \textquote{\emph{doloneia}}). In his address, Diomedes has called upon Odysseus, \textquote[\textquote{for wise above all is he in understanding} \emph{Il.}~10.247]{\textgreek{ἐπεὶ περίοιδε νοῆσαι}}.\footnote{Translation from \textcite{murray1928}. More literally, \textquote{since he who knows well thinks.}} \begin{verse}\SingleSpacing \textquote{\textgreek{ἀλλ' ἴομεν· μάλα γὰρ νὺξ ἄνεται, ἐγγύθι δ' ἠώς, \\ ἄστρα δὲ δὴ προβέβηκε, παροίχωκεν δὲ πλέων νὺξ \\ τῶν δύο μοιράων, τριτάτη δ' ἔτι μοῖρα λέλειπται.}} \\ (\emph{Il.}~10.251--253)\end{verse} \begin{verse} \SingleSpacing \textquote{But let's go; for night is quite completed and dawn is near, \\ and the stars have advanced, and more than two--thirds \\ of the night have passed by, and the third lot is left.} \\ (\emph{Il.}~10.251--253)\end{verse}\DoubleSpacing While Odysseus's planning comes in response to Diomedes's call for action, he is still making plans in reference to time. Between the poles of night and dawn (\textquote{\textgreek{νὺξ \ldots{} ἠώς}}), the stars within are referenced (\textquote{\textgreek{ἄστρα δὲ δὴ προβέβηκε}}) and their movement indicators of the passage of time (\textquote{\textgreek{τῶν δύο μοιράων, τριτάτη δ' ἔτι μοῖρα λέλειπται}}) in which they form their plans.\footnote{\textquote{\textgreek{πλέων νὺξ}} by itself here may mean \textquote{the great part of,} though with the genitives of comparison (\textquote{\textgreek{τῶν δύο μοιράων}}) also \textquote{more than} \parencite[177]{hainsworth1993}. See note at \textcite[177]{hainsworth1993} generally for the unclear passing of time here.} This example which Polybius may have in mind shows the hero looking into the future, forming a plan, and following it through. As the historian says, Odysseus is prepared for contingency, in so far as he immediately sets out upon the raid after this speech. Polybius's ideal commander is one who, like Odysseus, uses foresight (\textquote{\textgreek{προϊδέσθαι}}) and plan--making (\textquote{\textgreek{εἰς τὸ πολλὴν ἀπορίαν παρασκευάζειν}}) in order to best respond to contingency (\textquote{\textgreek{τὰ παρὰ δόξαν γινόμενα}} \emph{Hist.}~9.16.1--2).

%Odysseus, Odyssey 5
In the other example that Polybius seems to have in mind of Odysseus as a good planner, there is more than just navigation by stars, but also the use of critical foresight in the evasion of unexpected danger. Upon leaving Calypso, Odysseus's navigation upon the sea offers an example of his capacity of foresight, both literal and figurative. \begin{verse}\SingleSpacing \textgreek{αὐτὰρ ὁ πηδαλίῳ ἰθύνετο τεχνηέντως \\ ἥμενος· οὐδέ οἱ ὕπνος ἐπὶ βλεφάροισιν ἔπιπτε \\ Πληϊάδας τ' ἐσορῶντι καὶ ὀψὲ δύοντα Βοώτην \\ Ἄρκτον θ', ἣν καὶ ἄμαξαν ἐπίκλησιν καλέουσιν, \\ ἥ τ' αὐτοῦ στρέφεται καί τ' Ὠρίωνα δοκεύει, \\ οἴη δ' ἄμμορός ἐστι λοετρῶν Ὠκεανοῖο· \\ τὴν γὰρ δή μιν ἄνωγε Καλυψώ, δῖα θεάων, \\ ποντοπορευέμεναι ἐπ' ἀριστερὰ χειρὸς ἔχοντα.} \\ (\emph{Od.}~5.270--275)\footnote{\emph{Odyssey} text from \textcite{muhll1962}.}\end{verse} \begin{verse} \SingleSpacing Seated, however, he was skillfully guiding straight \\ with the steering oar; nor did sleep fall upon his eyes \\ looking at the Pleiads and late--setting Ploughmen \\ and the Bear, which they also call Wagon, \\ and circles itself and keeps an eye on Orion, \\ and alone is without baths in the Ocean; \\ for beautiful Calypso bid him \\ to pass over the sea while holding it on his left. \\ (\emph{Od.}~5.270--275)\end{verse} \DoubleSpacing In navigating, his sleepless eyes keep watch for any looming physical threats, and in this sense he stares into the future. The reference to Odysseus's eyes (\textquote{\textgreek{ἐπὶ βλεφάροισιν}}) underscores an implicit focalization of the four constellations through him. Steering a straight course signifies the expression of intention (\textquote{\textgreek{ἰθύνετο}}), the plan of which has been already set through advice from Kalypo (\textquote{\textgreek{μιν ἄνωγε Καλυψώ}}). The narrator emphasizes the skillful accomplishment of Odysseus's navigation as well (\textquote{\textgreek{τεχνηέντως}}). He keeps up this navigation over an extended period of time (over seventeen full days \emph{Il.}~278). Earlier, Odysseus shows technical proficiency when he takes cloth brought by Calypso and \textquote[\textquote{skillfully crafted them} \emph{Od.}~5.259]{\textgreek{ὁ δ' εὖ τεχνήσατο καὶ τά}} into sails. For the rest of \emph{Odyssey}~5, Odysseus engages in several other, quite similar deliberations (\emph{Od.}~5.297--312, 353--366, 406--425, 464--474). These share some identical features, such as direct discourse, introduction with \textquote{\textgreek{ὀχθήσας},} the \emph{thumos} as addressee, expression of emotion (e.g., \textquote{\textgreek{ὤ μοι}}), and \textquote{\textgreek{ἢ/ἦε}} constructions. The deliberations continue until he lands upon the shores of Phaeacia. There is some thematic tension, perhaps, between Odysseus's foresight with the planning/directing done by Calypso (signaled with \textquote{\textgreek{μιν ἄνωγε Καλυψώ}}).\footnote{Odysseus's following of Calypso (and later Circe, another female divinity) here mirrors his more frequent following of Athena, an observation made by \textcite{murnaghan1995}.} Nevertheless, in following a preset course according to the guide of these constellations, the narrator signals Odysseus's qualities of foresight and accomplishing of plans.

%stanford on Odysseus: 'taking the long view'
From the \emph{Odyssey} to the Sophocles's \emph{Philoctetes} and Horace's \emph{Epistles}~1.2, Odysseus's cunning has a long tradition in the ancient world and beyond. These deliberations are another element of his famous cleverness. Seeing this cleverness, and hence deliberations, as a result of his self--control, \textcite{stanford1963} offers an explanation of Odysseus's successful return home to Ithaca. It is by \textquote[{\cite[79]{stanford1963}}]{taking the long view} in wisdom and self--control that Odysseus gains control over environment and achieves his narrative goals.\footnote{\textquote[{\cite[79]{stanford1963}}]{He had a remarkable power of taking the long view, of seeing actions in their widest context, of disciplining himself to the main purpose in hand. Thus while other heroes at Troy are squabbling like children over questions of honour and precedence, Odysseus presses on steadily towards victory}. For scholarship on Odysseus's self--control and its importance to the \emph{Odyssey}, much of which is in response to Stanford, see \textcite{rose1975}, \textcite{dobbs1987}, \textcite{most1989}, \textcite{glenn1998}, and \textcite{scodel2004}.} Achilles and Zeus are not remembered for intelligence as Odysseus is, though there is perhaps something similar in their use of intelligence in achieving particular goals. Following Achilles's deliberation above, Athena persuades him to hold off, for he will gain three times more gifts at a future time (\emph{Il}.~1.212--214). This notion works well alongside the thesis of \textcite{scully1984} that Achilles's foreknowledge of his death \textquote{puts him in a category above other heroes} and \textquote[p.~23]{places Achilles alongside the divine}. In the two great Homeric texts, deliberation by characters are ways for the text to communicate an actor's mastery over other actors, his circumstances, and perhaps even the plot (in the case of Zeus and Odysseus's narrations at Phaeacia). Looking to Polybius and later Caesar, an actor's deliberation is something by which he exercises control over events around him. These two authors, we will see, share with Homer that their ideal commanders think critically about where they are and where they are going.

\subsection{Polybius}
\label{polyb-comm}
In looking at the similar \textquote{Odyssean} qualities of great generals and authors, I here look at exemplary discussions of each, first of ideal commanders and then of ideal historians, in the \emph{Histories}. I argue that Polybius's conception of rationalization provides for mastery over circumstances and ability to achieve the ends of their goals. Through deliberation, the ideal commander may accomplish military engagements over and against opponent men and even geography. As author, Polybius's authorial performance is that of a critical seeker of truth. He creates a narrative world in which he is an unchallenged master over events, both their construction into written stories and the motivations of minds of actors within the stories. I first look at how the text constructs ideal generals, then turn to how Polybius discusses himself as an ideal historian, whose qualities mirror those of a great general.

Polybius argues that armies succeed due to the rational deliberation of generals.\footnote{The\label{ft-polyb-biblio} two most directly relevant works are \textcite[][on Polybius's idea of agency, which success hinges on the state of mind and intentions of actor, and how well they assess a situation before doing anything]{podes1990} and \textcite[][on Polybius on ideal general, who wins with superior rationality and tactical knowledge]{poznanski1994}. Another important discussion of leaders' psychology is \textcite[210--229]{pedech1964}, especially discussion of the \textquote[p.~216--222]{\textfrench{héro raisonnable}}. See also \textcite[][p.~315, n.~121, considering the former two articles]{levene2010} on Polybius's rationalizing historical events (pp.~150--155), exposition of battles as explainable through rational analysis (pp.~181--293), and the utmost importance of rationality in commanders to the outcomes of battles (pp.~300--316).} He illustrates characters' rationality, or lack thereof, in historical examples as well as in several generalizing discussions of a commander's ideal qualities.\footnote{At least twice he directly ascribes the success of commanders to their rational calculations situations, for Hannibal at \emph{Histories} 3.47.6--48.12 and for Scipio Africanus at 10.2.5--8. In a lengthy digression in book nine, Polybius describes in general terms the characteristics of the ideal general (9.12--20), mentioning as brief illustrations three historical figures. He also singles out failures in foresight by Aratus at Cytherea, Cleomenes III at Megalopolis, and Nicias at Syracuse (9.17.1--9.19).} Polybius ascribes leaders' success to their rational approach to events. I will consider two deliberations by important figures, Scipio Africanus and Hannibal. In these reflections, their thought is forward looking and rational. Scipio Africanus and Hannibal, like Odysseus, share that they are more intelligent than they are lucky, and their success is a result of the activity of their minds. \blockquote[{\cite[216--217]{pedech1964}}]{\textfrench{Les grands héros de Polybe, à qui va son admiration sans réserve et aux quels il attribute une influence décisive; sont des hommes froids, positifs et calculateurs, des cerveaux qui raisonnent et ordonnent. Il les a si bien dépouillés de toute affectivité qu'ils paraissent taillés sur le même patron.}} This \textquote[]{\textfrench{héro raisonnable}} has a well--organized mind which operates according to principles of structured thought. As with observations about Odysseus's self--control contributing to his success in the \emph{Odyssey}, self--control plays a part in Polybius's conception of such a character, who must be the master of his emotions.\footnote{\textquote[{\cite[24]{poznanski1994}}]{\textfrench{À la base de la réussite du chef de guerre, il y a tout d'abord les qualités intrinsèques de l'individu; son portrait psychologique est celui d'un homme réfléchi, qui sait dominer ses sentiments}}.}

I will next look at the narrator's characterization of Scipio's rationality and then turn to close readings of his actual deliberations, as done above with Homeric deliberators. At the opening of narrative on events in Spain, the author disagrees with other historians' evaluation of Scipio, whom they \textquote[\textquote{portray as successful always, for the most part, unexpectedly and prospering over unexpected events by accident} \emph{Hist.}~10.2.5]{\textgreek{ἐπιτυχῆ τινα καὶ τὸ πλεῖον αἰεὶ παραλόγως καὶ ταὐτομάτῳ κατορθοῦντα τὰς ἐπιβολὰς παρεισάγουσι}}. Such people admire Scipio for being \textquote[\textquote{more divine and <therefore> more praiseworthy than those acting according to reason} \emph{Hist.}~10.2.6]{\textgreek{θειοτέρους εἶναι καὶ θαυμαστοτέρους \ldots{} τῶν κατὰ λόγον \ldots{} πραττόντων}}. Instead, the valuation should be the other way around. Scipio is the \emph{pragmatikos} man of action (\textquote{\textgreek{πραττόντων}}), whose activity is mental and rational in nature (\textquote{\textgreek{κατὰ λόγον}}). \blockquote[\emph{Hist.}~10.2.7]{\textgreek{τὸ δ' ἐπαινετὸν μόνον ἴδιον ὑπάρχει τῶν εὐλογίστων καὶ φρένας ἐχόντων ἀνδρῶν, οὓς καὶ θειοτάτους εἶναι καὶ προσφιλεστάτους τοῖς θεοῖς νομιστέον.}} \blockquote[\emph{Hist.} 10.2.7]{Solely praiseworthy are well--reasoning men and those posssesing mind, who should be considered most divine and most beloved by the gods.} This passage encapsulates Polybius's attitude toward human agency. For Scipio here, he downplays the importance of random chance (\textquote{\textgreek{τὰς ἐπιβολὰς}}) and fortune (or \emph{tychē}, here as \textquote{\textgreek{ἐπιτυχῆ}} and \textquote{\textgreek{ταὐτομάτῳ}}) in favor of mental planning (not \textquote{\textgreek{παραλόγως}}). Generally speaking, Polybius sets aside an important role for \emph{tychē}, so this particular downplaying for Scipio is all the more notable.\footnote{On \emph{tychē} in Polybius's, see \textcite[331--354]{pedech1964}, \textcite[136--140]{sacks1981}, and \textcite[195--201]{mcging2010}. For an overarching statement, consider the following: \textquote[{\cite[331]{pedech1964}}]{\textfrench{Quand l'historien a tenu compte, dans l'examen des causes, de la volonté humaine, du pouvoir de la parole et l'efficacité des institutions, il s'aperçoit qu'il reste encore de jeu dans le mécanisme de l'histoire. Une force incalculable vient s'interposer entre la cause et l'effect, déjouant les prévisions les plus rigoreuses et déroutant la dialectique de l'historien; c'est la fortune ou le hasard (τύχη, αὐτόματον)}}.} Who alone should be praised, he says, are those who think well and possess \emph{phrenes}, the very word used by Achilles in \emph{Iliad}~1 (\textquote{\textgreek{κατὰ φρένα}} 1.193, also examples in footnote~\ref{fthomdel}). Simply put, a commander's active thinking is crucial to his success, far more than other factors like fortune or good luck. Scipio, summarizes \textcite{mcging2010} on this passage, is remarkable for reason of \textquote[][.]{his shrewdness (\emph{agchinoia}), calculation (\emph{logismos}), and foresight (\emph{pronoia})} In its generalizing statements of characterization, the \emph{Histories} display Scipio as smart not lucky.

Turning to several of Scipio's actual deliberations, we see habits similar to those of Odysseus and other deliberators in the Homeric epic. Of the examples of Scipio's internal deliberation in Polybius's \emph{Histories}, I will look closely at the form and content of two (\emph{Hist.}~10.7.6--8.1 and 14.1.4--5).\footnote{The five clearcut examples that I can find are: \emph{Hist.}~10.7.6--8.1, 10.39.9, 11.22.1--3, 14.1.4--5, and 14.1.8.} The first example of Scipio deliberating will help to flesh out the qualities of his thinking in the \emph{Histories}. In book~10, in a reflection upon Scipio's character in reference to the capture of New Carthage, the narrator offers a portrait of the rational processes going through Scipio's mind (\emph{Hist.}~10.2.1--20.8).\footnote{See \textcite[192--196]{walbank1967}.} Considering how to pursue the Carthaginians in Spain, Scipio decides against attacking all three of its generals at once, but to seize the strategically valuable town of New Carthage (\emph{Hist.}~10.7.6--8.1). \begin{greek} \begin{enumerate} \SingleSpacing 
\item (7.6) νομίζων οὖν, 
  \begin{enumerate} 
  \item ἐὰν μὲν εἰς μάχην συνιέναι κρίνῃ τοῖς πολεμίοις, 
    \begin{enumerate} 
    \item τὸ μὲν πρὸς πάντας ἅμα κινδυνεύειν (ἐπισφαλὲς εἶναι) τελέως 
      \begin{enumerate}
      \item καὶ διὰ τὸ προηττῆσθαι τοὺς πρὸ αὐτοῦ 
      \item καὶ διὰ τὸ πολλαπλασίους εἶναι τοὺς ὑπεναντίους,
      \end{enumerate}
    \end{enumerate}
  \item (7.7) ἐὰν δὲ πρὸς ἕνα συμβαλεῖν σπεύδων, 
    \begin{enumerate}
    \item κἄπειτα τούτου φυγομαχήσαντος, 
      \begin{enumerate}
      \item ἐπιγενομένων δὲ τῶν ἄλλων δυνάμεων,
      \item συγκλεισθῇ που,
      \end{enumerate}
    \item κατάφοβος ἦν
      \begin{enumerate}
      \item μὴ ταῖς αὐταῖς Γναΐῳ τῷ θείῳ καὶ Ποπλίῳ τῷ πατρὶ περιπέσῃ συμφοραῖς.
      \end{enumerate}
    \end{enumerate}
  \end{enumerate}
\item (8.1) διὸ τοῦτο μὲν τὸ μέρος ἀπεδοκίμασε.
\end{enumerate} \DoubleSpacing \end{greek} \begin{enumerate} \SingleSpacing 
\item (7.6) He thought: 
  \begin{enumerate} 
  \item if he decided to begin battle with the enemies, 
    \begin{enumerate} 
    \item it would be dangerous to venture against all at once 
      \begin{enumerate}
      \item both on account of those previous to him having been bested 
      \item and on account of the opponents being many more in number; 
      \end{enumerate}
    \end{enumerate}
  \item (7.7) but if, hastening to engage with one,
    \begin{enumerate}
    \item whereupon <if he were to> flee battle, 
      \begin{enumerate}
      \item and other <Carthaginian> forces came after in support,
      \item he would then be hemmed in somewhere,  
      \end{enumerate}
    \item he was afraid 
      \begin{enumerate}
      \item lest he came upon the same events of his uncle Gnaeus and father Publius.
      \end{enumerate}
    \end{enumerate}
  \end{enumerate}
\item (8.1) Wherefore he rejected this path.
\end{enumerate} \DoubleSpacing 
As above in Homer, we find language of thinking (\textquote{\textgreek{νομίζων}}), adverbs and particles which structure and balance this thought (\textquote[][,]{\textgreek{οὖν}} \textquote[][,]{\textgreek{ἐὰν}} \textquote[][,]{\textgreek{μὲν \ldots{} δὲ}} \textquote{\textgreek{μὲν}} $\times$ 2), \textquote[][,]{\textgreek{ἐὰν δὲ}} \textquote[][,]{\textgreek{κἄπειτα}} \textquote[][,]{\textgreek{μὴ}} and \textquote{\textgreek{διὸ}}), and conditional or circumstantial statements (\textquote{\textgreek{ἐὰν}} $\times$ 2, \textquote[][,]{\textgreek{τούτου φυγομαχήσαντος}} and \textquote{\textgreek{ἐπιγενομένων δὲ τῶν ἄλλων δυνάμεων}}). Especially apparent in this tabular outline, there is a strong parallelism to the entire deliberation, which lends the impression of a well--balanced consideration of all possible options (\textquote{\textgreek{ἐὰν μὲν \ldots{} ἐὰν δὲ}}; \textquote{\textgreek{καὶ διὰ τὸ \ldots{} καὶ διὰ τὸ}}). This use of \textquote{\textgreek{μὲν \ldots{} δέ}} was seen above in Homeric deliberations (e.g., Achilles at \emph{Il.}~1.188–94 and Odysseus at \emph{Il.}~11.401–410), as was the introduction of emotion into the interior life of the deliberator, such as the fear apparent here (\textquote{\textgreek{κατάφοβος ἦν μὴ}}). Yet again, Scipio shows the foresight (especially in the ending of \textquote{\textgreek{περιπέσῃ συμφοραῖς}}) and active agency (such as wanting to attack all at once, \textquote{\textgreek{ἅμα κινδυνεύειν \ldots{} τελέως}}).\footnote{For more on foresight, see \emph{Hist.}~10.2.10--13, right before this passage, on the role of dreams and omens in popular contemporary understandings of Scipio.} The deliberation's contents include an appeal to precedent in past history, and one plausibly on Scipio's mind. This also works, though in a minor way, to elide the difference between Polybius's ideal commander and his ideal historian, both critical interpreters of the past. \textcite{walbank1967} joins the qualities of this deliberation to several other larger issues pertinent to the \emph{Histories}. Scipio's deliberation, he writes, is intended by Polybius to counter impressions of the \emph{mythos} Scipio created about himself, when he \textquote[][.]{deliberately represented the fruit of calculation as the work of divine powers} In this way the piece functions like Polybius on Hannibal's crossing of the Alps (\emph{Hist.}~3.47.6--48.12), looked at next, which counters those who subscribe to fanciful tales to explain the feat.

%11.22.1--3
Another example of Scipio deliberating comes before the battle of Ilipa at \emph{Histories} 11.22.1--3, when he thinks about how to reorganize his troops so as to surprise Hasdrubal. He considers two strategies -- Hasdrubal's and his own -- and then comes to a decision. \begin{greek} 
\begin{enumerate} \SingleSpacing 
\item (1) Κατὰ δὲ τὸν καιρὸν τοῦτον δυσὶ δοκεῖ κεχρῆσθαι στρατηγήμασιν ὁ Πόπλιος. (2) θεωρῶν γὰρ 
  \begin{enumerate}
  \item τὸν Ἀσδρούβαν ὀψὲ ποιούμενον τὰς ἐξαγωγάς, 
    \begin{enumerate}
    \item καὶ μέσους Λίβυας, 
    \item τὰ δὲ θηρία προτιθέμενον ἑκατέρων τῶν κεράτων,
    \end{enumerate}
  \item αὐτὸς εἰωθὼς τῇ μὲν ὥρᾳ προσανατείνειν, 
    \begin{enumerate}
    \item (3) τοὺς δὲ Ῥωμαίους μέσους ἀντιτάττειν τοῖς Λίβυσι,
    \item τοὺς δ' Ἴβηρας ἐπὶ τῶν κεράτων παρεμβάλλειν,
    \end{enumerate}
  \end{enumerate}
\item ᾗ προέθετο κρίνειν ἡμέρᾳ, 
  \begin{enumerate}
  \item τἀναντία τοῖς προειρημένοις ποιήσας 
  \item μεγάλα συνήργησε ταῖς σφετέραις δυνάμεσι πρὸς τὸ νικᾶν, 
  \item οὐκ ὀλίγα δ' ἠλάττωσε τοὺς πολεμίους.
  \end{enumerate}
\end{enumerate} \DoubleSpacing \end{greek} 
\begin{greek} 
\begin{enumerate} \SingleSpacing 
\item (1) At this moment Publius seems to have availed himself of two stratagems. (2) For seeing that 
  \begin{enumerate}
  \item Hasdrubal was for a long time making excursions,
    \begin{enumerate}
    \item Libyans being in the middle, 
    \item and putting the elephants in front of each wing, 
    \end{enumerate}
  \item he himself having been accustomed to wait until a late hour
    \begin{enumerate}
    \item (3) to set the middle Romans against the Libyans 
    \item and to draw the Iberians up alongside, 
    \end{enumerate}
  \end{enumerate}
\item on which day he decided to judge, 
  \begin{enumerate}
  \item having done the opposite to that previously ordered,
  \item he contributed toward the victory of his own forces 
  \item no less than he reduced the power of his enemies. 
  \end{enumerate}
\end{enumerate} \DoubleSpacing \end{greek}
 The deliberation is composed of two parts (or \textquote[][,]{stratagems} \textquote{\textgreek{δυσὶ δοκεῖ κεχρῆσθαι στρατηγήμασιν}}), being Hasdrubal's versus his own past oppositions of Roman/Iberian against Libyans. In showing balance between what Hasdrubal's plans would be versus his own, Scipio shows one hallmark of reasoned calculation. The entire sentence (\emph{Hist.}~11.22.2) is based off a clear antithesis between Hasdrubal's (1.a)) and Scipio's armies (1.b)) (\textquote{\textgreek{μεγάλα συνήργησε \ldots{} ὀλίγα δ' ἠλάττωσε}} and \textquote{\textgreek{ταῖς σφετέραις δυνάμεσι \dots{} τοὺς πολεμίους}}). Following the deliberation comes his decision (2.), where he decides (\textquote{\textgreek{κρίνειν}}) to put his Iberians in the center and Romans at the flanks. As with Odysseus's deliberations above, Scipio's come at a critical moment, both within the progression of the battle of Ilipa (\textquote{\textgreek{κατὰ δὲ τὸν καιρὸν τοῦτον}}) and  in the greater scheme of the war (this battle would permanently cripple the Carthaginians' foothold in Spain). As in the previous passages, this has verbs and expressions of thinking (\textquote{\textgreek{δοκεῖ κεχρῆσθαι στρατηγήμασιν}} and \textquote{\textgreek{προέθετο κρίνειν}}), including a visual one (\textquote[][,]{\textgreek{θεωρῶν}}), which in light of Polybius's characterization of Odysseus may bring to mind Odysseus scanning the sea's horizon for threats in the \emph{Odyssey}. Another visual element of this deliberation is the presentation of the opposing forces, and among each who faces whom. These verbs are in the active voice, denoting an active thinking agent, save for \textquote[][,]{\textgreek{προέθετο}} which articulates a self--reflecting actor (as if literally putting thought in front of himself). The visual nature of Scipio's observations and responses also indicate an actively observing and responsive general. The inferential \textquote{\textgreek{γὰρ}} does not play a strong role in the calculation itself, but does explain why he needed to deliberate at that moment. This being a kairotic moment, temporal concepts seem to be in mind with Scipio's trick, changing the organization of troops from what he has ordered (\textquote{\textgreek{τἀναντία τοῖς προειρημένοις}}) to what he does at this moment (\textquote{\textgreek{ᾗ \ldots{} ἡμέρᾳ}}). Overall, the long and complicated sentence represents a rich deliberative process occurring within Scipio's mind. Other examples of Scipio's deliberations show similar characteristics to Homeric heroes, like the presence of thinking terms (\textquote{\textgreek{ἡγεῖτο}} 10.39.9; \textquote{\textgreek{νομίσας}} 14.1.8), fear (\textquote{\textgreek{τῷ δεδιέναι}} 10.39.9), and adverbs (e.g., \textquote{\textgreek{οὖν}} 14.1.8). This brief survey of Scipio's internal deliberation shows, in action, what Polybius means in writing that Scipio was smarter than lucky; what \textcite{poznanski1994} means about Scipio's agency; and \textcite{pedech1964} and \textcite{podes1990} mean of the \textquote[][.]{\textfrench{héro raisonnable}}

Turning next to another major figure of the \emph{Histories}, I show that Polybius portrays Hannibal's agency through internal deliberations much like those just seen of Scipio. Like Scipio's, these portraits of rationality appear, in part, to derive from a tradition of Greek literary tradition of intelligence--in--action. More importantly, however, understanding another great commander's mind will help to show how Polybius elides the skill set of general and historian. Even more important to the argument of this chapter, I will later in the chapter propose that Caesar is adapting both this tradition of the deliberating commander and this Polybian precedent of eliding the skills actor and author.

Polybius takes a similar approach with Hannibal in taking issue with those who have claimed that he did not possess innate skills but had divine assistance. For instance, Polybius insists that appeals to the supernatural are not necessary to satisfactorily explain his crossing of the Alps. Historians, he says, have paradoxically portrayed (\textquote{\textgreek{παρεισάγοντες,}} as above at \emph{Hist.}~9.16.1--3 and 10.2.5) Hannibal as \textquote[\textquote{a general inimitable in courage and foresight}]{\textgreek{ἀμίμητόν τινα παρεισάγοντες στρατηγὸν καὶ τόλμῃ καὶ προνοίᾳ}}, yet at the same time \textquote[\textquote{most irrational} \emph{Hist.}~3.47.7]{\textgreek{ἀλογιστότατον}}. While they are correct in writing that he possessed the essential attribute of foresight, Polybius cannot let it stand that Hannibal be considered an irrational thinker or planner (as in the example of his crossing the Alps \emph{Hist.}~3.47.9). In his portrait of Hannibal, rationality is crucial. \blockquote[\emph{Hist.}~4.48.10--11]{\textgreek{(10) Ἀννίβας γε μὴν οὐχ ὡς οὗτοι γράφουσιν, λίαν δὲ περὶ ταῦτα πραγματικῶς ἐχρῆτο ταῖς ἐπι βολαῖς. (11) καὶ γὰρ τὴν τῆς χώρας ἀρετήν, εἰς ἣν ἐπεβάλετο καθιέναι, καὶ τὴν τῶν ὄχλων ἀλλοτριότητα πρὸς Ῥωμαίους ἐξητάκει σαφῶς, εἴς τε τὰς μεταξὺ δυσχωρίας ὁδηγοῖς καὶ καθηγεμόσιν ἐγχωρίοις ἐχρῆτο τοῖς τῶν αὐτῶν ἐλπίδων μέλλουσι κοινωνεῖν.}} \blockquote[\emph{Hist.}~4.48.10--11]{(10) Indeed Hannibal did not do as those say, but took up these plans very practically. (11) For he had scrutinized clearly the excellence of the land, to which he wanted to go, and the hostility of the people toward the Romans, and for the rough ground in between he used native guides and leaders who would take part in his hopes.} In his pragmatic behavior (\textquote{\textgreek{πραγματικῶς}}), Hannibal is literally a man of action, the sort Polybius praises elsewhere as the ideal commander, such as Odysseus (\textquote{\textgreek{τὸν ἄνδρα τὸν πραγματικὸν}} \emph{Hist.}~12.27.10) and Scipio, both men who \textquote[\textquote{\textgreek{τῶν κατὰ λόγον \ldots{} πραττόντων}}]{act according to reason}. Like Odysseus looking over the sea for threats, Hannibal inspects situations (\textquote{\textgreek{ἐξητάκει σαφῶς}}) for contingencies. The futurity of his mind's objects is indicated here by \textquote{\textgreek{ἐπεβάλετο,}} \textquote{\textgreek{τοῖς \ldots{} μέλλουσι,}} and \textquote{\textgreek{τῶν αὐτῶν ἐλπίδων,}} all of which point to action not yet complete. Closely and practically investigating how to cross the Alps, Hannibal exhibits the qualities of a rational and deliberating commander.\footnote{For a similar discussion, see \textcite[216--220]{pedech1964} of Hamilcar (p.~217), Hannibal, (pp.~217--219), Scipio (pp.~217), and Philopoemen (pp.~219--220); the last three of which are Polybius's \textquote{\textfrench{trois héros},} distinguished by \textquote[p.~220]{\textfrench{une logique rigoureuse de la réflection et de l'action}}.} 

Beyond these evaluative statements by the narrator, which demonstrate how the \emph{Histories} characterize Hannibal as a rational agent, I will next illustrate several of his internal self--reflections in action, thus showing  that he belongs, like Scipio, to the previously established type of the deliberating hero. Of the twelve internal deliberations by Hannibal in Polybius's \emph{Histories} that I can find, I will examine two examples which exhibit Hannibal's process of thinking at critical moments (\emph{Hist.}~3.17.4--8 and~3.70.9--12).\footnote{\emph{Hist.}~3.17.2--7, 3.42.5, 3.51.6, 3.52.4--6, 3.60.13, 3.70.9-12, 3.79.2--4, 3.100.1, 3.102.10, 3.104.1--3, 3.111.1, and 9.46.} In the first example, the narrator here explains Hannibal's thought process for attacking the Spanish town of Saguntum (\emph{Hist.}~3.17.4--8). With his foresight, weighing of evidence, and deductive logic, this portrait of Hannibal's mind puts him in league with the likes of Homer's Odysseus and Polybius's Scipio. \begin{greek} \begin{enumerate} \SingleSpacing 
\item (4) ᾗ τότε παραστρατοπεδεύσας Ἀννίβας ἐνεργὸς ἐγίνετο περὶ τὴν πολιορκίαν, πολλὰ προορώμενος εὔχρηστα πρὸς τὸ μέλλον ἐκ τοῦ κατὰ κράτος ἑλεῖν αὐτήν. 
  \begin{enumerate}
  \item (5) πρῶτον μὲν γὰρ ὑπέλαβε παρελέσθαι Ῥωμαίων τὴν ἐλπίδα τοῦ συστήσασθαι τὸν πόλεμον ἐν Ἰβηρίᾳ· 
  \item δεύτερον δὲ καταπληξάμενος ἅπαντας εὐτακτοτέρους 
    \begin{enumerate}
    \item μὲν ἐπέπειστο παρασκευάσειν τοὺς ὑφ' αὐτὸν ἤδη ταττομένους, 
    \item εὐλαβεστέρους δὲ τοὺς ἀκμὴν αὐτοκράτορας ὄντας τῶν Ἰβήρων,
    \end{enumerate}
  \item (6) τὸ δὲ μέγιστον, οὐδὲν ἀπολιπὼν ὄπισθεν πολέμιον ἀσφαλῶς ποιήσεσθαι τὴν εἰς τοὔμπροσθεν πορείαν. 
  \item (7) χωρίς τε τούτων 
    \begin{enumerate}
    \item εὐπορήσειν μὲν χορηγιῶν αὐτὸς ὑπελάμβανεν πρὸς τὰς ἐπιβολάς, 
    \item προθυμίαν δ' ἐνεργάσεσθαι ταῖς δυνάμεσιν ἐκ τῆς ἐσομένης ἑκάστοις ὠφελείας, 
    \item προκαλέσεσθαι δὲ τὴν εὔνοιαν τῶν ἐν οἴκῳ Καρχηδονίων διὰ τῶν ἀποσταλησομένων αὐτοῖς λαφύρων. 
    \end{enumerate}
  \end{enumerate}
\item (8) τοιούτοις δὲ χρώμενος διαλογισμοῖς ἐνεργῶς προσέκειτο τῇ πολιορκίᾳ, τοτὲ μὲν ὑπόδειγμα τῷ πλήθει ποιῶν \ldots{}
\end{enumerate} \DoubleSpacing  \end{greek} \begin{enumerate} \SingleSpacing 
\item (4) Then Hannibal, having camped around it, was energetically besieging the city, foreseeing many useful things to come from seizing it by force. 
  \begin{enumerate}
  \item (5) For first he understood that he would take away the Romans' hope of organizing a war in Iberia.
  \item Secondly, amazed that they all were rather well--organized,
    \begin{enumerate}
    \item he was convinced that he would <prepare> those now <posted> to him,
    \item and those of the Iberians autonomous holding fast,
    \end{enumerate}
  \item (6) and most of all, having left no one belligerent behind, would march forward safely. 
  \item (7) Aside from these <reasons>,
    \begin{enumerate}
    \item  he thought that he would raise funds for the expedition,
    \item and energize the eagerness <of the soldiers> with spoils to each,
    \item and call forth goodwill from the Carthaginians at home through the spoils to be dispatched to them. 
    \end{enumerate}
  \end{enumerate}
\item (8) Consulting these calculations he set upon the city energetically, now making himself an example to the many \ldots{}
\end{enumerate} \DoubleSpacing As in the previous examples, we see the characteristic formal features of deliberative or self--reflective behavior: adverbs and particles which structure a proof--like thought--process (\textquote[][,]{\textgreek{πρῶτον μὲν γὰρ \ldots{} δεύτερον δὲ \ldots{} τὸ δὲ μέγιστον}} \textquote{\textgreek{χωρίς τε τούτων}}, including \textquote{\textgreek{μέν \ldots{} δὲ}} series, embedded within larger points (1.b)i. \& ii. and (1.d)i., ii. \& iii); thinking vocabulary (\textquote[][,]{\textgreek{ὑπέλαβε}} \textquote[][,]{\textgreek{τὴν εὔνοιαν}} \textquote[][,]{\textgreek{τοιούτοις δὲ χρώμενος διαλογισμοῖς}} and \textquote{\textgreek{ὑπελάμβανεν}}); and foresight (\textquote{\textgreek{προορώμενος εὔχρηστα πρὸς τὸ μέλλον}}); and active agency (\textquote{\textgreek{Ἀννίβας ἐνεργὸς ἐγίνετο}} and \textquote{\textgreek{ἐνεργῶς}}). On Hannibal, Walbank writes that \textquote{what really interest} Polybius \textquote{are not details of the siege} but the reasons why he chooses the siege \parencite[329]{walbank1957}. With such thorough self--reflections, Hannibal is a deliberating hero like Scipio. Furthermore, in its context, Hannibal's deliberation comes immediately after the Roman senate's deliberations (\emph{Hist.}~3.16).\footnote{E.g., their \textquote{\textgreek{τοῖς λογισμοῖς}} and \textquote[\emph{Hist.}~3.16.5 and 7]{\textgreek{τοῖς διαλογισμοῖς}}. According to \textcite[327]{walbank1957}, \textquote[][.]{the parallelism with the Romans' calculations (16.7) is deliberate}} Hannibal deliberated well, as the siege was successful and lays important groundwork for the Second Punic War.

%Hist. 3.70.9--12
Later in \emph{Histories}~3, the text portrays Hannibal in another deliberation, here following those of the Roman consuls T. Sempronius Longus and P. Cornelius Scipio (\emph{Hist.}~3.70.9--12). \begin{greek} \begin{enumerate} \SingleSpacing \item (9) ὁ δ' Ἀννίβας παραπλησίους ἔχων ἐπινοίας Ποπλίῳ περὶ τῶν ἐνεστώτων κατὰ τοὐναντίον ἔσπευδε συμβαλεῖν τοῖς πολεμίοις, 
  \begin{enumerate} \SingleSpacing 
  \item θέλων μὲν πρῶτον ἀκεραίοις ἀποχρήσασθαι ταῖς τῶν Κελτῶν ὁρμαῖς, 
  \item (10) δεύτερον ἀνασκήτοις καὶ νεοσυλλόγοις συμβαλεῖν τοῖς τῶν Ῥωμαίων στρατοπέδοις, 
  \item τρίτον ἀδυνατοῦντος ἔτι τοῦ Ποπλίου ποιήσασθαι τὸν κίνδυνον, 
  \item τὸ δὲ μέγιστον, πράττειν τι καὶ μὴ προΐεσθαι διὰ κενῆς τὸν χρόνον.
  \end{enumerate}
\item (11) τῷ γὰρ εἰς ἀλλοτρίαν καθέντι χώραν στρατόπεδα καὶ παραδόξοις ἐγχειροῦντι πράγμασιν εἷς τρόπος ἐστὶν οὗτος σωτηρίας, τὸ συνεχῶς καινοποιεῖν ἀεὶ τὰς τῶν συμμάχων ἐλπίδας. 
\item (12) Ἀννίβας μὲν οὖν εἰδὼς τὴν ἐσομένην ὁρμὴν τοῦ Τεβερίου πρὸς τούτοις ἦν.
\end{enumerate} \DoubleSpacing  \end{greek} \begin{enumerate} \SingleSpacing \item (9) Hannibal, having thoughts similar to Scipio about the situation, hastened to engage with the enemy, 
  \begin{enumerate} \SingleSpacing 
  \item wishing first to avail himself of the unmixed impulses of the Celts,
  \item (10) second to engage with the untested and newly enlisted legions of the Romans,
  \item third to defuse Scipio's threat, 
  \item and most of all to do something and not lose this time fruitlessly. 
  \end{enumerate}
\item (11) Therefore, for someone who has brought an army into a foreign land and attempted with incredible deeds, there is this one way for safety -- the constant renewing of the hopes of allies.
\item (12) Thus Hannibal, who knew T. Sempronius Longus's assault against them would come. \end{enumerate} \DoubleSpacing Again, we see the characteristics of logical thought, like structuring adverbs (\textquote{\textgreek{μὲν πρῶτον \ldots{} δεύτερον \ldots{} τρίτον \ldots{} τὸ δὲ μέγιστον}} and \textquote{\textgreek{γὰρ}}) and terms of thinking (\textquote[][,]{\textgreek{ἔχων ἐπινοίας}} \textquote[][,]{\textgreek{θέλων μὲν πρῶτον}} and \textquote{\textgreek{εἰδὼς \ldots{} ἦν}}). As in the above examples from Homer and Scipio Africanus, the deliberation mediates between multiple possibilities and the one best course of action (\textquote{\textgreek{εἷς τρόπος}}). The activity of the thinking person is again exhibited through Hannibal's speed (\textquote{\textgreek{ἔσπευδε}} and \textquote{\textgreek{μὴ προΐεσθαι διὰ κενῆς τὸν χρόνον}}).\footnote{A parallelism with the above Hannibal example (\emph{Hist.}~3.17.4–8) is the use of future infinitives.} These two exemplary deliberations by Hannibal illustrate that he is able to deliberate in the manner I am calling the \textquote[][.]{deliberating hero} Furthermore, the characterization of Hannibal, like that of Scipio above, is that of a shrewd and intelligent leader who navigates his army through challenging circumstances by means of rational self--reflection.

% \subsubsection{Ideal commander $=$ ideal historian} 
Turning from Polybius's characterization of deliberating commanders, or heroes, to his philosophy of historiography, there are more similarities of the ideal general and writer. Polybius's ideal commander and historian share that they are both men of action (\textquote{\textgreek{τὸν ἄνδρα τὸν πραγματικὸν}}).\footnote{\textquote[{\cite[14--15]{mcging2010}}]{He becomes, to a certain extent, not just the recorder of events, but the hero, too—in terms of the epic poetry to which he likens history, not just Homer, but also Odysseus. The latter is the perfect man of action (\emph{pragmatikos}): versatile, widely traveled, observant, and \textquote{experienced in the wars of men and grievous waves} (Homer, \emph{Od.}~8.183). Polybius states his conviction “that the dignity of history demands such a man” (12.28.1), no doubt thinking of himself}. See \textcite[pp.~129--168, esp.~pp.~129--130]{mcging2010} on the significance of this passage to Polybius's biography and self--construction of himself as a Homeric hero.} These \textquote{pragmatic} positions of author and actor are mirrored by Polybius's arguments about the composition of history, in what he calls \textquote[\textquote{pragmatic history}]{\textgreek{πραγματικὴ ἱστορία}} done in a \textquote[\textquote{pragmatic genre}]{\textgreek{πραγματικὸς τρόπος}}.\footnote{For surveys of these terms, see \textcite[21--32]{pedech1964} and \textcite[171--186]{sacks1981}.} \textcite{pedech1964} defines the idea as follows. \blockquote[{\cite[32]{pedech1964}}]{\textfrench{\textgreek{Πραγματικὴ} précise et limite la portée d'{\textgreek{ἱστορία:}} c'est l'étude qui s'attache exclusivement aux événements publics et aux actes politiques et qui fait la matiere de la \textgreek{διήγησις τῶν πραγμάτων}}.} As \textquote{\textgreek{πραγματικὴ}} specifies and limits \textquote{\textgreek{ἱστορία},} it specifies and limits \textquote{\textgreek{ἀνήρ},} taking form in a man whose actions are directed at the accomplishment of action, or \textquote{\textgreek{τῶν πραγμάτων}.} In this elision between general and author, Polybius conceives of the ideal man as one possessing critical foresight, who is able, through his mental and physical activity, to demonstrate mastery over his environment in the accomplishment of his goals.\footnote{On the importance of the mind, and of rationality in particular (e.g., thinking  \textquote{\textgreek{κατὰ λόγον}}), to Polybius's construction of generals, see \textcite[210--229]{pedech1964}.}

%Polybius's mastery over context
In emphasizing the importance of rationality to the historian, and how he himself holds the qualities of a scrupulous investigator, Polybius implicitly claims to hold a mastery over context much as his deliberating heroes do. As indicated above about Odysseus, Polybius's ideal author of history is a man of action (\textquote{\textgreek{τὸν ἄνδρα τὸν πραγματικὸν}} \emph{Hist.}~12.27.10), as outlined at the opening of this chapter (12.27.4--11). In another instance, Polybius explains his own first--person fieldwork in going to where Hannibal crossed the Alps (\emph{Hist.}~3.47.6--48.12).\footnote{See 3.48.12 for a statment of his personal investigation. On this passage, see \textcite[pp.~81--82, 91]{mcging2010}.} Polybius discusses his own personal investigations of the passing and disputes other authors, whom he claims have made a twofold theological mistake.\footnote{See \textcite[381]{walbank1957} for an overview of to whom Polybius refers.} First, they construe the passing as so difficult that they must introduce gods into the narrative. \blockquote[\emph{Hist.}~3.47.8]{\textgreek{καταστροφὴν οὐ δυνάμενοι λαμβάνειν οὐδ' ἔξοδον τοῦ ψεύδους θεοὺς καὶ θεῶν παῖδας εἰς πραγματικὴν ἱστορίαν παρεισάγουσιν.}} \blockquote[\emph{Hist.}~3.47.8]{Unable to bring make a conclusion or end of lies, they introduce gods and the sons of gods to the history of events.} Second (as mentioned above), they make the commanders seem semi--divine or supernaturally (\textquote{\textgreek{θειοτέρους}} \emph{Hist.}~10.2.6) lucky (\textquote{\textgreek{ἐπιτυχῆ}} \emph{Hist.}~10.2.5). By inaccurately characterizing the geography and wrongly insisting upon Hannibal's foolishness (\emph{Hist.}~3.47.9), these historians must introduce divinities to their histories in order to create a plausible narrative, to help him either overcome the landscape or redeem his own bad decision.\footnote{\textquote[\textquote{For proposing the difficulties and roughness of the Alpine mountains as so great that not only horses, soldiers, and elephants, but even well--equipped infantry would not pass through easily; and likewise adding for us that the land was as so barren that, unless a some manifesting god or hero showed Hannibal the way, they all in great difficulty would have perished, <the historians> commonly fall upon both of the errors above} \emph{Hist.}~3.47.9]{\textgreek{ὑποθέμενοι γὰρ τὰς ἐρυμνότητας καὶ τραχύτητας τῶν Ἀλπεινῶν ὀρῶν τοιαύτας ὥστε μὴ οἷον ἵππους καὶ στρατόπεδα, σὺν δὲ τούτοις ἐλέφαντας, ἀλλὰ μηδὲ πεζοὺς εὐζώνους εὐχερῶς ἂν διελθεῖν, ὁμοίως δὲ καὶ τὴν ἔρημον τοιαύτην τινὰ περὶ τοὺς τόπους ὑπογράψαντες ἡμῖν ὥστ', εἰ μὴ θεὸς ἤ τις ἥρως ἀπαντήσας τοῖς περὶ τὸν Ἀννίβαν ὑπέδειξε τὰς ὁδούς, ἐξαπορήσαντας ἂν καταφθαρῆναι πάντας, ὁμολογουμένως ἐκ τούτων εἰς ἑκάτερον τῶν προειρημένων ἁμαρτημάτων ἐμπίπτουσι}}.} In their unrealistic portrayal of events, historians must come up with implausible narrative devices, such as the \emph{\textlatin{deus ex machina}}. \blockquote[\emph{Hist.}~3.48.8--10]{\textgreek{(8) ἐξ ὧν εἰκότως ἐμπίπτουσιν εἰς τὸ παραπλήσιον τοῖς τραγῳδιογράφοις. καὶ γὰρ ἐκείνοις πᾶσιν αἱ καταστροφαὶ τῶν δραμάτων προσδέονται θεοῦ καὶ μηχανῆς διὰ τὸ τὰς πρώτας ὑποθέσεις ψευδεῖς καὶ παραλόγους λαμβάνειν, (9) τούς τε συγγραφέας ἀνάγκη τὸ παραπλήσιον πάσχειν καὶ ποιεῖν ἥρωάς τε καὶ θεοὺς ἐπιφαινομένους, ἐπειδὰν τὰς ἀρχὰς ἀπιθάνους καὶ ψευδεῖς ὑποστήσωνται. πῶς γὰρ οἷόν τε παραλόγοις ἀρχαῖς εὔλογον ἐπιθεῖναι τέλος; (10) Ἀννίβας γε μὴν οὐχ ὡς οὗτοι γράφουσιν, λίαν δὲ περὶ ταῦτα πραγματικῶς ἐχρῆτο ταῖς ἐπιβολαῖς.}} \blockquote[\emph{Hist.}~3.48.8--10]{(8) Thus <the historians> stumble upon something similar to the tragedians. For the conclusions to their dramas need a god and machine (i.e., \emph{deus ex machina})\footnote{\textcite[382]{walbank1957}.} due to taking up untrue and unreasonable subjects, (9) and it is necessary that the historians suffer the same thing and make heroes and gods appear, since they laid down unlikely and untrue beginnings. For how is it possible to put a well--reasoned end to incalculable beginnings? (10) Hannibal was not as those <historians> write, but undertook these enterprises very practically.} The real problem with the historians who posit an un--rational Hannibal or other heroic commanders is that they are improperly robbed of their mastery over their minds, enemies, and greater context. By choosing improbable beginnings (\textquote{\textgreek{παραλόγοις ἀρχαῖς}}), reasonable laws of cause and effect cannot take the narrative to their proper conclusions (\textquote{\textgreek{εὔλογον ἐπιθεῖναι τέλος}}).\footnote{For attacks by Polybius against dramatic history, see \textcite[pp.~144--170, esp.~p.~144, n.~51 for other examples]{sacks1981}.} If the pragmatic commander is a man of mental and physical action, and pragamtic history is the narration of actions \parencite[so][31--32]{pedech1964}, then Polybius, by way of this parallelism, is setting himself up as a master over circumstances in a way not unlike that of a Scipio or Hannibal.

What exactly does this mastery entail? Polybius surely claims to hold a strong claim to the truth.\footnote{For Polybius's appeals to the need for truth and its benefits, see \textcite[pp.~139--144, esp.~ p.~139, n.~38 for examples]{sacks1981} and \textcite[p.~66 for other examples]{walbank1957}.} And among other historians, he has no compunctions about criticizing other writers (e.g., Timaeus at 12.12.3), going so far as to recommend to \textquote{criticize friends and praise enemies} in the service of the truth.\footnote{\textquote[\textquote{Wherefore one must not hesitate to accuse friends nor praise enemies} \emph{Hist}.~1.14.7]{\textgreek{διόπερ οὔτε τῶν φίλων κατηγορεῖν οὔτε τοὺς ἐχθροὺς ἐπαινεῖν ὀκνητέον}}. See \textcite[21--95 \emph{passim}]{sacks1981} for an overview of Polybius's treatment of Timaeus, and \textcite[214--216]{pedech1964} on him against other historians.} With no room for the supernatural, Polybius may create a world that is almost completely explainable, save for the periodic vagaries of chance, through understanding the actions of men. As an author of narrative, he also finds another sort of mastery over the creation of stories. Between his authorial persona as a rigorous thinker, an investigative reporter, and fierce critic of others; Polybius constructs a narrative world without room for ambiguity or doubt. He offers a world explainable through the observable actions of men, and not according to some inner essence (as he explains in the following sentence, \emph{Hist}.~1.14.8--9).\footnote{\textquote[{\cite[66]{walbank1957}}]{In general P.~prefers to judge, not the man as a whole, but his separate actions as they occur}.} The Odyssean, pragmatic man in the Polybian conception is characterized by this control, be it over events (by the deliberating hero) or stories (deliberating author). Polybius's heroes, Polybius himself as author, and Homer's Odysseus all share that, through critical deliberation, they achieve their ends, be they military success (e.g., Scipio), probable conclusion to a story (e.g., \textquote{\textgreek{εὔλογον ἐπιθεῖναι τέλος}}), or return home (\textquote{taking the long view}).

I would like to pull back from the Greco--Roman literary tradition and briefly consider Aristotle, whose concept of \emph{phronesis} (\textquote[][,]{wisdom} \textquote{practical wisdom} {\cite[207]{woods1992}}) may come to mind in this section's discussion of Homer and Polybius. And seeing that Caesar (along with other notable figures like Marcus~Cicero and the jurist \textlatin{Servius Sulpicius Rufus}) studied in Rhodes, where the Peripatetics had a long tradition, a consideration of Aristotelian foresight is not out of place.\footnote{On Cicero and \textlatin{Ser. Sulpicius Rufus} in Rhodes, see \textcite[9]{rawson1985}, and on Rhodian rhetoric, see \textcite[132--133]{vanderspoel2007}, \textcite[155]{connolly2007a}, and \textcite[133]{connolly2007}.} In a revisionist survey of its role in achieving of the good, \textcite[]{moss2011} defines \emph{phronesis} and the \emph{phronimos} (\textquote[][,]{wise} \textquote{intelligent} man {\cite[207]{woods1992}}) at face value, as Aristotle describes it in passages like the following from the \emph{Nicomachean Ethics}.\footnote{Many scholars have previously taken stances which make \emph{phronesis} and virtue (a related matter for Aristotle) intellectual states. For a survey of relevant passages about \emph{phronesis} in Aristotle, especially those in relation to virtue and the ends of action, see \textcite[\emph{passim}]{moss2011}. For bibliography of those who have taken the position that the intellect is crucial in identifying ends, see \textcite[p.~205, n.~2]{moss2011}.} \blockquote[\emph{Ethica Nicomachea} 6.12.6 $=$ 1144a6--9]{\textgreek{ἔτι τὸ ἔργον ἀποτελεῖται κατὰ τὴν φρόνησιν καὶ τὴν ἠθικὴν ἀρετήν· ἡ μὲν γὰρ ἀρετὴ τὸν σκοπὸν ποιεῖ ὀρθόν, ἡ δὲ φρόνησις τὰ πρὸς τοῦτον.}\footnote{Text from \textcite{bywater1894}.}} \blockquote[\emph{Ethica Nicomachea} 6.12.6 $=$ 1144a6--9]{Moreover, a man's proper function is accomplished according to \emph{phronesis} and ethical virtue: for virtue makes sure the end is right, and \emph{phronesis} makes things <point> toward this <end>.\footnote{Here I follow \textcite[366--367]{rackham1934} in translating \textquote{\textgreek{τὸ ἔργον}} as \textquote{a man's proper function.}}} In the deliberating process, this \textquote{practical wisdom} is not the logical reasoning faculty or practice that leads one to any particular end, but is an impulse that points one toward the general direction of what is good. For one to achieve this goal, one needs the \emph{logos} of deliberation to precisely determine the goal.\footnote{The most important positive definitions of \emph{phronesis} in Aristotle come at \textcite[243--251]{moss2011}.} Thus, \emph{phronesis} is a foundation from which one may then practice good decision--making or deliberation, as Aristotle makes clear earlier in the \emph{Ethics}. \blockquote[\emph{Ethica Nicomachea} 2.6.15 $=$ 1106b\-36--1107a2]{\textgreek{ Ἔστιν ἄρα ἡ ἀρετὴ ἕξις προαιρετική, ἐν μεσότητι οὖσα τῇ πρὸς ἡμᾶς, ὡρισμένῃ λόγῳ καὶ ᾧ ἂν ὁ φρόνιμος ὁρίσειεν.}} \blockquote[\emph{Ethica Nicomachea} 2.6.15 $=$ 1106b36--1107a2]{Therefore virtue is a habit of mind concerned with deciding, being for us in a median state which has been determined by \emph{logos} and by which the \emph{phronimos} may make decisions.\footnote{For a perhaps more satisfying translation, consider that of \textcite[244]{moss2011}: \textquote{Therefore virtue is a state issuing in decisions, consisting in a mean relative to us, determined by \emph{logos} and by that by which the \emph{phronimos} would determine it.}}} If we were to think about the deliberating hero in Aristotle's terms, we might construct something of a \emph{phronismos} hero, whose innate wisdom points him toward happiness, and whose \emph{logos} is the tool that assists him in the process of deliberating about how to come to the predetermined target. This Aristotelian hero has a goal, as Odysseus's was return to Ithaca or Hannibal's the destruction of Rome. For Caesar's depiction of himself as a deliberating hero, to which I turn next, there could be something similar to this bifurcation of deliberative talent, the characteristic that points one to generally desirable ends (e.g., the pacification of Gaul and defense of Rome), and the rationally fine--tuned calculations that allow the general to overcome any given impediment to his goal. In the next section, I will consider precisely how he uses his foresight and other deliberative faculties to point at the good as Caesar defines it.

\section{Caesar}
\label{ch2-delib-caesar}
The \emph{Bellum~Gallicum} should be understood in the tradition of the deliberating hero. Homer's heroes exercise their intelligence, in the form of self--reflection, display of their command over their context and, in the case of Zeus, mastery over plot. Polybius merges the qualities of the commander and author; those that make for good general are the same qualities that make for a good historian. Caesar goes a step further by crafting an individual who possess these Polybian qualities and who plays both roles of commander and historian. I here document, in several fully elaborated examples plus an overview of many others, how Caesar, as author and actor, exercises this dual control over narrative events and the narrative itself. By way of the collusion of actor and author in the \emph{Bellum~Gallicum}, the character Caesar's intellectual omnipotence, backed up by military victories, informs our understanding of the author (or implicit narrator) Julius Caesar, a figure whom no reader would mistake for a disinterested participant in the history's event.\footnote{See \textcite[150--155]{riggsby2006} on these two \textquote{voices} (first and third person) and how they operate in together, and for two examples of a colloquial \textquote{we,} where he argues the two figures merge in the very same sentence (\emph{BG}~5.13.4 and 7.25.1, \cite[152]{riggsby2006}).} I bring this new evidence about the deliberating general in support of tradition of scholarship that understands the \emph{Commentarii} as apologetic, defensive of Caesar's persistent expansion of empire, in nature.

What scholarship there has been on topics similar to Caesar's interior deliberations, has produced arguments along the following lines: the representation of a thinking Caesar creates for readers the impression of a fair or intelligent figure. The only significant study of Caesar's mental activity is \textcite{batstone1990}.\footnote{A shorter version of this argument, and only about the \emph{Bellum~civile}, also appears at \textcite[160--164]{batstonedamon2006}.} The piece considers concessive clauses and demonstrates how the text's syntax allows for the presentation of two sides of an event. Looking specifically at \textquote{\textlatin{etsi \ldots{} tamen}} clauses, \textcite{batstone1990} shows that the text draws \textquote{the reader into the mind of the actor and thinker} and thereby \textquote[351]{presents the conceder as a man of complex thought}.\footnote{\textquote[{\cite[351]{batstone1990}}]{The author may linger over what he claims is to be overlooked. Thus, \emph{etsi}, like the formal, hypotactic sentence in general, has an affinity for leisurely and complex consideration. It presents a process, not merely a result, and so may draw the reader into the mind of the actor and thinker. \emph{(Tam)etsi}, then, offers a complex event as a whole, and, inasmuch as the concessive thought is ascribed to the actor, it presents the conceder as a man of complex thought. Simple \emph{tamen}, on the other hand, emphasizes reversal in that it does not indicate concession until after the fact. Instead, it narrates a simple event and then takes away or denies the normal expectations that one would have had based upon the simple event}.} \textcite[193--195]{riggsby2006} has some remarks, following from \textcite{batstone1990}, about characterization of the actor Caesar. Riggsby's point here is that balanced thought presents Caesar as a self--controlled character possessing foresight. Of the examples he offers, some are internal deliberations (\emph{BG}~1.14.6, 3.22.4, 4.20.1), others not (5.21.2, 7.10.2, 7.33.1).\footnote{Riggsby's concern is with the types of arguments that the actor or narrator make with \textquote[][,]{\textlatin{tamen}} and how the narrator/actor reasons that strategic targets are preferable to smaller ones, readily accepts surrenders, and will engage in risky battles for political or ethical reasons; he thereby \textquote[194]{looks always to the \textquote{big picture}}. Another work that should be singled out is \textcite[48--52]{slonsky2008}, a masters thesis that builds on \textcite[192--195]{riggsby2006} and approaches as justificatory Caesar's internal deliberation at \emph{Bellum~Gallicum} 1.33.2--5 of whether to fight the Germans.} My evidence offers further support to the picture here, especially concerning Caesar's foresight. I move, however, in a different direction and argue that these deliberations, when understood in the tradition of the deliberating hero, are means by which the actor may, first, appear to exercise mastery over difficult situations and, second, craft an investigative character who uses deliberations as a means to justify his pursuits in Gaul.

%ex 1: BG 1.7.4--6
In two long and many brief examples, I next explain how the text carefully portrays the inner--workings of the actor Caesar's mind. He elucidates the problem, looks at two (or more) possible directions of action, facts on the ground that encourage or discourage each choice, and finally comes to a decision. Through this pattern of thought, the narrator portrays this general in charge of himself and that world around him. With foresight and planning like Odysseus, the well--reasoning mind of Scipio, and Polybius's bold spirit, Caesar stands out as the consummate deliberating general and shrewd military historian. At \emph{Bellum~Gallicum} 1.7.1, Caesar receives more information about the Helvetii's movements, and the first representation of deliberation by the general appears. Messengers from the Helvetii come to Caesar to report to him their alleged intention of crossing peacefully through the Roman province of Transalpine Gaul, and ask (\textquote{\textlatin{rogare}}) his permission to do so (\emph{BG}~1.7.3--4). In response to this information, the narrator explains Caesar's thoughts. \blockquote[\emph{BG}~1.7.4--5]{\textlatin{(4) Caesar, quod memoria tenebat L. Cassium consulem occisum exercitumque eius ab Helvetiis pulsum et sub iugum missum, concedendum non putabat; (5) neque homines inimico animo, data facultate per provinciam itineris faciundi, temperaturos ab iniuria et maleficio existimabat.}} \blockquote[\emph{BG}~1.7.4--5]{\textlatin{(4) Caesar, since he remembered that L. Cassius as consul was killed and his army defeated by the Helvetii and sent under the yoke, thought that <the request> should not be allowed; (5) and he did not think that men with hostile intent, once the ability of making way through the province had been given, would abstain from injury and wrongdoing.}} The main clause notes Caesar's thinking with two verbs (\textquote{\textlatin{putabat}} and \textquote{\textlatin{existimabat}}) about the situation. Within the initial subordinate clause, he falls into historical memory (\textquote{\textlatin{memoria tenebat}}) of L. Cassius's defeat by the Helvetii in 107 B.C.. This clause is marked as evidence (by the \textquote{\textlatin{quod}}) in support of his conclusion to refuse the request (\textquote{\textlatin{concedendum non}}). Though this explanation of Caesar's decision--making is brief, it contains an elaborate rhetoric that allows the narrator to explain and justify Caesar's actions. If broken down according to syntax, a rudimentary logical argument becomes apparent. \begin{latin} \begin{enumerate} \SingleSpacing \item Decision: \textquote{\textlatin{concedendum non putabat}}
  \begin{enumerate}
  \item Reason: \textquote{\textlatin{quod memoria tenebat}}
    \begin{enumerate}
    \item Fact: \textquote{\textlatin{L. Cassium consulem occisum}}
    \item Fact: \textquote{\textlatin{exercitumque eius ab Helvetiis pulsum}}
    \item Fact: \textquote{\textlatin{et sub iugum missum}}
    \end{enumerate}
  \item Reason: \textquote{\textlatin{neque homines \ldots{} temperaturos ab iniuria et maleficio existimabat}}
    \begin{enumerate}
    \item Fact: \textquote{\textlatin{inimico animo}}
    \item Condition: \textquote{\textlatin{data facultate per provinciam itineris faciundi}}
    \end{enumerate}
  \end{enumerate}
\end{enumerate} \DoubleSpacing \end{latin} The final decision is put in clear language (\textquote{\textlatin{concedendum non putabat}}). In support of this, there are two primary reasons, one concerning historical events of the past (\textquote{\textlatin{quod memoria tenebat}}) and another being informed speculation about the future (\textquote{\textlatin{homines \ldots{} temperaturos \ldots{} existimabat}}). For the negative historical precedent in Gaul, three distinct pieces of evidence are brought forward, each worse than the next: Cassius's death, defeat of the army, and subjugation of the army (in 107 B.C., \cite{kranerdittenbergermeusel1967}). For a general may die in battle but the army succeed; and an army may be defeated without undergoing the particularly humiliating exercise of being sent under the yoke. On Caesar and such events, \textcite{kraus2009} observes that the Roman people's \textquote[p.~168]{historically validated interests work in tandem with his own}. Concerning the future, Caesar postulates that the Helvetii will not behave as they say they will, giving several reasons why this is so. Foremost, there is the fact, or at least strong presumption, that the Helvetii have a hostile mind (\textquote{\textlatin{inimico animo}}). Also, the association of Gaul and its inhabitants as \textquote{\textlatin{inimici}} was common in Roman stereotypes.\footnote{Cicero writes that, aside from Gallia Narbonensis, Gaul \textquote[\textquote{was held by people either hostile to this empire or unloyal or unknown or certainly savage and barbarous and warlike} \emph{Prov.~cons.}~19]{\textlatin{a gentibus aut inimicis huic imperio aut infidis aut incognitis aut certe immanibus et barbaris et bellicosis tenebantur}} Elsewhere, Cicero speaks of \textquote[\textquote{most hostile Gaul} \emph{Phil.}~10.10]{\textlatin{inimicissimam Galliam}}, \textquote[\textquote{into hostile Gaul} \emph{Phil.}~10.21]{\textlatin{in Galliam \ldots{} inimicam}}, and \textquote[\textquote{most hostile Gallic <provinces>} \emph{Ad~fam.}~16.12.4]{\textlatin{Gallias \ldots{} inimicissimas}}.} Understood in reference to past Roman conflicts with the Helvetii, such as L.~Cassius's, this claim has further grounding. Caesar also considers the effect that express permission (\textquote{\textlatin{data facultate}}) would have on them.\footnote{According to \textcite{kranerdittenbergermeusel1967}, \textquote{\textlatin{data facultate}} has a conditional force, though it may work circumstantially too.} In his thinking, Caesar demonstrates that any allowance offered will be taken by these barbarians as license to injury and wrongdoing.

Though a brief decision--making process, the reasoning behind \emph{Bellum~Gallicum} 1.7.6 has been expressed with some care. The narrator announces Caesar's decision, which here is a trick played against the Helvetii. \blockquote[\emph{BG}~1.7.6]{\textlatin{Tamen, ut spatium intercedere posset, dum milites quos imperaverat convenirent, legatis respondit diem se ad deliberandum sumpturum; si quid vellent, ad Idus Apriles reverterentur.}}  \blockquote[\emph{BG}~1.7.6]{Nevertheless, in order that a space of time intercede, while soldiers whom he had ordered arrive, to the legates he responded that he would take a day for deliberating; if they should want anything, then they would return on April 13.} The final sentence does several things: demonstrates another option (to do nothing, if only briefly) with \textquote{\textlatin{tamen}}; why this should be done with purpose clause \textquote{\textlatin{ut \ldots{} posset}}); and announces Caesar's final decision through expression of verbal action (\textquote{\textlatin{respondit}}). Caesar in fact mentions the importance and difficulty of decision--making in his deceitful response. Though Caesar has already made his decision, he claims that he needs more time to deliberate (\textquote{\textlatin{deliberandum}}). Considered as a whole, this passage shows the fundamental characteristics, of vocabulary and syntactical complexity, of Caesar's interior decision--making. The parallels with the deliberations in the \emph{Iliad}, \emph{Odyssey}, and Polybius's \emph{Histories} are evident, too. We can see an implicit either/or construction, accomplished with conditional statement; comparative statements to indicate final preference; and the introduction of emotion as a motivating force to deliberation.

%ex 2: BG 7.10
Before discussing the purpose of these deliberations in the narrative, I will consider one more example in fine detail. An example of how elaborate Caesar's deliberation can be is \emph{Bellum~Gallicum} 7.10.1--3, where the general weighs whether to take Roman troops out of winter quarters dangerously early, in order to defend the allied town of Gorgobina. In this passage, complex syntax fashions a detailed argument, based upon facts and inference, of the best course of action. The narrator explains Caesar's plan to make way toward the land of the Lingones so as to foil any plot of Vercingetorix (\emph{BG}~7.9.4). Vercingetorix hears of this movement and begins to attack the town of Gorgobina (\emph{BG}~7.9.6). Although not expressed, Caesar must be understood as learning of this, as the knowledge of it causes him to consider the newly arisen problem. \blockquote[\emph{BG}~7.10.1--3]{(1) Magnam haec res Caesari difficultatem ad consilium capiendum adferebat, si reliquam partem hiemis uno loco legiones contineret, ne stipendiariis Haeduorum expugnatis cuncta Gallia deficeret, quod nullum amicis in eo praesidium videret positum esse; si maturius ex hibernis educeret, ne ab re frumentaria duris subvectionibus laboraret. (2) praestare visum est tamen omnes difficultates perpeti, quam tanta contumelia accepta omnium suorum voluntates alienare. (3) itaque cohortatus Haeduos de supportando commeatu praemittit ad Boios, qui de suo adventu doceant hortenturque, ut in fide maneant atque hostium impetum magno animo sustineant.} \blockquote[\emph{BG}~7.10.1--3]{(1) This matter brought to Caesar great difficulty for taking counsel. If he were to contain the legions in one place for the rest of the winter, he feared that, after tributaries of the Haedui had been taken by siege, all of Gaul would defect, since <Gaul> would see there was placed no defense in him for allies; if he were to lead <the legions> out somewhat early from winter quarters, he feared that he would suffer with respect to the grain supplies due to difficult transportation. (2) It seemed better, nevertheless, to endure  all difficulties, than, after such a great insult had been suffered, to estrange the good--will of all his own people. (3) Therefore, having encouraged the Haedui for the purpose of supplying the provisions, he sent ahead to the Boii <messengers>, that they tell of his coming and encourage <the Boii>, in order that <the Boii> remain in good faith and resist the attack of the enemy with great courage.} Whereas the deliberation at \emph{Bellum~Gallicum} 1.7.4--6 goes from decision to reasons defending it, this moves from problem to possible solutions. Hypotaxis works for the development of a sophisticated decision--making process. The piece begins with the announcement that Caesar has a major problem (\textquote{\textlatin{magnam \ldots{} difficultatem}}) that makes deliberation difficult (\textquote{\textlatin{ad consilium capiendum}}), in the form of an apodosis of a contrary--to--fact conditional statement. \begin{latin} \begin{enumerate} \SingleSpacing
\item Problem: \textlatin{Magnam haec res Caesari difficultatem ad consilium capiendum adferebat,}
  \begin{enumerate}
  \item Option: \textlatin{si reliquam partem hiemis uno loco legiones contineret,}
    \begin{enumerate}
    \item Drawback: \textlatin{ne \ldots{} cuncta Gallia deficeret,}
      \begin{enumerate}
      \item Circumstance: \textlatin{stipendiariis Haeduorum expugnatis}
      \item Reason: \textlatin{quod nullum amicis in eo praesidium videret positum esse;}
      \end{enumerate}
    \end{enumerate}
  \item Option: \textlatin{si maturius ex hibernis educeret,}
    \begin{enumerate}
    \item Drawback: \textlatin{ne ab re frumentaria duris subvectionibus laboraret.}
    \end{enumerate}
  \end{enumerate}
\end{enumerate} \DoubleSpacing \end{latin} Caesar considers two solutions to this problem, each of which are expressed as protaseis to a conditional sentence (options a) and )b). The indicative apodosis makes the problem vivid and conveys that it is already at hand.\footnote{\textquote{In the apodosis of a condition contrary to fact the past tenses of the Indicative may be used to express what was \emph{intended}, or \emph{likely}, or \emph{already begun}. In this use, the Imperfect Indicative corresponds in time to the Imperfect Subjunctive \ldots{}} \textcite[§517.b., p.~329]{greenough1903}.} The imperfective \textquote{\textlatin{adferebat}} reflects the prolonged nature of Caesar's deliberation, like many other verbs of deliberation. In the course of considering these two options, Caesar mounts factual evidence and, from the evidence, draws logical inferences. The text gives evidence why each could potentially turn out badly. Subordinated, in grammar and argument, to the options a) and b) are two fearing clauses which express a drawback to each option (\textquote[][,]{\textlatin{ne \ldots{} deficeret}} \textquote{\textlatin{ne \ldots{} laboraret}}).\footnote{These are in fact both fearing clauses, signified by \textquote{\textlatin{difficultas}}: \textquote[{\cite{kranerdittenbergermeusel1967}}]{\textgerman{\emph{ne deficeret} von einem in \emph{difficultatem adferebat} liegenden Begriff der Furcht abhängig, weil er befürchten mußte, daß --}}. For a similar sentence, see 7.35.1 (\textquote{\textlatin{in magnis difficultatibus res, ne \ldots{}}}). On fearing clauses without verbs of fearing, see \textcite[146]{woodcock1959}.} The fearing clauses here are clearly focalized through the actor Caesar, underscoring that this thinking is done by the general and not the narrator. Why Caesar fears the first option, of not venturing out, is given some further elaboration. First, a condition (\textquote{\textlatin{stipendiariis \ldots{} expugnatis}}) is supposed for the sake of deliberation. Then, the text explains why the Gauls would revolt (\textquote{\textlatin{quod \ldots{} esse}}). The result, that Caesar has a very difficult decision to decide upon, remains unchanged, though the reason why is explained, based on what appears a reasonable general fact, that allies expect protection from their \textquote{\textlatin{amici}} (\textquote{friends}). If and when this expectation (\textquote{\textlatin{stipendiariis \ldots{} expugnatis}}) is broken, his alliances with pacified Gauls could be eroded. The proof of 1.b) is much less elaborated, leaving it to be inferred that grain is in short supply in this region of Gaul at that time of year. In considering the two possible courses of action, Caesar mounts factual and inferred evidence to construct two different conditional suppositions. Including fear into Caesar's deliberation, the \emph{Bellum~Gallicum} acts not entirely dissimilarly to how \textcite{lendon1999} characterizes Caesar's battle descriptions as distinct from the Xenophon's and Polybius's, in that Caesar ascribes the utmost importance to soldiers' psychology and bravery. %Myles: ``Polybius on soliders' bravery -- See ration Polyb 10.11.18, 3.64.11, 3.109.5, 15.14.5; cf. 10.13.10 hormh
 To himself, too, Caesar allows room for the causative importance of emotion (or fear, at least) in the explanation of military affairs.\footnote{For summary of \textcite{lendon1999}, see page~\pageref{lendon-sum}.}

Next, the deliberation moves beyond separation of options to the direct comparison of them to one another.  \begin{latin} \begin{enumerate} \SingleSpacing
\item Comparison of option a: \textlatin{praestare visum est tamen omnes difficultates perpeti,}
  \begin{enumerate}
  \item Option b: \textlatin{quam \ldots{} omnium suorum voluntates alienare.}
    \begin{enumerate}
    \item Circumstance: \textlatin{tanta contumelia accepta}
    \end{enumerate}
  \end{enumerate}
\end{enumerate} \DoubleSpacing \end{latin} At \emph{Bellum~Gallicum} 7.10.2, Caesar compares option b) (\textquote{\textlatin{praestare \ldots{} perpeti}}) to option a) (\textquote{\textlatin{quam \ldots{} alienare}}). One last piece of circumstantial information is offered as to why option a is a less desirable choice: because the Romans would be at the receiving end of Vercingetorix's \textquote{insult} (\textquote{\textlatin{contumelia accepta}}). At \emph{Bellum~Gallicum} 1.7.4--6, the reasoning is really a defense of a decision that has been made. In this deliberation, the problem is placed up front, and final decision is delayed. The effect is an especially strong impression that Caesar strove to make the best decision based on facts and good reasoning. Similarly here, the narrator lays out both courses that Caesar may take, so as to show his clear consideration of the situation.

Caesar's final decision comes with the narrator's explanation of what he did in light of the decision--making. Added to this are several phrases pointing to Caesar's action and intentions. Caesar encourages the Haedui, sends messengers ahead, who are given specific words by which to encourage the Boii. \begin{latin} \begin{enumerate} \SingleSpacing
\item Decision: \textlatin{itaque cohortatus Haeduos de supportando commeatu praemittit ad Boios,}
  \begin{enumerate}
  \item Delegation: \textlatin{qui de suo adventu doceant hortenturque,}
    \begin{enumerate}
    \item Intention: \textlatin{ut in fide maneant atque hostium impetum magno animo sustineant.}
    \end{enumerate}
  \end{enumerate}
\end{enumerate} \DoubleSpacing \end{latin} The  decision (\textquote{\textlatin{itaque \ldots{Boios}}}) is a direct result of the actor Caesar's thought. The \textquote[\textquote{therefore}]{\textlatin{itaque}} expresses a \textquote[\emph{OLD} 1.a.]{result or inference}. \textcite{gildersleeve1895} put the force of \textquote{\textlatin{itaque}} even stronger than inference, but \textquote[§500]{\emph{facts} that follow from the preceding statement}.\footnote{For an idea of how crucial an \textquote{\textlatin{itaque}} can be in the \emph{Commentarii}, see \textcite[186]{raaflaub2009} and \textcite[267]{henderson1996} on how Caesar shows a mendacious tendency in mentioning why he called in his troops from Gaul to Ariminum (\emph{BC}~1.11.4).} Two purpose clauses follow, one embedded within the other. The first is a relative purpose clause (\textquote{\textlatin{qui \ldots{} hortenturque}}). It expresses Caesar's orders to his messengers, which they are to repeat to the Boii, that the Romans were coming in support. The purpose of the messengers' words to the Boii are also elaborated (\textquote{\textlatin{ut \ldots{} sustineant}}). These three sentences at \emph{Bellum~Gallicum} 7.10.1--3 showcase Caesar's well--considered thinking. Here again we can see the basic formal features, such as either/or, comparatives, emotion, and conditional statements. This example and that above of internal deliberation show the particular ways that logical thought appears in the \emph{Bellum~Gallicum}. As a deliberating general, his mastery over difficulties and contingency as a result of the complexity of his rational behavior.

%ex 3: many more examples
Deliberations like the previous two examples are quite common in the \emph{Bellum~Gallicum}. In looking for just several common verbs of thought found in the above examples of the previous two sections, there are ten instances of Caesar thinking with \textquote[][,]{\textlatin{putare}}\footnote{\emph{BG}~1.7.5, 1.14.2, 1.33.2, 1.35.2, 1.46.3, 3.10.3, 5.46.4, 6.2.3, 7.10.1--3, and 7.73.2.} twenty--four of \textquote[][,]{\textlatin{existimare}}\footnote{\emph{BG}~1.7.6, 1.23.1, 1.33.4, 1.37.4, 1.37.4, 1.38.2, 1.41.3, 1.47.2--3, 2.2.5, 4.5.1, 4.6.5, 4.13.3, 4.16.1--4, 4.17.2, 4.22.2, 4.36.2, 5.24.6, 5.49.6, 6.1.3, 6.5.5--6, 7.7.4, 7.33.1--2 (twice), and 7.54.2.} and three with a form of \textquote[][.]{\textlatin{ad consilium capiendum}}\footnote{\emph{BG}~4.5.1, 4.13.3, and 7.27.1.} Of these, the lengthier deliberations occur at important moments in the narrative, when Caesar must decide whether or not to engage in a significant military venture, for instance against the Helvetii (\emph{BG}~1.7.4--6), Suebi (1.33.2--5 and 1.37.4), Belgae (2.2.5), Veneti (3.10.1--3), Germans (4.5.1, 4.13.3, and 4.16.1--4), Britons (4.20.1--2), Eburones (5.46.4), Gauls and Germans (6.2.3), and Gauls under Vercingetorix (7.33.1--2). Caesar usually, but not always, decides to wage war. The most significant example is his deciding not to enter Germania at \emph{BG}~4.16.1--4.\footnote{There are some interesting features to this decision not to enter Germany. See \textcite{krebs2006} on Caesar's appeal to geography and Darius's defeat in Scythia (they were lead into a swamp), how Caesar's \emph{celeritas} would end in Germania.} There are several other extended deliberations that do not occur before battle preparations, but in the middle of events.\footnote{\emph{BG}~1.46.3 (not fighting within a meeting with Ariovistus), 4.22.2 (considering leniency for Morini), 6.1.1--3 (on Italian levies in general preparation for war), 7.10.1--3 (on whether Gauls will revolt), and 7.54.2 (reflecting on Haedui deceit).} These mid--campaign deliberations act as confirmations of the initial decision to go to war, thus further justifying Caesar's partially-implemented plan (e.g., \emph{BG}~7.54.2). Overall, internal deliberations explain why Caesar must wage war in Gaul the way that he does.

%position -- rework, perhaps; this was originally meant to follow the apologetic deliberations part
A brief look at the position of these deliberations within larger narratives show that the stories move the story along by giving persuasive reasons why he must take up a particular campaign. For example, in Caesar's first deliberation about the Suebi, he considers (\textquote[][,]{\textlatin{putaret}} \textquote[][,]{\textlatin{arbitrabatur}} etc.) the dangers that the Germans pose to Italy.\footnote{See section~\ref{bg-concilium} for a close reading, and on thinking language in particular, footnote~\ref{bg-1-33}.} With the conclusion to the deliberation, that Ariovistus could not be ignored (\textquote{\textlatin{ut ferendus non videretur}} \emph{BG}~1.33.5), expansion into the northern Gaul is justified. Another critical moment in the text comes early in book 3, where Caesar faces a rebellion led by the Veneti. The beginning of this deliberation weighs his options. \blockquote[\emph{BG}~3.10.1]{\textlatin{Erant hae difficultates belli gerendi, quas supra ostendimus, sed multa tamen Caesarem ad id bellum incitabant.}} \blockquote[\emph{BG}~3.10.1]{These were the difficulties, which we have shown above, of waging war, but nevertheless many things were urging Caesar to this war.} In this deliberation, Caesar decides to move closer to war. The entire deliberation (\emph{BG} 3.10.1--3) shows similarities in vocabulary to the above two examples, such as \textquote[at \emph{BG}~7.10.1--2]{\textlatin{difficultates}}, \textquote[at \emph{BG}~1.7.6]{\textlatin{tamen}}, \textquote[at \emph{BG}~1.7.5]{\textlatin{iniuria}}, \textquote[at \emph{BG}~7.10.3]{\textlatin{itaque}}, \textquote[at \emph{BG}~1.7.4]{\textlatin{putavit}}, and two gerundives (\textquote{\textlatin{partiendum}} and  \textquote{\textlatin{distribuendum}}). There are similarities of thought, such as Caesar considering their reaction through the eyes of Gauls: \textquote[\textquote{most of all he feared lest, if this part <of Gaul> were neglected, then the other nations would think they could to the same} \emph{BG}~3.10.2]{\textlatin{in primis ne hac parte neglecta reliquae nationes sibi idem licere arbitrarentur}}. Another similarity to the deliberations in the two lengthy examples (\emph{BG}~1.7 and~7.10) is that they are well--reasoned acts of persuasion. The Gauls, Caesar recalls, have a history of injury to Roman citizens (\textquote{\textlatin{iniuria retentorum equitum Romanorum}} \emph{BG}~3.10.3), rebelling after surrendering (\textquote{\textlatin{rebellio facta post deditionem}}), and conspiracy (\textquote{\textlatin{defectio datis obsidibus}}). Caesar considers another reason, a consideration of other Gauls' perception of events (\textquote{\textlatin{ne \ldots{} reliquae nationes \ldots{} arbitrarentur}}). The confirmations of this reason are two, one based on Gallic nature: \textquote[\textquote{since he knew that nearly all Gauls are eager for revolution and are easily excited to war}]{\textlatin{cum intellegeret omnes fere Gallos novis rebus studere et ad bellum mobiliter celeriterque excitari}}; and another on universal human nature: \textquote[\textquote{however all men by nature are incited with a zeal for liberty, and hate the condition of servitude} \emph{BG}~3.10.3]{\textlatin{omnes autem homines natura libertatis studio incitari et condicionem servitutis odisse}}. Finally, the result of the deliberation comes last (\textquote{\textlatin{itaque \ldots{} putavit}}). The deliberative qualities of these arguments remain fundamental to the way that Caesar's venturing into and conquest of Gaul is explained by the \emph{Bellum~Gallicum}.

In this chapter's conclusion, next, I will to tie together the significance of Caesar's adoption of the deliberating hero.

\section{Conclusion}
\label{dec-conc}
%chapter summary
Homer, Polybius, and Caesar do something similar with their heroes. They portray their leading men, partially or exclusively military commanders, as experts of the sword and mind. The deliberations by Achilles, Zeus, and Odysseus share important formal characteristics, like being perplexed, declaring in language their problem to their \emph{thumos} (or similar), considering two or more options, noting in subordinate clauses consequences of each option, and often expressing a final decision with a comparison expressing what course of action is best. Many of these same formal features appear  in Caesar's deliberations in the \emph{Bellum~Gallicum}. So too do some of the particular deliberative habits of the clever Odysseus, who persistently scrutinizes the horizon for possible threats and opportunities for gain. In this particular respect, I considered how Polybius extols these qualities of Odysseus to such an extent that he embeds these actions into the heart of the \emph{Histories}, of which both ideal and real commanders (e.g., Hannibal and Scipio) and ideal and real historians (i.e., himself) exemplify the qualities. Then, I showed that this obsessive attentiveness and venturing spirit is alive and well in the \emph{Bellum~Gallicum}'s hero, the \textquote{Caesar} character of its narratives. These two aspects of deliberation -- the logical consideration of options and venturing spirit -- appear in actor Caesar and implicit narrator Caesar. Divergences in this tradition of the deliberating hero -- author, audience, genre, type of leader, constitution of army, etc.  -- are perhaps without number. But one similarity across the texts is that these deliberations express to its readers the thinker's mastery over circumstances.\footnote{For an important work on the influence of Homeric tactics upon actual Roman battle tactics, see \textcite{lendon2005}.} Like an Odysseus, we observe the actor Caesar using foresight, making plans, and confronting dilemmas with critical thought. We know, of course, that Caesar is not only an actor but also a teller of stories about himself. As such, he has a sort of control over the narrative destiny of the version of himself in the story. Like Zeus, to some degree Achilles, and Polybius, the narrating Caesar has control over the plot of his own work. These different parallelisms amount, in the \emph{Bellum~Gallicum}, to two complimentary forms of mastery.

I would like to consider a little more the message of Caesar's deliberative command. I will here consider several conclusions that may be drawn from this chapter: first, as an analogy of Caesar's deliberations to stasis theory; second, as a demonstration of a disciplined mind; and third, the historiographical authority brought to the text through the \emph{Bellum~Gallicum}'s adoption of the deliberating hero. 
 %1: stasis theory (1/2)
 First, Caesar, in his moments of internal deliberation, is alone, without any viewers upon him nor any to listen to him, save the readers of the text. Readers cannot engage or interact with these deliberations, but merely observe fair and balanced hypotaxis. Caesar models good thinking for his readers. In the \emph{Bellum~Gallicum}, the act of deliberative weighing, essential to intersubjective communication, lacks the messy human problems of conflict, compromise, and mutual understanding. The utility of these deliberations is not that they to lead us to the best answer, but to a belief that Caesar's private deliberations are sufficient for coming to what otherwise may be done in a plurality. In this manner, the two--part mastery of author and actor might be profitably understood analogously to \emph{stasis} theory, which is laid out in treatises like the anonymous \emph{Ad Herennium} and Cicero's \emph{De inventione}, as a system which aids an orator in best forming a persuasive argument around particular set of ideas.\footnote{On \emph{stasis} theory and ancient passages, see \textcite[39--42][on Aristotle]{solmsen1941a}, \textcite[172--178][on post--Aristotelian rhetorics]{solmsen1941b}, \textcite[pp.~13--29, 93--100, and 130--132][on Aristotle, the Hellenistic period, and Cicero, respectively]{wisse1989}, \textcite[29--32][on \emph{Ad~Her.}~and \emph{De~inv.}]{corbeill2002}, \textcite[43--49][especially on Aristotle, Isocrates, and Hermogenes]{habinek2005}, \textcite[69--76][on the \emph{Ad~Her.}~and \emph{De~inv.}]{connolly2007}.} On Roman rhetorical texts, \textcite[69--76]{connolly2007} considers Roman rhetorical texts as facets of political theories, about how to achieve political unity and the harmonious running of the Roman republic: \emph{stasis} theories \textquote[{\cite[72]{connolly2007}}]{foreground the capacity of language to resolve disputes that could otherwise divide and damage the political community}. On display, I suggest, in the \emph{Bellum~Gallicum}'s portraits of rationality is an argument that Caesar has a similar capacity to reconcile opposing sides, be they conflicting demands of a situation or political factions.

%stasis theory (2/2)
Not forcing us, his readership, to accept his decisions, the rhetoric of Caesar's deliberations persuade his audience that the best possible decision has been made. The qualities of his deliberations perhaps signal more than a justification of his activity in Gaul but also of his leadership style. Being aware of history as it unfolded after the composition of the \emph{Bellum~Gallicum} (i.e., Caesar becoming \emph{\textlatin{dictator perpetuus}} and partial founder of the Pricipate), it is easy to speculate that the work functions to persuade his Roman readership of his political acumen, remarkable not for his ability to advocate for one partisan faction but judiciously see all sides of a debate, process their consequences, and make the best possible decision. \blockquote[{\cite[74]{connolly2007}}]{In their sketch of schemas of possible, observable, demonstrable, plausible knowledges, \emph{Rhetorica ad Herennium} and \emph{de Inventione} reveal the will to bridge the gap between \textquote{consensual knowledge} -- the public agreement about what words mean and how their use relates to decisions in the world -- and the rationalizing drive of philosophy.} In Caesar's modeling of good thought, an implicit message appears to his readers that they do not need their current political system and all its deliberative messiness. The intersubjective institution of the senate has been collapsed into this one exceptional person, whose deliberative action could promise to set the republic on the right course, for the right reasons, and according to the right plan.

%2: disciplining the mind
Second, through Caesar's showpieces of deliberative brilliance, the \emph{Bellum Gallicum} also highlights its hero's discipline of mind, another trait rhetorical treatises promoted. While the correct use of \emph{staseis} will regulate political discourse, a pedagogical handbook gives opportunity for its student to master his own mind \parencite[45--46]{habinek2005}.\footnote{\textquote[{\cite[45]{habinek2005}}]{The distinct impression left by the surviving treatises is that the production of a tekhne or ars is itself a kind of rhetorical performance, a demonstration of mastery either on the part of a pupil seeking to enter the ranks of practitioner (e.g., Cicero with \emph{De Inventione}) or on the part of a teacher trawling for pupils}.} On Roman handbooks, \textcite[71]{connolly2007} writes that this \textquote{disciplining the mind and voice} aims to join eloquence to a better regulated civic sphere.\footnote{\textquote[{\cite[71]{connolly2007}}]{Not surprisingly, then, law enforcement through language is the central theme. Cicero classifies eloquence as “a kind of civic reason,” “part of civil science” (\emph{\textlatin{ratio civilis quaedam}}, \emph{\textlatin{civilis scientiae pars}}, \emph{Inv.}~1.6). The Auctor of the \emph{\textlatin{Rhetorica ad Herennium}} promises to cover only matters that are relevant to the ratio or system of public speaking (\emph{\textlatin{pertinere}}, 1.1.2), which deals with practices proper to citizenship and consensus (\emph{\textlatin{ad usum civilem \ldots{} cum adsensione auditorum}}, 1.2.2)}.} As with an implicit argument that he is able to make the best decision between factions, Caesar's masterful deliberating also showcases the discipline of his mind and thus how he is able to consistently arrive at these decisions, not only due to (perhaps) proper training but a deep structure of a well trained mind.

Third and finally, I will consider several ways that the deliberating hero trope impacts the authoritativeness of the \emph{Bellum~Gallicum}'s version of events. A certain omniscience of the narrator becomes implied through the methodical inspection of situations by the deliberating actor Caesar. That is, we see the general acting like an historian, and we as readers project the values of the deliberating actor onto the putatively objective narrator. The deliberating actor Caesar therefore may lend authority to the \emph{Bellum~Gallicum}'s version of events, as he adopts more than just a \textquote{been there, done that} pose to lend his work authority, but one of \textquote{been there, deliberated that.} As Polybius lends his narratives authenticity by inspecting events first--hand (and in the first person), Caesar likewise considers events, only for him in the third person as an actor within the story. With Rome flush with versions of events in Gaul (evident in, for example, Ariovistus's communication with Roman aristocrats and Q.~Cicero's extant letters to his brother), the deliberating general and author work together for the \emph{Bellum~Gallicum}'s establishment of a truthful account of the Gallic wars.\footnote{For more on Caesar's and contemporaries' official communications (e.g., Cicero's proconsular letters from Cilicia, \emph{Ad~fam.}~15.1 and 15.2) with Rome, see \textcite[338--349]{osgood2009}.} Of the examples above, the Polybian elision of author and actor is perhaps most apparent when the actor Caesar reflects upon the Helvetii in 107 B.C. (\emph{BG}~1.7.4--5) and thus becomes an historian himself. Because of the elision of actor and author in the \emph{Bellum~Gallicum}, the deliberating general becomes an embodiment of the historian's mind at work, reducing obligations for the historian to do any first--person deliberation himself. My observations here reflect and are informed by the subject of Chapter~\ref{ch-pres} of this dissertation, that there is a bifurcation of a narrating Caesar that reveals shrewd thinking inside the mind of the outwardly naive actor Caesar. Holding the dual roles of historian (as author and implicit narrator) and commander, Caesar transmits a message that he has critically thought through events, and the dry objectivity of the narrator's textual style may thus be preserved.

The importance of these deliberations by Caesar become more evident when considered in the next chapter, how this ability to think is uniquely his in the \emph{Bellum~Gallicum}. As crucial as Caesar's deliberation is to the Roman army, equally so is the army's unthinking obedience. In one particular breakdown of the army's system of communication, command, and obedience, the text offers us valuable insights into the inner mechanics of its unthinking Roman army.

\begin{comment}
%cut from Polybius section
%14.1.4--5
%dlcorr: This looks almost there, but still needs some work on the details, in my view.  The biggest problem is in the new Scipio section.  The trouble with the passage that you quote from Polybius 14 is that it is more complex than you imply, and the reasoning isn't wholly to Scipio's credit.  Your structure obscures the fact that he is deliberating about two entirely separate matters - (i) the possibility of detaching Syphax from the Carthaginian alliance; and (ii) whether there might be an opportunity to capture his camp.  Moreover, Scipio's reasoning on (i) is entirely ill-founded, despite his confidence:
%Walbank: 14.1-10 `Scipio in Africa'
Another example of Scipio deliberating comes at \emph{Histories}~14.1.4--5, when, in Africa, he considers how best to persuade Syphax to return into alliance with Rome. \begin{greek} 
\begin{enumerate} \SingleSpacing 
\item (4) οὐ γὰρ ἀπεγίνωσκε 
  \begin{enumerate}
  \item καὶ τῆς παιδίσκης αὐτὸν ἤδη κόρον ἔχειν, 
    \begin{enumerate}
    \item δι' ἣν εἵλετο τὰ Καρχηδονίων,
    \end{enumerate}
  \item καὶ καθόλου τῆς πρὸς τοὺς Φοίνικας φιλίας 
    \begin{enumerate}
    \item διά τε τὴν φυσικὴν τῶν Νομάδων ἁψικορίαν 
    \item καὶ διὰ τὴν πρός τε τοὺς θεοὺς καὶ τοὺς ἀνθρώπους ἀθεσίαν.
    \end{enumerate}
  \item (5) ὢν δὲ περὶ πολλὰ τῇ διανοίᾳ 
  \item καὶ ποικίλας ἔχων ἐλπίδας ὑπὲρ τοῦ μέλλοντος 
    \begin{enumerate}
    \item διὰ τὸ κατορρωδεῖν τὸν ἔξω κίνδυνον 
      \begin{enumerate}
      \item τῷ πολλαπλασίους εἶναι τοὺς ὑπεναντίους,
      \end{enumerate}
    \end{enumerate}
  \item ἐπελάβετό τινος ἀφορμῆς τοιαύτης.
  \end{enumerate}
\end{enumerate} \DoubleSpacing \end{greek}
\begin{enumerate} \SingleSpacing 
\item (4) For he did not depart from his judgment that %οὐ γὰρ ἀπεγίνωσκε 
  \begin{enumerate}
  \item he had tired of the girl, %καὶ τῆς παιδίσκης αὐτὸν ἤδη κόρον ἔχειν, 
    \begin{enumerate}
    \item on account of whom he chose the Carthaginians %δι' ἣν εἵλετο τὰ Καρχηδονίων, 
    \end{enumerate}
  \item and generally <tired of> the friendship of with the Phoenicians %καὶ καθόλου τῆς πρὸς τοὺς Φοίνικας φιλίας 
    \begin{enumerate}
    \item both on account of the fickle nature of Numidians %διά τε τὴν φυσικὴν τῶν Νομάδων ἁψικορίαν 
    \item and on account of their faithlessness to gods and men. %καὶ διὰ τὴν πρός τε τοὺς θεοὺς καὶ τοὺς ἀνθρώπους ἀθεσίαν.
    \end{enumerate}  
  \item (5) Since he was <thinking> about many things in his mind %ὢν δὲ περὶ πολλὰ τῇ διανοίᾳ (`since his mind was much distracted' -- Walbank) 
  \item and had various hopes about what was to come %καὶ ποικίλας ἔχων ἐλπίδας ὑπὲρ τοῦ μέλλοντος
    \begin{enumerate}
    \item on account of fearing a battle in the open country %διὰ τὸ κατορρωδεῖν τὸν ἔξω κίνδυνον (`a battle in the open country' -- Walbank)
      \begin{enumerate}
      \item because of the enemy's greater numbers opposed, %τῷ πολλαπλασίους εἶναι τοὺς ὑπεναντίους,
      \end{enumerate}
    \end{enumerate}
  \item he laid hold of an opportunity of the following sort. %ἐπελάβετό τινος ἀφορμῆς τοιαύτης.
  \end{enumerate}
\end{enumerate} \DoubleSpacing 
 As with the examples of deliberation in Homer, this brief deliberation is full of thinking words (\textquote[][,]{\textgreek{ἀπεγίνωσκε}} \textquote[][]{\textgreek{ὢν δὲ περὶ πολλὰ τῇ διανοίᾳ}} and \textquote{\textgreek{καὶ ποικίλας ἔχων ἐλπίδας}}); adverbs, particles, and prepositions that structure and balance the thought (\textquote[][,]{\textgreek{γὰρ}} \textquote[][,]{\textgreek{δι' ἣν}} \textquote[][,]{\textgreek{διά τε \ldots{} καὶ διὰ}} and \textquote{\textgreek{διὰ τὸ κατορρωδεῖν}}); and causal participles (\textquote{\textgreek{ὢν δὲ \ldots{} καὶ ποικίλας ἔχων}} \cite[426]{walbank1967}). As Odysseus looks to the future, so too does Scipio (\textquote{\textgreek{τοῦ μέλλοντος}}), plus something similar in his hopes (\textquote{\textgreek{ἐλπίδας}}), which indicate a future-- and goal--oriented  mental activity. Finally, the activity of Scipio's mental activity is evident in the active voice verb which begins the deliberative piece (\textquote{\textgreek{ἀπεγίνωσκε}}). His thought shows hallmarks of rational calculation, such as a summoning of evidence (1.a) and 1.b)), general principles which support the validity of the evidence (e.g., the ethnic stereotypes of 1.b).i. \& ii.), cause-and-effect relationships of the evidence and conclusions (\textquote{\textgreek{ὢν δὲ περὶ πολλὰ τῇ διανοίᾳ \ldots{} ἐπελάβετό}}). Finally, there is an emotional stimulus which spurs some of his thinking (\textquote{\textgreek{διὰ τὸ κατορρωδεῖν}}), though Scipio's fear is not put forward as clearly as Homeric heroes' anxiety (\textquote{\textgreek{ὀχθήσας}}) or fear clauses by Caesar. Here considering whether and why to continue his pursuit of Syphax's friendship, Scipio's thinking here conforms to the form of deliberative thought seen above in Homer and below in Caesar. %make sure that this is corrected
 Other examples of Scipio's deliberations show similar characteristics, like the presence of thinking terms (\textquote{\textgreek{ἡγεῖτο}} 10.39.9; \textquote{\textgreek{νομίσας}} 14.1.8), fear (\textquote{\textgreek{τῷ δεδιέναι}} 10.39.9), looking words (\textquote{\textgreek{θεωρῶν}} 11.22.2), and related adverbs (e.g., \textquote{\textgreek{γὰρ}} and \textquote{\textgreek{αὐτὸς}} 11.22.2; \textquote{\textgreek{οὖν}} 14.1.8). This brief survey of two of text's showcase of Scipio's internal deliberation shows in action what Polybius means in writing that Scipio was smarter than lucky; what \textcite{poznanski1994} means about Scipio's agency; and \textcite{pedech1964} and \textcite{podes1990} of \textquote[][.]{\textfrench{héro raisonnable}}
\end{comment}

\chapter{External deliberation}%3333
\label{ch-delib-subs}% meier1995 p. 244--6 on Roman soldiers and their emotions

In a discussion of master--servant relations in the \emph{Politeia}, Aristotle indulges in a moment of fantasy, where man--made tools are imagined as sorts of automata that both understand their master's will and are able to act without any human effort. \blockquote[\emph{Politeia} 1.2.5, 1253b]{\textgreek{εἰ γὰρ ἠδύνατο ἕκαστον τῶν ὀργάνων κελευσθὲν ἢ προαισθανόμενον ἀποτελεῖν τὸ αὑτοῦ ἔργον, \ldots{} οὐδὲν ἂν ἔδει οὔτε τοῖς ἀρχιτέκτοσιν ὑπηρετῶν οὔτε τοῖς δεσπόταις δούλων.}} \blockquote[\emph{Politeia} 1.2.5, 1253b]{If all tools, ordered or understanding orders, were able to accomplish their work, \ldots{} then there would be need neither of servants by directors nor need of slaves by masters.} In the world as it is, servants and slaves physically perform what their directors and masters intend. In this passage, an automaton would perform a master's mind without any need for commands to be ordered.\footnote{Inside the ellipsis: \textquote[\textquote{just as the statues of Daedalus or tripods of Hephaestus, which \textquote[][,]{automatons} the poet says, \textquote[][,]{entered the assembly of the gods} as if the shuttle separated the wool and picks plucked the lyre}]{\textgreek{<καὶ> ὥσπερ τὰ Δαιδάλου φασὶν ἢ τοὺς τοῦ Ἡφαίστου τρίποδας, οὕς φησιν ὁ ποιητὴς αὐτομάτους θεῖον δύεσθαι ἀγῶνα, οὕτως αἱ κερκίδες ἐκέρκιζον αὐταὶ καὶ τὰ πλῆκτρα ἐκιθάριζεν}}. Text from \textcite{ross1957}.} The laborer seems like an afterthought or unfortunate necessity of the fulfillment of an idea (\textquote{\textgreek{ἀποτελεῖν τὸ αὑτοῦ ἔργον}}). What binds these classes of actors is a two--part process: First, masters give orders (\textquote{\textgreek{κελευσθὲν}}) and, second, slaves interpret them (\textquote{\textgreek{προαισθανόμενον}}). Caesar's forces are partially like Aristotle's automata: on the one hand, they must be commanded (like a traditional slave), but on the other hand (like the automata) they (with a few noteworthy exceptions) unthinkingly do what they are told. 

Normally, in the \emph{Bellum~Gallicum}, dissemination of the master's plans occurs without a hitch, with subordinate actors doing what they are told (Chapter~\ref{ch-pres}). In the occasional instances when communication breaks down, however, subordinate officers must decide what to do without explicit orders from their general. An interesting feature of the \emph{Bellum~Gallicum} is that these subordinates, when forced to make decisions on their own, often choose to do so collaboratively with other lead officers. Instead of mimicking Caesar's internal deliberative methods (Chapter~\ref{ch-delib}), deliberation is distributed among actors. In these deliberations, the individual officers line up on two sides of a debate, thus embodying the two--sided considerations that Caesar has been shown to make when debating with himself. 

For example, early in book~3, Caesar assigns Ser. Sulpicius Galba to the Alps, so as to keep it open to traders (\emph{BG}~3.1.1--2). Before leaving, Caesar has also authorized some specific decision--making responsibility to him, that he may decide whether to winter the troops in the mountains, \textquote[\textquote{if he thought it necessary} \emph{BG}~3.1.3]{\textlatin{si opus esse arbitraretur}}. Galba does decide to winter there, initially thinking nothing is to be feared (\textquote{\textlatin{nihil de bello timendum existimaverat}}).\footnote{\textquote[][.]{He had thought nothing of war ought to be feared} For the similarity, to Caesar's thought process, of this construction, of \textquote{\textlatin{existimare}} with a gerund see \emph{Bellum~Gallicum}~1.7.} But after only two days in camp, he realizes that the Gauls have revolted and that his troops were under serious threat because the fort's defenses were incomplete and there was a lack of grain (\emph{BG}~3.3.1). Needing to act fast, Galba quickly calls a \emph{consilium}, in which \textquote[\textquote{he began to ask for opinions} \emph{BG}~3.3.1]{\textlatin{sententias exquirere coepit}}. As often with Caesar's deliberations (e.g., \emph{BG}~7.10.1--3), there are two sides to a debate.\footnote{Perhaps significantly, Caesar's deliberation at \emph{Bellum~Gallicum} 7.10.1--3 is somewhat similar in subject, in their both being about whether to leave or remain in winter quarters.} For each, factual evidence is offered: the Gauls had retreated to high in the mountains and there was no hope for support or supplies for the Romans (\emph{BG}~3.3.2). Some were of the opinion that they should flee and march to safety (\textquote{\textlatin{nonnullae \ldots{} sententiae}} \emph{BG}~3.3.3), while the majority (\textquote{\textlatin{maiori parti}} \emph{BG}~3.3.4) chose to defend the camp. Evidently a majoritarian event, the decision stands and a defensive battle is then fought (\emph{BG}~3.4--6.5). The battle is successful, though the narrator comments that afterward \textquote[\textquote{Galba did not want to tempt fortune often} \emph{BG}~3.6.4]{\textlatin{saepius fortunam temptare Galba nolebat}}. The implication of this seems, that, while having been successful, his success was not due to his good judgment. In book~4, Q.~Cicero's success after some dubious decision--making is mostly ascribed to \emph{fortuna} (\emph{BG}~6.42.1).\footnote{See \textcite[95--96]{riggsby2006} on Caesar's reprimand of Q.~Cicero.} Galba even reflects (\textquote{\textlatin{meminerat}}) on how he came into the situation with one plan (\textquote{\textlatin{alio consilio}}), but found an unforeseen reality (\textquote{\textlatin{aliis occurrisse rebus videbat}} \emph{BG}~3.6.4). Since prescience is one of Caesar's strongest qualities in the \emph{Bellum~Gallicum} \parencite[pp.~193 and 248 n.~5]{riggsby2006}, and since he never has any regrets for his decisions, it seems that there are very different expectations for how Caesar and his subordinates make decisions. On the one hand, a subordinate should mimic, in that Galba should have had better foresight and planning; but there is a different expectation about how, as a subordinate officer, he should make decisions.

Though they are not common in the \emph{Bellum~Gallicum}, instances of subordinates' independent decision--making are noteworthy because they demonstrate that there is a special deliberative protocol which applies to all but the general. There are five examples of collaborative, independent decision--making performed in Caesar's communicative absence by subordinate officers. The\label{del-list} deliberations occur at the following: \emph{Bellum~Gallicum} 3.3 (under Galba's leadership, 3.1--6); 3.23.8 (under P. Licinius Crassus's leadership, 3.20--26); 5.28.2--31.3 (led by Cotta and Sabinus, 5.26--37); 6.36 (under Q.~Cicero, 6.36--42); and 7.60.1 (led by Labienus, 7.57--62).\footnote{Q.~Cicero elsewhere reads a letter of Caesar's \textquote[\emph{BG}~5.48.9]{\textlatin{in conventu militum}} though there is no problem addressed. Labienus in his \emph{consilium} does not in fact resolve any dispute. In the \emph{Bellum~civile}, there are two major episodes in which subordinates deciding independently, 2.1--15 (by C. Trebonius) and 2.23--42 (Curio).} In all of these examples, it is evident that communication with Caesar is impossible or not feasible in face of imminent threat. For Galba and Crassus, Caesar has already departed to Italy (\emph{BG}~3.1.1); for Cotta and Sabinus the threat is too immediate (in two days 5.27.8) and, moreover, some think Caesar had already returned to Italy (5.29.2); for Q.~Cicero, Caesar is presumed to be very distant (6.36.1); and for Labienus, Caesar is a ways off, somewhere between Gergovia and the land of the Senones (7.51.1; cf. 7.53.4). In all but one (Q.~Cicero's), the group meeting is called a \emph{consilium}. In Labienus's \emph{consilium}, there is no disagreement spoken by officers, so thus no deliberation. Otherwise, all these episodes are quite similar. In addition to these deliberations, there are several other examples of imperfect communication with Caesar in which the officers face difficult decisions, but \emph{consilia} are not triggered.\footnote{Q.~Cicero's command in book~5 (\emph{BG}~5.39--52) is unique in that his communications with Caesar are for a period difficult, but never so bad that a \emph{consilium} is called. Q.~Cicero's primary decision, done independently, is to reject out of hand the same lie (for this, see section~\ref{cotta-sabinus}) that Ambiorix brings to Cotta and Sabinus: \textquote[\textquote{it was not the custom of the Roman people to accept any terms from an armed enemy} \emph{BG}~5.41.7]{\textlatin{non esse consuetudinem populi Romani ullam accipere ab hoste armato condicionem}}. This statement is very close to what Caesar says to enemies elsewhere (\emph{BG}~2.32.1 and 4.13.1). Likewise, Labienus, in intermittent communication with Caesar (\emph{BG}~5.46--58 \emph{passim}), decides upon certain actions himself, with no \emph{consilium} necessitated.} This demonstrates that Caesar's communicative absence must be significant if a \emph{consilium} is to be called. There is even one instance of panicked soldiers making a collaborative decision, choosing between two options.\footnote{They debate two options.: \textquote[\textquote{Some thought \ldots{}, others made the case that \ldots{}. The veteran soldiers did not approve this} \emph{BG}~6.40.2--4]{\textlatin{alii \ldots{} censent, \ldots{} alii \ldots{} ferant casum. hoc veteres non probant milites}}. These arguments contain adverbs common in arguments, like \textquote[][,]{\textlatin{ut}} \textquote[][,]{\textlatin{quoniam}} and \textquote[][.]{\textlatin{ita}} For the semblance of these arguments to deliberative proofs, see \emph{Bellum~Gallicum} 1.7. Their decision--making is faulty, however, because they lack experience (\textquote{\textlatin{nullo \ldots{} usu rei militaris}}).} Roman officers of the \emph{Bellum~Gallicum} operate according to a set of protocols that usually ensure their success when their leader's mind is unavailable.

It should be noted that there is a fair amount of independent action, though what separates them from these examples is that their accomplishment of Caesar's initial delegation is not problematized by the text. For example, Sabinus (\emph{BG}~3.17--19), Q.~Cicero (5.40--48), and Labienus  (book 5.46--58 \emph{passim} and 7.57--62, which, despite the \emph{consilium}, operates without confronting any unresolvable problem). These actors partake in a host of actions, such as moving around (\textquote{\textlatin{Labienus revertitur}} \emph{BG}~7.62.2), making plans (Sabinus's trick, 3.18), addressing soldiers (\textquote{\textlatin{Sabinus suos hortatus}} 3.19.1), giving orders (\textquote{\textlatin{Sabinus signum dat}} 3.19.1), and sending letters (\textquote{\textlatin{mittuntur \ldots{} a Cicerone litterae}} 5.40.1). Labienus at one point even debates with himself.\footnote{\textquote[\emph{BG}~7.59.3]{\textlatin{Labienus \ldots{} sibi capiendum consilium, atque antea senserat, intellegebat neque iam \ldots{} cogitabat}}.} They all operate far from Caesar's physical proximity, yet do not encounter any contingencies which Caesar has not yet accounted for in his orders for them, or they may communicate with their general if necessary (e.g., Q.~Cicero \emph{BG}~5.40.1 and 48.3--9). Unlike these examples of independent action, independent decision--making occurs at moments when Caesar's orders do not account for a situation at hand.

In the instances of independent decision--making by subordinate officers, plans go sev\-ere\-ly badly only once, during \emph{legati} Cotta and Sabinus's stewardship over part of the army. Though things go somewhat awry with Q.~Cicero's camp in book~6, the Cotta and Sabinus episode stands out as unique in the scope of the military disaster (nearly all of an entire legion and five cohorts lost) and amount of text dedicated to the entire episode -- twelve chapters with almost four to the \emph{consilium} itself.\footnote{See \emph{Bellum~Gallicum} for total soldiers, and 5.37.6--7 for extent of Roman deaths.} Compared to any other episodes or meetings led by subordinates, their failure stands out in the \emph{Bellum~Gallicum}. Section~\ref{cotta-sabinus} looks at the former disaster, looking to determine why exactly this deliberation goes wrong. As a unique example of how things are not supposed to be done, this highly contentious \emph{consilium} offers a view into what makes a proper collaborative deliberation work well. I argue that the misstep that Sabinus makes is in not successfully controlling his fear, leading to an emotional incontinence which spreads to his audience. By contrasting  Sabinus's fear to Caesar's during his internal deliberations, it becomes clear that the problem is not fear, but a failure to regulate it.

Due to the thematic importance of collaborative decision--making in Xenophon's \emph{Anabasis}, as well as its likely influence upon Caesar's \emph{Commentarii}, I next turn to Xenophon's autobiographical work (\ref{delib-xen}). This section shows that the \emph{Anabasis} may be a model for the \emph{Bellum~Gallicum} in how a leader's mind may completely control an army, though in each text the leader's plans are communicated differently. I consider several noteworthy convergences and divergences of collaborative decision--making and philosophies of leadership. Leadership and deliberation in the \emph{Anabasis} may inform the template for these matters in the \emph{Bellum~Gallicum}, but they become expressed in different ways. In the \emph{Anabasis}, decision--making after book~2 is always collaborative, though there is little or no debate within assemblies. In both works, it is one leader's plan that directs the army, but the authorization of plans is different. Plans in the \emph{Bellum~Gallicum}, save the five exceptions of independent deliberation, originate in Caesar's mind, and are then distributed through his army. As a general, Caesar might be understood to posses a quality like \emph{phronesis}, the pre--rational foresight that I discussed at the end of section~\ref{polyb-comm}. If Caesar possesses \emph{phronesis}, then his subordinate officers certainly do not, for they are bid to not generate ideas outside of their general's will, and when they do attempt deliberation without guidance, it is clear that they lack the entire host of skills (\emph{phronesis}, \emph{logos}, emotional modeling) critical to successful deliberation.\footnote{See \textcite[419--426]{yack2006} on Aristotle on political deliberation, to which Yack claims other comments on individual deliberation are applicable (p.~420, cf. \emph{Rhet.}~1357).} Xenophon's deliberative bodies are retained, therefore, in the \emph{Bellum~Gallicum}, but relocated from the center of the network (i.e., a general) to its nodes (i.e., subordinate officers and soldiers).\footnote{For an understanding of Caesar's army as a network in the \emph{Bellum~Gallicum} and how the representation of it as such informs the structure of narrative, consider \textcite[49--56]{rambaud1966}: \textquote[{\cite[51]{rambaud1966}}]{\textfrench{César organise des opérations rayonnantes, les rapports sur l'activité des corps détachés se multiplient, et l'insertion des ces comptes rendus renouvelle la structure des livres}}.} The conclusion (\ref{ind-conc}) expands upon some of the chapter's observations and explains their significance in light of this dissertation's first two chapters, specifically how the text portrays communication and mutual understanding.

\section{Cotta and Sabinus}
\label{cotta-sabinus}
Because of faulty decision--making, L. Aurunculeius Cotta and Q. Titurius Sabinus lead the cohorts under their command to ruin (\emph{BG}~5.26--37). In an unanticipated situation and unable to receive any orders from Caesar, they lead a deliberative assembly that comes to a decision that proves ruinous to the Romans. The failure of this collaborative deliberation is Sabinus's inability to control his emotion, which insidiously spreads through the legion. The episode is revealing of the importance of Caesar's communicative presence, and the protocols by which subordinates should operate when lacking their general's orders. 

Several scholars have written on this episode, mostly focusing on how the attention given to Sabinus's mistakes is done in order to put the blame on the legate instead of general.\footnote{\textquote{The\label{distancing-bibliog} narrative was obviously intended to soften the impact of the disaster and distance Caesar himself from blame. \ldots{} Caesar tried to shift the blame onto his legate, but few if any of his contemporaries were fooled and sources see this as \emph{his} defeat} \parencite[300]{goldsworthy2006}. Situating this episode in the large context of Caesar's general representation of subordinates \parencite[73--76]{adcock1956}, Adcock writes that, generally, \textquote[{\cite[74]{adcock1956}}]{the operations of \emph{legati} are described so that the military quality of their actions, their \emph{consilia}, so far as these are their own and not Caesar's at one remove, can be appraised, but that is all}. Powell writes that the author gave Sabinus's rhetoric a popular tone in order to not put off some of the Optimates class: \textquote{If \ldots{} Caesar has largely made up this dialogue, he has shown Sabinus in a way unlikely to appeal to Optimates, some of whom might otherwise have taken the lead in exploiting the tale of defeat. For Sabinus is inciting the common soldiers against their other legate, Cotta, and ultimately against the orders and reputation of the commander--in--chief, Caesar. Optimates had bad views about this kind of demagogy} \parencite[118--119]{powell1998}.} As an author, Caesar was surely interested in distancing himself from this failure within the army. According to \textcite{riggsby2006}, Cotta and Sabinus both fail because they do not act like Caesar: \textquote{the legates did not provide even a semblance of authority} nor \textquote[{\cite[94]{riggsby2006}}]{provide a model of authoritative leadership}. To the contrary, this section demonstrates that the \emph{legati} are correct in not imitating Caesar and, instead, deliberating as a group. If anyone is to be blamed, it is Sabinus alone; and the reason for this blame is his improper deliberative and communicative practices.\footnote{Concerning blame, the narrator, later in book~5, puts the blame on Sabinus alone. In the episode of the two \emph{legati} (\textquote{\textlatin{de casu Sabini et Cottae}} \emph{BG}~5.52.4), he then address Q.~Cicero's men, telling them \textquote[\textquote{that the loss happened through the fault and rashness of the \emph{legatus}} \emph{BG}~5.52.6]{\textlatin{quod detrimentum culpa et temeritate legati sit acceptum}}. \textcite[p.~235, n.~53]{riggsby2006} acknowledges this passage, but considers it a revision of Caesar's initial opinion.} More importantly, however, this section shows that the real reason for faulty decision--making is the intrusion of unregulated emotion into the deliberative process. Though the emotion of fear plays a crucial role in Caesar's perception and decision--making, it gets out of hand in this episode, so much so that Sabinus ruins the rationality of the minds of those around him.

%2: summary
Caesar tells a story of independent decision--making by subordinates in twelve chapters. The plot develops as follows. Ambiorix and Catuvolcus, Gallic leaders of the Eburones who had previously been friendly to the Romans, are spurred by Indutiomarus of the Treveri into attacking the camp of Sabinus and Cotta, who had earlier  (\emph{BG}~5.24.4--5) been assigned by Caesar to the land of the Eburones (\emph{BG}~5.26). After a brief skirmish, Ambiorix tells the Roman knight C. Arpineius that he is in fact a friend of Caesar and had attacked the camp for reason of internal pressure from his own people and from a general movement in Gaul for freedom (\emph{BG}~5.27). German mercenaries, Ambiorix claims, would arrive in two days' time. In response to this news, Cotta and Sabinus call a \emph{consilium}, at which a heated argument arises (\emph{BG}~5.28.9--31.3). Against Cotta's persuasive arguments to remain in camp, the assembly is worn down by Sabinus's aggressive haranguing, that they should make a run for Q.~Cicero's camp before enemy hordes besiege them. As the Romans flee, they are attacked and a chaotic battle ensues (\emph{BG}~31.4--37). When all seems dire for the Romans, Sabinus, against the recommendation of Cotta, treats for peace with Ambiorix (\emph{BG}~5.36.1--37.2). In this meeting with the Gallic leader, Sabinus is killed (\emph{BG}~5.37.2) and Cotta falls in battle shortly thereafter (\emph{BG}~5.37.4).

%3: consilium/Cotta
While the narrative sets up the dilemma with the same terminology as for Caesar, these two \emph{legati} use a different procedure to approach the problem. They arrange themselves into a \emph{consilium} of leaders of higher ranks, and adhere to particular principles of collaborative and deliberative decision--making. The \emph{consilium} arises in a manner much like Caesar's interior deliberations do elsewhere in the text. First, Cotta and Sabinus hear from a Gaul through intermediate Roman messengers.\footnote{Once Ambiorix's words are passed from C. Arpineius and Q. Junius to Cotta and Sabinus: \textquote[\textquote{Arpinius and Junius reported what they had heard to the legates} \emph{BG}~5.28.1]{\textlatin{Arpinius et Iunius, quae audierant, ad legatos deferunt}}. For examples of messages being reported to Caesar with \textquote[][,]{\textlatin{deferre}} see, in book~5 alone, \emph{BG}~5.6.6, 5.25.3, 5.45.4, and 5.49.2.} In response, we peer into the minds the Romans. \blockquote[\emph{BG}~5.28.1]{\textlatin{Illi repentina re perturbati, etsi ab hoste ea dicebantur, tamen non neglegenda existimabant maximeque hac re permovebantur, quod civitatem ignobilem atque humilem Eburonum sua sponte p. R. bellum facere ausam vix erat credendum.}} \blockquote[\emph{BG}~5.28.1]{Thrown into confusion by the suddenness of the matter, although it was spoken by an enemy, they nevertheless thought that it must not be ignored and were moved by it most of all because it scarcely seemed believable that an idle and humble state of the Eburones would make bold undertakings against the Roman people on their own volition.} Already, we see the hallmarks of Caesar's representation of thought,  explored at \emph{Bellum~Gallicum} 1.7. There is a verb of thinking combined with a gerund (\textquote{\textlatin{non neglegenda existimabant}}) and complex, subordinated thought (\textquote{\textlatin{etsi \ldots{} tamen}}).\footnote{See above at \emph{Bellum~Gallicum} 1.7 (section~\ref{ch2-delib-caesar}, p.~\pageref{ch2-delib-caesar}) for discussion of this formula and its significance to Caesar's deliberations.} Like the emotional language looked at in Chapter~\ref{ch-delib}, the narrator expresses the emotional responses of the \emph{legati} too (\textquote{\textlatin{illi \ldots{} perturbati}} and \textquote{\textlatin{permovebantur}}). There is even, before any debate arises, logical structures explaining why they are afraid (\textquote{\textlatin{quod \ldots{} credendum}}; \textquote{\textlatin{etsi \ldots{} tamen}}). Next, the narrator presents group--based decision--making in the \emph{consilium} as the forum for subordinates to decide what to do: \textquote[\textquote{therefore they deferred the matter to a \emph{consilium} and a great controversy arose among them} \emph{BG}~5.28.2]{\textlatin{itaque ad consilium rem deferunt magnaque inter eos existit controversia}}. Despite the resemblance to the setup to Caesar's deliberations, nobody as an individual thinks about the matter himself, but it is moved to a group assembly. The shades of \emph{consilium}, explored above at~\ref{bg-consilium}, are revealing, too, since the word can refer to the activity at a meeting (\textquote{debate,} \textquote{discussion,} \textquote{deliberation} \emph{OLD} def.~1), what is said at a meeting (\textquote{advice,} \textquote{council} \emph{OLD} def.~2), and the meeting itself (\textquote{a deliberative or advisory body,} \textquote{council} \emph{OLD} def.~3).\footnote{Similarly, \textcite{meusel1887} divides the word according to the act of \textquote[][,]{deliberation}, the \textquote{deliberative assembly} in which deliberation occurs, and the \textquote{counsel} that emerges from such assemblies (defs.~A, B, and C).} Not only will the deliberation include the two leaders, but an unnamed number of other officers too.\footnote{\textcite{kranerdittenbergermeusel1967} emphasize that \emph{\textlatin{inter eos}} means that the meeting is beyond just the two \emph{legati}. \textquote[{\cite{kranerdittenbergermeusel1967}}]{\textgerman{\emph{\textlatin{inter eos}} könnte auf das in \emph{\textlatin{deferunt}} liegende Subject (legati, Sabinus et Cotta) gehn, bezieht sich aber wegen des \emph{\textlatin{magna exsistit controversia}} wohl auf die durch \emph{\textlatin{consilium}} bezeichneten Mitglieder des Kriegsrates}}.} The \textquote{deferring} (\textquote{\textlatin{deferunt}}) implies also the inclusion of officers lower in the chain of command.\footnote{\textquote{To refer for a decision (to), put (before)} (\emph{OLD}, s.~v.~\textquote{\textlatin{defero}}, def.~10.b.).} Finally the \textquote{\textlatin{itaque}} denotes a natural and seamless movement of the decision from the legates to assembly. In response to a problem, these \emph{legati} arrange a deliberative assembly, where they behave much in the same way that Caesar deliberates internally.

%4: Cotta
The meeting itself and the speeches, for and against striking camp, are deliberative in nature. Like deliberative oratory, the arguments fall between at least two options (\emph{Ad~Her.}~3.2). Cotta's plan is to remain where they are and not engage directly with the enemy. He says, along with other military leadership, that they should remain where they are and that no martial activity should occur without direct orders from Caesar. \blockquote[\emph{BG}~5.28.3]{\textlatin{L. Aurunculeius compluresque tribuni militum et primorum ordinum centuriones nihil temere agendum neque ex hibernis iniussu Caesaris discedendum existimabant}} \blockquote[\emph{BG}~5.28.3]{L. Aurunculeius and many of the military tribunes and centurions of the first order thought that nothing rash should be done nor should they depart from the winter quarters without an order of Caesar.} As seen in the narrator's description of Caesar's internal deliberations, their thinking is recorded along with what they think must be done (\textquote{\textlatin{agendum \ldots{} discedendum existimabant}}). Emphasizing the reasonable nature of Cotta's argument, he and the others choose to do nothing rashly (\textquote{\textlatin{nihil temere}}). Their argument takes the form of a proof, according to the principles called for in at \emph{Ad~Herennium} 3.2. For example, they offer reasons (i.e., \emph{rationes}) for remaining. The first they seem to take for granted, that nothing should be done without Caesar's order (\textquote{\textlatin{iniussu Caesaris}}). The text presents four more reasons: a fortified camp could hold back the Germans, they had enough grain to last the winter, help would eventually come from Caesar, and they should not do anything based on the words of an enemy (\emph{BG}~5.28.4--6). Several of these arguments are supported with adverbs suitable for a proof, some of which guide their thinking through complicated subordination: \textquote[][,]{\textlatin{quantasvis \ldots{} etiam}} \textquote[][,]{\textlatin{quod}} \textquote[][,]{\textlatin{interea}} and \textquote[][.]{\textlatin{postremo}} From its initial characterization and that of Cotta's arguments, the meeting is deliberative and with rationally argued speeches. In the absence of the general, this \emph{consilium} performs in a group exactly what Caesar does within his own mind.

%5: Sabinus
In response to this majority opinion, Sabinus, seemingly alone, disagrees (\emph{BG}~5.29--30). Fear, over which Sabinus lacks control, seems to corrupt the quality of his rational arguments, and leads to a corruption of the deliberative assembly. His arguments take the form of rational proofs, though the arguments themselves patently misuse evidence. In the first speech (\emph{BG}~5.29), he offers five reasons why they should move to Q.~Cicero's camp: (a) he thought Caesar had left, (b) the Germans were close and (c) angry, (d) Gaul was angry, and (e) Ambiorix would only take up such a plan if he were certain to win (5.29.2--5). To several of these \emph{rationes}, Sabinus gives supporting evidence (i.e., \emph{rationis confirmationes}. For example, to his thinking that Caesar had left for Italy, he explains two separate pieces support, that (a) the Carnutes would not have killed their Roman--sympathizing leader Tasgetius, nor (b) would the Eburones have been so bold in their recent attack (\emph{BG}~5.29.2). The second of these is supported by several conditions (\textquote{\textlatin{si ille adesset}} and \textquote{\textlatin{tanta contemptione}}). In explanation of his fourth point, that \textquote[\textquote{Gaul was on fire}]{\textlatin{ardere Galliam}} (\emph{BG}~5.29.4), he explains that they Gauls are angry because they have been disgraced so many times by the Romans (\textquote{\textlatin{tot contumeliis acceptis sub populi Romani imperium}}) and, because of the disgrace, of their faded military glory (\textquote{\textlatin{superiore gloria rei militaris ex<s>tincta}} \emph{BG}~5.29.4). Further, there are several tokens common to rational argumentation in the \emph{Bellum~Gallicum}, such as an adverb (\textquote{\textlatin{postremo}} \emph{BG}~5.29.5), repeated conditionals (\textquote{\textlatin{si \ldots{} si}} \emph{BG}~5.29.6), and several rhetorical questions (\emph{BG}~5.29.6-7). Finally, his mental activity is described by the narrator, as with: \textquote[\textquote{he thought Caesar had departed to Italy} \emph{BG}~5.29.2]{\textlatin{Caesarem arbitrari profectum in Italiam}}. There are also several interesting moments which show a certain reflexivity to Sabinus's speech. He explicitly states the evidence he is and is not considering: \textquote[\textquote{he was not looking at the enemy as fabricator of the story, but at the situation}]{\textlatin{sese non hostem auctorem, sed rem spectare}} \emph{BG}~5.29.3).\footnote{On \textquote{\textlatin{rem}}: \textquote[{\cite{kranerdittenbergermeusel1967}}]{\textgerman{die Sachlage, die tatsächlichen Verhältnisse}}.} In his disagreement, Sabinus repeats the word \emph{consilium} (as in \textquote{plan}), to show that he is evaluating one against another, namely that of Cotta's (\textquote{\textlatin{Cottae \ldots{} consilium}}) compared to that of their enemy (\textquote{interficiendi Tasgetii consilum} and \textquote{\textlatin{Ambiorigem \ldots{} consilum \ldots{} descendisse}} \emph{BG}~5.29.2, 5). In a dynamic response Sabinus's arguments take the form of a rational argument, but closer scrutiny reveals that it is not, or rather should not be, persuasive.

The text distinguishes Sabinus in several ways, through the narrator's description, the introduction of emotional language into his words, and misuse of evidence (which seems to show that his rationality has been corrupted). Though the message of the story is that such unchecked fear results in bad outcomes, it needs to also be noticed that, in this passage, it is Sabinus's uncontrolled emotion which causes others to be persuaded by him.
%a narrator's language
 The narrator's characterization of Sabinus indicates that he lacks emotional self--regulation.\footnote{On this speech in general, \textcite{kranerdittenbergermeusel1967} characterize Sabinus as timid and imprudent: \textquote[][.]{\textgerman{Überhaupt zeigt die ganze Darstellung den Titurius als einen ängstlichen, unbessonnenen Menschen, der törichterweise den Worten des Feindes glaubt, in der Gefahr \ldots{} gleicht den Kopf verliert und nur auf die Rettung seines lieben Lebens bedacht ist}}} In his first objection to Cotta's plan, he is said to be \textquote[\textquote{\textlatin{clamitabat}}]{crying out}.\footnote{Meusel construes \textquote{\textlatin{clamitans}} here very strongly as \textquote[{\cite{kranerdittenbergermeusel1967}}]{\textgerman{laut schreiend}}.} That to shout in such a manner is not appropriate, consider that nowhere does the verb \emph{clamitare} appear elsewhere in the \emph{Corpus Caesarianum}, save once when Dumnorix, one of the most anti--Roman enemies of the \emph{Bellum~Gallicum}, begs for his life moments before being killed by Caesar's men.\footnote{By connection to this verb, Dumnorix's pathetic behavior may rub off on our interpretation of Sabinus: \textquote[\textquote{Once summoned, he repeatedly began to resist and to defend himself and to beg on account of the loyalty of his men, often shouting repeatedly that he was a free man and belonged to a free state} \emph{BG}~5.7.8]{\textlatin{Ille autem revocatus resistere ac se manu defendere suorumque fidem implorare coepit, saepe clamitans liberum se liberaeque esse civitatis}}.} The intensity of his reply is signaled by \textquote[][,]{\textlatin{contra ea}} which is an uncommon intensification of \textquote{\textlatin{contra}} \parencite{kranerdittenbergermeusel1967}.\footnote{This kind of response is exceptional in the \emph{Bellum~Gallicum}: \textquote{\textlatin{contra ea}} does not appear anywhere else in the \emph{Corpus Caesarianum}.} The disagreement is called an \textquote[\textquote{\textlatin{disputatione}} \emph{BG}~5.30.1]{argument}, and then put more strongly by the members of the \emph{consilium} as \textquote[\textquote{\textlatin{sua dissensione et pertinacia}} \emph{BG}~5.31.2]{dissension and obstinacy}, as well as \textquote[\emph{BG}~5.31.3]{\textlatin{disputatione}}. Finally, that his second speech is spoken \textquote[\textquote{in a louder voice} \emph{BG}~5.30.1]{\textlatin{clariore voce}} shows that he has moved beyond the art of persuasion and into another sort of persuasion (here, in appeal to the power of the soldiers).\footnote{\textquote[{\cite{kranerdittenbergermeusel1967}}]{\textgerman{Was ihren Gründen an Gewicht und Überzeugungskraft abgeht, suchen sie durch die Kraft ihrer Stimme, durch Schreien zu ersetzen}}.} In contrast, consider how Cotta's side, in their disagreement, \textquote[\textquote{were resisting sharply}]{\textlatin{acriter resisteretur}}, which, though a strong expression of disagreement, does not have strong emotional ring to it. In the \emph{Bellum~Gallicum}, \textquote{\textlatin{acriter}} is a common adverb to characterize the fighting of soldiers in battle.\footnote{E.g., \emph{BG}~1.26.1, 1.52.3, 2.10.1, 3.21.1, 4.26.1, 5.15.4, and 5.17.3.} The narrator characterizes Sabinus as an emotional speaker.

%b emotional language in speech
Also contributing to the portrait of Sabinus's emotionality, his speech is full of emotional language. His characterization of the situation is rather extreme, threatening that the majority's plan will end up in \textquote[\emph{BG}~5.29.1]{\textlatin{aliquid calamatis}}. The expression is one of angry emotion, so far not present within the \emph{consilium}.\footnote{The \textquote{\textlatin{aliquid}} emphasizes the impending disaster: \textquote[{\cite{kranerdittenbergermeusel1967}}]{\textgerman{Sabinus macht es gerade so, wie viele unserer Volksredner: was ihren Gründen an Gewicht und Überzeugungskraft abgeht, suchen sie durch die Kraft ihrer Stimme, durch Schreien zu ersetzen}}.} Toward the end of his first speech, he prioritizes fear in his calculations: \textquote[\textquote{in which <plan>, if there were not a present danger, certainly hunger from a long siege ought to be feared} \emph{BG}~5.29.7]{\textlatin{in quo si non praesens periculum, at certe longinqua obsidione fames esset timenda}}. Contrast how Cotta said that \textquote[\textquote{\textlatin{nihil temere agendum}} \emph{BG}~5.28.3]{nothing rash ought to be done}, while Sabinus insists that everything ought to be feared and action be immediate. %maybe transition statement

In the second speech, he denigrates the reasoning of the others, telling them, \textquote[\textquote{prevail, if you so want} \emph{BG}~5.30.1]{\textlatin{vincite \ldots{} si ita vultis}}, as if their desire is stronger than their logic.\footnote{The \textquote{\textlatin{vultis}} clearly insinuates that they are relying on opinion, not facts: \textquote[{\cite{kranerdittenbergermeusel1967}}]{\textgerman{so setz denn eure Meinung durch}}.} In contrast to their alleged mental weakness, Sabinus claims that, \textquote[\textquote{nor am I the one who, among you, fears the danger of death} \emph{BG}~5.30.2]{\textlatin{neque is sum \ldots{} qui gravissime ex vobis mortis periculo terrear}}. Due to Sabinus's behavior, the \emph{consilium} comes to the wrong decision. Though Cotta remains unpersuaded, he relents to Sabinus for the sake of unity. As shown, his emotional and poorly--reasoned arguments lead the decision--makers astray. He also violates the procedures of the Caesarian assembly by appealing to the soldiers (\textquote[][,]{\textlatin{clariore voce, ut magna pars militium exaudiret}} \textquote{with a louder voice, so that a great part of soldiers would hear} \emph{BG}~5.30.1), instead of the proper audience of the two leading \emph{legati}, \emph{tribuni militum}, and \emph{primorum ordinum centuriones} (\emph{BG}~5.28.3).\footnote{The composition of this \emph{consilium} is nearly identical to that in book~1 (section~\ref{bg-consilium}).} Unlikely as it may seem, Sabinus makes an appeal to reason, but this is in fact another aspect of corruption. He says \textquote[\textquote{these men know \ldots{} they will demand a reckoning/explanation (\emph{ratio}) explanation from you} \emph{BG}~5.30.2]{\textlatin{hi sapient \ldots{} abs te rationem reposcent}}. Making appeals to their knowledge (\textquote{\textlatin{sapient}}) is a statement that they are able to construct and evaluate reasons (\textquote{\textlatin{rationem}}) too. Sabinus has not lost interest in reason, but has maligned that of others and through the introduction of outsiders has corrupted the meeting.

%c misuse of evidence
Finally, Sabinus's particular reasons for moving camp seem not to hold up under closer scrutiny. Commentators have pointed out two central problems concerning his claim about the Carnutes and Gauls' extinguished glory. The first reason in his argument for leaving is because the Carnutes had killed Tasgetius, who was allied with Caesar. They would only have done something so bold, implies Sabinus, if they were confident of Caesar's departure to Italy (\emph{BG}~5.29.2). This information, however, has not come from Caesar, so Sabinus may be presumed to have learned this from Ambiorix.\footnote{\textquote[{\cite{kranerdittenbergermeusel1967}}]{\textgerman{Woher wußte Titurius dies? Caesar scheint seinen Legaten nichts davon mitgeteilt zu haben (c. 25, 3. 4); also haben Titurius und Cotta es wohl von den Eburonen erfahren}}.} Second, Sabinus claims that \textquote[\textquote{Gaul was burning} \emph{BG}~5.29.4]{\textlatin{ardere Galliam}} because their \textquote[\textquote{former military glory had been extinguished} \emph{BG}~5.29.4]{\textlatin{superiore gloria rei militaris exstincta}}. \textcite{kranerdittenbergermeusel1967} object to the argument because, they say, Gaul had long ago lost their military glory.\footnote{\textquote[{\cite{kranerdittenbergermeusel1967}}]{\textgerman{Diese Behauptung des Titurius ist unhaltbar: die Gallier hatten ihren Kriegsruhm längst eingebüßt: in ihren Kämpfen mit den Römern in Italien und in der Narbonensis und in den Kämpfen mit den Germanen}}. Statements } There may be some gray area here, for while the Gauls surely resented Roman presence and victories (e.g., the many conspiracies, as at \emph{BG}~2.1), the narrator makes it clear that much of Gaul (especially the south--eastern part) had been thoroughly subdued.\footnote{\textquote[\textquote{Gradually, accustomed to being conquered and defeated by many battles, so that they do not compare themselves to <Germans> in \emph{virtus}} \emph{BG}~6.24.6]{\textlatin{paulatim adsuefacti superari multisque victi proeliis ne se quidem ipsi cum illis virtute comparant}}.} An argument, however in support of a \textquote{burning Gaul} is their collective Gallic determination to revolt in \emph{Bellum~Gallicum} 7.\footnote{As shown in \textcite[177--181]{jervis2001}, Vercingetorix instills in the Gauls a \textquote{\emph{libertas} or death} mentality, though nothing of the sort appears before earlier than book~7.} Whatever the force, weak or just mediocre, of this second piece of supporting evidence, Sabinus does not make a strong case overall for moving camp. This, combined with the unchecked emotion that informs his rhetoric, is central in the eventual breakdown of the deliberative process.

Sabinus's argument prevails, but not through a formal persuasive process. The \emph{consilium} itself completely breaks down (\textquote{\textlatin{consurgitur ex consilio}} \emph{BG}~5.31.1) and participants on both sides of the debate call for reconciliation: \textquote[\textquote{begged that they not drag the matter down into the greatest danger due to their disagreement} \emph{BG}~5.31.1]{\textlatin{orant ne sua dissensione et pertinacia rem in summum periculum deducant}}. The expression of Cotta's acquiescence is revealing in that it does not denote a change of mind, but an ending of the dissension: \textquote[\textquote{finally Cotta, moved, yielded; Sabinus's opinion won} \emph{BG}~5.31.3]{\textlatin{tandem dat Cotta permotus manus, superat sententia Sabini}}.\footnote{In the other instance of independent deliberation by a subordinate, Q.~Cicero is \textquote{moved} in a very similar way (\textquote{\textlatin{permotus vocibus}} \emph{BG}~6.36.1) by the voices of soldiers, who demand to be let outside a fort.} \textcite{kranerdittenbergermeusel1967} note the origin of \textquote{\textlatin{manus dare}} in surrendering gladiators, meaning that Cotta gives in but is not persuaded.\footnote{\textquote[{\cite{kranerdittenbergermeusel1967}}]{\textgerman{\textquote{Ergibt sich} wie ein Besiegter ohne weitere Gegenwehr; ursprüglich wohl von dem Gladiator, der sich für besiegt erklärt und jede weitere Gegenwehr aufgibt}}.} In the end, the actors privilege unanimity over a majority opinion or utility: \textquote[\textquote{the matter was easy, \ldots{} if only all feel and think one thing} \emph{BG}~5.31.2]{\textlatin{facilem esse rem, \ldots{} si modo unum omnes sentiant ac probent}}. But despite the prevailing of Sabinus's side of the argument, the contagion of his emotion has been let loose upon the army. The narrator later says that, among the soldiers, \textquote[\textquote{all sorts of things were made up, whereby they could not remain without danger and danger was increased from the fatigue of the soldiers' watches} \emph{BG}~5.31.5]{\textlatin{omnia excogitantur, quare nec sine periculo maneatur et languore militum et vigiliis periculum augeatur}}. While, according to narrator, they later fight bravely (e.g., \emph{BG}~5.35.1 and 4), the flight is was filled \textquote[\textquote{shouting and crying}]{\textlatin{clamore et fletu}}. Cotta and some others put a high premium on unanimity, so much so that he finally acquiesces. Though quite similar in setup to Caesar's internal deliberations, and similar in argumentation, the development of this \emph{consilium} is quite different from what is seen in Chapter~\ref{ch-delib}.

In the conclusion of the episode, during the battle, the actors retain their characteristics. Sabinus loses his head and is revealed to be irrational and cowardly. \blockquote[\emph{BG}~5.33.1]{\textlatin{Tum demum Titurius, ut qui nihil ante providisset, trepidare et concursare cohortesque disponere, haec tamen ipsa timide atque ut eum omnia deficere viderentur.}} \blockquote[\emph{BG}~5.33.1]{Then finally Titurius, as he had not foreseen anything before, was trembling and running around and positioning cohorts, yet all these very actions seemed to be done apprehensively and as if he lost his mind.} The very words and concepts he threw at others are now applied to him by the narrator. He is timid (\textquote{\textlatin{trepidare}} and \textquote{\textlatin{timide}}) and he seems to have lost all reason (\textquote{\textlatin{eum omnia deficere}}). The narrator contrasts Cotta (\textquote{\textlatin{at Cotta}}) who had foreseen the posibility of an attack (\textquote{\textlatin{qui cogitasset haec posse in itinere accidere}} \emph{BG}~5.33.2). In the end, Sabinus unwisely attempts a colloquy with Ambiorix and in so doing is killed (\emph{BG}~5.37.2).

The power of this example comes from its parallelism to Caesar's deliberative practices as much as from its differences. In this unique instance of failed communication within the army, others try to perform the job that Caesar does by himself. Just like Caesar, they receive information, perceive a difficult situation, deliberate on both sides of a difficult question, come to a final decision, and finally act upon it. The key differences are found in the extroversion of the deliberation, that the two positions are advocated by two speakers, and how emotion, namely fear, informs or interferes with rational deliberative process.

Having laid out the basics of both types of deliberation, I have several comments to make about Caesar's implicit message about emotion in the deliberative process. On the surface, there may appear to be a contradiction in this message, for as shown in section~\ref{ch2-delib-caesar}, Caesar uses his own fear as a catalyst for interrogation of an issue, while in the case of Sabinus fear distorts and eventually overwhelms his and everyone else's mind. That Sabinus's emotional appeals are persuasive should not be surprising, since a speaker's character and emotional appeals were equal partners to \emph{logos} in much of the rhetorical tradition.\footnote{Noteworthy exceptions to this rule are the rhetorical handbooks of the Hellenistic period, exemplified by \emph{Ad~Her.}~and \emph{De~inv.}~\parencite[77--93]{wisse1989}. \emph{Ethos} and \emph{pathos} find a revival in the \emph{De oratore}, published in 55 B.C. during Caesar's composition of the \emph{Bellum~Gallicum}; on this generally, see \parencite[pp.~222--249 and 250--300, on \emph{ethos} and \emph{pathos}, respectively]{wisse1989} and \parencite{wisse2002a}.} However, the disastrous ending of the Cotta and Sabinus episode reveals just the opposite argument. Consider, for example, Cicero in the \emph{De oratore}, which assigns an important and positive role to emotional modeling by the successful orator.\footnote{See \emph{De~oratore}~2.185--216 (esp.~2.189--190) for an account of the importance of emotional role--modeling to be done by the effective orator.} \blockquote[\emph{De~or.}~2.190]{\textquote{\textlatin{neque est enim facile perficere ut irascatur ei, cui tu velis, iudex, si tu ipse id lente ferre videare; neque ut oderit eum, quem tu velis, nisi te ipsum flagrantem odio ante viderit; neque ad misericordiam adducetur, nisi tu ei signa doloris tui verbis, sententiis, voce, vultu, conlacrimatione denique ostenderis.}}} \blockquote[\emph{De~or.}~2.190]{\textquote{For it is not easy to make a judge angry at whom you wish, if you yourself seem to take the matter lightly; nor that he hate whom you wish, unless he first sees you yourself burning with hate; nor will he be led to compassion, unless you show to him the signs of your grief with words, opinions, voice, visage, and finally lamentation.}} Antonius, the speaker here, says that if an orator is to persuade by means of emotion, then he must seem (\textquote{\textlatin{videare}}) to them to posses the emotional state he wishes to impart. Simply put, if the speaker is aflame (\textquote{\textlatin{flagrantem}}) then his emotion will spread like fire (\textquote{\textlatin{quae nisi admoto igni ignem concipere possit}}).\footnote{The passage continues: \textquote[\textquote{For just as no material is so easily inflamed as that which is able to catch fire without fire, so too there is no mind, able to be inflamed, so ready for comprehending the force of an orator except when [the orator] himself becomes inflamed and burning with respect to it [the mind]} \emph{De~or.}~2.190]{\textlatin{ut enim nulla materies tam facilis ad exardescendum est, quae nisi admoto igni ignem concipere possit, sic nulla mens est tam ad comprehendendam vim oratoris parata, quae possit incendi, nisi ipse inflammatus ad eam et ardens accessory}}.} Caesar's portrait of Sabinus's failure to self--regulate works with and against this rhetorical tradition. In going against the grain, the implicit message is that Sabinus's style of oratory results in disastrous decisions, at least in the context of the Roman army. Caesar's own deliberations in the \emph{Bellum~Gallicum}, which I consider next, offer a clear alternative to Sabinus's emotional--oratorical style.

Caesar's deliberations find a limited but important role for emotion. His temperate decision--making includes a use of fear used to the end of foresight. For instance, \emph{Bellum~Gallicum} 7.10.1 has two fearing clauses (\textquote{\textlatin{ne \ldots{}} deficeret} and \textquote{\textlatin{ne \ldots{} laboraret}}).\footnote{Though lacking verbs of fearing, such clauses of fearing depend \textquote[{\cite[146]{woodcock1959}}]{on a general idea of anxiety inherent in the context}.} Verbs of fearing with Caesar as subject are common in the \emph{Bellum~Gallicum}.\footnote{For example, of total proper fearing clauses with \textquote[][,]{\textlatin{ne}} 16 occur with \textquote{\textlatin{vereri}} and 5 with \textquote{\textlatin{timere}} \parencite[s.~v.~{\textquote[][,]{ne}} II.A.a)β)εε)]{meusel1887}. \textquote{\textlatin{Vereri}} is otherwise inflected with Caesar over seven times (2.11.2, 2.23.3, 4.5.1, 5.5.4, 5.6.5, 5.9.1, and 6.29.1), and \textquote{\textlatin{timere}} more than that \parencite{meusel1887}.} In addition to fear, \textcite[438--440]{oldsjo2001} catalogs many verbs of \textquote[][,]{mental state} most of which characterize Caesar not as a bland thinker, but one with thoughts and desires (e.g., \textquote[][,]{\textlatin{velle}} \textquote[][,]{\textlatin{confidere}} \textquote{\textlatin{mirari}}). The way that thoughts and feelings match up in Caesar's deliberation offers some insight into his persuasive characteristics. As a deliberating character, Caesar's mental activity is a part of his \emph{ethos}; his use of the emotion, if kept in check, as in the fearing clauses in which he is prone to think, he appears a reasonable leader and sensitive to the many dangers Gaul presents. Sabinus's lack of foresight (\textquote{\textlatin{qui nihil ante providisset}} \emph{BG}~5.33.1) is not a failure to eliminate fear from his rational processes, but not fail to limit and harness it to spur him into rational deliberation.

\section{Deliberation in Xenophon}
\label{delib-xen}

Xenophon's \emph{Anabasis} is interesting to consider in light of collaborative de\-cis\-ion--mak\-ing in the \emph{Bellum~Gallicum}, because Xenophon's text has often been thought to be a precedent, or at least precursor, to Caesar's \emph{Commentarii}. Deliberation in each text show some remarkable convergences and divergences. The chief difference, simply put, is that collaborative decision--making is common and important in the \emph{Anabasis}, while in the \emph{Bellum~Gallicum} it is uncommon. Another important difference is that, within collaborative discourse,  there is rarely if ever any disagreement in deliberative assemblies in the \emph{Anabasis}; but in the \emph{Bellum~Gallicum}, deliberation is agonistic, as that between Cotta and Sabinus. Despite these differences, however, there is an important commonality in each text's message about the importance of one--man authority and obedience. While presenting collaborative decision--making in different ways and for different narratives, Xenophon and Caesar nevertheless offer a remarkably similar take on leadership.

%on anab--bg relationship 1/2
Comparisons between Xenophon's \emph{Anabasis} and Caesar's \emph{Commentarii} are frequently made on the basis of several commonalities, genre and subject matter. Regardless of how one labels their genre, both texts are third--person autobiographical memoirs, the authors write of themselves in the third person, and they share a simple prose style. As recordings of an author's own \emph{res gestae}, both works were, in antiquity, associated with the \emph{commentarius} genre.\footnote{For the most recent comprehensive survey of Caesar's relationship to the \emph{commentarius}, see \textcite[133--155]{riggsby2006}. \textcite{bomer1953} offers the most complete survey of ancient \emph{commentarii} and Greek \textgreek{ὑπομνήματa}. See \textcite[144--146]{batstonedamon2006} for an appraisal that downplays the influence of Xenophon's third--person narrative in the \emph{Bellum~civile}, but argues that the device offers similar ends to both authors, \textquote[{\cite[145]{batstonedamon2006}}]{to construct and enjoy the opportunities \ldots{} for self--praise and self--justification}. Of Xenophon's writings on Socrates, for instance, A.~Gellius calls them \textquote[\textquote{\emph{commentarii} of the words and deeds of Socrates} \emph{Noctes Atticae} 14.3.5]{\textlatin{dictorum atque factorum Socratis commentarios}}. On the popularity of Xenophon in at Rome, see most recently \textcite[362--366]{bartley2008}, which builds upon Quintillian 12.10.39 and \textcite{munscher1920}. As to whether Julius Caesar had read the \emph{Anabasis}, I follow \textcite{bartley2008}, who argues that it is highly probably, though the question \textquote[{\cite[380]{bartley2008}}]{cannot be answered by direct testimony. However, the evidence \ldots{} concerning the popularity of other works of Xenophon among the literary circles in which Caesar moved, the striking similarity in topic material, the lack of other famous military commentaries at that period and the similarities in style suggest that the influence should be accepted}.} For example, in the \emph{Brutus}, Q.~Catulus's autobiographical writings (\textquote{\textlatin{de consulatu et de rebus gestis suis}}) bear the hallmarks of a \emph{commentarius} \parencite[228]{bomer1953} and are said to be written \textquote[\textquote{in an easy and Xenophontean manner of speech} \emph{Brut.}~1.132]{\textlatin{Xenophonteo genere sermonis}}.\footnote{\emph{Brutus} text from \textcite{malcovati1970}.} Catulus's autobiography seems similar in autobiographical content and style, of which the \emph{Commentarii} are described similarly as \textquote[\textquote{stripped of all ornament of speech}]{\textlatin{omni ornatu orationis \ldots{} detracta}} and \textquote[\textquote{with pure and clear brevity} \emph{Brut.}~262]{\textlatin{pura et inlustri brevitate}}.\footnote{\textquote[{\cite[362]{albrecht1997}}]{The \emph{commentarius} had its Roman roots in the official reports made by magistrates, but may also be explained by Greek models. An ordered collection of material intended for literary elaboration was called \textgreek{ὑπόμνημα}. Xenophon acted as literary model both for \emph{commentarii} (in the case of Caesar), and for autobiography (in the case of \textlatin{Lutatius~Catulus})}. For Cicero's claim that Caesars's \emph{Commentarii} (\textquote{\textlatin{commentarios quosdam scripsit rerum suarum}}) were deemed too good to be touched, see \emph{Brutus} 262. On a direct comparison of prose style between the \emph{Anabasis} and \emph{Commentarii}, see \textcite[373--379]{bartley2008}. For brief discussions of similarities of Xenophon's and Caesar's prose style, see also \textcite[51]{leeman1963} and \textcite[346]{cleary1985}.} In subject matter, the \emph{Bellum~Gallicum} and \emph{Anabasis} concern historical military ventures that take place far from home; and the authors, as characters, play significant roles as leaders of their army.\footnote{\textquote[{\cite[164]{kraus2009}}]{The very decision to write a seven-book third-person narrative about adventures in a foreign land must be to some extent itself an intertextual gesture, aligning Caesar’s text with Xenophon’s \emph{Anabasis} and even, by extension, with Xenophon’s continuation of Thucydides, the \emph{Hellenika}}. For a comparison, between the \emph{Anabasis} and \emph{Bellum~Gallicum}, of autobiographical manner and battle scenes see \textcite[366--373]{bartley2008}. On \emph{Anabasis} as an Odyssean \emph{nostos} story, see \textcite[159--195]{purves2010}. On the unknown, foreign elements of the \emph{Bellum~Gallicum} in a late Republican context, see \textcite[47--71]{riggsby2006}.} Concerning the question of whether Caesar in fact read the \emph{Anabasis}, I follow \textcite{bartley2008} and \textcite{kraus2009}, who, in light of a burden of evidence, presume that the \emph{Bellum~Gallicum} \textquote[{\cite[164]{kraus2009}}]{must be to some extent itself an intertextual gesture}. \textcite[362]{bartley2008} makes the case even more strongly, asking, \textquote{why should he not have read that work by Xenophon which seems to have the closest possible link with his own writings?} Concerning genre and subject, the \emph{Anabasis} and \emph{Bellum~Gallicum} share some significant traits.

%on anab--bg relationship 2/2
Beyond genre and subject, comparisons of the \emph{Anabasis} and \emph{Bellum~Gallicum} are uncommon. \textcite{lendon1999} stands out as an exception, seeing important parallels in how each work treats troop morale. Both authors are very concerned with \textgreek{ψυχή} and \textgreek{φόβος}, or \emph{animus} and \emph{timor} \parencite[295--296]{lendon1999}.\footnote{\textquote[{\cite[295--296]{lendon1999}}]{The outlines of Caesar's understanding of \emph{animus} -- simply a translation of the way Xenophon uses the word \textgreek{ψυχή} -- are very similar to, and probably borrowed from, the Greeks. Both gather all forms of low morale -- from quiet discouragement, to defeatism, to desperate irrational panic -- into one broad functional category: \textgreek{φόβος}, \textgreek{ἀθυμία}, and synonyms in Xenophon, \emph{terror} and \emph{timor} in Caesar}.} As leaders, Xenophon and Caesar manage their soldiers' minds (e.g., the \emph{consilium} discussed above at section~\ref{bg-consilium}). This point of contact between the \emph{Anabasis} and \emph{Bellum~Gallicum} is helpful in this chapter's argument about Caesar's representation of leadership. As I hope to show here, decision--making in the \emph{Anabasis}, though collaborative, is top--down, in a manner not unlike the \emph{Bellum~Gallicum}. These leaders' intervention into the psychic life of their soldiers, as demonstrated by \textcite{lendon1999} and the first chapter of this dissertation, is a parallel phenomenon to the single--minded nature of deliberation in both texts.

In light of these three points of contact between the \emph{Anabasis} and \emph{Bellum~Gallicum}, I would like to briefly survey deliberative assemblies in the \emph{Anabasis}, and in so doing demonstrate the two salient differences of collaborative decision--making in it. First, decision--making in groups is common in the \emph{Anabasis}, whereas it is in the \emph{Bellum~Gallicum} it only occurs in exceptional circumstances. Second, while disagreement is typical to the collaborative decision--making in the \emph{Bellum~Gallicum}, in the \emph{Anabasis} disagreement is slight. But despite these two significant divergences, in their representation of deliberative assemblies, both works share a common portrait of plan--making as the result of one man's mind. Scholars have typically seen assemblies in the \emph{Anabasis} within the army--as--\emph{polis} topos. Since at least Gibbon, the political, deliberative nature of Xenophon's army has been observed as remarkable. \blockquote[{\cite[pp.~373--374, ch.~24]{gibbon1846}}]{Instead of tamely resigning themselves to the secret deliberation and private views of a single person, the united councils of the Greeks were inspired by the generous enthusiasm of a popular assembly; where the mind of each citizen is filled with the love of glory, the pride of freedom, and the contempt of death.} While scholars have tempered such a strong claim about the Cyreans as a moving city, the particular language that Gibbon uses is helpful.\footnote{On Gibbon on Xenophon, see \textcite[31]{hornblower2008}. \textcite{nussbaum1967} is generally taken as having taken the strongest position of the army--as--\emph{polis} interpretation (\cite[16--17]{dalby1992} and \cite[9--10]{lee2007}). For overviews of revisions to this basic position, see \textcite[30--32]{hornblower2008} and \textcite[especially pp.~9--11]{lee2007}. The most significant departure is by \textcite[see especially pp.~80--108]{lee2007}, which plays down the role of all--army meetings and prioritizes the decision--making by smaller groups of soldiers. For a bibliography of positions concerning the importance of assemblies in the \emph{Anabasis}, see \textcite[p.~10, n.~40]{lee2007}.} In terminology that applies to the \emph{Bellum~Gallicum} as much as the \emph{Anabasis}, Gibbon draws a strong distinction between interior (\textquote{secret deliberation and private views of a single person}) and collaborative decision--making (\textquote{united councils}). The distinction applies to authority as well as deliberation, in that the Greeks do not follow one individual (\textquote{tamely resigning themselves}) but instead the \textquote[][.]{popular assembly} In several examples, I next illustrate how group deliberation in the \emph{Anabasis} is the default mode of decision--making, and is non--confrontational and top--down. 

From the very moment of their independence from the generals who led them to Babylon, the Greeks initiate collaborative decision--making. Leadership falls into two distinct sorts in the \emph{Anabasis}, that of the first two books and that following books~3--7. \textcite[65--95]{dillery1995}, considering the unity of soldiers, helpfully divides the \emph{Bellum~Gallicum} into four phases (books 1--2, 3--4, 5--6, and 7). The most profound distinction, though, lies between books~1--2 and 3--7. At the close of \emph{Anabasis}~2, the five generals of Cyrus's forces are tricked into meeting Tissaphernes and are then killed (\emph{Anab.}~2.5.31).\footnote{In a hostile territory, on the run, and without a traditional top--down command structure, their newfound situation at the opening of book~3 remains in play for the remainder of the \emph{Anabasis}. \textquote[{\cite[70]{dillery1995}}]{One of the most dramatic changes between the army of the first phase and the army of the second is its sense of unity. Before, the units were kept separate and functioned independently, under the command of their own leaders; indeed, at times they almost fought one another. None of this divisiveness is present in the second phase; in fact at the command level there is almost complete unanimity, with Xenophon reporting only one disagreement between himself and Chirisophus (4.6.3; however cf. 3.3.11 and 4.1.19), the major decision makers among the leaders of the army. As Xenophon realizes, the goal is survival, and the survival of the individual depends on the survival of the whole, or as he puts it, \textquote{we are all in need of common safety} (3.2.32)}. For a not dissimilar division, see also \textcite[880--883]{howland2000}.} Immediately following their independence from their previous commanders and circumstances, collaborative decision--making is the norm, and dissent is minimal to the point of absence. Xenophon has a revelatory dream, in which he realizes what should be done (\emph{Anab.}~3.1.11--14). He communicates his plan to other Athenian captains, that the greatest threat to themselves is the disorganization following from a weak chain of command (\emph{Anab.}~3.1.1--25). All the captains (\textquote{\textgreek{οἱ δὲ ἀρχηγοὶ \ldots{} πάντες}} \emph{Anab.}~3.1.26), save one, insist that Xenophon take the lead.\footnote{\emph{Anabasis} text from \textcite{marchant1904}.} The dissenter disagrees with Xenophon's position that flight is the best method, saying instead that they should instead appeal to Tissaphernes for help (\emph{Anab.}~3.1.26). Before able to say more than a sentence, Xenophon interrupts (\textquote{\textgreek{μεταξὺ ὑπολαβὼν}} \emph{Anab.}~3.1.27) and berates him (3.1.27--30), followed by Agasias doing the same (3.1.31). Apollonides, this lone dissenter, is ridiculed for being a Boeotian and for his pierced ears, then driven out of the meeting (\textquote{\textgreek{ἀπήλασαν}} \emph{Anab.}~3.1.32). Even in the \emph{Bellum~Gallicum}'s most acrimonious debate, between Cotta and Sabinus, dialog never devolves into such base ridicule and silencing as here. The extreme nature of this rejection of oppositional opinions is better understood in light of this passage's similarity, and likely allusion to, Odysseus and of Thersites in \emph{Iliad}~2.\footnote{See \textcite{rinner1978} on parallels of \emph{Anabasis} 3.1--2 to \emph{Iliad}~2.} This very first instance of collaborative decision--making sets expresses the homogeneity of thought that characterizes nearly all decisions in the \emph{Anabasis}. There is here one very brief moment of disagreement, but the severity with which the outlier is treated only underscores the absence of two--sided deliberation among the text's actors.

Following this decision, they follow Xenophon's recommendation and reappoint the deceased generals and other officers (3.1.32). The leadership in attendance of the next meeting (\emph{Anab.}~3.1.33--47) are the newly appointed generals, lieutenants, and captains.\footnote{The positions and their translations adopted here are: \textgreek{ἄρχοντες} (\textquote{commanders}), \textgreek{στρατηγός} (\textquote{general}), \textgreek{ὑποστράτηγος} (\textquote{lieutenant}), \textgreek{λοχαγός} (\textquote{captain}), and \textgreek{στρατιώτης} (\textquote{soldiers}).} Within this group, Xenophon repeats what he had said earlier, agreement is unanimous, with no dissenting view whatsoever, and five replacement commanders are chosen. In these two initial meetings, collaborative decision--making allows for no dissension. The thinking is the plan of one person, Xenophon.

There follows a third group meeting, this time of the generals presenting Xenophon's plan to all the \textquote{Ten Thousand} soldiers (\emph{Anab.}~3.2). Along with Xenophon, several speak, though they all counsel essentially the same advice that Xenophon had initially devised: they should not bargain with nor confront directly Tissaphernes, but flee in pursuit of \textquote[\textquote{safety} \emph{Anab.}~3.2.8 and 3.2.15; cf. 3.1.26]{\textgreek{σωτηρίας}}. This is not to say that, for example, Xenophon's speech is identical to what he has said twice earlier, but that the fundamental plan has remained unchanged.\footnote{There is a greater precision to Xenophon's plan at the larger assembly, where, according to \parencite[70]{dillery1995}: \textquote{at the conclusion of the same speech Xenophon identifies three \emph{separate} goals that might possibly motivate the Ten Thousand -- desire to return home, desire to live and desire for riches (3.2.29) -- a distinction that suggests that soteria can mean precisely \textquote{staying alive.} Survival for Xenophon, however, meant not just living but living honourably and free from the control of the enemy (3.1.43, 3.2.3). But at another level the notion of \textquote{safe--return,} specifically to homes in Greece, is also clearly meant when reference is made to surviving.}} Another aspect of the homologous nature of the army, in addition, to the single--minded plan, is the unanimous voting. At this assembly, three majority votes are taken, and the soldiers are each time unanimous: \textquote[\textquote{having heard all the soldiers made obeisance to the god with one motion} 3.2.9]{\textgreek{ἀκούσαντες δ' οἱ στρατιῶται πάντες μιᾷ ὁρμῇ προσεκύνησαν τὸν θεόν}}, \textquote[\textquote{all lifted up their hand} 3.2.33]{\textgreek{ἀνέτειναν πάντες}}, and \textquote[\textquote{no one spoken against. \ldots{} These were approved} \emph{Anab.}~3.2.38]{\textgreek{οὐδεὶς ἀντέλεγεν. \ldots{} ἔδοξε ταῦτα}}. From the evidence of first three meetings of the \emph{Anabasis}, we can see that collaborative decision--making is, after book~2, the default method of making decisions, and that, when deliberating, there are next to no dissenting views. From beginning to end, the plan is Xenophon's alone. The possibility of alternate choices is twice brought up towards the end of his speech, in his recommendation for the position of commanders: \textquote[\textquote{if anyone sees something else better, let let it be done} \emph{Anab.}~3.2.37]{\textgreek{εἰ μὲν οὖν ἄλλο τις βέλτιον ὁρᾷ, ἄλλως ἐχέτω}} and \textquote[\textquote{if anyone sees something better, let him speak} \emph{Anab.}~3.2.28]{\textgreek{εἰ δέ τις ἄλλο ὁρᾷ βέλτιον, λεξάτω}}. Nobody, however, speaks up with another plan (\textquote{\textgreek{οὐδεὶς ἀντέλεγεν}}). Instead, one individual's plan, here Xenophon's, is expressed uncorrupted from one to many, and voted without any significant alteration.\footnote{\textcite[59--98]{dillery1995} offers a compelling explanation of the idealistic tendency to an \textquote{ideal community} in the \emph{Anabasis} He locates Xenophon's panhellenism in a context fourth--century thinking as that in Isocrates's \emph{Panegyricus} (on which, see in particular pp.~54--58). In this context, a high premium is placed upon Greeks, in opposition to non--Greeks, \textquote[translation of \emph{Panegyricus} 173, taken from {\cite[54]{dillery1995}}]{to be of one mind (\emph{homonoesai}) until we gain our profits from the same source and venture risks against the same enemy}. See also \textcite[880--883]{howland2000} on the community's unity and moments of dissonance.}

Throughout the entirety of the \emph{Anabasis}, deliberative assemblies are common, but controversy in them not. Rare disagreements are immediately mollified, and on several occasions, dissenters thrown out of an assembly. Once deliberations are over, the text represents the Greeks as unanimous in their opinion. These patterns of deliberation work together to make the expression of a leader's mind, usually by Xenophon's or his Lacedaemonian colleague Cheirisophos's, without interference. Deliberative assemblies occur among officers (e.g., \emph{Anab.}~3.1.33--47) and soldiers (e.g., 3.2).\footnote{Examples of all--army assemblies, like that at \emph{Anab.}~3.2, at 5.1.1--13, 5.4.19--21, 5.6.1--11, 5.6.22--34, 5.8.1--26, 6.1.24--33, 6.2.4--8, 6.4.10--11, 6.4.17--19, 7.3.10--14, and 7.6.7--41. There are smaller meetings of generals and superior officers, like the one at \emph{Anab.}~3.1.33--47, at 3.3.11-20, 4.1.19--22, 4.6.7--20.} Similar to the above two assemblies at \emph{Anabasis} 3.1.33--47 and 3.2, the voting members are unanimous in their opinion, and disagreement is nonexistent or extremely minimized. On occasion, when there is a difference of opinion, there is no dialogic debate whereby the best decision wins out, but disagreement arises only from under--informed actors.\footnote{Meetings during which disagreement of some sort occurs include \emph{Anabasis} 4.6.7--20, 5.6.22--34, 5.8.1--26, 6.2.4--8, and 7.6.7--41.} For an example of how the text represents differing opinions, consider \emph{Anabasis} 4.6.7--20, where the Cyreans must march uphill where enemy soldiers are positioned. Cheirisophos thinks  (\textquote{\textgreek{ἐμοὶ \ldots{} δοκεῖ}}) they should move that day or the next (\emph{Anab.}~4.6.8). Cleanor does not express his idea as a disagreement, but refinement of what Cheirisophos said, thinking (\textquote{\textgreek{ἐμοὶ \ldots{} δοκεῖ}}) to move that day (\emph{Anab.}~4.6.9). Xenophon disagrees, but the difference again takes the form of another refining of the previous. He thinks (\textquote{\textgreek{ἐγὼ δ' οὕτω γιγνώσκω}} \emph{Anab.}~4.6.9) that they should move that day, and try to secure a favorable position while so doing. From this point, there is no more discussion of the plan, but the Athenian and Spartan trade several lighthearted jokes (4.6.14--16). The first two speakers are mild in their relationship to their opinions (\textquote{\textgreek{ἐμοὶ \ldots{} δοκεῖ}}), while Xenophon's statement (\textquote{\textgreek{ἐγὼ δ' οὕτω γιγνώσκω}}) is best understood as \textquote[][,]{I think} with the sense of actively forming a judgment.\footnote{See \emph{LSJ} def.~II for use with \textquote[][.]{\textgreek{οὕτω}} \textcite[393]{matherhewitt1962} recommends a translation of \textquote[][.]{have this opinion}} There is no vote in endorsement of the final proposal, but agreement is implicit in several leading commanding hoplites volunteering to lead the effort, and from the narrators comment that they \textquote{agreed} \textquote[\emph{Anab.}~4.6.20]{\textgreek{ταῦτα συνθέμενοι}}.\footnote{The agreement with \textquote{\textgreek{συντίθημι}} suggests not a settlement, but planning that was \textquote[\emph{LSJ} def.~II]{put together constructively}. Troop unity is stated more strongly, e.g. \textquote[\textquote{all lifted up their hand} \emph{Anab.}~5.6.33]{\textgreek{ἀνέτειναν ἅπαντες}} and \textquote[\textquote{nobody spoke against} \emph{Anab.}~7.3.14]{\textgreek{οὐδεὶς ἀντέλεγεν}}.} This passage's disagreement, as mild as it is, is the only disagreement between Cheirisophos and Xenophon in the entire \emph{Anabasis} \parencite[70]{dillery1995}. The actors here express their minds mildly, avoiding contrariness, and offer their thoughts as modifications to preexisting ideas.

In another all--army assembly (\emph{Anab.}~7.6.7--41), Xenophon converts the ire of a disagreeable soldiery to another, thwarting their plan. When Lacedaemonians suggest following them in seeking vengeance against Tissaphernes, the soldiers are enthusiastic, but there comes a problem for Xenophon, when \textquote[\textquote{\textgreek{εὐθὺς ἀνίσταταί τις τῶν Ἀρκάδων τοῦ Ξενοφῶντος κατηγορήσων}} \emph{Anab.}~7.6.8]{right away one of the Arcadians stood up and accused Xenophon} who had led them to partner with Seuthes (7.6.9--10). Two others also accuse Xenophon, who then speaks in defense of his planning (\emph{Anab.}~7.6.11--38), reminding the soldiers that they themselves had universally endorsed their current course: \textquote[\textquote{all of you said to go with Seuthes, and all voted on this} \emph{Anab.}~7.6.14]{\textgreek{πάντες μὲν ἐλέγετε σὺν Σεύθῃ ἰέναι, πάντες δ' ἐψηφίσασθε ταῦτα}}. Next, Charminos speaks in support of Xenophon (\emph{Anab.}~7.6.39). The anti--Xenophon argument morphs into accusations against Seuthes. Eurylochos's position is motivated by the same ire the soldiers initially had against Xenophon's leadership, but he now puts the blame of Seuthes alone (\emph{Anab.}~7.6.40). Finally, one of Xenophon's friends recommends assaulting Seuthes's associate Heracleides for withholding pay from them, and on the grounds of his Thracian ethnicity (\emph{Anab.}~7.6.41). Xenophon's problems with his soldiers have vanished, and the decision about where to go indefinitely deferred. In both of these two examples from the flight narrative (4.6.7--20 and 7.6.7--41) of the \emph{Anabasis}, as well as those at its very inception (\emph{Anab.}~3.1.33--47 and 3.2), contrary views are not expressed agonistically and assembly members think uniformly.

This brief survey of deliberation in the \emph{Anabasis} shows that debate in it tends to be non--confrontational and decisions unanimous. Their frequency, too, indicates that collaborative decision--making is essential to the functioning of the army. Due to lack of friction within deliberations, proposals by Xenophon and other leaders pass unchanged. \textcite{dillery1995} offers a partial explanation of this phenomenon, saying that the Cyreans' decision--making is top--down during times of crisis, but soldier--located when danger is not impending.\footnote{\textquote[{\cite[78]{dillery1995}}]{The command structure of the Ten Thousand throughout the second phase was very much \textquote[][,]{top-down} that is to say the common soldiers had little input in deciding what the army did; decisions had to be made at the command level as each new crisis was encountered. But once the immediate danger of the enemy and the difficult terrain are behind them, the ultimate power of decision gravitates back towards the soldiers where it was in the first phase}.} This may be the case, but even in times of relative safety, when soldiers have fullest input into decisions, an individual leader's proposal is not significantly altered. The most vivid illustration of this comes at \emph{Anabasis} 3.1.11, when the actor Xenophon has a dream of his father's house struck by lightening. The narrator enters into his mind (\textquote[][,]{\textgreek{ἔννοια αὐτῷ ἐμπίπτει}} \textquote{a notion fell upon him} \emph{Anab.}~3.1.13) while Xenophon interprets the meaning of the dream (3.1.13--14). It is within this exegesis that the Greeks' core defensive strategy emerges, as he becomes mindful that: \textquote[\textquote{nobody prepares nor cares how we will defend ourselves} \emph{Anab.}~3.1.14]{\textgreek{ὅπως δ' ἀμυνούμεθα οὐδεὶς παρασκευάζεται οὐδὲ ἐπιμελεῖται}}. The key point of concern that must be cared about (\textquote{\textgreek{ἐπιμελεῖται}}) is the Greeks' protection of themselves (\textquote{\textgreek{ὅπως δ' ἀμυνούμεθα}}). For Xenophon, bringing his thinking (\textquote{\textgreek{ἔννοια}}) to expression must pass some decision--making process, yet the deliberations are idealized enough that the original vision remains intact.

All this is not to say that there are not moments of dissension or disagreement. For example, early in book~7, the Greek soldiers, believing that they are being deceived, revolt against the plans by the generals to lead them to Seuthes (\emph{Anab.}~7.1.13--17). Xenophon, however, is able to calm them through a speech (\emph{Anab.}~7.1.25--31).

The observations of this section find corroboration with a theory of leadership articulated by the narrator and actor Xenophon. At the conclusion of book~2, the narrator offers sketches of the five murdered generals (\emph{Anab.}~2.6.12--30). The emphasis is upon their ability to instill obedience in their men. Clearchos was so severe \textquote[\textquote{that soldiers were disposed to him as children to a teacher} \emph{Anab.}~2.6.12]{\textgreek{ὥστε διέκειντο πρὸς αὐτὸν οἱ στρατιῶται ὥσπερ παῖδες πρὸς διδάσκαλον}}. Proxenos was fair, but was \textquote[\textquote{not able to inspire respect for nor fear of himself in his soldiers} \emph{Anab.}~2.6.19]{\textgreek{οὐ μέντοι οὔτ' αἰδῶ τοῖς στρατιώταις ἑαυτοῦ οὔτε φόβον ἱκανὸς ἐμποιῆσαι}}. Menon held the confidence of his troops (\textquote[][,]{\textgreek{τὸ δὲ πειθομένους τοὺς στρατιώτας παρέχεσθαι}} \textquote{furnishing persuaded soldiers} \emph{Anab.}~2.6.27), but only achieved this through sharing in their wrongdoing. Nothing is said about Agias and Socrates the Achaean but that they commanded respect and nobody found fault with them (\emph{Anab.}~2.6.27). The narrator's judgments of these leaders pertains to their ability to achieve obedience in their inferiors. A similar preoccupation with leadership and obedience is found in Xenophon's programmatic instructions to the army at \emph{Anabasis} 3.1--2. He commands the new leaders to be \textquote[\textquote{\textgreek{τῶν λοχαγῶν ἄριστοι καὶ τῶν στρατηγῶν ἀξιοστρατηγότεροι}} \emph{Anab.}~3.1.24]{the best of captains and most worthy generalship of the generals}. The new superior--inferior relationship, he says, calls for attentiveness in leaders (\textquote{\textgreek{τοὺς ἄρχοντας ἐπιμελεστέρους}}) and orderliness and persuasion in soldiers (\textquote{\textgreek{τοὺς ἀρχομένους εὐτακτοτέρους καὶ πειθομένους}} \emph{Anab.}~3.2.30). Xenophon here is setting up the conditions of authority and obedience that explain how the Cyreans operate throughout their flight to safety. The idealization of absolute obedience sheds light on the narrator's representation of the top--down and (nearly) conflict--free deliberative meetings. Despite the many opportunities for soldiers to contribute to decision--making (which they occasionally take up, e.g. 7.1.25--31), Xenophon is clear about the value of each, as when he tells the leaders that \textquote[\textquote{without leaders nothing beautiful nor good happens} \emph{Anab.}~3.1.38]{\textgreek{ἄνευ γὰρ ἀρχόντων οὐδὲν ἂν οὔτε καλὸν οὔτε ἀγαθὸν γένοιτο}}. Xenophon's originating dream, facilitated by cooperative deliberators, produces idealized group decision--making. The result is a portrait of a nearly frictionless translation of mental to physical action.

\section{Conclusion}
\label{ind-conc}

This chapter has demonstrated that collaborative decision--making occurs with different frequency and in different ways in the \emph{Bellum~Gallicum} and \emph{Anabasis}. As seen in section~\ref{cotta-sabinus}, group deliberations are rare, but can be very contentious when they do occur. The Cotta and Sabinus episode stands out in length and message as an example of decision--making gone awry, whereby the author conveys an important message about how deliberation among subordinates should work: there should be strong control over the constitution of a meeting's audience and a regulation of its fear, which easily spins out of control. Section~\ref{delib-xen} portrays just the opposite in the \emph{Anabasis}, that while common, the deliberative process is, even at its most divisive, conciliatory and constructive. In both texts, the leader's plan moves unhindered through his army, though through different processes. What the \emph{Anabasis} does through idealized deliberations, the \emph{Bellum~Gallicum} does through idealized communication. Despite their differences, there is striking commonality in both texts, in how their top--down expression of a leader's mind to all those below on the chain of command. If Caesar was familiar with the \emph{Anabasis} and looked to it as a model while writing his \emph{Commentarii}, it seems that he has re--purposed deliberative assemblies for another end.

In each text, the represented armies are expressions of their generals' minds. Chapters~\ref{ch-pres} and~\ref{ch-delib} of this dissertation offer an explanation of how Caesar's mind becomes expressed as action in the \emph{Bellum~Gallicum}. Chapter~\ref{ch-pres} explains the idealized state of communication which permeates the Roman army and allows for Caesar's unimpeded expression of mind to all of its members. When a panic spreads among the soldiers, Caesar's personal intervention gains control over their minds. Chapter~\ref{ch-delib} details the originating source of the general's plans. I detail how much energy the text invests in explaining how Caesar absorbs information, and the protocols by which he processes it, whereby he arrives to the best of all possible decisions. Caesar's deliberations, which ought to occur in a communal context, are completely internalized within the actor Caesar.

Considering Caesar's interest elsewhere in suppressing collaborative discourse, it is remarkable that subordinates turn to this form of decision--making when isolated from their general. In so representing these assemblies the way it does, the text undermines this sort of collectivity by showing how these subordinates cannot handle the process. While there is much that Caesar's soldiers do well, some (e.g., Sabinus) cannot police themselves, nor others (e.g. Cotta) maintain control over them. Other independent decision--making are non--catastrophic due to the subordinates' collaborative decision--making that recommends as little action as possible, such as remaining in camp like Galba (\emph{BG}~3.3) and Q.~Cicero (6.36). Caesar alone, who uses his fear response to heighten rationality, has control over himself.\footnote{Crassus fought because the only option was starvation (\emph{BG}~3.23.8). Only in Labienus's \emph{consilium} does the leader not seek approval (\emph{BG}~7.60.1). The reason for this may be because of his elevated political stature (e.g., he was tribune of the \emph{plebs} in 63).} The deliberative assemblies, especially Cotta and Sabinus's, offer false promises of intersubjectivity, which are undercut so deeply that the text's message is that communicative reason is not a viable option in the world of Caesar's leadership.

The message of the Cotta and Sabinus passage is twofold. On the one hand, his plan to leave the camp is confirmed as a bad one due to its disastrous end. But, of course, Sabinus \textquote{wins} in the debate proper. So the \emph{consilium} is not only a testament to the danger of emotional deliberation, but also to its power. The \emph{Commentarii} express a similar ethical message in their narrative voice, which has elsewhere been observed to avoid judgments of characters, but instead express evaluative statements for the actor Caesar. \textcite[150--155]{riggsby2006} demonstrates the focalization of negative uses of \textquote{\textlatin{barbarus}} through the actor Caesar, while the narrator keeps to the neutral sense of the word. Similarly, in the \emph{Bellum~civile}, \textcite[pp.~148--154 and 154--156]{batstonedamon2006} observe that the narrator does not ascribe judgmental adjectives (like \textquote{\textlatin{timidus}}) to individuals, but prefers abstract nouns like \textquote[p.~154]{\textlatin{timor}}. This also speaks to the narrator's strong inclination not to essentialize actors, but to speak of the forces upon them. 

Thinking back to Aristotle's view of slaves and automata, and the two--part process it endorses, Caesar's subordinates look like automata, in that they ideally have Caesar's commands present within them and ideally interpret the protocols correctly. But in Caesar's absence, Sabinus shows that subordinates and soldiers may easily slide from a mechanistic ideal into something more human and difficult to control. This alternative and negative example of deliberation that the \emph{Bellum~Gallicum} offers underscores how critical Caesar's presence -- physical, communicative, and when necessary psychic -- is to the army and its endeavors.

\cleardoublepage
\clearpage
\phantomsection
\addcontentsline{toc}{chapter}{Conclusion}
%\chapter[ toc-title ][ head-title ]{ title }
%\chapter*[ head-title ]{ title }
\chapter*[Conclusion]{Conclusion}%cccc
\label{ch-conclusion}
\setcounter{footnote}{0}
This dissertation has so far avoided one subject that has consistently fascinated scholars of the \emph{Commentarii} -- their veracity, status as propaganda, \emph{\textfrench{déformations}}, \emph{\textgerman{Tendenz}}, etc.. Over the past sixty years, scholars, agreeing that there is some sort of politically motivated distortion in Julius Caesar's storytelling, have outdone one another in claims about the extent of these distortions or the subtlety of these distortions. The most radical critique has come from \textcite{rambaud1966}, that both texts are artfully and consciously crafted in every way so as to mislead readers. In an influential critique of this thesis, \textcite{collins1972} refutes Rambaud on the \emph{Bellum~Gallicum} but confirms his thesis for the \emph{Bellum~civile}. More recent research departs from this topic, mostly studying how the rhetoric of the text -- its genre, third--person narration, lack of judging adjectives, etc. -- subtly conveys messages about the contents of stories and how a reader should understand them.\footnote{Examples \label{ft-simple-narr} of such scholarship, seen throughout this dissertation, are \textcite{batstone1990}, \textcite{batstone1991}, the edited volumes of \textcite{welchpowell1998} and \textcite{cairnsfantham2003}, \textcite{lieberg2003}, \textcite{kraus2005}, \textcite[esp.~pp.~207--214]{riggsby2006}, \textcite{kraus2007}, \textcite{bartley2008}, and \textcite[159--165]{kraus2009}.} Nevertheless, when writing about Julius Caesar, historical matters are difficult to avoid. I especially like the approach by \textcite[207--214]{riggsby2006}, who delays and confines questions of propaganda until the very final pages of his monograph. For the very reasons that, in a review, \textcite{melchior2006} criticizes Riggsby -- who \textquote{feels compelled in the final few pages to enter into the debate on propaganda and ideology in Caesar's writing} -- I approve of this choice to separate and subordinate the topic of propaganda to what is most important -- the \emph{Bellum~Gallicum}'s stories and storytelling techniques.\footnote{In fuller explanation: \textquote[{\cite{melchior2006}}]{Although this is to some degree unavoidable given the history of Caesarian scholarship, the accusation of an agenda feels flat. Riggsby endorses the label \textquote[][,]{propaganda} but only after defining it with broad strokes}.} 
%anti-teleological
%If we do try to understand the \emph{Commentarii} as somehow political writings, the introduction to \textcite{welchpowell1998} gives an important warning against teleological thinking \textquote[{\cite[ix]{welch1998b}}]{of a Caesar with limited choices and and pre--determined future}. I suggest a similar approach which appreciates that Julius Caesar did not, while writing the \emph{Bellum~Gallicum} in the 50's BC, have foreknowledge of his autocracy in the 40's and certainly not Augustan rule or the Principate.\footnote{The most recent comprehensive account of the publication date of the \emph{Bellum~Gallicum} is in \textcite{wiseman1998}, which argues that the work was gradually written and incrementally published during Caesar's years in Gaul.} From as early as Suetonius's biographies of the emperors, Caesar has commonly been seen as the end of one system (the Roman republic) and the beginning of another (the empire).\footnote{For another imperial vision of the Republic and Caesar, see \textcite[67--101]{gowing2005} on Lucan's portrait of Caesar as a destroyer of memory.} The recent call by \textcite{flower2010}, to reappraise periods of Roman history, falls within this line of thought.\footnote{In an effort to move away from Republic/Principate dichotomies, Flower proposes there multiple republican periods (six, to be exact) which progres gradually to first triumvirate of 59--53 B.C. (p.~33). For Flower's thinking on periodization, see \textcite[18--34]{flower2010}.}
 By briefly approaching this question of distortion with some new information, I would like to follow the lead of Riggsby and consider how the \emph{Bellum~Gallicum} may have interacted with its historical context. Based on this dissertation's three chapters, I here attempt to briefly illustrate how the text may have operated within its historical context, specifically in what appear to be its efforts to preempt or respond to concerns about Caesar's unprecedented proconsulship in Gaul. I first illustrate scholarship's positions on the apologetic nature of the \emph{Commentarii}, then turn to one way in which the \emph{Bellum~Gallicum} appears to behave apologetically, in its apparent response to contemporary concerns of unconstitutional abuses of \emph{imperium} by means of extraordinary delegations of power. In portraying himself as a general to whom \emph{imperium} has been properly delegated, and as a leader who uses \emph{imperium} skilfully and within proper limits, I argue that the \emph{Bellum~Gallicum} may accurately be located within this Roman discourse about the proper use of proconsular power.

% apology 
An important debate in scholarship on the \emph{Commentarii} about whether the \emph{Bellum~Gallicum} is a defense of Caesar's proconsulship, and debate continues about its self--justificatory nature. Some have claimed that in writing the text, Caesar was not apologetic or self--justificatory.\footnote{See \textcite[esp.~p.~923]{collins1972}, \textcite[178--183]{brunt1978}, \textcite[27--29]{hall1998}, \textcite[73--74]{lieberg1998}, and \textcite[saying Caesar needed minimal justification]{brown1999}. \textcite[22--25]{klotz1910} and \textcite[esp.~pp.~216--217]{schlicher1936} write that book~1 is self--justificatory, but the rest of the \emph{Bellum~Gallicum} less so. See \textcite[p.~243 n.~2]{riggsby2006} for some of more anti--justification bibliography.} The usual reason offered by this point of view is that, to the Roman mentality, the expansion of empire (\cite[926--927]{collins1972}) needed no apology, especially if an opponent were a threat to Roman security (\cite{brunt1978}).\footnote{On Cicero's \emph{De provinciis consularibus}: \textquote[{\cite[183]{brunt1978}}]{On this sort of principle no war that Rome could fight against foreign peoples who might some day be strong enough to fight attack her could be other than defensive. There is no indication in the speech that this view was contested. Those who wished to relieve Caesar of his command evidently argued not that his campaigns had been unjust or unnecessary, but that the war was already over}. For more on Cicero and just war theory, see \textcite[158--161]{riggsby2006}.} On the other side of this debate are far more scholars, who think Caesar's actions demanded justification at Rome, and that he wrote the \emph{Bellum~Gallicum} as an apology.\footnote{\textcite[111--133]{rambaud1966}, \textcite[esp.~pp.~184--185]{gardner1983}, \textcite[246--251]{meier1995}, \textcite[pp.~10, 33, and 144--146]{batstonedamon2006}, \textcite[175--189]{riggsby2006}, and \textcite[247--251]{vasaly2009}. See \textcite[195--198]{marincola1997} on early Roman autobiography and its self--justificatory content, as well as the genre's (largely unknown) influence upon Caesar's \emph{Bellum~Gallicum}.} Arguments of this sort generally claim that Caesar overextended his governorship of Gallia Narbonensis and Cisalpina, so that some justification was necessary. In a similar vein, a justificatory narrative is important to the argument of \textcite{rambaud1966}, that Caesar's writing shamelessly distorts events for the purpose of self--promotion.\footnote{\textquote[111]{\textfrench{Dans les \emph{Commentaires}, César représente le cours de l'histoire suivant une logique qui lui es propre et sert ses intérêts. Tous ne l'ont pas vu, d'autant plus que le dépouillement du style fait illusion, et semble purement narratif, étant peu chargé de ces particules logiques chères aux orateurs. C'est que l'écrivain ne voulait pas se donner l'appearence de demounted}}.} This dissertation, especially Chapter~\ref{ch-delib}, supports the thesis that the \emph{Bellum~Gallicum} was written to justify his actions. For Caesar's internal deliberations come at critical times in the narrative, when the actor must decide whether or not to engage in battle, and persuade the reader why Caesar's decision was a good one. These deliberations, as evident in their highly structured form looked at in section~\ref{ch2-delib-caesar}, show that this Roman general has both accounted for endless minutiae yet has his eyes on the larger goal of security for Rome and allied states. More broadly, Chapter~\ref{ch-delib} demonstrates how Caesar has adopted the tradition of the deliberating hero and adapted it to his own narrative ends, in this case a defense of the rationally demonstrable necessity for all of his proconsular actions.

%Cicero and commentators
The plainness of the \emph{Bellum~Gallicum}'s style has long played a role in understandings of its rhetorical aims. Cicero's praise of Caesar's style in the \emph{Brutus} is perhaps most interesting, confirming Caesar's plain style and indirectly speaking to its rhetorical intentions and/or results.\footnote{See also Suetonius, \emph{Divus Iulius} 56 for a later consideration of comments made by Cicero, Aulus Hirtius, and Asinius Pollio. Among Cicero's work, noteworthy are also the extant letters from Caesar to Cicero: \emph{Ad Att.}~172a (9.6a, Loc: \textlatin{in itinere Arpis Brundisium}; Date: c.~iii Non.~Mart.~49), 185 (9.16.2--3), 199b (10.8b, Loc: \textlatin{Massiliam iter faciens}; Date: xv Kal.~Mai.~49); and from Cicero to Caesar: \emph{Ad~fam.}~26 (7.5, Loc: Romae; Date: Apr.~54), 317, (13.15, Date: c.~Mai.(?) 45), 316 (13.16, Loc: Romae(?); Date: ex~45 vel in~44), \emph{Ad~Att.}~178a (9.11a, Date: xiv aut xiii Kal.~Apr.~49). Letter numbers, locations, and dates come from \textcite{shackletonbailey1965} and \textcite{shackletonbailey1977}.} For instance, Brutus's famous observation that the \emph{Commentarii} \textquote[\textquote{are naked, plain, and charming, stripped of all adornment, like a garment} \emph{Brut.}]{\textlatin{nudi sunt, recti et venusti, omni ornatu orationis tamquam veste detracta}}, means that he does not consider them \textquote[{\textquote{history}}]{\textlatin{historiam}}, which would be written by later embellishers. In light of this dissertation, it is significant that Brutus's declaration of the \emph{Commentarii} as non--history lies in what he calls \textquote[\textquote{\textlatin{ornatu orationis}}]{adornment of speech}, which must refer to the activity of a narratorial agent, since there is no shortage of embellishment within, to choose a few examples from this dissertation, Caesar's speeches (e.g., to his soldiers), dialog (e.g., with various Gauls or Ariovistus), thought (e.g., deliberations), and representation of subordinates' self--coordination (e.g., independent deliberations). Chapters~1 and~2 have addressed the important role that the work's narratorial persona takes, and how this influences the goals of the work. Chapter~1 shows how the narrator places cynical thought within the mind of the actor Caesar, thus not needing to editorialize himself. In Chapter~\ref{ch-delib}, I argue that the qualities of the deliberating actor Caesar are projected upon the work's implicit narrator. Thus, the shrewd deliberations that I analyze there become the property of the narrator, though again without having to disrupt the dry qualities of the factual dictation of events. If I were to use Brutus's words, I would say that the \emph{Bellum~Gallicum} could be understood as embellish--able only by the \textquote[{\textquote[][,]{tasteless}} \textquote{inept}]{\textlatin{ineptis}}, while the mentally sound (\textquote{\textlatin{sanos}}) will appreciate Caesar's mental sophistication. This dissertation, and specifically its analysis of actor/author divisions, helps to better explain Romans' inability to reproduce the \emph{Bellum~Gallicum} into history. If, due to its paradoxically unfinished and complete state, perfect in its imperfection, Caesar could enjoy unique control over the apologetic stories about himself.\footnote{There have been many variations in interpretation of Caesar's simple style. See footnote~\ref{ft-simple-narr} (p.~\pageref{ft-simple-narr}) for bibliography, and also Syme's following quote about its interactions within the stylistic economy of autobiography: \textquote{The Republican politician adopted and patronized men of letters to display his magnificence and propagate his fame. The monarchic Pompeius possessed a domestic chronicler, the eloquent Theophanes of Mytilene. Caesar, however, was his own historian in the narratives of the Gallic and Civil Wars, and his own apologist -- the style of his writings was effective, being military and Roman, devoid of pomp and verbosity; and he skilfully made out that his adversaries were petty, vindictive and unpatriotic} \parencite[459--460]{syme1939}.} 

% Hirtius's speaks to a similar paradox (what he calls \textquote{admiratio,} \textquote{admiration}): that the \emph{Bellum~Gallicum} have \textquote[\textquote{both an easy and highest written elegance, and the truest knowledge of explaining his own plans} \emph{BG}~8.pr.7]{\textlatin{cum facultas atque elegantia summa scribendi, tum verissima scientia suorum consiliorum explicandorum}}, yet \textquote[\textquote{how easily and quickly he finished them} \emph{BG}~8.pr.6]{\textlatin{quam facile atque celeriter eos perfecerit}}. As Hirtius constructs this thought, despite the alleged ease of their composition (\textquote{\textlatin{facile}}) mirrors the fluidity of their stylistics (\textquote{\textlatin{facultas atque elegantia summa scribendi}}), the \emph{Commentarii} attain the goal of historiography, the truest knowledge (\textquote{\textlatin{verissima scientia}}). Hirtius seems to implicitly confirm my distinctions between simple narration and elaborate actors' minds with the subject, or \textquote{\textlatin{scientia,}} of the \emph{Commentarii} being \textquote[][.]{\textlatin{suorum consiliorum explicandorum}} Pollio also observes a speed to Caesar's writing, though is critical of it for this reason. According to Suetonius, Pollio wrote that, assuming Caesar did not lie, this speed led to the introduction of errors into the manuscript, due to Caesar's lack of followup or credulity of others' reports.\footnote{\textquote[\textquote{Asinius Pollio thinks that they were composed with insufficient diligence and without enough whole truth, since Caesar trusted too easily actions by others and those by himself he, due to lapse of judgment or memory, published them incorrectly; he thinks that they would have been rewritten and corrected} \emph{Iul.}~56.4]{\textlatin{Pollio Asinius parum diligenter parumque integra ueritate compositos putat, cum Caesar pleraque et quae per alios erant gesta temere crediderit et quae per se, uel consulto uel etiam memoria lapsus perperam ediderit; existimatque rescripturum et correcturum fuisse}}.} The errors that Pollio seems to have in mind are of the contents of stories (e.g, events like the Cotta and Sabinus episode, which Caesar could have only known through others' reports) and not the 

 %What Brutus does not explain, it seems, is why the \emph{Commentarii} cannot be embellished (at least by \textquote[][,]{\textlatin{sanos ldots{} homines}} \textquote{sane men}). A prime reason, I suggest, lies in the fact that there is already a tremendous degree of embellishment within the stories and specifically the actors' minds. So despite the (well--crafted) simplicity of the narrative, due to a reader's projection of complexity (of shrewdness, deliberation, etc.) from Caesar the actor upon Caesar the implicit narrator, what might potential for conversion into \emph{historia} is in fact impossible. 

 %Hirtius's explanation also paradoxical -- quickly written yet elegant (Iul. 56.3); Pollio thinks their flaw is lack of followup (ref Polybian--style critique) ... he believed others' reports too quickly and perhaps was untruthful himself (another Polybius-relatable idea) (Iul. 56.4)

% Caesar's style share that they generally praise (Cicero and Hirtius) or criticize (Pollio) the \emph{Commentarii}, all three invoke how their plain style relates to their truthfulness. 
%next divide these three up into my terms/ideas

% Caesar was also a master communicator, as Cicero attests that the \emph{Commentarii} are \textlatin{\emph{nudi ...?  recti et venusti}}, \textquote{naked, plain, and charming} (\emph{Brutus} 262).\footnote{See \textcite[pp. 14--15 and pp. 99--111]{krostenko2001} on the Catullan tenor of \textlatin{\emph{venustus}} here.} 

%perhaps the most compelling ancient reading belongs to that of \textlatin{Asinius Pollio}, who was of the opinion that Caesar, either purposefully or naively, was too trusting of the reports of his legates.\footnote{\emph{Testimonium} preserved by Suetonius: \textquote[\emph{Iul.}~41.4]{\textlatin{}}.}
 
%from oldass proposal

%Suetonius
%Cicero seizes upon two characteristics to describe C. Iulius Caesar -- his love of power and his skill as a communicator. Testifying to the former, he reports that Caesar always had ready the following lines of Euripides: \textquote{If indeed it is necessary to do wrong, concerning tyranny it is best to do wrong, though the good must act piously in other respects.}\footnote{Spoken by Eteokles in the \emph{Phoinissai} 524--525. \textlatin{Testimonia} at \emph{De officiis 3.82} and Suetonius \emph{Divus Iulius} 30.} 

%jccorr: 3) Your conclusion is still very short, and I’m left wondering what happened to your thoughts on the unspoken deep connection between Habermasian communicative rationality and the exercise of irresistible authority. If you prefer not to elaborate these views at greater length, I would ask that you at least consider how your assessment of Caesar aligns itself with contemporary representations of Caesar by Cicero in his letters and in the Brutus and in Caesar’s own letters (preserved in ad Att. 9 and 10).  
%%notes: talk about Caesar's representation as hyper--rational by Cicero in letters, Caeasr's own letters (support my argument with `touches'), and esp.~in Brutus; 
%kjcorr: I have not addressed Cicero/Caesar's letters. Doing an analysis about whether and how they are plain or objective is too much work; and it would be a distraction -- I should keep my focus on how ppl (scholars, me, and even ancients) see Caesar's style in the BG. Even if he had a dry style in his letters, what importance would it have for our understanding of the issues at hand?


\begin{comment}
to Caesar
26 sb
ad fam. book 7, letter 5, section sa, line 1 (NB. Loc: Romae; Date: Apr. 54)

317
ad fam. book 13, letter 15, section sa, line 1 (NB. Date: c. Mai.(?) 45)

316
ad fam. book 13, letter 16, section sa, line 1 (NB. Loc: Romae(?); Date: ex. 45 vel in. 44)

ad Att. 178a
book 9, letter 11a, section sa, line 1 (NB. Date: xiv aut xiii Kal. Apr. 49)

to Cicero
ad Att. 172a
book 9, letter 6a, section sa, line 1 (NB. Loc: in itinere Arpis Brundisium; Date: c. iii Non. Mart. 49)

ad Att.185
book 9, letter 16, section 2, line sa

ad Att.199b
book 10, letter 8b, section sa, line 1 (NB. Loc: Massiliam iter faciens; Date: xv Kal. Mai. 49 (§2))

\end{comment}

Building on this observation that Caesar appears to justify his excursions into Gaul and beyond, I conclude by attemping to situate the \emph{Bellum~Gallicum} in a contemporary Roman debate concerning the expansion of delegations of \emph{imperium} in the administration of Rome's growing empire. Caesar's text, I conclude, functions as an argument that he uses power in a constitutional way, concerning both the origin of his power and his use of it. % extraordinary delegations
 As such, Caesar's writings could find home within discourses about extraordinary delegations of power toward the end of the Republic. One possible inquiry which has informed some of the background of this dissertation is the crises created by extraordinary delegations of \emph{imperium} that historians have identified as a crucial phenomenon in the dissolution of the Republic. Extraordinary delegations of power are basically of two sorts: \textquote[{\cite[280]{ridley1981}}]{first, military command with a lower or civil magistracy; second, military command with no office, i.e. as a \emph{privatus}}. Though these extraordinary grants of power can be found going far back into Rome's history, there was a dramatic increase of these following Sulla's reforms.\footnote{On historical and constitutional developments of extraordinary delegations of \emph{imperium}, see especially \textcite[esp.~pp.~634--637, also 557--570, 630--639]{brennan2000}, but also \textcite[II: 1.647--662]{mommsen1887}, \textcite{ehrenberg1953}, \textcite[534--543]{gruen1974}, \textcite[59]{lintott1981}, \textcite[281, 290--292]{ridley1981}, \textcite{richardson1991}, and \textcite[114--115]{lintott1999}.} With its concern in conveying Caesar's all--powerful control over his army, I suggest that the \emph{Bellum~Gallicum} goes beyond a typical ancient representation of a general leading an army. Caesar's control over the army and his mental self--control too make sense when understood in an environment permeated by anxieties about the \emph{lex Gabinia}, by which Pompey's unelected subordinates fought battles at their own discretion,\footnote{The \emph{lex Gabinia} (67 B.C.) authorized Pompey to appoint and delegate himself \emph{legati pro praetore}, \emph{legati} who would pursue pirates throughout the Mediterranean, distant enough from him that they could not refer to their leader for immediate decisions (\emph{OCD}\textsuperscript{3}).} and other legal and extra--legal accumulations of power by a handful of individuals. This dissertation's first chapter shows how, even at a physical distance, Caesar's communicative presence establishes his command over all of his men. Chapter~\ref{ch-delib} shows that he has adapted Greek literature to emphasize something similar, his mind's control of the army and events. In what was a noteworthy and novel proconsulship in so many ways (duration, size of army, wealth accrued, etc.), Caesar, as author and actor, takes great pains to make himself appear solely in control, as the one and only decider in his army. So too in Chapter~\ref{ch-delib-subs}, according to his army's protocols, soldiers do not need to think for themselves. In this sense, the \emph{Bellum~Gallicum} could be responding to concerns about the delegation of \emph{imperium} in the provinces, and an argument for how responsibly Caesar exercises it. As seen in comparison to Xenophon's \emph{Anabasis}, Caesar's one--man leadership over his army is not unique in the ancient literary tradition, though the thoroughness of his active control over his men's communication is exceptionally strong. By repressing or downplaying instances of officer--driven decision making, the \emph{Bellum~Gallicum} appears to enter into discourse on extraordinary delegations by situating its hero squarely on the side of Roman tradition.

% senate and people, imperium
Caesar not only delegates action to his subordinate officers, as seen in all three chapters of this dissertation, he is himself authorized to legitimately act by the Roman people and senate. Though, obviously, the Roman senate authorized Caesar as proconsul, throughout the \emph{Bellum~Gallicum} the author reminds readers of the delegative authority given to himself by the Roman senate, and his support by and of the people.\footnote{On constitutional delegations of proconsular power, see \textcite[54--58]{lintott1981} and \textcite[113--115]{lintott1999}. For research into Roman terms for and conceptions of power, see on \emph{auctoritas} \textcite{heinze1925}, \textcite{arendt1958}, and \textcite[pp.~10--20 and \emph{passim}]{galinsky1996}; and on \emph{imperium} \textcite{richardson1991}, \textcite{lincoln1994}, and \textcite{drogula2007}.} His successes are their successes, as evident in their \emph{\textlatin{supplicationes}} (public rituals of thanksgiving), by which the \emph{Bellum~Gallicum} portrays a reciprocal relationship between the proconsul Caesar and the authorizing senate.\footnote{See \emph{Bellum~Gallicum} 2.35.4, 4.38.5, and 7.90.4. On the importance of \emph{\textlatin{supplicationes}} in the \emph{Bellum~Gallicum}, see \textcite[340]{osgood2009}, with bibliography. On those decreed by the senate, \textcite[95--96]{orlin1997}.} Senatorial decrees to actors other than Caesar play a role in the text, too, as in \emph{Bellum~Gallicum}~1, where Caesar debates with Ariovistus the meaning and value of senatorial decrees (here that of \emph{amicus}).\footnote{For a summation of the legal issues at hand, see \textcite[57--58]{lintott1981}, who sees Caesar's protection of all Gaul, from the Germans, as legitimate. References to Caesar's own men increase as the work develops, leading \textcite{welch1998} to conclude that increased showcasing of subordinate officers \textquote[{\cite[102]{welch1998}}]{reflects an author in political as well as military turmoil, far more consciously naming and acknowledging the members of his own class}.} In sections~\ref{bg-legati} and~\ref{bg-conloquium}, I showed that Caesar operates as one who performs the valuable service of correcting foreign allies' interpretations when necessary. As \emph{legati} were delegated by popular authority, though under the guidance of an \emph{imperator}, Caesar depicts himself as a proper overseer of those whom they have elected. He in turn gives authorization to his subordinates for the purpose of accomplishing particular goals (especially apparent in chapters~\ref{ch-delib} and~\ref{ch-delib-subs}), while still maintaining near--perfect control over them (Chapter~\ref{ch-pres}). As a representative of the senate, might the \emph{Bellum~Gallicum} be an argument that Caesar is uniquely suited to reflect and mediate between political power balances at Rome, between the popular assembly and senate? When reflecting upon the legal and political origins of an \emph{imperator} or \emph{legatus}'s delegated power, the \emph{Bellum~Gallicum}'s portrait of communication and interactional relationships has significance that makes sense in light of this conclusion's thesis, that the \emph{Bellum~Gallicum} reflects an historically embedded construction of \emph{imperium} and Caesar's preeminent use of it.

In this conclusion, I have pointed to some probable and promising contexts in which the \emph{Bellum~Gallicum} may be historically situated, based upon the findings of this dissertation's three chapters. This \emph{commentarius}, I argue, functions as an argument, a testament to its author's leadership capacities. In what pretend to be dry annual reports, Caesar makes a case for himself, that he has appropriately and expertly executed the power which has been delegated to him, evident in the narrative pains he takes to demonstrate himself as a faithful servant of the Roman state. Beyond these \textquote{real} efforts, that is how Caesar explains what he has done, Caesar with his army in the \emph{Bellum~Gallicum} is also an analogy, perhaps like Cato's ship of state, illustrating what kind of he leader is; the political interactions that he commands with his subordinates, allies, and enemies; and the precise execution with which he wields the \emph{\textlatin{imperium Romanum}}. Exploiting the literary imagination of his readership (Homer, Xenophon, and Polybius, among others), any of Caesar's goals as an author, propagandistic or not, are not accomplished through a complex portrait of a single man, but of coordinated interaction, of complex processes of labor, of which Caesar is at the same moment just one of many participants, yet one critically important and irreplaceable.

\begin{comment}
% The concept of delegated power is found within thought nearly contemporary to the \emph{Commentarii}. Explaining the etymologies of \emph{imperator} and \emph{legatus} (though of course speciously by our standards), Varro writes that the general's authority comes by means of a delegation from the Roman people, and that of emissary legates from the people and senate. \blockquote[\emph{\textlatin{DLL}} 5.87]{\textlatin{imperator ab imperio populi qui eos, qui id attemptasse<n>t, †oppressi<t> hostis. legati qui lecti publice, quorum opera consilioque uteretur peregre magistratus, quive nuntii senatus aut populi essent.}\footnote{Text from \textcite{goetzschoell1910}, but following emendation of \textquote{\textlatin{oppressi<t>}} in \textcite{kent1938}.}} \blockquote[\emph{DLL} 5.87]{The general (\emph{\textlatin{imperator}}) <was so called because> from the power (\emph{\textlatin{imperium}}) of the people; who [the general] †oppressed enemies who had attacked it [power]. Legates (\emph{\textlatin{legati}}) <were so called because> they were chosen (\textquote{\textlatin{lecti}}) publicly, whose labor and advice magistrates would use while abroad, or who would be messengers of the senate and people.} Thus, an \emph{imperator} like Caesar, and at least in etymology his \emph{legati}, may legitimately act because he has been given \emph{imperium} by the Roman state.
%put this somewhere
%\footnote{The \emph{legati} that Varro has in mind seem to be the ambassador type, though the various types of \emph{legati} \parencite[][identifies three, envoys, lieutenants, and ambassadors]{broughton195152} emerge from a common background \parencite[see][for histories of the \emph{legatus}]{schleussner1978,thomasson1991,brennan2000}. For research into Roman terms for and conceptions of power, see on \emph{auctoritas} \textcite{heinze1925}, \textcite{arendt1958}, and \textcite[pp.~10--20 and \emph{passim}]{galinsky1996}; and on \emph{imperium} \textcite{richardson1991}, \textcite{lincoln1994}, and \textcite{drogula2007}.} 

%kyle: models of leadership, not venues of this thought
%	I am very doubtful that you can draw so direct a line between the Bellum Gallicum and ‘one man rule’.  Surely the Romans of the Republic always made a distinction between the absolute authority over his army exercised by a general with legitimate imperium on the one hand, and the role of a single aristocrat within the polity of his aristocratic peers on the other.
%what is my point to this paragraph?% commented out for now, but may be good material here
%Considering the \emph{Bellum~Gallicum}'s interest in one--man rule, especially strong in administrative and diplomatic matters, this text should perhaps be considered as an argument for -- or an experiment concerning -- Caesar as a would--be authoritarian. Along these lines, I will next take the conclusions of this dissertation and briefly consider several ways that Caesar may have been adapting the \emph{Bellum~Gallicum} in response to contemporary anxieties about \emph{imperium} and his use of it, evident in its construction of, first, Caesar's relationship to the senate and people, (affirmations of his obedience to those from whom his \emph{imperium} originates); second, justifications of his proconsular adventures (his responsible use of \emph{imperium}); and third, illustrations of use of his un--extraordinary use power, unlike several of his contemporaries (the sole and perfect use of \emph{imperium} within his army). In the way that Cato's ship of state (\emph{De~sen.}~17, looked at the Introduction) is a metaphor for one--man leadership through a division of mental and physical labor, I suggest something similar here, that Caesar's leadership of his army is a metaphor for how he would lead the Roman state. %dlcorr: The trouble with your claim that Caesar’s army is a metaphor on a par with the ‘ship of state’ is that the ‘ship of state’ metaphor was standard since archaic Greece and instantly recognisable; you have not shown any comparable cultural recognisability for your claimed ‘army of state’ metaphor.
% By this view, the rhetorical end of the \emph{Bellum~Gallicum} would be convincing its readers that the leader figured in its narratives would be a great political leader and some particular ways how. This conceptualization of autocracy is cast according to communicative and deliberative concepts. The text's proposed leader comes to good decisions by talking with himself and not deliberative bodies, and his plans become actualized through predictable power hierarchies which disseminate what their leader wants done.

 Caesar explains who he is through his subordinates and his army's procedures as much as through his own activities in the text. Attempts, therefore, to understand the \emph{Bellum~Gallicum} as propaganda (\emph{\textfrench{déformation}}, \emph{\textgerman{Tendenz}}, etc.), there must be a full appreciation of the entirety of the elements of the text's narrative.

%syme1939 Chapter 5: the Caesarian Party (59-?77) -look here and compare with gruen1974 section on "Caesar's legates"

To study representations of the use of \emph{imperium} is to study the generals and their subordinates who exercise this power. Further, to study representations of generals' relationship to \emph{imperium} is to study their relationship to communication, for texts contemporary to these innovative delegations frequently create strong correlations between a general's use of power and his use, usually as an alternative, of communication. The handful of contemporary (80 B.C.--14 A.D.) texts that treat innovations to the use of \emph{imperium} directly and in a sustained manner are, \ldots{}
\end{comment}

\begin{appendices}

\chapter{License}
\label{license}
\SingleSpacing
Found here are the human--readable Commons Deed (section~\ref{commons-deed})\footnote{The Commons Deed for this license is also available at %\\
 \href{http://creativecommons.org/licenses/by-nc-sa/3.0/}{http://creativecommons.org/licenses/by-nc-sa/3.0/}.} and complete Legal Code for the Creative Commons Attribution--NonCommercial--ShareAlike 3.0 Unported License \ccbyncsa{} (section~\ref{legal-code}).\footnote{The Legal Code is also available at \href{http://creativecommons.org/licenses/by-nc-sa/3.0/legalcode}{http://creativecommons.org/licenses/by-nc-sa/3.0/legalcode}.}

\section{Commons Deed}
\label{commons-deed}

\begin{itemize}
\item \textbf{You are free:}
   \begin{itemize}
   \item \textbf{to Share} -- to copy, distribute and transmit the work
   \item \textbf{to Remix} -- to adapt the work
   \end{itemize}
\item \textbf{Under the following conditions:}
   \begin{itemize}
   \item \textbf{Attribution} -- You must attribute the work in the manner specified by the author or licensor (but not in any way that suggests that they endorse you or your use of the work). 
   \item \textbf{Noncommercial} -- You may not use this work for commercial purposes. 
   \item \textbf{Share Alike} -- If you alter, transform, or build upon this work, you may distribute the resulting work only under the same or similar license to this one. 
   \end{itemize}  
\item \textbf{With the understanding that:}
   \begin{itemize}
   \item \textbf{Waiver} -- Any of the above conditions can be \textbf{\underline{waived}} if you get permission from the copyright holder.
   \item \textbf{Public Domain} -- Where the work or any of its elements is in the public domain under applicable law, that status is in no way affected by the license. 
   \item \textbf{Other Rights} -- In no way are any of the following rights affected by the license: 
      \begin{itemize}
      \item Your fair dealing or \textbf{\underline{fair use}} rights, or other applicable copyright exceptions and limitations;
      \item The author's \textbf{\underline{moral rights}};
      \item Rights other persons may have either in the work itself or in how the work is used, such as \textbf{\underline{publicity}} or privacy rights.
      \end{itemize}
   \item \textbf{Notice} -- For any reuse or distribution, you must make clear to others the license terms of this work. The best way to do this is with a link to this web page. 
   \end{itemize}  
\end{itemize}

\section{Legal Code}
\label{legal-code}

THE WORK (AS DEFINED BELOW) IS PROVIDED UNDER THE TERMS OF THIS CREATIVE COMMONS PUBLIC LICENSE (``CCPL'' OR ``LICENSE''). THE WORK IS PROTECTED BY COPYRIGHT AND/OR OTHER APPLICABLE LAW. ANY USE OF THE WORK OTHER THAN AS AUTHORIZED UNDER THIS LICENSE OR COPYRIGHT LAW IS PROHIBITED.

BY  EXERCISING ANY RIGHTS TO THE WORK PROVIDED HERE, YOU ACCEPT AND AGREE TO BE BOUND BY THE TERMS OF THIS LICENSE. TO THE EXTENT THIS LICENSE MAY BE CONSIDERED TO BE A CONTRACT, THE LICENSOR GRANTS YOU THE RIGHTS CONTAINED HERE IN CONSIDERATION OF YOUR ACCEPTANCE OF SUCH TERMS AND CONDITIONS.

\begin{enumerate}[1.]
\item \textbf{Definitions}
   \begin{enumerate}[a.]
   \item \textbf{``Adaptation''} means a work based upon the Work, or upon the Work and other pre--existing works, such as a translation, adaptation, derivative work, arrangement of music or other alterations of a literary or artistic work, or phonogram or performance and includes cinematographic adaptations or any other form in which the Work may be recast, transformed, or adapted including in any form recognizably derived from the original, except that a work that constitutes a Collection will not be considered an Adaptation for the purpose of this License. For the avoidance of doubt, where the Work is a musical work, performance or phonogram, the synchronization of the Work in timed--relation with a moving image (``synching'') will be considered an Adaptation for the purpose of this License.
   \item \textbf{``Collection''} means a collection of literary or artistic works, such as encyclopedias and anthologies, or performances, phonograms or broadcasts, or other works or subject matter other than works listed in Section 1(g) below, which, by reason of the selection and arrangement of their contents, constitute intellectual creations, in which the Work is included in its entirety in unmodified form along with one or more other contributions, each constituting separate and independent works in themselves, which together are assembled into a collective whole. A work that constitutes a Collection will not be considered an Adaptation (as defined above) for the purposes of this License.
   \item \textbf{``Distribute''} means to make available to the public the original and copies of the Work or Adaptation, as appropriate, through sale or other transfer of ownership.
   \item \textbf{``License Elements''} means the following high--level license attributes as selected by Licensor and indicated in the title of this License: Attribution, Noncommercial, ShareAlike.
   \item \textbf{``Licensor''} means the individual, individuals, entity or entities that offer(s) the Work under the terms of this License.
   \item \textbf{``Original Author''} means, in the case of a literary or artistic work, the individual, individuals, entity or entities who created the Work or if no individual or entity can be identified, the publisher; and in addition (i) in the case of a performance the actors, singers, musicians, dancers, and other persons who act, sing, deliver, declaim, play in, interpret or otherwise perform literary or artistic works or expressions of folklore; (ii) in the case of a phonogram the producer being the person or legal entity who first fixes the sounds of a performance or other sounds; and, (iii) in the case of broadcasts, the organization that transmits the broadcast.
   \item \textbf{``Work''} means the literary and/or artistic work offered under the terms of this License including without limitation any production in the literary, scientific and artistic domain, whatever may be the mode or form of its expression including digital form, such as a book, pamphlet and other writing; a lecture, address, sermon or other work of the same nature; a dramatic or dramatico--musical work; a choreographic work or entertainment in dumb show; a musical composition with or without words; a cinematographic work to which are assimilated works expressed by a process analogous to cinematography; a work of drawing, painting, architecture, sculpture, engraving or lithography; a photographic work to which are assimilated works expressed by a process analogous to photography; a work of applied art; an illustration, map, plan, sketch or three--dimensional work relative to geography, topography, architecture or science; a performance; a broadcast; a phonogram; a compilation of data to the extent it is protected as a copyrightable work; or a work performed by a variety or circus performer to the extent it is not otherwise considered a literary or artistic work.
   \item \textbf{``You''} means an individual or entity exercising rights under this License who has not previously violated the terms of this License with respect to the Work, or who has received express permission from the Licensor to exercise rights under this License despite a previous violation.
   \item \textbf{``Publicly Perform''} means to perform public recitations of the Work and to communicate to the public those public recitations, by any means or process, including by wire or wireless means or public digital performances; to make available to the public Works in such a way that members of the public may access these Works from a place and at a place individually chosen by them; to perform the Work to the public by any means or process and the communication to the public of the performances of the Work, including by public digital performance; to broadcast and rebroadcast the Work by any means including signs, sounds or images.
  \item \textbf{``Reproduce''} means to make copies of the Work by any means including without limitation by sound or visual recordings and the right of fixation and reproducing fixations of the Work, including storage of a protected performance or phonogram in digital form or other electronic medium.
   \end{enumerate}
\item \textbf{Fair Dealing Rights.} Nothing in this License is intended to reduce, limit, or restrict any uses free from copyright or rights arising from limitations or exceptions that are provided for in connection with the copyright protection under copyright law or other applicable laws.
\item \textbf{License Grant.} Subject to the terms and conditions of this License, Licensor hereby grants You a worldwide, royalty--free, non--exclusive, perpetual (for the duration of the applicable copyright) license to exercise the rights in the Work as stated below:
   \begin{enumerate}[a.]
   \item to Reproduce the Work, to incorporate the Work into one or more Collections, and to Reproduce the Work as incorporated in the Collections;
   \item to create and Reproduce Adaptations provided that any such Adaptation, including any translation in any medium, takes reasonable steps to clearly label, demarcate or otherwise identify that changes were made to the original Work. For example, a translation could be marked ``The original work was translated from English to Spanish,'' or a modification could indicate ``The original work has been modified.'';
   \item to Distribute and Publicly Perform the Work including as incorporated in Collections; and,
   \item to Distribute and Publicly Perform Adaptations.
   \end{enumerate}

The above rights may be exercised in all media and formats whether now known or hereafter devised. The above rights include the right to make such modifications as are technically necessary to exercise the rights in other media and formats. Subject to Section 8(f), all rights not expressly granted by Licensor are hereby reserved, including but not limited to the rights described in Section 4(e).

\item \textbf{Restrictions.} The license granted in Section 3 above is expressly made subject to and limited by the following restrictions:
   \begin{enumerate}[a.]
   \item You may Distribute or Publicly Perform the Work only under the terms of this License. You must include a copy of, or the Uniform Resource Identifier (URI) for, this License with every copy of the Work You Distribute or Publicly Perform. You may not offer or impose any terms on the Work that restrict the terms of this License or the ability of the recipient of the Work to exercise the rights granted to that recipient under the terms of the License. You may not sublicense the Work. You must keep intact all notices that refer to this License and to the disclaimer of warranties with every copy of the Work You Distribute or Publicly Perform. When You Distribute or Publicly Perform the Work, You may not impose any effective technological measures on the Work that restrict the ability of a recipient of the Work from You to exercise the rights granted to that recipient under the terms of the License. This Section 4(a) applies to the Work as incorporated in a Collection, but this does not require the Collection apart from the Work itself to be made subject to the terms of this License. If You create a Collection, upon notice from any Licensor You must, to the extent practicable, remove from the Collection any credit as required by Section 4(d), as requested. If You create an Adaptation, upon notice from any Licensor You must, to the extent practicable, remove from the Adaptation any credit as required by Section 4(d), as requested.
   \item You may Distribute or Publicly Perform an Adaptation only under: (i) the terms of this License; (ii) a later version of this License with the same License Elements as this License; (iii) a Creative Commons jurisdiction license (either this or a later license version) that contains the same License Elements as this License (e.g., Attribution--NonCommercial--ShareAlike 3.0 US) (``Applicable License''). You must include a copy of, or the URI, for Applicable License with every copy of each Adaptation You Distribute or Publicly Perform. You may not offer or impose any terms on the Adaptation that restrict the terms of the Applicable License or the ability of the recipient of the Adaptation to exercise the rights granted to that recipient under the terms of the Applicable License. You must keep intact all notices that refer to the Applicable License and to the disclaimer of warranties with every copy of the Work as included in the Adaptation You Distribute or Publicly Perform. When You Distribute or Publicly Perform the Adaptation, You may not impose any effective technological measures on the Adaptation that restrict the ability of a recipient of the Adaptation from You to exercise the rights granted to that recipient under the terms of the Applicable License. This Section 4(b) applies to the Adaptation as incorporated in a Collection, but this does not require the Collection apart from the Adaptation itself to be made subject to the terms of the Applicable License.
   \item You may not exercise any of the rights granted to You in Section 3 above in any manner that is primarily intended for or directed toward commercial advantage or private monetary compensation. The exchange of the Work for other copyrighted works by means of digital file--sharing or otherwise shall not be considered to be intended for or directed toward commercial advantage or private monetary compensation, provided there is no payment of any monetary compensation in con--nection with the exchange of copyrighted works.
   \item If You Distribute, or Publicly Perform the Work or any Adaptations or Collections, You must, unless a request has been made pursuant to Section 4(a), keep intact all copyright notices for the Work and provide, reasonable to the medium or means You are utilizing: (i) the name of the Original Author (or pseudonym, if applicable) if supplied, and/or if the Original Author and/or Licensor designate another party or parties (e.g., a sponsor institute, publishing entity, journal) for attribution (``Attribution Parties'') in Licensor's copyright notice, terms of service or by other reasonable means, the name of such party or parties; (ii) the title of the Work if supplied; (iii) to the extent reasonably practicable, the URI, if any, that Licensor specifies to be associated with the Work, unless such URI does not refer to the copyright notice or licensing information for the Work; and, (iv) consistent with Section 3(b), in the case of an Adaptation, a credit identifying the use of the Work in the Adaptation (e.g., ``French translation of the Work by Original Author,'' or ``Screenplay based on original Work by Original Author''). The credit required by this Section 4(d) may be implemented in any reasonable manner; provided, however, that in the case of a Adaptation or Collection, at a minimum such credit will appear, if a credit for all contributing authors of the Adaptation or Collection appears, then as part of these credits and in a manner at least as prominent as the credits for the other contributing authors. For the avoidance of doubt, You may only use the credit required by this Section for the purpose of attribution in the manner set out above and, by exercising Your rights under this License, You may not implicitly or explicitly assert or imply any connection with, sponsorship or endorsement by the Original Author, Licensor and/or Attribution Parties, as appropriate, of You or Your use of the Work, without the separate, express prior written permission of the Original Author, Licensor and/or Attribution Parties.
   \item For the avoidance of doubt:
      \begin{enumerate}
         \item \begin{midsloppypar} \textbf{Non--waivable  Compulsory License Schemes.} In those jurisdictions in which the right to collect royalties through any statutory or compulsory lic\-en\-sing scheme cannot be waived, the Licensor reserves the exclusive right to col\-lect such royalties for any exercise by You of the rights granted under this License; \end{midsloppypar}
         \item \textbf{Waivable Compulsory License Schemes.} In those jurisdictions in which the right to collect royalties through any statutory or compulsory licensing scheme can be waived, the Licensor reserves the exclusive right to collect such royalties for any exercise by You of the rights granted under this License if Your exercise of such rights is for a purpose or use which is otherwise than noncommercial as permitted under Section 4(c) and otherwise waives the right to collect royalties through any statutory or compulsory licensing scheme; and,
         \item \textbf{Voluntary License Schemes.} The Licensor reserves the right to collect royalties, whether individually or, in the event that the Licensor is a member of a collecting society that administers voluntary licensing schemes, via that society, from any exercise by You of the rights granted under this License that is for a purpose or use which is otherwise than noncommercial as permitted under Section 4(c).
         \end{enumerate}
   \item Except as otherwise agreed in writing by the Licensor or as may be otherwise permitted by applicable law, if You Reproduce, Distribute or Publicly Perform the Work either by itself or as part of any Adaptations or Collections, You must not distort, mutilate, modify or take other derogatory action in relation to the Work which would be prejudicial to the Original Author's honor or reputation. Licensor agrees that in those jurisdictions (e.g. Japan), in which any exercise of the right granted in Section 3(b) of this License (the right to make Adaptations) would be deemed to be a distortion, mutilation, modification or other derogatory action prejudicial to the Original Author's honor and reputation, the Licensor will waive or not assert, as appropriate, this Section, to the fullest extent permitted by the applicable national law, to enable You to reasonably exercise Your right under Section 3(b) of this License (right to make Adaptations) but not otherwise.
   \end{enumerate}

\item \textbf{Representations, Warranties and Disclaimer}

\begin{midsloppypar}
UNLESS OTHERWISE MUTUALLY AGREED TO BY THE PARTIES IN WRITING AND TO THE FULLEST EXTENT PERMITTED BY APPLICABLE LAW, LICENSOR OFFERS THE WORK AS--IS AND MAKES NO REPRESENTATIONS OR WARRANTIES OF ANY KIND CONCERNING THE WORK, EXPRESS, IMPLIED, STATUTORY OR OTHERWISE, INCLUDING, WITHOUT LIMITATION, WARRANTIES OF TITLE, MERCHANTABILITY, FITNESS FOR A PARTICULAR PURPOSE, NONINFRINGEMENT, OR THE ABSENCE OF LATENT OR OTHER DEFECTS, ACCURACY, OR THE PRESENCE OF ABSENCE OF ERRORS, WHETHER OR NOT DISCOVERABLE. SOME JURISDICTIONS DO NOT ALLOW THE EXCLUSION OF IMPLIED WARRANTIES, SO THIS EXCLUSION MAY NOT APPLY TO YOU.
\end{midsloppypar}

\item \textbf{Limitation on Liability.} EXCEPT TO THE EXTENT REQUIRED BY APPLICABLE LAW, IN NO EVENT WILL LICENSOR BE LIABLE TO YOU ON ANY LEGAL THEORY FOR ANY SPECIAL, INCIDENTAL, CONSEQUENTIAL, PUNITIVE OR EXEMPLARY DAMAGES ARISING OUT OF THIS LICENSE OR THE USE OF THE WORK, EVEN IF LICENSOR HAS BEEN ADVISED OF THE POSSIBILITY OF SUCH DAMAGES.

\item \textbf{Termination}
   \begin{enumerate}[a.]
   \item This License and the rights granted hereunder will terminate automatically upon any breach by You of the terms of this License. Individuals or entities who have received Adaptations or Collections from You under this License, however, will not have their licenses terminated provided such individuals or entities remain in full compliance with those licenses. Sections 1, 2, 5, 6, 7, and 8 will survive any termination of this License.
   \item Subject to the above terms and conditions, the license granted here is perpetual (for the duration of the applicable copyright in the Work). Notwithstanding the above, Licensor reserves the right to release the Work under different license terms or to stop distributing the Work at any time; provided, however that any such election will not serve to withdraw this License (or any other license that has been, or is required to be, granted under the terms of this License), and this License will continue in full force and effect unless terminated as stated above.
   \end{enumerate}
\item \textbf{Miscellaneous}
   \begin{enumerate}[a.]
   \item Each time You Distribute or Publicly Perform the Work or a Collection, the Licensor offers to the recipient a license to the Work on the same terms and conditions as the license granted to You under this License.
   \item Each time You Distribute or Publicly Perform an Adaptation, Licensor offers to the recipient a license to the original Work on the same terms and conditions as the license granted to You under this License.
   \item If any provision of this License is invalid or unenforceable under applicable law, it shall not affect the validity or enforceability of the remainder of the terms of this License, and without further action by the parties to this agreement, such provision shall be reformed to the minimum extent necessary to make such provision valid and enforceable.
   \item No term or provision of this License shall be deemed waived and no breach consented to unless such waiver or consent shall be in writing and signed by the party to be charged with such waiver or consent.
   \item This License constitutes the entire agreement between the parties with respect to the Work licensed here. There are no understandings, agreements or representations with respect to the Work not specified here. Licensor shall not be bound by any additional provisions that may appear in any communication from You. This License may not be modified without the mutual written agreement of the Licensor and You.
   \item The rights granted under, and the subject matter referenced, in this License were drafted utilizing the terminology of the Berne Convention for the Protection of Literary and Artistic Works (as amended on September 28, 1979), the Rome Convention of 1961, the WIPO Copyright Treaty of 1996, the WIPO Performances and Phonograms Treaty of 1996 and the Universal Copyright Convention (as revised on July 24, 1971). These rights and subject matter take effect in the relevant jurisdiction in which the License terms are sought to be enforced according to the corresponding provisions of the implementation of those treaty provisions in the applicable national law. If the standard suite of rights granted under applicable copyright law includes additional rights not granted under this License, such additional rights are deemed to be included in the License; this License is not intended to restrict the license of any rights under applicable law.
   \end{enumerate}
\end{enumerate}

%\end{comment}

\end{appendices}

\newpage

\clearpage

\backmatter

\begin{OnehalfSpace}
\setlength{\bibitemsep}{\onelineskip}
\printbibliography

\end{OnehalfSpace}

\end{document}
